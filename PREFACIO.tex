\chapter*{Prefácio}

\addcontentsline{toc}{chapter}{Prefácio}

\hedramarkboth{Prefácio}{}

Em \emph{Cioran, l'hérétique}, biografia intelectual do ensaísta romeno
--- e filósofo dostoievskiano --- Emil Cioran, o jornalista francês
Patrice Bollon faz uma breve e aguda observação que pode servir como
porta de entrada para este \emph{Dostoiévski e a dialética: fetichismo
da forma, utopia como conteúdo}, de Flávio Ricardo Vassoler. Depois de
afirmar que na França (e, poderíamos acrescentar, na maioria dos países
ocidentais) não é comum que se veja em Dostoiévski mais do que um grande
romancista e um genial psicólogo das profundezas da alma humana, Bollon
lembra que, para leitores e intelectuais russos como Chestov e Berdiaev,
o escritor é considerado um pensador metafísico, derivando de sua visão
do homem uma filosofia política que, de resto, é considerada
``reacionária''.

A rica fortuna crítica de Dostoiévski no Brasil e a vigorosa escola de
tradutores do russo que aqui se formou, tendo como ponto de partida
justamente a publicação das obras do autor de \emph{Crime e castigo},
não escapa à observação de Bollon --- o que torna ainda mais preciosa a
contribuição de Vassoler. Originalmente tese de doutorado do escritor e
pesquisador paulista, \emph{Dostoiévski e a dialética} extrai da
ficção do romancista e contista russo, bem como de seus ensaios, cartas
e textos jornalísticos ou memorialísticos, uma ``teleologia teológica''
como horizonte de superação ainda inconcluso --- à época do escritor e em
nossa própria época --- das aporias de sua forma de pensar a política (em
especial o socialismo) e o cristianismo (em especial a confissão
ortodoxa).

Sem deixar de fazer um \emph{close reading} de diferentes passagens da
obra de Dostoiévski, dentro de uma linhagem de intérpretes que irradia
de Boris Schnaiderman e Paulo Bezerra, Vassoler identifica nele uma
antropologia de tipo teológico-metafísico que, por seu caráter
escatológico (no sentido de doutrina sobre o destino último da
humanidade e sobre o juízo final), tem implicações éticas e pragmáticas:
suas personagens encarnam ações e valores antitéticos que, no âmbito dos
vetores ocidentalistas/racionalistas e eslavófilos/espiritualistas do
contexto histórico e ideológico da Rússia tsarista, apontam
simultaneamente para uma tentação niilista e para um desejo de redenção
que permanecem abertos, irresolvidos. Mas que --- eis uma primeira
afirmação, tão polêmica quanto original, nessa obra repleta de
iluminações polêmicas e originais --- já apontam para a síntese dialética
superadora das contradições de nosso mundo cindido, de nossa existência
alienada da totalidade, na forma de uma utopia histórico-transcendental
\emph{realizável}.

Dostoiévski utopista, Dostoiévski profeta, nada disso é novidade na
seara da recepção do escritor --- e Vassoler, a rigor, não endossa tal
imagem, em geral mais retórica do que conceitualmente consistente. A
diferença é que, ao detectar elementos utópicos ou proféticos em
Dostoiévski, ele reivindica ``o teor de verdade do conteúdo de suas
obras como a articulação de uma filosofia da história''. Ou seja, a
utopia, em Dostoiévski, só permanece sendo um ``não-lugar'' (conforme o
significado da expressão cunhada por Thomas More há 500 anos) porque as
condições objetivas para seu acontecer ainda não vieram ao ponto crítico
de fusão e transformação --- mas elas podem vir, ou fatalmente virão,
quando então o hesitante vaticínio de Lukács, ao final de \emph{A Teoria
do romance,} terá se cumprido: Dostoiévski como Homero ou Dante de um
mundo novo, o arauto do momento em que deixamos ``o estado da absoluta
pecaminosidade'' (o pecaminoso, aqui, entendido não em seu significado
catequético, mas como dimensão da Queda bíblica que, no plano literário,
fez do romance uma ``epopeia do mundo abandonado por deus'', segundo a
fórmula do pensador húngaro).

Ao mesmo tempo, as tais condições objetivas de superação de nosso mundo
cindido --- que este livro de Vassoler nos convida a enxergar em
Dostoiévski --- são tudo, menos objetivas: dependem daquilo que ele chama
de ``cicatrização do espírito'', por sua vez uma vivência que pertence à
ordem supramundana ou, no mínimo, à esfera da decisão soteriológica
ilustrada pela parábola de \emph{O sonho de um homem ridículo}.

A utopia de Dostoiévski seria então realizável \emph{porque} histórica e
transcendental \emph{porque} dependente de um voluntarismo espiritual ---
proposição que coloca Vassoler numa dupla perspectiva: de um lado,
materialista-dialética, explicitamente baseada na Teoria Crítica de
Adorno; de outro, religiosa, espiritualista ou mesmo espírita, já que um
dos autores mobilizados por seu arsenal teórico é ninguém menos do que
Allan Kardec.

Essa é outra proposição audaciosa do livro, considerando que o criador
do espiritismo não goza de prestígio intelectual nem entre aqueles que
leem Dostoiévski em chave estritamente literária, nem entre aqueles que,
assimilando o escritor russo ao campo da filosofia da história ou da
filosofia da religião, descartam o kardecismo como doutrina e matéria de
fé sem ossatura teológica. E fato é que o próprio Dostoiévski faz
considerações ambíguas sobre o espiritismo em seu \emph{Diário de um
escritor}, no qual há três artigos (datados do ano de 1876) que
transitam da sátira à dúvida, mas nos quais ele também ridiculariza
aqueles que descartam de modo automático qualquer experiência mística ou
de transcendência.

Num livro que dialoga tanto com Hegel, Marx, Lukács e Adorno quanto com
teóricos de literatura como (e em primeiro lugar) Mikhail Bakhtin,
Vassoler não está colocando a doutrina de Kardec no mesmo registro de
decodificação do real, mas no registro de um alargamento do existente
que aparece nas cogitações de Dostoiévski e de suas personagens.
Trata"-se, antes, de localizar na obra do escritor um movimento de
negação determinada ora do socialismo (utópico e científico), ora do
irracionalismo religioso, para então atingir a superação dialética que
opera uma cicatrização das feridas espirituais sem as quais não seria
possível fazer triunfar, no tempo da história e do mundo, aquela
teleologia teológica que tem como premissa inegociável a dimensão
supramundana do homem: ``O teor de verdade de igualdade e justiça do
socialismo'', escreve Vassoler, ``é rearticulado com o princípio cristão
de eternidade da alma --- segundo Dostoiévski, a base substancial para a
constituição da eticidade objetiva para além do niilismo que só faz
sentenciar que \emph{se Deus não existe, tudo é permitido} ---, de modo a
que (\ldots) seja possível vislumbrar a filosofia da história como o ímpeto
de reconciliação da utopia''.

Por aí se vê o quanto a leitura dialética de Dostoiévski proposta por
Vassoler não apenas se coloca na esteira daquela mirada que entrevê no
escritor um pensador e um filósofo da história como, ao iluminar os
conteúdos emancipatórios de sua utopia histórico"-transcendental, vai na
contramão da imagem de um engajamento reacionário que a ele se atrelou a
despeito do caráter revolucionário de seu ``romance polifônico''. E a
menção ao conceito de Mikhail Bakhtin, formulado em \emph{Problemas da
poética de Dostoiévski}, para descrever a miríade de vozes que habita os
livros do escritor serve aqui para ressaltar o verdadeiro \emph{tour de
force} que Vassoler realiza em relação ao teórico russo, num corpo a
corpo com as noções bakhtinianas de polifonia e dialogismo.

Pois se, num primeiro momento, a ``impossibilidade de criar uma nova
totalidade a partir da voz particular de uma determinada personagem''
coloca Bakhtin em consonância com Lukács e sua visão do romance
dostoievskiano como expressão do colapso histórico do mundo objetivo (no
qual seria igualmente impossível atingir uma nova totalidade a partir da
perspectiva fragmentada das personagens do escritor), num segundo
momento, Vassoler entra em diatribe com Bakhtin e, munido da Teoria
Crítica, argumenta que Bakhtin, ao ontologizar ou essencializar a
dialogia, proscreve a dimensão temporal (o elemento mesmo dos
desenvolvimentos e das transformações dialéticas) e transforma os
romances de Dostoiévski em verdadeiras catedrais polifônicas, estáticas
no tempo e sem devir histórico. Com isso (e conforme o subtítulo do
presente livro) a forma dialógica se transmuta num fetichismo da forma
que, na ausência de uma percepção dos movimentos (transcendentais,
espirituais) de resistência à reificação existentes em Dostoiévski,
oblitera o acesso a seus conteúdos utópicos e emancipadores, dentro de
um processo no qual Vassoler --- em mais um momento de inflexão polêmica
--- aproxima o dialogismo de Dostoiévski à dinâmica marxista de alienação
do trabalho, fetichização da mercadoria e criação de mais"-valia (que se
transforma em ``menos-valia'' no subsolo e na estepe da vida danificada
das personagens dostoievskianas).

A reflexão de Vassoler é suficientemente clara, obsessivamente
minuciosa, para que suas reviravoltas teóricas fiquem apenas indicadas,
sem maior aprofundamento, nesta apresentação de \emph{Dostoiévski e a
dialética.} Faltaria ressaltar, entretanto, algo que talvez só o
privilégio de conhecer o autor pessoalmente permita perceber nas
entrelinhas de sua escrita: uma paixão pela discussão de ideias e uma
urgência na leitura de Dostoiévski que colocam suas altercações com
leitores e teóricos dostoievskianos no mesmo diapasão das famosas
``cenas de conclave'' (expressão de Leonid Grossman) em que as
personagens do escritor russo defendem pontos de vista como se disso
dependesse a salvação do mundo e a redenção de suas almas. Mas creio que
mesmo o leitor que não conheça de perto Flávio Ricardo Vassoler poderá
sentir, após percorrer as páginas incandescentes deste livro, aquilo que
pode ser resumido pelas seguintes palavras: ardor e coragem intelectual.

\begin{flushright}
\emph{Manuel da Costa Pinto}
\end{flushright}
