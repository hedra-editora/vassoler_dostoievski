%!TEX root=./LIVRO.tex

\chapter*{}

\vspace*{\fill}

%\begin{flushright}
\ \ \quad \ 
\begin{minipage}[c]{0.84\textwidth}
\scriptsize\emph{As Segunda e Terceira Frentes Bielorrussas irromperam no Leste da
Prússia, já território alemão, numa orgia de vingança: 2 milhões de
mulheres alemãs seriam estupradas nos meses seguintes. \emph{Os soldados
russos chegaram a violar mulheres russas recém-libertadas de campos
nazistas} [grifo cúmplice de Svidrigáilov, Rogójin, Stavróguin, do
homem do subsolo e do homem ridículo]. Ora, você leu Dostoiévski, é
claro. Vê como a alma humana é complicada? Então, imagine um homem que
lutou de Stalingrado a Belgrado -- por milhares de quilômetros de sua
própria terra devastada, por cima dos cadáveres de seus camaradas e
entes mais queridos. Como pode um homem assim reagir normalmente? E o
que há de tão medonho em se divertir com uma mulher depois de tais
horrores?}

\smallskip
\hspace*{\fill}-- Ióssif Vissariónovitch Djugachvili,\\
\hspace*{\fill} também conhecido como Stálin\footnotemark
\end{minipage}
\footnotetext{Simon Sebag Montefiore,
  \emph{Stálin: A corte do tzar Vermelho}. São Paulo: Companhia das
  Letras, 2006, p. 532.}

\bigskip

\, \ 
\begin{minipage}{0.84\textwidth}
\scriptsize\emph{Eu vi a verdade, eu vi e sei que as pessoas podem ser belas e felizes,
sem perder a capacidade de viver na terra. Não quero e não posso
acreditar que o mal seja o estado normal dos homens.}

\smallskip
\hspace*{\fill}-- Homem ridículo, ``O sonho de um homem ridículo''\footnotemark
\end{minipage}
\footnotetext{Tradução de Vadim Nikitin. São Paulo: Editora 34, 2003, p. 121 e p. 122.}
%\end{flushright}

\thispagestyle{empty}