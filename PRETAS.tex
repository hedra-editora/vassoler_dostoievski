
\textbf{Flávio Ricardo Vassoler}, escritor e professor, é doutor em Teoria Literária e Literatura Comparada pela \versal{FFLCH-USP}, com pós-doutorado em Literatura Russa pela Northwestern University (\versal{EUA}). É autor das obras literárias \emph{O Evangelho segundo Talião} (nVersos, 2013) e \emph{Tiro de Misericórdia} (nVersos, 2014), além de ter organizado o livro de ensaios \emph{Fiódor Dostoiévski e Ingmar Bergman: O niilismo da modernidade} (Intermeios, 2012). Escreve, periodicamente, para o caderno literário ``Aliás'', do jornal \emph{O Estado de S. Paulo}. Mantém a página virtual Portal Heráclito (\emph{www.portalheraclito.com.br}).

\textbf{Dostoiévski e a dialética: fetichismo da forma, utopia como conteúdo} constitui uma análise absolutamente original sobre a obra do escritor russo Fiódor Dostoiévski, dividida em duas partes dialeticamente articuladas: uma tese, em que a interpretação canônica da ``catedral polifônica'', consagrada pelo crítico russo Mikhail Bakhtin, é posta em cheque à luz de um constante diálogo com a Teoria Crítica frankfurtiana, especialmente Adorno, e com a forma mercadoria analisada por Marx; e uma antítese, que pensa as complexas relações de Dostoiévski com o socialismo e o cristianismo, contrapondo aí polos como Allan Kardec e Hegel, Cristo e Marx, Bakhtin e Adorno etc.

\textbf{Manuel da Costa Pinto} é mestre em Teoria Literária e Literatura Comparada pela Universidade de São Paulo (\versal{USP}), jornalista e crítico literário. Foi um dos criadores da Revista \versal{CULT}, da qual foi editor por seis anos. Autor dos livros \emph{Albert Camus: um elogio do ensaio} (Ateliê, 1998), \emph{Literatura brasileira hoje} (Publifolha, 2004) e \emph{Paisagens interiores e outros ensaios} (B4 Editora, 2012). Organizador das antologias \emph{A Inteligência e o cadafalso} (Record, 1998), de Albert Camus, \emph{Antologia comentada da poesia brasileira do século 21} (Publifolha, 2006) e \emph{Crônica brasileira contemporânea} (Editora Moderna, 2013), entre outras.






