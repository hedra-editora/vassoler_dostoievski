
\textbf{Flávio Ricardo Vassoler} é doutor em Letras pela \versal{FFLCH-USP}, com estágio doutoral junto à
  Northwestern University (\versal{EUA}). É autor das obras literárias \emph{Tiro
  de Misericórdia} (nVersos, 2014) e \emph{O Evangelho segundo Talião}
  (nVersos, 2013) e organizador do livro de ensaios \emph{Dostoiévski e
  Bergman: o niilismo da modernidade} (Intermeios, 2012).

\textbf{Dostoiévski e a dialética: fetichismo da forma, utopia como conteúdo} põe em xeque a interpretação mais famosa da obra de Fiódor Dostoiévski, eternizada pelo crítico soviético Mikhail Bakhtin em \emph{Problemas da poética de Dotoiévski}, para quem a obra do russo comporia uma ``catedral polifônica", ou seja: um concerto de vozes em harmonia erigida em bases sólidas. Para Vassoler, a ideia de harmonia vai de encontro ao caráter contraditório, dialético, da chamada catedral dostoievskiana. No cerne de seu argumento, o autor vai “aos porões”, ao que subjaz a essa construção harmoniosa para nela “insuflar ar dialético” e assim enxergar sua contraditoriedade latente. As vozes em harmonia de Dostoiévski comporiam, em sua investigação do indivíduo, um diagnóstico do capitalismo insurgente, em que o sujeito é também súdito: esse coro estaria abafado, enforcado pelas cordas vocais do subsolo. Vassoler então ressignifica a catedral e pensa uma polifonia dialética, vozes que contraporiam cristianismo e socialismo, Allan Kardec e Hegel, Cristo e Marx, Bakhtin e Adorno, contradições cuja síntese, a catedral, seria fruto de oposições e não mais de harmonia.  

\textbf{Manuel da Costa Pinto} é mestre em Teoria Literária e Literatura Comparada pela Universidade de São Paulo (\versal{USP}), jornalista e crítico literário. Foi um dos criadores da Revista \versal{CULT}, da qual foi editor por seis anos. Autor dos livros \emph{Albert Camus: um elogio do ensaio} (Ateliê, 1998), \emph{Literatura brasileira hoje} (Publifolha, 2004) e \emph{Paisagens interiores e outros ensaios} (B4 Editora, 2012). Organizador das antologias \emph{A Inteligência e o cadafalso} (Record, 1998), de Albert Camus, \emph{Antologia comentada da poesia brasileira do século 21} (Publifolha, 2006) e \emph{Crônica brasileira contemporânea} (Editora Moderna, 2013), entre outras.






