%!TEX root=./LIVRO.tex

%\textbf{DOSTOIÉVSKI E A DIALÉTICA}

%\textbf{FETICHISMO DA FORMA, UTOPIA COMO CONTEÚDO}

%\textbf{FLÁVIO RICARDO VASSOLER}

%EPÍGRAFE!!!
%\begin{quote}
%As Segunda e Terceira Frentes Bielorrussas irromperam no Leste da
%Prússia, já território alemão, numa orgia de vingança: 2 milhões de
%mulheres alemãs seriam estupradas nos meses seguintes. \emph{Os soldados
%russos chegaram a violar mulheres russas recém"-libertadas de campos
%nazistas} {[}grifo cúmplice de Svidrigáilov, Rogójin, Stavróguin, do
%homem do subsolo e do homem ridículo{]}. Ora, você leu Dostoiévski, é
%claro. Vê como a alma humana é complicada? Então, imagine um homem que
%lutou de Stalingrado a Belgrado -- por milhares de quilômetros de sua
%própria terra devastada, por cima dos cadáveres de seus camaradas e
%entes mais queridos. Como pode um homem assim reagir normalmente? E o
%que há de tão medonho em se divertir com uma mulher depois de tais
%horrores?

%Ióssif Vissariónovitch Djugachvili,

%também conhecido como Stálin\footnote{Simon Sebag Montefiore,
%  \emph{Stálin: A corte do tzar Vermelho}. São Paulo: Companhia das
%  Letras, 2006, p. 532.}

%Eu vi a verdade, eu vi e sei que as pessoas podem ser belas e felizes,
%sem perder a capacidade de viver na terra. Não quero e não posso
%acreditar que o mal seja o estado normal dos homens.

%Homem ridículo, ``O sonho de um homem ridículo''\footnote{\textsuperscript{*}
%  Tradução de Vadim Nikitin. São Paulo: Editora 34, 2003, p. 121 e p.
%  122.}\textsuperscript{*}
%\end{quote}

%Sumário%

%Preâmbulo, 11%

%Parte I: Tese%

%Dostoiévski e o fetichismo da forma mercadoria, 18%

%Capítulo 1%

%\begin{quote}
%Mediações para a constituição da dialética polifônica de Fiódor
%Dostoiévski, 19
%\end{quote}%

%\begin{enumerate}
%\def\labelenumi{\arabic{enumi}.}
%\item
%  Introdução, 19
%\end{enumerate}%

%\begin{quote}
%1.2. O caminho através das aporias dialógico"-polifônicas de Mikhail
%Bakhtin, 23
%\end{quote}%

%\begin{enumerate}
%\def\labelenumi{\arabic{enumi}.}
%\setcounter{enumi}{2}
%\item
%  Pressupostos dialógicos para o devir dialético, 33
%\end{enumerate}%

%\begin{quote}
%1.4. A polifonia como um momento da dialética negativa, 45%

%Capítulo 2%

%A dialética polifônica de Fiódor Dostoiévski, 61%

%2.1. No princípio eram a dicção -- e a contradição, 61%

%2.2. Marx e a arquitetura do subsolo, 64%

%2.3. A forma dinheiro escava as galerias subterrâneas, 78%

%2.4. O Evangelho segundo o Capital: Mitgefangen, mitgehangen; presos
%juntos, juntos enforcados, 86%

%2.5. Do leito do subsolo, a extração de mais"-valia se reverte em
%menos"-valia, 97
%\end{quote}%

%Parte II: Antítese%

%O conteúdo em Dostoiévski como a cicatrização do espírito rumo à
%utopia?, 121%

%\begin{quote}
%Capítulo 3%

%O prenúncio da teologia como teleologia --%

%e da teleologia como teologia, 122%

%3.1. Preâmbulo, 122%

%3.2. O cálculo utilitário à luz e às sombras da história, 127%

%3.3. A utopia como degredo siberiano, 135%

%3.3.1. História de proveta, socialismo de estufa, 137%

%3.3.2. O socialismo siberiano da Casa dos Mortos, 144%

%3.4. O prenúncio do Sermão da Estepe para além da montanha nacional, 161
%\end{quote}%

%\textbf{Capítulo 4}%

%\textbf{O Sermão da Estepe:}%

%\begin{quote}
%Ivan Karamázov e a filosofia dostoievskiana da história, 180%

%4.1. Se no princípio era o Verbo, como conjugar Suas aporias?, 180%

%4.2. Para além do Sermão da Montanha, 182%

%4.3. O Sermão da Estepe, 187
%\end{quote}%

%Capítulo 5%

%\textbf{A utopia como cicatrização do espírito, 204}%

%\begin{quote}
%\textbf{5.1. Preâmbulo, 204}%

%\textbf{5.2. Só há uma ação niilista verdadeiramente séria: o suicídio,
%207}%

%\textbf{5.3. A mão que fere é a mesma mão que pode curar, 214}%

%\textbf{5.4. Só há um problema existencial verdadeiramente sério: a
%eternidade, 221}%

%\textbf{5.4.1. Um espectro ronda Dostoiévski, o espectro do espiritismo,
%225}%

%\textbf{5.5. Do Éden à queda, 234}%

%\textbf{5.6. Da queda à reconciliação?, 246}
%\end{quote}%

%\textbf{Considerações finais, 269}%

%\textbf{Referências bibliográficas, 276}

\chapter{Preâmbulo}

O título deste livro, \emph{Dostoiévski e a dialética: fetichismo da
forma, utopia como conteúdo}, procura demonstrar que a análise possui,
dialeticamente, duas partes que se entrechocam, dois momentos
fundamentais em que a história e suas contradições despontam por meio da
imanência da obra dostoievskiana: o fetichismo da forma, a forma como
conteúdo histórico reificado, a forma como diagnóstico de época, e a
utopia como conteúdo, o conteúdo como filosofia da história, o conteúdo
como ímpeto de cicatrização do espírito e reconciliação, o conteúdo como
promessa de superação (\emph{Aufhebung}) e síntese. Sobrevoemos, então,
os momentos de constituição dialeticamente contraditória deste livro.

A primeira parte (tese), ``Dostoiévski e o fetichismo da forma
mercadoria'', é composta por dois capítulos: ``Mediações para a
constituição da dialética polifônica de Dostoiévski'' (capítulo 1) e ``A
dialética polifônica de Fiódor Dostoiévski'' (capítulo 2).

No capítulo 1, a análise da obra de Dostoiévski nos leva às tensões e às
afinidades eletivas que enredam duas vertentes: a polifonia proposta
pelo crítico russo Mikhail Bakhtin (1895-1975), em \emph{Problemas da
poética de Dostoiévski}, e a dialética materialista, sobretudo a partir
da \emph{Teoria estética}, do filósofo alemão Theodor Adorno
(1903-1969).

Para a tentativa de totalização das aporias do concerto polifônico, as
discussões fomentadas pela disciplina ``A~Teoria estética de Theodor
Adorno'', ministrada pelo Prof. Dr. Jorge de Almeida\footnote{Curso
  ministrado junto ao Programa de Pós"-Graduação em Teoria Literária e
  Literatura Comparada da Faculdade de Filosofia, Letras e Ciências Humanas da \versal{USP} entre 22/08/12 e 29/11/12.}, orientador de minha pesquisa de doutorado no Brasil, se mostraram fundamentais. As~reflexões adornianas sobre o potencial dialeticamente crítico da obra de arte para
apreender, questionar e reconfigurar os sentidos (e os ressentimentos)
de seu período histórico e dos conflitos em devir só fizeram insuflar ar
redivivo para que pudéssemos voltar a movimentar as insuficiências de
\emph{Problemas da poética de Dostoiévski.}

Municiado por tais mediações, o capítulo 2 se propõe, então, a
estruturar a dialética polifônica de Dostoiévski, de modo a reaproximar,
por analogia, a dialogia da dialética por meio da mimese em relação ao
movimento da forma mercadoria, tal como o pensador alemão Karl Marx
(1818-1883) o desenvolveu nos primeiros capítulos do volume 1 de \emph{O
capital.} Sobretudo a partir de um \emph{close reading} de
\emph{Memórias do subsolo}, o movimento do capital (D -- M -- D') --
movimento tautológico que Marx chamou de \emph{sujeito automático} --
nos fornecerá os elementos para superar as limitações da polifonia,
limitações que Bakhtin, com muita honestidade intelectual, reconhece
desde o início de sua obra sobre Dostoiévski. Assim, entenderemos como a
totalidade dialética dostoievskiana nos poderá fornecer um diagnóstico
niilista de época e como a forma \emph{subterrânea}, por meio de sua
constituição imanente, mimetiza e ressignifica a reificação
histórico"-social que o capitalismo reproduz, de modo que as personagens
sejam \emph{sujets}, isto é, ao mesmo tempo \emph{sujeitos} e
\emph{súditos}, de suas vozes continuamente enforcadas pelas cordas
vocais do subsolo.

Nesse sentido, podemos dizer que a primeira parte deste livro, ao
aproximar a polifonia da Teoria Crítica e Bakhtin de Adorno e Marx,
estabelece uma leitura que tenta alcançar uma síntese dialética em
relação a problemas e questões que são estudados pela fortuna crítica de
Dostoiévski desde a publicação da primeira versão de \emph{Problemas da
poética de Dostoiévski}, em 1929.

Antes de passarmos à apresentação da segunda parte de nosso trabalho (a
antítese), é fundamental frisar que nosso contato com a fortuna crítica
dostoievskiana teve início, no Brasil, a partir das obras de Boris
Schnaiderman, fundador dos estudos de língua e literatura russa na
Faculdade de Filosofia, Letras e Ciências Humanas da \versal{USP}. Obras como
\emph{Dostoiévski: prosa e poesia}\footnote{São Paulo: Perspectiva,
  1982.} e \emph{Turbilhão e Semente: ensaios sobre Dostoiévski e
Bakhtin}\footnote{São Paulo: Duas Cidades, 1983.} me fizeram refletir
sobre a arquitetura poética de Dostoiévski em estreitíssimo diálogo com
as apreensões e aporias bakhtinianas. Ademais, ensaios como
``Dostoiévski: a ficção como pensamento''\footnote{In: A. Novaes (Org.),
  \emph{Artepensamento}. São Paulo: Companhia das Letras, 1994.} e ``Os
paradoxos políticos de um gigante do pensamento''\footnote{In:
  \emph{Caderno 2}, \emph{O~Estado de São Paulo}. São Paulo, 26 de março
  de 2000.} e uma entrevista como ``Para Boris Schnaiderman, autor é o `escritor"-filósofo por excelência'"\footnote{In: \emph{Caderno Mais!}, Folha de São Paulo. São Paulo, 6 de maio de 2001.} me levaram a pensar
sobre as relações e contradições envolvendo arte e história, ética e
estética, literatura e política em Dostoiévski. A~partir de Boris
Schnaiderman, pude encontrar as obras do professor e tradutor Paulo
Bezerra -- \emph{Dostoiévski:} Bóbok\emph{, tradução e análise do
conto}\footnote{São Paulo: Editora 34, 2005.} e ``Mundos desdobrados,
seres duplicados''\footnote{In: \emph{Caderno de literatura e cultura
  russa: Dostoiévski}. São Paulo: Ateliê Cultural, 2008.} --, da
professora e tradutora Maria de Fátima Bianchi -- \emph{Os caminhos da
razão e as tramas secretas do coração: a representação da realidade em} A~dócil\emph{, de Dostoiévski}\footnote{Dissertação de mestrado,
  \versal{FFLCH-USP}. São Paulo: 2001.}, \emph{O~``sonhador'' de} A
senhoria\emph{, de Dostoiévski}\footnote{Tese de doutorado, \versal{FFLCH-USP}.
  São Paulo: 2006.} e ``O domínio do homem pelo homem na novela \emph{A senhoria}, de Dostoiévski''\footnote{In: \emph{Revista de literatura e
  cultura russa.} Volume 2, número 2. São Paulo: 2013, pp. 35-54.} --,
do professor e tradutor Bruno Barretto Gomide -- \emph{Da estepe à
caatinga: o romance russo no Brasil (1887-1936)}\footnote{São Paulo:
  Edusp, 2011.} e \emph{Antologia do pensamento crítico russo
(1802-1901)}\footnote{São Paulo: Editora 34, 2013.} -- e da professora e
tradutora Aurora Fornoni Bernardini -- ``A~longa provação de
Dostoiévski''\footnote{Resenha da obra: Dostoiévski, Fiódor.
  \emph{Recordações da casa dos mortos.} São Paulo: Nova Alexandria,
  2006.}, ``O~grande inquisidor e \emph{Os irmãos Karamázov}''\footnote{In:
  Novinsky, A. W.; Carneiro, M. L. T. (Orgs.). \emph{Inquisição: ensaios
  sobre mentalidade, heresias e arte.} São Paulo: \versal{USP}/Expressão e
  Cultura, 1992, pp. 682-691.}\emph{, ``}Questões de forma e modernidade
em Gógol e Dostoiévski''\footnote{In: Mutran, M. H. e Chiampi, I. (orgs), \emph{A~questão da modernidade}. Caderno 1, Departamento de Letras Modernas, \versal{FFLCH/USP}, 1993.} e ``Dostoiévski: criação, poesia e crítica''\footnote{In: \emph{Caderno de literatura e cultura russa:
  Dostoiévski}. São Paulo: Ateliê Cultural, 2008.}.

No exterior, o contato com a fortuna crítica dostoievskiana se deu,
sobretudo, a partir da leitura de todos os volumes da revista
\emph{Dostoevsky Studies}, periódico da International Dostoevsky
Society, publicados desde 1980 -- às vésperas, portanto, do centenário
da morte do escritor russo (1981) -- até 2014. Autores como Birgit
Harress, Carol Apollonio, Deborah Martinsen, Gary Saul Morson, Gerald J.
Sabo, Igor Vólguin, Joseph Frank, Karen Stepânian, Lyudmila Parts,
Michael Holquist, Michael Katz, Predrag Cicovacki, René Wellek, Robert
Bird, Robert L. Belknap, Robert Louis"-Jackson, Robin Feuer Miller,
Rudolf Neuhäuser, Sophie Ollivier, Steven Cassedy, Susan McReynolds,
Victor Terras, Zsuzsanna Bjørn Andersen, entre outros, são presenças
ubíquas em nossas discussões. Se quisermos utilizar uma imagem bastante
cara a este livro, os autores que orbitam ao redor da \emph{Dostoevsky
Studies} nos ajudaram sobremaneira a escavar as galerias do subsolo
dostoievskiano.

Entre os autores que participam das discussões da revista
\emph{Dostoevsky Studies}, enfatizo o contato que mantive com a Profª.
Drª. Susan McReynolds, docente do Departamento de Línguas e Literaturas
Eslavas da \emph{Northwestern University}, em Evanston, nos Estados
Unidos, durante o período em que realizei meu doutorado sanduíche sob
sua orientação -- de setembro de 2014 a agosto de 2015. Membro do
conselho editorial da \emph{Dostoevsky Studies}, ex"-presidente da North
American Dostoevsky Society e ex"-secretária executiva da International
Dostoevsky Society, a Profª. Drª. Susan McReynolds me trouxe prolíficas
recomendações tanto nos cursos que ministrou e dos quais fui monitor --
\emph{Russian Prose of the Nineteenth Century} (de setembro a novembro
de 2014), \emph{Dostoevsky's} Crime and Punishment (de janeiro a março
de 2015) e \emph{Russian Prose of Late Nineteenth and Early Twentieth
Centuries} (de abril a junho de 2015) -- quanto em nossas reuniões de
orientação.

Passemos, então, à apresentação da segunda parte (antítese), ``O
conteúdo em Dostoiévski como a cicatrização do espírito rumo à
utopia?'', que é composta por três capítulos: ``O~prenúncio da teologia
como teleologia -- e da teleologia como teologia'' (capítulo 3),
``Sermão da Estepe: Ivan Karamázov e a filosofia dostoievskiana da
história'' (capítulo 4) e ``A~utopia como cicatrização do espírito''
(capítulo 5).

Se, na primeira parte da tese, procuramos apresentar Dostoiévski como um
leitor fundamental da reificação das relações humanas por meio do
movimento da forma subterrânea em estreita conexão com o conteúdo
precipitado historicamente pelo fetichismo da mercadoria, na segunda
parte procuramos estruturar a filosofia da história que transpassa a
obra do escritor russo, filosofia da história que procura superar,
dialeticamente, a máxima atribuída a Ivan Karamázov, personagem crucial
de \emph{Os irmãos Karamázov}: "se Deus não existe, tudo é permitido".
Trata"-se de uma segunda tomada de posição deste livro em face da fortuna
crítica de Dostoiévski, tomada de posição que procura apreender as
tensões e contiguidades dialéticas de vozes dostoievskianas cruciais,
tais como teísmo e ateísmo, socialismo e cristianismo, niilismo e
redenção. Assim, podemos dizer que a dialética poético"-histórica que
movimenta a obra de Dostoiévski se comunica tanto à estruturação deste
livro como um todo contraditório quanto às minhas tomadas de posição
como pesquisador.

No capítulo 3, os diálogos dostoievskianos envolvendo as tensões e
afinidades eletivas entre socialismo e cristianismo nos fazem
correlacionar as discussões estabelecidas em \emph{Recordações da casa
dos mortos}, \emph{Notas de inverno sobre impressões de verão},
\emph{Memórias do subsolo}, \emph{Crime e castigo}, \emph{O~idiota} e
\emph{Os demônios.} Começamos, de tal forma, a pavimentar o caminho que
nos fará entender como a teleologia historicamente dialética pode se
imbricar com a teologia em meio à obra de Dostoiévski.

O capítulo 4, por sua vez, apresenta a análise de dois capítulos
fundamentais de \emph{Os irmãos Karamázov}: ``A~revolta'' e ``O~grande
inquisidor''. A~meu ver, tais capítulos podem ser considerados uma
grande síntese das reflexões de Dostoiévski sobre as (im)possibilidades
de redenção utópica da história, a (in)existência de Deus e a
reconciliação entre imanência e transcendência. Trata"-se de entender
como Dostoiévski compreendia o movimento que faz com que o Sermão da
Montanha, proferido por Jesus Cristo, possa descer até a estepe da
história humana para fazer com que o oferecimento da outra face seja o
norte quintessencial para a cicatrização do espírito.

Assim, no quinto e último capítulo -- algo como um prenúncio de síntese
histórica a dialogar com a máxima de que a arte, a resvalar a utopia, é
uma promessa de felicidade --, analisamos o conto ``O~sonho de um homem
ridículo'' como a descida da montanha transcendental até a estepe
imanente da história a partir de dois diálogos fundamentais: o primeiro
se dá com a \emph{Filosofia da História}, do filósofo alemão Georg
Wilhelm Friedrich Hegel (1770-1831), precisamente a partir do conceito
de cicatrização do espírito, isto é, por meio da noção de que a
racionalidade histórica, em seu movimento dialético e astucioso, tem o
potencial de promover novas (e dolorosas) sínteses a partir de rupturas
e cisões -- algo que, segundo Hegel (e Dostoiévski), a percepção que se
atém apenas ao caráter insuportável do sofrimento não conseguiria
apreender.

A influência de Hegel sobre Dostoiévski -- em termos mais globais, a
influência do idealismo alemão sobre a literatura e o pensamento russos
oitocentistas -- é enfatizada por Susan McReynolds (2002), para quem

\begin{quote}
os estudiososos de Dostoiévski concordam que o escritor alcançou a
maturidade intelectual em um ambiente envolto pelas ideias de Kant,
Schelling e Hegel; independentemente, então, dos textos que ele leu, as
ideias de tais filósofos permeiam a sua arte e o seu pensamento. A
filosofia alemã frequentemente chegava a Dostoiévski por meio de suas
relações com intelectuais russos, como Vissarion Bielínski, que dominava
boa parte dos textos originais (p. 94).
\end{quote}

O segundo diálogo fundamental -- diálogo derivado da dialética
envolvendo Hegel e Dostoiévski -- se dá com o educador francês Hippolyte
Léon Denizard Rivail (1804-1869), mais conhecido como Allan Kardec, o
codificador da doutrina espírita. Coube primeiramente a Rudolf
Neuhäuser, professor da Universidade de Klagenfurt, na Áustria,
apresentar as influências que o espiritismo, Kardec e autores correlatos
podem ter exercido sobre o escritor russo, no artigo ```O~sonho de um
homem ridículo': topicalidade como um dispositivo literário'', publicado
na edição de 1993 da revista \emph{Dostoevsky Studies}\footnote{``The
  Dream of a Ridiculous Man: Topicality as a Literary Device''. New
  Series. Volume 1, Number 2. Sem menção a lugar. Charles Schlacks, Jr.,
  Publisher, 1993, pp. 175-190.}, artigo com o qual estabeleceremos
extensas discussões. Vale dizer, então, que, com a aproximação de
Dostoiévski e Kardec mediada tanto por Neuhäuser quanto pelo diálogo
precípuo com a dialética hegeliana, esta pesquisa estabelece outro
posicionamento diferencial em relação a autores da fortuna crítica
dostoievskiana como Nikolai Berdiaev, Karen Stepânian, Gerald J. Sabo,
entre outros, que tendem a situar o escritor russo nos limites estritos
da teologia ortodoxa. A~cicatrização do espírito sobre a qual Hegel e
Dostoiévski discorreram, a meu ver, tem o potencial de sondar aspectos
do além"-mundo que, conforme discutiremos, aproximam a viagem
intergaláctica e transcendental do homem ridículo das reflexões de
Kardec.

Sem mais, comecemos a escavar o subsolo dostoievskiano, tendo sempre em
mente a admoestação que Bakhtin auscultou diante da esfinge de
Dostoiévski: ``Parece que todo aquele que penetra no labirinto do
romance polifônico não consegue encontrar a saída e, obstaculizado por
vozes particulares, não percebe o todo'' (\versal{BAKHTIN}, 2008, p. 51). Se o
todo é o dialeticamente falso e estilhaçado, isto é, se o todo é o
movimento que, por meio de suas fraturas e escombros, tem o ímpeto pela
reconciliação rumo a uma totalidade verdadeira, a promessa de felicidade
da utopia deve se precaver contra o canto de Circe das personagens
dostoievskianas. Se a razão já pôde vislumbrar paragens para além do
fetichismo e da reificação, o homem do subsolo, representante"-mor das
personagens dostoievskianas, bem poderia sentenciar: a promessa de
reconciliação, se é o que há de vir, até hoje desponta como o que ainda
não veio.

\part[Parte \versal{I} -- Tese]
{Tese\\[\bigskipamount] 
      \large Dostoiévski e o fetichismo da forma mercadoria}

\hedramarkboth{Parte \versal{I} - Tese}{}

%\chapter{Dostoiévski e o fetichismo da forma mercadoria}

\, \ 
\begin{minipage}{0.84\textwidth}
\scriptsize\emph{Parece que todo aquele que penetra no labirinto do romance polifônico
não consegue encontrar a saída e, obstaculizado por vozes particulares,
não percebe o todo.}

\smallskip
\hspace*{\fill}-- Mikhail Bakhtin, \emph{Problemas da poética de Dostoiévski}\footnotemark
\end{minipage}
\footnotetext{Tradução
  de Paulo Bezerra. Rio de Janeiro: Forense Universitária, 2008, p. 51.}

\bigskip

\, \ 
\begin{minipage}{0.84\textwidth}
\scriptsize\emph{Quanto mais consciência eu tinha do bem e de tudo o que é ``belo e
sublime'', tanto mais me afundava em meu lodo e tanto mais capaz me
tornava de imergir nele por completo. Porém, o traço principal estava em
que tudo isso parecia ocorrer"-me não como que por acaso, mas como algo
que tinha de ser. (\ldots) Chegava a ponto de sentir certo prazerzinho
secreto, anormal, ignobilzinho quando às vezes, em alguma horrível noite
de Petersburgo, regressava ao meu cantinho e me punha a lembrar com
esforço que, naquele dia, tornara a cometer uma ignomínia e que era
impossível voltar atrás. Remordia"-me então em segredo, dilacerava"-me,
rasgava"-me e sugava"-me, até que o amargor se transformasse, finalmente,
em certa doçura vil, maldita e, depois, num prazer sério, decisivo! Sim,
num prazer, num prazer! (\ldots) O~prazer provinha justamente da
consciência demasiado viva que eu tinha da minha própria degradação;
vinha da sensação que experimentava de ter chegado ao derradeiro limite;
de sentir que, embora isso seja ruim, não pode ser de outro modo; de que
não há outra saída; de que a pessoa nunca mais será diferente, pois,
ainda que nos sobrassem tempo e fé para isto, certamente não teríamos
vontade de fazê"-lo e, mesmo se quiséssemos, nada faríamos neste sentido,
mesmo porque em que nos transformaríamos?}

\smallskip
\hspace*{\fill}-- Homem do subsolo, \emph{Memórias do subsolo}\footnotemark
\end{minipage}
\footnotetext{Tradução de Boris Schnaiderman.
São Paulo: Editora 34, 2004, p. 19;
  pp. 19-20; p. 20.}

\bigskip

\, \ 
\begin{minipage}{0.84\textwidth}
\scriptsize\emph{O reflexo religioso do mundo real somente pode desaparecer quando as
circunstâncias cotidianas da vida prática reapresentarem para os homens
relações transparentes e racionais entre si e com a natureza. A~figura
do processo social da vida, isto é, do processo da produção material,
apenas se desprenderá do seu místico véu nebuloso quando, como produto
de homens livremente socializados, ela ficar sob seu controle consciente
e planejado. Para tanto, porém, se requer uma base material da sociedade
ou uma série de condições materiais de existência, que, por sua vez, são
o produto natural de uma evolução histórica longa e penosa.}

\smallskip
\hspace*{\fill}-- Karl Marx, \emph{O capital}\footnotemark
\end{minipage}
\footnotetext{Tradução de
  Flávio Kothe. São Paulo: Nova Cultural, 1998, p. 76.}

\thispagestyle{empty}

\chapter*{Capítulo 1\\
\bigskip
\emph{Mediações para a constituição da dialética polifônica de Fiódor
Dostoiévski}}

\addcontentsline{toc}{chapter}{Capítulo 1\\\scriptsize{\emph{Mediações para a constituição da dialética polifônica de Fiódor
Dostoiévski}}}
\hedramarkboth{Capítulo 1}{}

\section{1.1. Introdução}

É de conhecimento geral dos estudiosos de Fiódor Dostoiévski que a obra
\emph{Problemas da poética de Dostoiévski}, de Mikhail Bakhtin, é
reconhecida como um dos mais importantes trabalhos sobre o escritor
russo. Conforme veremos ao longo deste capítulo, até mesmo críticos de
tendências diferentes como René Wellek e Paul de Man reconhecem que a
dialogia e a polifonia erigidas por Bakhtin são conceitos fundamentais
que emanam da poética de Dostoiévski.

A dialogia pode ser entendida como a identidade relacional das
personagens, de modo que não haja egos e vozes delimitados sem a
existência da alteridade. As~vozes das personagens de Dostoiévski sempre
pressupõem a existência e o contraponto do outro, o que significa que,
para Bakhtin, a dialogia está relacionada tanto com a coexistência
quanto com a equipolência. A~coexistência se refere à nossa condição
dialógica, ela se refere à noção de que, em termos ontológicos, o eu só
consegue pensar sobre o mundo porque ele ou ela se comunica, porque nós
existimos juntos, porque a possibilidade de dizer \emph{eu} pressupõe a
\emph{nossa} existência. A~equipolência é o desdobramento da condição
dialógica: uma vez que não há um ego sem a existência do outro, a
alteridade não é apenas oposta à minha existência como uma diferença e
um contraponto, mas ela é parte de mim mesmo, o que implica uma
existência equipolente e ética das vozes para a expressão da identidade
relacional do ego.

A partir do concerto de vozes dialógicas, Bakhtin tenta estabelecer a
totalidade polifônica de Dostoiévski. Mas as noções de coexistência e
equipolência que, em Bakhtin, são derivadas da dialogia, implicam uma
liberdade determinada e inusitada para as personagens de Dostoiévski. A
liberdade das personagens é determinada, relativa e não absoluta, uma
vez que Bakhtin está lidando com as noções de totalidade poética e plano
autoral. A~liberdade determinada e relativa permite que as partes
contradigam o todo, como se as vozes pudessem até mesmo falar contra o
autor. Mas, como Bakhtin considera a noção de plano autoral um aspecto
fundamental da poética de Dostoiévski, uma importante aporia vem à tona:
como a liberdade dialógica, determinada, relativa e potencialmente
centrífuga das personagens poderia coexistir com o plano autoral
aglutinador e centrípeto sem estilhaçar a possibilidade de uma
totalidade poética, isto é, sem trazer a contradição para o seio do
sistema polifônico de Dostoiévski?

As realizações de Bakhtin se tornaram incontroversas -- e até mesmo
dogmáticas -- entre os críticos e os leitores de Dostoiévski. Tal
aceitação do teor de verdade das análises de Bakhtin é deveras merecida,
já que a noção de dialogia de fato ilumina as obras de Dostoiévski. No
entanto, a aceitação universal da dialogia e da polifonia como conceitos
centrais da poética de Dostoiévski tende a obscurecer algumas dimensões
da arte do escritor, uma vez que Bakhtin, com grande honestidade
intelectual, reconhece logo no início de seu livro que \emph{Problemas
da poética de Dostoiévski} ``não tem a pretensão de atingir a plenitude
na abordagem dos problemas levantados, sobretudo questões complexas como
o problema do romance polifônico \emph{integral}'' (2008, p. 2).

Neste momento, é possível mencionar a ideia central deste capítulo: um
diálogo entre Mikhail Bakhtin e a Teoria Crítica como um caminho a
conduzir da dialogia e da polifonia à dialética de Dostoiévski. A
chamada Teoria Crítica, posteriormente conhecida como Escola de
Frankfurt, foi fundada no início da década de 1930 e buscou estabelecer
uma crítica multidisciplinar à dinâmica entre cultura, política e sociedade no contexto europeu e norte"-americano da sociedade industrial.
A~partir de bases filosóficas e sociológicas relacionadas à dialética de Hegel e Marx -- e também por meio de um forte diálogo com a psicanálise
--, um dos principais propósitos da Teoria Crítica era explicar a
reprodução do capitalismo tardio por meio da sociedade totalmente
administrada e analisar as contradições que os movimentos emancipatórios
enfrentavam para superar a reificação social. Desde o início, Theodor
Adorno e Max Horkheimer, os principais intelectuais da Escola de
Frankfurt, se mostraram profundamente críticos tanto em relação ao
Terceiro Reich quanto em relação à União Soviética como sociedades
antípodas porém igualmente totalitárias. Mas, na \emph{Dialética do
Esclarecimento} (1944), Adorno e Horkheimer analisaram o surgimento de
Hollywood, a indústria cultural norte"-americana, como o mecanismo
totalitário e ideológico do capitalismo tardio para implementar a
administração social em meio à (e contra a) democracia\footnote{
  Para estudos clássicos e detalhados sobre a Escola de Frankfurt,
  conferir Rolf Wiggershaus, \emph{The Frankfurt School: Its
  History, Theories, and Political Significance}. Cambridge: The \versal{MIT}
  Press, 1995, e Martin Jay, \emph{The Dialectical Imagination: A
  History of the Frankfurt School and the Institute of Social Research,
  1923-1950}. Oakland: University of California Press, 1996.}\label{para-estudos-cluxe1ssicos-e-detalhados-sobre-a-escola-de-frankfurt-conferir-rolf-wiggershaus-the-frankfurt-school-its-history-theories-and-political-significance.-cambridge-the-mit-press-1995-e-martin-jay-the-dialectical-imagination-a-history-of-the-frankfurt-school-and-the-institute-of-social-research-1923-1950.-oakland-university-of-california-press-1996.}.
A~crítica à arte reduzida e reproduzida como mercadoria levou a Teoria
Crítica a analisar o teor verdadeiro e emancipatório das obras de modo a
expressar as contradições que a modernidade capitalista traz à luz. É
neste momento que as análises dialéticas e estéticas da Teoria Crítica
-- especialmente a \emph{Teoria estética} e outros ensaios de Theodor
Adorno com os quais discutiremos ao longo deste capítulo -- estabelecem
diálogos com a poética de Dostoiévski. Eu reconheço que tal percuso de
análise não é muito comum entre os estudiosos do escritor russo, uma vez
que György Lukács\footnote{Lukács não estava diretamente vinculado à
  Escola de Frankfurt, mas a sua \emph{Teoria do romance} (1914-15) é
  deveras importante tanto para Adorno e Horkheimer quanto para os
  conceitos desenvolvidos ao longo deste capítulo. A~\emph{Teoria do
  romance} termina com uma proposta de analisar a obra de Dostoiévski em
  um novo livro, algo que Lukács não fez porque, no momento em que o
  autor ia dar início a seu trabalho, a Revolução Russa irrompe e,
  então, todo o seu entusiasmo se volta para o novo movimento.}, Theodor
Adorno e Max Horkheimer não são leitores comumente associados à fortuna
crítica de Dostoiévski, mas o propósito deste capítulo é estabelecer que
a dialética e o movimento da contradição podem iluminar alguns problemas
e aporias que se encontram no âmago da dialogia e da polifonia de
Bakhtin, de modo a que possamos melhor compreender a poética de
Dostoiévski.

Conforme veremos ao longo deste capítulo, as colisões entre as partes
dialógicas -- as altercações entre as personagens e as altercações
dentro de uma mesma personagem com vozes múltiplas e contraditórias --
são recorrentes em meio à poética de Dostoiévski. Mas, então, as aporias
de Bakhtin nos trazem questões mais complexas: como é possível
reconciliar a tensão das partes com o plano autoral da totalidade sem
levar a própria noção de totalidade a um colapso diante da liberdade
determinada e relativa das personagens, liberdade que tende a
contradizer o plano autoral e a poética como um todo? O~que impediria
essas personagens dialógicas e relativamente autônomas de constituir
universos centrífugos e monadológicos que apenas orbitariam ao redor de
si mesmos? Como essas constelações particulares de personagens se
articulariam em um movimento centrípeto com o universo total das obras?

À luz da dificuldade que a dialogia bakhtiniana apresenta para superar
tais contradições, nós poderíamos adaptar a primeira frase do
\emph{Manifesto comunista}, de Marx e Friedrich Engels, para dizer que
um espectro ronda Bakhtin, o espectro da Teoria Crítica com sua crítica
dialética à teoria tradicional. De acordo com Max Horkheimer, ``uma
exigência fundamental, que todo sistema teórico tem que satisfazer,
consiste em estarem todas as partes conectadas ininterruptamente e
livres de contradição'' (1980b, p. 118). Em termos bastante gerais, de
acordo com a crítica de Adorno e Horkheimer, a teoria tradicional se
baseia em uma lógica sistêmica, em meio à qual as partes são
hierarquicamente relacionadas e dispostas para a formação do
todo\footnote{A~crítica dialética à teoria tradicional será propriamente
  desenvolvida, ao longo deste capítulo, como os limites dos
  pressupostos epistemológicos e ideológicos de Bakhtin. Para discussões
  mais detalhadas sobre a crítica à teoria tradicional propriamente
  dita, conferir \emph{Dialética do Esclarecimento}. Rio de Janeiro:
  Jorge Zahar Editor, 1985; Max Horkheimer. \emph{Conceito de
  Esclarecimento.} In: \emph{Os pensadores.} São Paulo: Abril Cultural,
  1980b, pp. 89-116; \emph{Teoria Tradicional, Teoria Crítica.} In:
  \emph{Idem}, pp. 116-154; \emph{Filosofia e Teoria Crítica}. In:
  \emph{Idem}, pp. 155-161.}. Apesar do fato de Bakhtin ter feito muitos
avanços com sua tentativa de estabelecer a dialogia como a base de uma
totalidade polifônica, a meu ver, se quisermos estabelecer a trajetória
que poderá guiar as aporias de Bakhtin das vozes dialógicas à totalidade
como um concerto polifônico, o universo epistemológico e ideológico de
Bakhtin relacionado à teoria tradicional e sistêmica não nos poderá
ajudar. Eu tentarei desenvolver esta análise cujas contradições Bakhtin
não pôde expressar, mas cujo caminho o crítico russo talvez não tenha
percorrido por causa do contexto político e repressivo da União
Soviética e da distância prudente que Bakhtin precisou manter em relação
à dialética e ao movimento da contradição, as chaves essenciais que a
Teoria Crítica nos fornece para superarmos as aporias da poética
polifônica de Dostoiévski.

\section{1.2. O caminho através das aporias dialógico"-polifônicas de Mikhail
Bakhtin}

Para começarmos a trilhar o caminho através das aporias da dialogia e da
polifonia de Bakhthin, é importante notar como autores de diferentes
perspectivas teóricas reconhecem o mérito das análises de Bakhtin e
dialogam com sua teoria poética sobre Dostoiévski.

René Wellek, um importante representante do \emph{New Criticism},
reconhece a importância de \emph{Problemas da poética de Dostoiévski}
para a compreensão da obra do escritor russo:

\begin{quote}
\emph{Problemas da poética de Dostoiévski}, de Mikhail Bakhtin, é
considerado, com justiça, um dos livros mais estimulantes e originais da
vasta literatura sobre Dostoiévski. (\ldots) Bakhtin afirma que Dostoiévski
criou um tipo de romance totalmente novo, romance que o crítico chama de
``polifônico'': um tipo literário formado por vozes independentes que
são completamente iguais, se tornam sujeitos de si mesmas e não obedecem
à posição ideológica do autor (1980, p. 31).
\end{quote}

Wolf Schmid, um autor que se relaciona tanto com a eslavística quanto
com os estudos de narratologia, também reconhece a importância de
Bakhtin para os estudos sobre Dostoiévski e discute seus fundamentos:

\begin{quote}
A técnica perspectivista de Dostoiévski está relativamente bem
pesquisada. Agradecemos a Bakhtin e a seus seguidores pelas profundas
análises a respeito das relações realmente tensas entre as instâncias
narrativas e suas vozes. (\ldots) A~descoberta de Bakhtin afirma: a palavra
na obra de Dostoiévski, considerando"-se seus diferentes níveis de
utilização -- tanto como figuras de linguagem quanto como o texto do
narrador -- é profundamente ``dialógica''. (\ldots) De acordo com Bakhtin,
o dialogismo implica uma tensão entre duas importantes posições, as
quais manifestam seus discursos simultaneamente (2002, p. 65).
\end{quote}

Michael Holquist, renomado estudioso da obra de Dostoiévski, afirma que
Mikhail Bakhtin apreende Fiódor Dostoiévski como o autor que, ao erigir
o romance polifônico, rompe com e supera as categorias do monologismo
artístico.

\begin{quote}
O autor de um romance pode manipular o outro não apenas como um outro,
mas como um \emph{sujeito}, um \emph{eu} (\emph{self}). É~isso, na
verdade, o que os grandes escritores sempre fizeram, mas o exemplo
paradigmático provém de Dostoiévski, autor que, com alto grau de
realização, permite que suas personagens tenham o s\emph{tatus} de um
``eu'', que pode se contrapor às reivindicações de seu outro autoral.
Isso fez com que Bakhtin se sentisse compelido a cunhar o termo especial
``polifonia'' para descrever o fenômeno (1990, p. 34).
\end{quote}

As características da monologia diriam respeito à estruturação
estático"-sistêmica dos elementos do todo em função do autor como
eminência parda -- tendência que a Teoria Crítica chamaria de teoria
tradicional. As~partes seriam imbricadas tendo em vista uma construção
hierárquica que transformaria as personagens em extensões da visão
autoral. Em oposição a tal perspectiva, Paul de Man, autor relacionado
ao desconstrucionismo, afirma, em diálogo com Bakhtin, que

\begin{quote}
as personagens de Dostoiévski não são vozes da identidade autoral, mas
vozes da alteridade radical. Isso não ocorre pelo fato de,
diferentemente de Dostoiévski, as vozes serem ficcionais, mas porque a
alteridade \emph{é} a realidade delas. O~princípio de realidade coincide
com o princípio da alteridade (1989, p. 109).
\end{quote}

Se mencionarmos as próprias palavras de Bakhtin, é possível dizer que,
em meio ao plano artístico de Dostoiévski, ``suas personagens principais
\emph{são}, em realidade, \emph{não apenas objetos do discurso do autor,
mas os próprios sujeitos desse discurso diretamente significante}''
{[}grifos do autor{]} (2008, p. 5). O~crítico russo estabelece a
primazia da polifonia, de modo que Dostoiévski possa ser apreendido a
partir de seus próprios princípios poéticos, uma vez que a monologia
artística não conseguiria entender o trabalho do escritor:

\begin{quote}
Do ponto de vista de uma visão monológica coerente e da concepção do
mundo representado e do cânone monológico da construção do romance, o
mundo de Dostoiévski pode afigurar"-se um caos, e a construção dos seus
romances, algum conglomerado de matérias estranhas e princípios
incompatíveis de formalização. Só à luz da meta artística central de
Dostoiévski por nós formulada podem tornar"-se compreensíveis a profunda
organicidade, a coerência e a integridade de sua poética (\versal{BAKHTIN}, 2008,
p. 6).
\end{quote}

Dessa maneira, podemos ver a incomum unanimidade ao redor de Bakhtin e
suas análises, uma vez que Wellek, Schmid, Holquist e De Man estão
vinculados a perspectivas teóricas muito diferentes. Mas essa
concordância pode obscurecer algumas aporias muito importantes que estão
no âmago da análise de Bakhtin sobre Dostoiévski. A~partir de agora,
revisaremos a dialogia e a polifonia de Bakhtin a fim de descobrir seus
potenciais e limites para explicar a poética de Dostoiévski. Assim, é
importante apresentar algumas questões para que vejamos alguns
princípios que, a meu ver, não foram suficientemente desenvolvidos por
Bakhtin, tais como: em que consistiria o plano artístico do escritor?
Sobre quais bases Bakhtin assenta \emph{Problemas da poética de
Dostoiévski? }

Nossa revisão começa com a configuração dialógica da personagem. De
acordo com Bakhtin, a dialogia pressupõe a horizontalidade das vozes,
sua equipolência.

\begin{quote}
Uma poética dialógica deve ser capaz, antes de mais nada, de identificar
e correlacionar as relações entre os pontos de vista; ela deve ser
adequada à complexa arquitetura que enforma o ponto de vista do autor
sobre suas personagens, o ponto de vista das personagens sobre o autor
e, ademais, o ponto de vista de cada uma dessas instâncias sobre si
mesmas (\versal{HOLQUIST}, 1990, p. 162).
\end{quote}

Bakhtin elenca uma miríade de exemplos extraídos das obras
dostoievskianas para demonstrar como a voz narrativa não apenas se
estrutura em função do outro, mas, em sua própria articulação, já
pressupõe a alteridade como réplica e, no limite, altercação. Assim, o
homem do subsolo, protagonista (e paradoxalista) das \emph{Memórias do
subsolo}, sempre se dirige a leitores pressupostos que ora são
desprezados e (quase) preteridos, ora fazem com que o
narrador"-personagem quase se cale diante da pequenez que sente por conta
da onipresença alheia. Eis um fragmento"-síntese da dubiedade (e da
duplicidade) identitária do homem do subsolo:

\begin{quote}
Tenho, por exemplo, um terrível amor"-próprio. Sou desconfiado e me
ofendo com facilidade, como um corcunda ou um anão, mas, realmente, tive
momentos tais que, se me acontecesse receber um bofetão, talvez até me
alegrasse com o fato. (\ldots) Eu vos inquieto, faço"-vos mal ao coração,
não deixo ninguém dormir. Pois não durmais, senti vós também, a todo
instante, que estou com dor de dentes. Para vós, eu já não sou o herói,
que anteriormente quis parecer, mas simplesmente um homem ruinzinho, um
\emph{chenapan} (\versal{DOSTOIÉVSKI}, 2000a, p. 20; p. 27).
\end{quote}

O homem do subsolo, como personagem dostoievskiana paradigmática nesse
sentido, congrega em seu devir a voz do eu"-outro em sua própria
trajetória relacional. Se observarmos as demais obras do escritor pelo
prisma dialógico, veremos que a unidade múltipla do homem do subsolo é
im/explodida para que mais personagens venham à tona a partir do caráter
relacional de suas vozes.

Pensemos, por exemplo, no par Raskólnikov e Svidrigáilov, de \emph{Crime
e castigo.} Ambos germinaram em meio à erosão moderna que solapa Deus e
o decálogo de Moisés como princípios inquestionáveis e basilares. O
pobretão e, portanto, ex"-estudante de Direito Raskólnikov quer verificar
na práxis se consegue suplantar o \emph{não matarás} para lançar mão de
seus projetos de reforma da humanidade. Svidrigáilov, aristocrata
lascivo, entrevê possibilidades (ou melhor, aventuras) algo distintas (e
mais luxuriosas) a partir do esfacelamento da instância moral: o
estupro, a extorsão e a pedofilia. É~bem verdade que os homicídios de
Raskólnikov e as perversões de Svidrigáilov são empiricamente distintos,
mas o caráter relacional das infrações eclode a partir da mesma aporia
historicamente configurada. Nos diálogos limítrofes entre Raskólnikov e
Svidrigáilov, cujo alvo da extorsão libidinal fora ninguém menos que
Avdótia Románovna, \emph{a irmã de Raskólnikov}, os leitores sentimos o
quanto a intelecção e as decisões de ambos estão entrelaçadas de modo
que a dualidade das personagens nos remeta à unidade primordial da
crise. Dostoiévski nos faz refletir, subterraneamente, sobre os matizes
de classe diversos das duas personagens: Raskólnikov, intelectual filho
da \emph{cisão} (\emph{raskol}) de classe, lança mão do assassínio da
velha usurária para quem empenhara seus últimos pertences como que a
fazer terra arrasada da sociedade que não o incluía, ao passo que o
senhorial Svidrigáilov percorre o corredor polonês do desejo como que a
referendar o prazer do ego abastado contra as serviçais que se irmanam a
Raskólnikov. O~pobre (se) aniquila, o rico referenda, mas ambas as
``transvalorações'' partem da mesma vacuidade divina que agora arremessa
os órfãos uns contra os outros sem que nada possa refreá"-los.

Agora nós podemos dar início à aproximação entre Mikhail Bakhtin e a
Teoria Crítica como uma mediação que nos levará da dialogia e da
polifonia à dialética. Comecemos com uma comparação entre a noção
dialógica de identidade para Bakhtin e a crítica de Adorno e Horkheimer
ao ego burguês. De acordo com Michael Holquist, ``a dialogia
{[}milita{]} tanto contra a monologia, a ilusão de corpos hermeticamente
fechados em si mesmos ou psiques isoladas em meio ao individualismo
burguês, quanto contra a noção de identidades estanques e estáticas''
(1990, p. 90). Verdadeiras encarnações de crises e ideias (ideias em
crise), as personagens dostoievskianas -- e os leitores projetados e
pressupostos -- não se estruturariam como vozes solipsistas, como ``os
muros sem janelas das mônadas de suas próprias pessoas'' (\versal{ADORNO} e
\versal{HORKHEIMER}, 1985, p. 178), ``como se o indivíduo pudesse realizar o seu
destino com suas emoções e sentimentos, como se a interioridade do
indivíduo ainda pudesse fazer algo sem mediação {[}social{]}'' (\versal{ADORNO},
1980, p. 270). Nestes fragmentos, Adorno e Horkheimer não estão lendo
Dostoiévski, mas suas noções são muito próximas à lógica bakhtiniana que
apreende as personagens dostoievskianas como fluxos relacionais, vozes
recíprocas de um concerto.

O problema com a dialogia começa, a meu ver, com uma importante
contradição: como o diálogo equipolente é uma estrutura essencial para a
constituição da identidade das personagens, ele não pode ser considerado
um elemento contingente da poética de Dostoiévski. Uma vez que o crítico
russo quer estabelecer a coexistência do plano autoral -- a noção
fundamental de totalidade -- com as vozes dialógicas a partir de suas
próprias perspectivas, o homem do subsolo, Raskólnikov e Svidrigáilov
não são apenas criaturas de Dostoiévski. A~voz do autor pode colidir com
e ser contradita pelas próprias vozes e ideias de suas personagens. As
personagens não são apenas partes cuja soma enforma a poética como uma
totalidade sistêmica, já que elas também adquirem uma liberdade
determinada e relativa diante do todo ao qual pertencem, fato que
estabelece uma relação bastante complexa em meio à totalidade.

\begin{quote}
A impressionante independência interior das personagens dostoievskianas
foi alcançada através de meios artísticos determinados. Trata"-se, antes
de mais nada, da liberdade e independência que elas assumem na própria
estrutura do romance em relação ao autor, ou melhor, em relação às
definições comuns exteriorizantes e conclusivas do autor. Isto,
obviamente, não significa que a personagem saia do plano do autor. Não,
essa independência e liberdade integram justamente o plano do autor.
Esse plano como que determina de antemão a personagem para a liberdade
(relativa, evidentemente) e a introduz como tal no plano rigoroso e
calculado do todo (\versal{BAKHTIN}, 2008, p. 12).
\end{quote}

Dois aspectos contraditórios entre si despontam da colocação
bakhtiniana: a liberdade determinada das personagens e sua relação com o
\emph{plano rigoroso e calculado do todo}, plano que nos leva à
instância autoral que as vozes das personagens chegam a interpelar.
Conforme já vimos, a colisão entre as partes dialógicas -- a altercação
entre as personagens e a altercação dentro de uma mesma personagem com
vozes múltiplas e contraditórias -- é algo recorrente em meio à poética
de Dostoiévski. Mas, então, surge a questão: como conciliar a tensão das
partes com o plano autoral sem estilhaçar a própria noção de totalidade
em face da liberdade determinada e relativa das personagens que
contradiz a poética como um todo? O~que impediria essas personagens
relativamente autônomas de constituir universos monadológicos e
centrífugos que apenas orbitariam ao redor de si mesmos? Como essas
constelações particulares de personagens se articulariam com o universo
total das obras?

Conforme mencionamos na introdução deste capítulo, para que a dialogia e
a polifonia bakhtinianas consigam superar suas aporias, é essencial
deixar de lado os pressupostos da teoria tradicional, uma vez que a
completude hierárquica e sistêmica não poderia estar mais distante da
perspectiva bakhtiniana de conceber a liberdade determinada, relativa e
centrífuga das personagens em face de Dostoiévski como a base para uma
nova forma literária. No~entanto, \emph{enunciar} o surgimento de uma
revolução -- em outras palavras, \emph{pressupor} uma nova totalidade
polifônica sem realizá"-la teórica e analiticamente -- não é equivalente
a estabelecer a dinâmica real da poética de Dostoiévski. De qualquer
forma, Bakhtin não estava sozinho ao entrever as aporias da tentativa
poética de configurar uma totalidade. Eis que o jovem György Lukács e
sua \emph{Teoria do romance} se envolvem com os problemas epocais de
modo a perceber que a forma não consegue configurar uma totalidade sem
se enredar às aporias históricas.

\begin{quote}
Uma totalidade simplesmente aceita não é mais dada às formas; eis por
que elas têm ou de estreitar e volatilizar aquilo que configuram, a
ponto de poder sustentá"-lo, ou são compelidas a demonstrar polemicamente
a impossibilidade de realizar seu objeto necessário e a nulidade
intrínseca do único objeto possível, introduzindo assim no mundo das
formas a fragmentariedade da estrutura do mundo. (\ldots) A~forma do
romance, como nenhuma outra, é uma expressão do desabrigo do
transcendental. (\ldots) Com o colapso do mundo objetivo, também o sujeito
torna"-se um fragmento; somente o eu permanece existente, embora também
sua existência dilua"-se na insubstancialidade do mundo em ruínas criado
por ele próprio. Essa subjetividade a tudo quer dar forma e, justamente
por isso, consegue espelhar apenas um recorte. (\ldots) Uma unidade pode
perfeitamente vir à tona, mas nunca uma verdadeira totalidade (2000, p.
36; pp. 37-38; p. 52; p. 54).
\end{quote}

Bakhtin e Lukács constatam a possibilidade (momentânea) de unidade para
a constituição do todo formal. No mundo moderno e fragmentado, o sujeito
apenas alcança uma unidade contingente. Conforme Lukács afirma, o
colapso histórico do mundo objetivo -- algo que Dostoiévski
interpretaria como uma interpenetração envolvendo a morte de Deus, o
relativismo ético e o individualismo capitalista -- significa a
impossibilidade de uma nova totalidade a partir de uma perspetiva
subjetiva, arbitrária e fragmentada, ou, para usarmos as noções de
Bakhtin, a impossibilidade de criar uma nova totalidade a partir da voz
particular de uma determinada personagem. Assim, a articulação recíproca
da personagem dialógica com o concerto polifônico permanece problemática
e contraditória. Se Bakhtin considera Dostoiévski o fundador de um novo
gênero, é preciso demonstrar \emph{como} a polifonia articula as vozes
das personagens sem arregimentá"-las, de maneira que a expressão do
fragmento não corresponda à sua subsunção com vistas à constituição do
todo hierárquico. \emph{Como} Bakhtin poderia superar tal contradição,
de maneira que o sistema polifônico, de acordo com suas próprias
palavras, viesse a fundar uma nova totalidade \emph{integral? }

Bakhtin apresenta os procedimentos que muitos estudiosos de Dostoiévski
utilizam para (tentar) superar a dissonância fundamental do concerto
polifônico: eles isolam uma voz ou algumas vozes com suas respectivas
ideias e tentam provar que uma abordagem particular pode sintetizar as
obras de Dostoiévski. É~assim que deparamos com esquerdistas, niilistas,
eslavófilos, cristãos, cristãos ortodoxos, ultranacionalistas,
existencialistas, irracionalistas etc. a advogar que o \emph{conteúdo}
específico de suas vozes sintetiza a totalidade da forma dostoievskiana.
Para tais correntes, basta isolar uma determinada tendência como voz
hegemônica e submeter toda a miríade restante de ideias ao crivo da
ideologia hegemônica. Se Dostoiévski não fosse um escritor dialógico; se
Dostoiévski não fosse o escritor cujas aporias dialógicas movem a
totalidade poética através de suas contradições, conforme tentaremos
demonstrar, os problemas que Bakhtin enfrenta já estariam resolvidos.
Ocorre que a honestidade intelectual o impede de apreender Dostoiévski
como o fundador de um novo gênero poético a partir da velha noção de
sistema legada pela teoria tradicional. Seria necessário, então, abrir
mão da sistemática configurada \emph{a priori} para se embrenhar em meio
à logicidade outra do material dostoievskiano. Mas como utilizar o
conceito de plano artístico se a dialogia das vozes equipolentes não
apresentava, em termos imediatos, a articulação das personagens
centrífugas com o todo romanesco? Estariam certas, então, as análises
monológicas da obra de Dostoiévski, para as quais o escritor havia
composto uma verdadeira colcha de retalhos à iminência do completo
esgarçamento?\footnote{Para uma leitura de Dostoiévski a partir de
  parâmetros que não são próprios à sua configuração poética -- uma
  leitura que, portanto, acaba por enquadrar o autor em categorias
  monológicas que reduzem suas perspectivas criativas --, conferir a
  análise do escritor russo Vladimir Nabokov. In: Fabio Brazolin
  Abdulmassih, \emph{Aulas de literatura russa -- F.M. Dostoiévski por
  V. Nabokov: por que tirar Dostoiévski do pedestal?} Dissertação de
  mestrado, \versal{FFLCH-USP}. São Paulo: 2010.}

Para muitos críticos vinculados a determinadas nuances ideológicas das
obras de Dostoiévski\footnote{A seleção de autores e temas em questão
  pretende apenas apresentar um breve panorama da polissemia dos estudos
  dostoievskianos em diálogo com a logicidade contraditória da obra do
  escritor. Para leituras eminentemente religiosas da obra de
  Dostoiévski, conferir Nikolai Berdiaev, \emph{L'esprit de
  Dostoïevski}. Paris: Éditions Saint"-Michel, 1929; Paul Evdokimov,
  \emph{L'orthodoxie}. Paris: \versal{DDB}, 1979; V. Ivanov, \emph{Dostoïevski,
  tragédie, mythe, religion.} Paris: Éditions des Syrtes, 2000; Karen
  Stepânian, ``\emph{Os irmãos Karamázov:} a hosana de Dostoiévski''.
  In: \emph{Caderno de Literatura e Cultura Russa: Dostoiévski}. São
  Paulo: Ateliê Editorial, 2008; Natália Arsentieva, ``El sueño de un
  hombre ridículo: el viaje hacia la verdad\emph{''.} In: \emph{Idem};
  Luiz Felipe Pondé, \emph{Crítica e profecia: a filosofia da religião
  em Dostoiévski.} São Paulo: Editora 34, 2003. Para uma leitura que
  aproxima Dostoiévski da teoria nietzscheana em relação aos
  questionamentos da moralidade e ao niilismo, conferir Lev Shestov,
  \emph{Dostoevsky and Nietzsche: The Philosophy of Tragedy}. Ohio: Ohio
  University Press, 1969. Para apreensões mais voltadas às questões
  nacionais em Dostoiévski, conferir Igor Volguin, ``A~devolução do
  bilhete: paradoxos da autoconsciência nacional''\emph{.} In:
  \emph{Caderno de Literatura e Cultura Russa: Dostoiévski}; Valentin E.
  Khálizev, ``Ivan Karamázov como mito russo do início do século \versal{XX}''.
  In: \emph{Idem}. Para apreensões mais voltadas aos problemas da forma
  dostoievskiana, conferir Leonid Grossman, \emph{Dostoiévski artista.}
  Rio de Janeiro: Civilização Brasileira, 1967; Paulo Bezerra, ``Mundos
  desdobrados, seres duplicados''. In: \emph{Caderno de Literatura e
  Cultura Russa: Dostoiévski}; Aurora Fornoni Bernardini, ``Dostoiévski:
  criação, poesia e crítica''\emph{.} In: \emph{Idem}; Boris
  Schnaiderman, \emph{Dostoiévski: prosa e poesia}. São Paulo:
  Perspectiva, 1982.}, é possível \emph{estancar o movimento}
dostoievskiano para dizer, por exemplo, que o amoralismo homicida de
Raskólnikov configura o núcleo das tensões. Tal concepção subordina os
demais momentos dostoievskianos -- as vozes de contraposição -- a essa
ideia, de maneira que o diálogo é sustado em função de determinada
posição. Segundo Bakhtin, tais críticos não percebem que as vozes se
configuram como teses sempre à espera das antíteses recíprocas. Assim,
ao niilismo de Raskólnikov se contrapõe a abnegação cristã de Sônia
Marmieládova, a ex"-prostituta que acompanhará o protagonista de
\emph{Crime e castigo} à Sibéria para ajudá"-lo a expiar suas faltas em
busca da completa regeneração. Sônia, por sua vez, não se veria como a
síntese absoluta que estancaria o devir. A~nova síntese se apresenta,
justamente, como uma nova tese a ser denegada por uma antítese
subsequente. Se cruzarmos as fronteiras das obras, a voz de Sônia
Marmieládova se ouvirá replicada pelo jovem Hippolit (\emph{O~idiota}),
que se vê emparedado pela natureza ao descobrir que a tuberculose não
lhe deixará mais de três semanas de vida. Que poderia a bela abnegação
de Sônia fazer diante da doença que coage Hippolit a assistir ao próprio
naufrágio?

\begin{quote}
Se eu tivesse o poder de não nascer, certamente não aceitaria a
existência nessas condições escarnecedoras. Mas eu ainda tenho o poder
de morrer, ainda que eu entregue o que já foi composto. Não é grande o
poder, tampouco é grande a rebeldia. (\ldots) A~natureza limitou a tal
ponto minha atividade com as suas três semanas de sentença que o
suicídio talvez seja a única coisa que eu ainda tenha tempo de começar e
terminar por minha própria vontade. (\ldots) Nós humilhamos demasiadamente
a providência, atribuindo"-lhe os nossos conceitos, movidos pelo despeito
de não podermos compreendê"-la. Porém, se ademais é impossível
compreendê"-la, então, repito, também é difícil responder por aquilo que
não é dado ao homem compreender (\versal{DOSTOIÉVSKI}, 2002, pp. 465-466).
\end{quote}

Se continuarmos a acompanhar o movimento interromanesco de teses e
antíteses em constante entrechoque, o moribundo Hippolit se verá
consolado pelo Príncipe Míchkin, a fusão dostoievskiana de Jesus Cristo
e Dom Quixote, até que a bondade avassaladora da encarnação de \emph{o
idiota} depare com o protótipo do super"-homem, o \emph{demônio} Nikólai
Stavróguin, revolucionário niilista e pedófilo que, no ápice do regozijo
pela ruptura com a norma, recebe o seguinte conselho do clérigo Tíkhon:

\begin{quote}
O desejo de martírio e auto"-sacrifício apodera"-se do senhor; domine
também esse desejo, desista desses folhetos {[}revolucionários{]} e de
sua intenção e assim vencerá tudo. Desvele seu orgulho e seu demônio!
Acabará triunfando, atingirá a liberdade\ldots (\versal{DOSTOIÉVSKI}, 2004, pp.
685-686).
\end{quote}

Segundo o clérigo Tíkhon, a liberdade redentora alcançaria o
\emph{Übermensch} (super"-homem) demoníaco, se Stavróguin se entregasse
de corpo e alma ao monastério.

A sequência vertiginosa, escatológica e contraditória de teses e
antíteses só para por aqui para tentarmos articular o sentido das
contraposições para além do entrechoque. Queremos uma visão panorâmica
em relação à poética de Dostoiévski para estabelecermos as mediações
entre as vozes em diálogo e as cordas vocais que as fazem falar.
``Parece que todo aquele que penetra no labirinto do romance polifônico
não consegue encontrar a saída e, obstaculizado por vozes particulares,
não percebe o todo'' (\versal{BAKHTIN}, 2008, p. 51). Mas, ora, a
\emph{enunciação} de que há um todo não significa a sua revelação, assim
como o mero ímpeto de se descobrir a saída do labirinto não nos liberta
do cárcere. Permanece rediviva, então, a tensão do jovem Lukács, para
quem ``o romance é a epopeia de uma era para a qual a totalidade
extensiva da vida não é mais dada de modo evidente, para a qual a
imanência do sentido da vida tornou"-se problemática, mas que ainda assim
tem por intenção a totalidade'' (2000, p. 55).

Em \emph{Sobre a questão judaica}, Marx, lembrando o Hegel do prefácio à
\emph{Fenomenologia do Espírito}, traz à tona dois momentos dialéticos
que nos podem ajudar a refletir sobre o emparedamento da totalidade pela
contradição. ``A~formulação de uma questão é a sua resolução. (\ldots) Como
se resolve uma oposição? Tornando"-a impossível'' (2010, p. 37). Pensemos: o que está \emph{impossibilitando} a superação da contradição em meio à constituição sistêmica? E~mais: se a formulação de uma questão
pressupõe toda a articulação do problema, de modo que a resposta deriva
sempre da pergunta, é preciso lançar a aporia contra si mesma para
torná"-la impossível, isto é, para fazê"-la mover"-se.

Se a história se transfigurou; se a história, mediação imanente para a
constituição da forma, se tornou mais contraditória ao relegar o
indivíduo ao desterro transcendental e à coação do capitalismo tardio,
nós poderíamos perguntar: se o mundo estático e estamental da nobreza
feudal se desmanchou no ar, por que a crítica de Dostoiévski precisa
lançar mão do sistema hierárquico de modo necessário e inequívoco? Por
que Bakhtin deveria prescindir da contradição como se ela fosse um mero
erro? Se Bakhtin e Lukács entreviram, a partir da logicidade do material
dostoievskiano, que ``mundo contingente e indivíduo problemático são
realidades mutuamente condicionantes'', talvez devamos arremessar ``o
romance, em contraposição à existência em repouso na forma consumada dos
demais gêneros, (\ldots) como algo em devir, como um processo'' (\versal{LUKÁCS},
2000, p. 79; p. 72). Tal ideia coincide com a seguinte colocação de
Michael Holquist (1990):

\begin{quote}
A dialogia é uma forma de arquitetura, a ciência geral de ordenação das
partes em um todo. Em outras palavras, a arquitetura é a ciência das
relações. (\ldots) Ademais, Bakhtin enfatiza que uma relação nunca é
estática, mas está sempre em processo de ser feita ou desfeita (p. 29).
\end{quote}

Bakhtin tateia pelo novo dostoievskiano, reconhece seus fragmentos, mas
não se dá conta de que se trata de escombros, de que a totalidade já não
está dada \emph{a priori}, de que é preciso não apenas contemplar os
entrechoques das partes (os diálogos), mas descobrir aquilo que os
movimenta. No limite da descoberta sobre a totalidade polifônica,
Bakhtin recua e volta a pensar em termos de um sistema coeso, quando a
dinâmica dostoievskiana demanda a reflexão a contrapelo da
\emph{integralidade. }

Eis, então, que a \emph{Teoria estética} de Theodor Adorno insufla ar
contraditório no concerto polifônico que não sabe como se estruturar de
forma total para sugerir que ``a dissonância é a verdade da harmonia'' e
``a desintegração é a verdade da arte integral'' (2012, p. 171; p. 465).
No limite, Adorno poderia perguntar a Bakhtin: e se ``a arte de elevada
pretensão tende a ultrapassar a forma como totalidade e desemboca no
fragmentário?'' (2012, p. 225). E~se o fragmentário insinuar um novo
movimento da totalidade para além da tradicional estruturação estática?
Integração e desintegração, dissonância e harmonia não seriam momentos
que se excluiriam de forma absoluta, mas polos relacionais que se
\emph{superariam} por meio da pressuposição recíproca. A~integração
entraria em ruínas em meio à desintegração para que uma voz redentora
caminhasse pelos escombros para insinuar aos humilhados e ofendidos de
Dostoiévski que a matéria putrefata não apenas já soergueu casas e
memórias como pode, ainda uma vez, voltar a fornecer abrigos.
\emph{Contra} a reconciliação, vozes antitéticas que insistem em
chafurdar no lodo do sadismo -- no limite, no charco da automutilação
sadomasoquista -- querem convencer as teses positivas de que só é
possível se esgueirar pelo subsolo. Assim, ao invés de um todo
fulgurante que se eleva como um palácio, a poética de Dostoiévski
rastejaria como (e com) o movimento contraditório dos escombros.

Para que possamos estabelecer as devidas mediações que nos levarão aos
escombros como síntese do movimento dostoievskiano, precisamos, antes de
mais nada, reconstituir a crítica bakhtiniana à dialética para entender
por que o crítico russo não totalizou -- e/ou não pôde totalizar -- o
concerto polifônico como quebra da espinha tonal. Então entreveremos os
limites da dialogia que hipostasia uma equipolência de vozes (ainda)
inexistente em meio à realidade histórica para a apreensão de uma
poética da contradição.

\section{1.3. Pressupostos dialógicos para o devir dialético}

Um passo atrás para darmos dois à frente: acompanhemos de forma mais
pormenorizada o movimento dialógico que, no limite, se configura como
dialogismo velado.

\begin{quote}
Imaginemos um diálogo entre duas pessoas no qual foram suprimidas as
réplicas do segundo interlocutor, mas de tal forma que o sentido geral
não tenha sofrido qualquer perturbação. O~segundo interlocutor é
invisível, suas palavras estão ausentes, mas deixam profundos vestígios
que determinam todas as palavras presentes do primeiro interlocutor.
Percebemos que esse diálogo, embora só um fale, é um diálogo sumamente
tenso, pois cada uma das palavras presentes responde e reage com todas
as suas fibras ao interlocutor invisível, sugerindo fora de si, além dos
seus limites, a palavra não"-pronunciada do outro. (\ldots) Em Dostoiévski
esse diálogo velado ocupa posição muito importante e sua elaboração foi
sumamente profunda e sutil (\versal{BAKHTIN}, 2008, p. 226).
\end{quote}

Em correlação com a análise bakhtiniana, eis as primeiras palavras do
homem do subsolo:

\begin{quote}
Sou um homem doente\ldots %espaço
Um~homem mau. Um homem desagradável. Creio que
sofro do fígado. Aliás, não entendo níquel da minha doença e não sei, ao
certo, do que estou sofrendo. Não me trato e nunca me tratei, embora
respeite a medicina e os médicos. Ademais, sou supersticioso ao extremo;
bem, ao menos o bastante para respeitar a medicina. (Sou suficientemente
instruído para não ter nenhuma superstição, mas sou supersticioso.) Não,
se não quero me tratar, é apenas de raiva. Certamente não compreendeis
isto (\versal{DOSTOIÉVSKI}, 2000a, p. 15).
\end{quote}

Imaginemos o autonomeado paradoxalista do subsolo sobre um palco -- ou
melhor, junto a um púlpito. A~princípio, suas palavras parecem esgarçar
o hedonismo de nossos tempos, já que o narrador subterrâneo vai se
flagelando cada vez mais diante dos outros (vós) pressupostos à sua
fala. Doente, mau, desagradável. Quando diz que não entende nada sobre
sua doença e que não sabe de que sofre, o homem do subsolo se vê
bastante diminuído em relação aos interlocutores espectrais. Quando
afirma ser supersticioso, então, os leitores esclarecidos tendem a se
sentir diante de um bárbaro medieval. Mas eis que, em uma virada
irônico"-dialética bem própria à personagem cuja trajetória será
analisada com minúcia no próximo capítulo, o homem do subsolo revela que
seu respeito pela medicina se deve, na verdade, à extrema superstição de
sua índole. Ora, como associar a ciência médica à superstição sem
zombar, veladamente, dos interlocutores invisíveis que, até então,
poderiam se sentir superiores ao irracionalismo do narrador? E~para
arrematar a virada na queda de braço, o protagonista há pouco néscio se
torna ``suficientemente instruído'' -- mais do que os leitores? --, mas,
ainda assim, sua superstição aceita a medicina. Ao~fim e ao cabo, os
interlocutores espectrais não temos a capacidade de entender por que a
raiva do homem do subsolo não lhe permite tratar"-se -- somos,
finalmente, rebaixados diante da personagem que a princípio só fazia
flagelar"-se. O~diálogo velado permite ao protagonista, então,
``substituir com sua própria voz a voz de outra pessoa'' (\versal{BAKHTIN}, 2008, p. 245), mas, neste momento, descobrimos algumas dificuldades para
aceitarmos a pressuposição ética de que as vozes em diálogo são
efetivamente equipolentes. Ora, trata"-se não de um diálogo constituído
de forma liberal, em que as colocações são levadas em consideração com
igual valia e força de expressão, mas de um verdadeiro duelo ao longo do
qual a suscetibilidade da personagem dostoievskiana estimula e flagela
os interlocutores pressupostos. E~mais: como os interlocutores são
projetados, suas características próprias sequer despontam em meio ao
diálogo velado que, nesse sentido, revela seu caráter eminentemente
solipsista. Como a (auto)flagelação do homem do subsolo tende a se
voltar contra os demais, teríamos que inverter o sinal ético da dialogia
bakhtiniana para falar sobre vozes equi"-impotentes.

Ao entrever as relações danificadas em meio ao universo ficcional do
escritor russo, Bakhtin apreende Dostoiévski como um crítico do
capitalismo como a base social e histórica para o individualismo e as
aporias éticas. Nesse sentido, vale apresentar a citação a seguir para
que desvelemos outro importante aspecto de contradição entre Bakhtin e a
Teoria Crítica:

\begin{quote}
O capitalismo criou as condições para um tipo especial de consciência
permanentemente solitária. Dostoiévski revela toda a falsidade dessa
consciência, que se move em um círculo vicioso. Daí a representação dos
sofrimentos, das humilhações e do \emph{não"-reconhecimento} do homem na
sociedade de classes. Tiraram"-lhe o reconhecimento e privaram"-no do
nome. Recolheram"-no a uma solidão forçada, que os insubmissos
transformaram em uma \emph{solidão altiva} (passar sem o reconhecimento,
sem os outros) (\versal{BAKHTIN}, 2008, p. 323).
\end{quote}

Bakhtin bem reconhece os fundamentos histórico"-sociais da egolatria
mutilada das personagens dostoievskianas. Sendo assim, é importante
perguntar: como \emph{partir} de uma ontologia dialógica se o
capitalismo faz a equipolência das vozes regredir a uma nova versão da
guerra de todos contra todos? A~Teoria Crítica, nosso par dialógico que
tenta superar as aporias bakhtinianas, sugere que, talvez, ``o interesse
pela totalidade estética quisesse objetivamente ser o interesse por uma
organização adequada da totalidade'' (\versal{ADORNO}, 2012, p. 27). Mas como bem
sabem o jovem Lukács e a Teoria Crítica a acompanhar o negativo
historicamente configurado, ``toda tentativa de configurar o utópico
como existente acaba apenas por destruir a forma sem criar realidade''
(\versal{LUKÁCS}, 2000, p. 160).

Como consideramos a história o fundamento e a dinâmica das ideias e da
criação artística, não podemos deixar de observar o contexto de
publicação e disseminação de \emph{Problemas da poética de Dostoiévski.}
``O~desejo compreensível e quase instintivo de não ser tão explícito ao
escrever para o público soviético'' (\versal{MORSON}, 1986, p. 83): os
intelectuais estavam efetivamente premidos pela política do Partido a
ditar os rumos daquilo que poderia ou não ser discutido. Além disso, os
desdobramentos da União Soviética poderiam sugerir a Bakhtin concepções
ambíguas sobre o devir da dialogia. Por um lado, a dicotomia ideológica
que se estabeleceu com a Guerra Fria tornaria a \versal{URSS} um campo de disputa
para que o socialismo mostrasse sua constituição efetivamente
democrática e dialógica para além da ditadura do proletariado que há
muito se revertera em ditadura do Partido. Não à toa, então, a segunda
versão de \emph{Problemas da poética de Dostoiévski} voltou a ser
publicada somente em 1963, dez anos após a morte de Stálin. Por outro
lado, a ontologia dialógica poderia ser lida como um diatribe
\emph{velada} ao socialismo de caserna que não liberava o potencial
emancipatório da utopia como equipolência social efetiva. Assim, a
aporia para a constituição do romance polifônico total, não resultante
``da contingência do talento e transparente em sua necessidade,
aponta{[}ria{]} para o social'' (\versal{ADORNO}, 1980a, p. 267). Poderíamos,
então, concluir que a completude do todo polifônico seria a tolerância
para além da ditadura, a liberação da multiplicidade de vozes em
contraposição à propaganda oficial e ao cerceamento da
liberdade\footnote{Tais apreciações estético"-políticas abrem
  perspectivas para investigações histórico"-documentais que procurem
  comprovar a diatribe social de Bakhtin em correlação não apenas com a
  lógica do material polifônico, mas também com \emph{diálogos
  necessariamente velados} -- a polícia política à espreita -- por meio
  de cartas, anotações, gravações e pronunciamentos em círculos de
  pesquisas. O~escopo deste livro me leva ao acompanhamento da
  transformação histórico"-formal de Dostoiévski a partir da teoria
  dialógico"-polifônica como um momento da dialética negativa. A
  descoberta histórico"-documental de que \emph{Problemas da poética de
  Dostoiévski} também poderia ser lido como uma crítica fundamental aos
  desdobramentos da \versal{URSS} apresenta, por um lado, um importante marco de
  reflexão sobre as profundas imbricações entre estética e política; por
  outro lado, no entanto, o devir histórico nos mostra que tal
  perspectiva datada projetaria a ontologia do diálogo como condição
  fundamental dos homens a despeito da reificação que deformava a
  alteridade como instrumento da classe dominante e como adversário para
  a altercação cotidiana por meio da democracia. Parece"-me, então, que
  as premissas de Bakhtin precisam ser continuamente arremessadas contra
  si mesmas para que a limitação histórica da dialogia nos faça
  percorrer o negativo da ontologia que não se realizou como utopia.}.

\begin{quote}
Ser significa comunicar"-se pelo diálogo. Quando termina o diálogo, tudo
termina. Daí o diálogo, em essência, não poder nem dever terminar. No
plano da sua concepção de mundo utópico"-religiosa, Dostoiévski transfere
o diálogo para a eternidade, concebendo"-o como um eterno co"-júbilo, um
eterno co"-deleite, uma eterna con"-córdia. No plano do romance isso se
apresenta como inconclusibilidade do diálogo, apresentando"-se
primariamente como a sua infinidade precária (\versal{BAKHTIN}, 2008, p. 293).
\end{quote}

O diálogo, então, tenderia ``para o outro infinito'' (\versal{IDEM}, p. 265),
para um concerto eterno em que as vozes não cessariam de se expressar --
como se o maestro, Deus, não as regesse à revelia de suas
características precípuas. ``Se procurarmos uma imagem para a qual como
que tendesse todo esse mundo, essa imagem seria a Igreja como comunhão
de almas imiscíveis, onde se reúnem pecadores e justos'' (\versal{IDEM}, pp.
29-30). Vemos, então, que Bakhtin transforma a totalidade em uma noção
de eternidade que coincidiria com a tolerância de um deus que permitisse
a progressão incessante dos diálogos. Assim, no plano romanesco,
apreendemos a série dialógica que se prolonga por se prolongar, sem que,
para Bakhtin, haja uma teleologia nos debates, uma noção de formação,
transformação e superação (\emph{Aufhebung}).

\begin{quote}
A categoria fundamental da visão artística de Dostoiévski não é a de
formação, mas a de \emph{coexistência} e \emph{interação}. Dostoiévski
via e pensava seu mundo predominantemente no espaço, e não no tempo.
(\ldots) Toda a matéria semântica que lhe era acessível e a matéria da
realidade, ele procurava organizar em um tempo sob a forma de
confrontação dramática e procurava desenvolvê"-la extensivamente (\versal{IDEM},
p. 31).
\end{quote}

Chegamos, então, à efetiva encruzilhada que até este momento entrelaçou
a polifonia e a dialética, mas que, a partir de agora, as fará trilhar
caminhos distintos. Estamos preparando o caminho para uma possível
superação (\emph{Aufhebung}) da polifonia pela dialética, mas antes
precisamos desdobrar a noção de que a confrontação das personagens
dostoievskianas se desenvolve \emph{extensivamente}, isto é, em meio a
um espaço que cristaliza o \emph{devir do tempo.}

Formação demanda tempo. Coexistência e interação, espaço e tolerância. É
bem verdade que o Sermão da Montanha de Jesus Cristo eleva o
oferecimento da outra face ao inimigo como mandamento essencial do amor
ao próximo. Assim, o respeito pela condição do outro transforma a
coexistência em perdão. Mas então deveríamos perguntar por que o homem
do subsolo e Raskólnikov sofrem em demasia. O homem do subsolo alardeia
aos quatro cantos que é mau e desagradável, que lhe apetece assistir ao
sofrimento alheio, que seu amor próprio doentio -- aliado a uma
autoflagelação não menos patológica -- \emph{deforma} as relações com os
demais como sucessivas e inequívocas quedas de braço. Porém, se assim
fosse indefinidamente, por que ele tentaria se redimir diante de Liza, a
prostituta? Retomemos brevemente o contexto da narrativa: o protagonista
sai de seu subsolo para encontrar antigos ``colegas'' dos tempos de
escola. Todos o desprezam por sua atual condição -- o baixíssimo
salário, a moradia indevida, as roupas decrépitas. Mas o homem do
subsolo ainda pode se escorar em sua condição intelectual mais elevada.
O jantar que marca a despedida de um dos convivas para uma província
longínqua da Rússia -- um pedágio necessário para que Zvierkóv alcance
novos patamares na carreira pública -- logo se transforma em um
verdadeiro duelo entre o achincalhado homem do subsolo e os demais que
não sabem por que o paradoxalista teria sido convidado.

A narrativa progride vertiginosamente, de modo que a personagem se
mutile cada vez mais na medida em que percebe que os outros a desprezam.
Quando um ``colega'' sugere que o ordenado do protagonista é baixo, o
homem do subsolo vai e revela o quanto ganha; quando tudo já está se
desfazendo, quando todos o abandonam para ir a um bordel, o
paradoxalista, no ápice da desolação e da humilhação, se ajoelha diante
do anfitrião do jantar para -- pedir dinheiro emprestado. Dostoiévski
apresenta sua maestria em fundir pontos de virada narrativa a cicatrizes
da alma. Tudo isso faria parte da condição contraditória que Bakhtin
aceita como coexistência das vozes imiscíveis e equipolentes. Ocorre que
o homem do subsolo, demasiado humano, vai ao bordel ao qual teriam se
dirigido seus ``colegas'' para tirar a limpo toda aquela situação
humilhante. Lá chegando, não encontra os convivas, mas a jovem
prostituta Liza, oriunda da belíssima cidade medieval de Riga.
Dostoiévski sabe utilizar como ninguém a simbologia da dominação. Se a
Rússia de fato nutre um complexo de inferioridade e um ressentimento
historicamente geridos em relação à Europa Ocidental, os Países Bálticos
-- Letônia, cuja capital é Riga, Estônia e Lituânia --, às margens da
fronteira tsarista e soviética, sempre sofreram as invectivas do
imperialismo russo. Assim, os \emph{civilizados} ``colegas'' do homem do
subsolo, que o humilharam há pouco, agora fornecem o furor para que o
ressentimento subterrâneo seduza e humilhe a \emph{prostituta da
Letônia.} E assim teríamos a transferência inequívoca da dor para o
outro, mas, em um primeiro momento, o homem do subsolo lança mão de uma
conquista livresca para ganhar a confiança de Liza e doutriná"-la a
deixar, de uma vez por todas, a vida no meretrício. A jovem se vê
cativada e revela ao protagonista que chegara a receber, dias antes, uma
carta de um estudante que, insciente em relação à sua vida desgraçada, a
queria cortejar. Liza poderia então casar"-se! O homem do subsolo tenta
sufocar a compaixão que sente e, num rompante, dá seu endereço para Liza
a fim de que os dois continuem a conversar em um local mais humano. O
desfecho do imbróglio de fato leva Liza, alguns dias depois, à casa de
nosso protagonista, e o momento da chegada da jovem não poderia ser mais
dostoievskiano: o homem do subsolo acabara de ser humilhado por seu
\emph{criado}, o altivo Apolón, que, munido de efetiva consciência de
classe -- reformista, não revolucionária --, se recusava a fazer os
serviços até que seu salário atrasado lhe fosse pago. Assim, ainda uma
vez, a humilhação deve ser paga com mais humilhação. Então, o homem do
subsolo começa a injuriar Liza e passa a lhe dizer que tudo aquilo que
havia recomendado a ela no bordel não passava de um engodo para
seduzi"-la e torná"-lo altivo perante seus olhos. O homem do subsolo se
faz trêmulo, diz e se desdiz e, no ápice da fúria que se mescla ao
torpor, começa a chorar. Ora, mais um motivo para despertar sua
suscetibilidade doentia, porque ele chora diante de uma prostituta, ele
se rebaixa diante da mais vil das criaturas socialmente proscritas. Mas
eis que a maestria de Dostoiévski atua ainda uma vez para que a
dialética e o devir irrompam do seio da contradição: Liza não se mostra
uma desalmada que simplesmente relega o homem do subsolo. Ela se condói
pelo outro e começa a chorar junto com o protagonista como que a tomar
sua dor para si. Então, o homem do subsolo, o ressentido por excelência,
o ardiloso, o maquiavélico, o sádico, o masoquista, expele a contradição
e mostra que seu \emph{devir} não coincide consigo mesmo:

\begin{quote}
-- Não me deixam\ldots Eu não posso ser\ldots bondoso! -- mal proferi; em
seguida fui até o divã, caí nele de bruços e passei um quarto de hora
soluçando, presa de um verdadeiro acesso de histeria. Ela deixou"-se cair
junto a mim, abraçou"-me e pareceu petrificar"-se naquele abraço (2000a,
p. 140).
\end{quote}

De fato, o transcorrer da narrativa calará esse ímpeto de
\emph{superação} do homem do subsolo: o protagonista humilhará Liza ao
tentar pagar a prostituta pela visita. Com essa atitude vil, o
paradoxalista consegue esmaecer a compaixão de Liza, e a moça o
abandona. Mas, ainda assim, a contradição se movimentou: o homem do
subsolo não está satisfeito com sua condição. Ele quer mais, ele quer se
transformar, ele quer se superar. Não conseguir é um grande dilema, mas
a noção de \emph{desdobramento da contradição} está efetivamente
presente em Dostoiévski.

Rompamos ainda uma vez a fronteira das obras e alcancemos o dilema
visceral de Raskólnikov. Cometido o crime, o castigo passa a
dilacerá"-lo. Não se trata propriamente do medo ególatra de ser punido e
de ver cerceado o ímpeto hedonista e utilitário do eu. Trata"-se do
remorso, do arrependimento, da impossibilidade de Raskólnikov ser um
indiferente moral como Napoleão. O duplo homicídio o fustiga, ele não
consegue suportar o fardo de ter aspergido sangue alheio, de ter
infringido o \emph{não matarás.} Então, Dostoiévski transforma a
prostituta Liza de \emph{Memórias do subsolo} na Sônia Marmieládova de
\emph{Crime e castigo.} O pai de Sônia, o bufão Marmieládov, sustenta
seu alcoolismo às custas do meretrício da filha. A despeito de ser
humilhada pela madrasta, Sônia ainda sustenta a casa do pai. Mas seu
coração não foi maculado, e, como Liza diante do homem do subsolo, ela
não pode conter a compaixão ao saber que Raskólnikov havia matado a
usurária Alióna Ivánovna e sua irmã a golpes de machada. Sônia implora
para Raskólnikov se entregar, ele precisaria pedir perdão à terra que
maculara, beijar o chão diante do Mercado do Feno\footnote{O Mercado do
  Feno, em São Petersburgo, teria como equivalente o Mercado Municipal,
  em São Paulo. Trata"-se de um amplo mercado em uma região central e
  pauperizada, como a que abriga o Mercadão na capital paulistana.}, em
uma encruzilhada. Ao fim, Raskólnikov de fato se entrega e é condenado a
trabalhos forçados na Sibéria. Sônia está lá, ao seu lado, para
consolá"-lo e acompanhá"-lo em sua transformação. O término do romance nos
apresenta algo além da coexistência bakhtiniana, o prenúncio da
transformação \emph{ao longo do tempo}, o sentido da superação:

\begin{quote}
Ela {[}Sônia{]} passou todo esse dia intranquila e à noite chegou até a
adoecer. Mas ela estava tão feliz que quase se assustava com a sua
felicidade. Sete anos, \emph{apenas} sete anos! No começo da sua
felicidade, em outros instantes, os dois estavam prontos para considerar
esses sete anos como sete dias. Ele não sabia nem que essa \emph{nova
vida} {[}grifo meu{]} não lhe sairia de graça, que ainda deveria pagar
caro por ela, pagar por ela com um grande feito no futuro\ldots

Mas aqui já começa outra história, \emph{a história da renovação gradual
de um homem, a história do seu paulatino renascimento, da passagem
progressiva de um mundo a outro, do conhecimento de uma nova realidade,
até então totalmente desconhecida} {[}grifo meu{]}. Isto poderia ser o
tema de um novo relato -- mas este está concluído (2001, p. 561).
\end{quote}

É bem verdade que o corte abrupto de Dostoiévski não nos permite
acompanhar a transformação de Raskólnikov. Mas, \emph{aqui}, é esboçado
um sentido para além da condição torpe do utilitarismo e do hedonismo,
algo que, como vimos, também já irrompera em \emph{Memórias do subsolo.}
A \emph{con"-córdia}, para retomarmos uma categoria que Bakhtin atribui à
coexistência dostoievskiana, agora pressupõe a discórdia, de modo que
uma nova condição possa despontar. Ainda assim, continuemos a explorar a
análise bakhtiniana para levarmos o diálogo a falar para além de si
mesmo.

\begin{quote}
Essa tendência sumamente obstinada a ver tudo como coexistente, perceber
e mostrar tudo em contiguidade e simultaneidade, como que situado no
espaço e não no tempo, leva Dostoiévski a dramatizar no espaço até as
contradições e etapas interiores do desenvolvimento de um indivíduo,
obrigando as personagens a dialogarem com seus duplos, com o diabo, com
seu \emph{alter ego} e com sua caricatura (\ldots). Essa mesma
particularidade de Dostoiévski explica o fenômeno habitual das
personagens duplas em sua obra. (\ldots) Dostoiévski procurava converter
cada contradição interior de um indivíduo em dois indivíduos para
dramatizar essa contradição e desenvolvê"-la extensivamente. Essa
particularidade tem sua expressão externa na propensão do escritor pelas
cenas de massa, em sua tendência a concentrar em um lugar e em um tempo
-- contrariando frequentemente a verossimilhança pragmática -- o maior
número possível de pessoas e de temas, ou melhor, concentrar em um
instante a maior diversidade qualitativa possível. Daí a tendência a
seguir no romance o princípio dramático da unidade do tempo. Daí a
rapidez catastrófica da ação, o movimento em turbilhão, o dinamismo
(\versal{BAKHTIN}, 2008, pp. 31-32).
\end{quote}

A espacialização do tempo, segundo Bakhtin, cindiria as personagens em
duplos para que o homicida Raskólnikov deparasse com o estuprador
Svidrigáilov como um sentido outro para a sua desconstrução da
transcendência. A~erosão da divindade solaparia as raízes das ações
éticas e faria germinar o niilismo. Isso parece claro com relação às
personagens \emph{decaídas.} Mas e quanto às personagens redentoras? De
fato, Liza se apieda pelo homem do subsolo, e Sônia toma como missão de
sua vida acompanhar o condenado Raskólnikov. Então poderíamos perguntar
a Bakhtin: deveríamos falar apenas em \emph{con"-córdia} e em respeito ao
outro tal como ele se apresenta? Nesse caso, parece"-me que Bakhtin não
levou às últimas consequências o princípio da dialogia. Por um lado, é
verdade que a espacialização do tempo transforma a unidade das
personalidades em cisão identitária espacializada para que as ideias
encarnadas possam se confrontar em meio à ``rapidez catastrófica da
ação''.

\begin{quote}
No plano da cosmovisão abstrata, essa particularidade se manifestou na
escatologia política e religiosa do romancista, em sua tendência a
aproximar os ``fins'', a tateá"-los no presente, a vaticinar o futuro
como já presente na luta das forças coexistentes (\versal{IDEM}, pp. 33-34).
\end{quote}

Mas, então, seria possível perguntar: por que a escatologia
dostoievskiana não poderia espacializar as personagens como encarnações
de temporalidades diversas? Por que a decadência de um Raskólnikov não
poderia se confrontar com Sônia Marmieládova como o \emph{hic et nunc}
de seu devir? E~reciprocamente: por que Sônia Marmieládova não poderia
decair de sua condição moralmente elevada para chafurdar no lodo
relativista de Raskólnikov? As~premissas de Bakhtin não podem responder
a essas perguntas, porque o princípio de \emph{con"-córdia} e
\emph{coexistência} inviabiliza o devir e a superação (\emph{Aufhebung})
ao longo do tempo como se a transformação dos homens e mulheres fosse
uma síntese coercitiva e monológica do autor -- e, no limite, de Deus.
Isso é algo que Paul de Man expressa quando, em diálogo com Bakhtin, o
autor distancia a dialogia da dialética:

\begin{quote}
Longe de aspirar ao \emph{telos} de uma síntese ou de uma resolução,
como poderia ser o caso nos sistemas dialéticos, a função da dialogia é
pensar através da radical exterioridade ou heterogeneidade de uma voz em
relação ao outro, incluindo"-se aí o próprio escritor. Ela ou ele não se
encontra, nesse sentido, em uma situação privilegiada em relação às suas
personagens (1989, p. 109).
\end{quote}

A noção de teleologia moral -- e, ainda uma vez no limite, histórica --
pareceria tanto a Bakhtin quanto a Paul de Man um estancamento do
diálogo, e a totalidade \emph{integral} do crítico russo, então,
coincidiria com a Igreja abarrotada de justos e, sobretudo, de infiéis a
conviverem pela dissonância do diálogo. Novamente, Deus seria a entidade
que, com suprema tolerância, impediria o esgarçamento do todo tenso e
fragmentado. Vemos, então, que a imagem bakhtiniana de Dostoiévski
configura uma síntese original, é preciso admitir, a envolver céu e
inferno. Não haveria o paraíso dos privilegiados cindido do terror de
todos os condenados. Os~planos se interpenetrariam para transformar os
justos em caridosos e os condenados em sujeitos da compaixão. Nossa
investigação terminaria aqui se não arremessássemos Dostoiévski contra
Bakhtin para que o autor prolongasse o devir da escatologia.

Eis, então, algumas questões importantes que derivam das aporias de
Bakhtin relacionadas à perspectiva crítica sobre a totalidade poética de
Dostoiévski: o sentido do todo está relacionado apenas à proposição do
diálogo? E~se os decaídos exterminarem todos os justos? E~se os justos
deixarem de sê"-lo? Não haverá um sentido que os faça evoluir? A
compaixão deverá respeitar a tudo e a todos? Por que o Deus da
tolerância não admite que os seres humanos possamos nos aproximar do
Criador para além de nossa condição atual?

Todas essas questões mostram o limite da ontologia dialógica que
proscreve o tempo, o desenvolvimento e a síntese como instâncias
fundamentais para a cicatrização das personagens de Dostoiévski --
aquilo que, mais adiante neste livro, chamaremos de cicatrização do
espírito. ``Dostoiévski sempre retrata o homem no \emph{limiar} da
última decisão, no momento de \emph{crise} e reviravolta completa -- e
não"-predeterminada -- de sua alma'' (\versal{BAKHTIN}, 2008, p. 69). Ora, mas que
reviravolta seria essa se, para o crítico russo, não há princípio de
formação, transformação e superação em Dostoiévski?

Então, nós podemos formular ainda mais questões para nos prepararmos
para a passagem das aporias de Bakhtin para aquilo que compreendemos
como o movimento dialético em Dostoiévski: as crises das personagens
dostoievskianas poderiam ser os pontos de virada para seus
desenvolvimentos? A~noção que Bakhtin descreve como as vozes
espacializadas das personagens em simultaneidade poderia representar o
devir? O~homem do subsolo conseguiria ver através de Liza o que ele
poderia viver no futuro? Sônia poderia apresentar a redenção para
Raskólnikov? O~outro dialógico, então, apresentaria a crise como um
devir para além de si mesmo -- o futuro, a cicatrização da crise, seria
um tempo simultâneo, um tempo enraizado no espaço. Mas, então,
conseguiríamos apreender mais aporias nos argumentos de Bakhtin: se as
personagens não devem evoluir, se o sentido apenas diz respeito à
aceitação do outro em sua condição presente, o que aconteceria se o
outro, como muitas personagens homicidas de Dostoiévski o fazem, não
pudesse conviver com a minha diferença? Seria então preciso extirpá"-lo,
eliminá"-lo, uma vez que a tolerância não conseguiria lidar com tal fato?

A cosmovisão utópico"-religiosa que Bakhtin projeta para Dostoiévski não
investiga a eternidade e não pode descobrir a importância fundamental
das crises para a evolução humana, pois o postulado de que as vozes são
equipolentes e, sobretudo, \emph{imiscíveis} faz com que a
\emph{con"-córdia} bloqueie o aprendizado recíproco com o respeito pela
condição da personagem tal como ela se apresenta. Chegamos, então, a uma
contradição intrínseca à noção de vozes \emph{imiscíveis:} a dialogia se
vê limitada por um diálogo imiscível que não pode falar dentro do outro
para transformá"-lo, mas apenas para espreitá"-lo. Se assim fosse, por que
Liza se apiedaria tanto pelo homem do subsolo senão para retirá"-lo de
sua condição ressentida e ajudá"-lo a viver para além de sua dor? Por que
Sônia se entregaria à missão de acompanhar Raskólnikov rumo à Sibéria
senão pelo ímpeto de fazer com que o duplo homicida expiasse sua falta e
se tornasse um novo homem?

O caráter estático e espacializado da ontologia dialógica faz Bakhtin
entrar em contradição consigo mesmo ao apreender que

\begin{quote}
a personagem de Dostoiévski em nenhum momento coincide consigo mesma.
(\ldots) A~ela não se pode aplicar a forma de identidade: A~é idêntico a A.
(\ldots) A~vida autêntica do indivíduo só é acessível a um enfoque
\emph{dialógico}, diante do qual ele responde \emph{por si mesmo} e se
revela livremente (\versal{IDEM}, p. 57; p. 67).
\end{quote}

Não é possível entender que a personagem dostoievskiana não coincida
consigo mesma, a não ser que ela possa se transformar ao longo do tempo.
Ao usarmos as noções de Bakhtin, poderíamos pensar que a duplicação da
personagem permite que o eu se veja como eu"-outro -- mas, justamente, se
esse outro é imiscível, o hífen identitário que liga o eu ao outro é
tênue e, no limite, pode ser subsumido para que as personagens, contra a
crítica de Bakhtin ao individualismo, voltem a se transformar em
mônadas. Não à toa, o tom de contradição se instaura quando Bakhtin
afirma que o indivíduo responde ``\emph{por si mesmo} e se revela
livremente''. Ora, a dialogia questiona precisamente esse \emph{si
mesmo}; e mais: como alguém chafurdado na dor e no remorso pode se
revelar livremente sem a ajuda de um outro que se apiede pelo sofrimento
alheio e se proponha a colaborar para a sua transfiguração? A~dialogia
não está em condições de explicar que a duplicação das personagens
dostoievskianas\footnote{Duplicação que, no limite, pode se dar em
  relação à própria identidade da personagem, que se cinde em duas ou
  mais vozes como a esquizofrenia literariamente configurada -- falamos,
  por exemplo, de Goliádkin, protagonista de \emph{O Duplo}. Tradução de
  Paulo Bezerra. São Paulo: Editora 34, 2013.} pode ser a fusão espacial
do tempo como uma imagem especular do devir: a cicatrização do homem do
subsolo e de Raskólnikov poderá entrever em Liza e em Sônia um sentido
de superação da recalcitrância no ressentimento e no mal. Para tanto,
seria necessário apreender a dialogia como um momento da dialética, de
modo que a uma voz"-tese se contrapusesse uma antítese que a fizesse
refletir e viver de outra forma. Não se trataria simplesmente de uma
síntese do autor, uma vez que a própria voz de Dostoiévski se imiscuiria
no processo, como se o escritor também fosse se transformando em meio à
e ao longo da imbricação com a lógica do material artístico que vai
compondo e que também o recompõe. Neste momento, se Bakhtin acompanhasse
a lógica das obras de Dostoiévski para apreender o caráter relacional do
sujeito e do objeto artísticos imbricados ao processo criativo, a
instância do autor não coincidiria com o Dostoiévski empírico. Essa nova
aporia faz a Teoria Crítica observar, ainda uma vez, as insuficiências
da dialogia bakhtiniana. Então, o movimento imanente do material poético
nos coloca diante do ``fato de que o sujeito artístico pratica métodos
cujos resultados concretos não pode prever'' (\versal{ADORNO}, 2012, p. 45). A
liberdade das personagens e do autor seria determinada pelo processo de
constituição do todo, por suas idas e vindas, pelo devir recalcitrante e
contraditório. Já não falaríamos apenas de coexistência, mas de
superação (\emph{Aufhebung}); já não falaríamos apenas de uma
\emph{con"-córdia} tolerante e estática, mas de uma \emph{con"-córdia}
(tese) com vistas a uma discórdia antitética que levaria as personagens
a novos patamares de consciência -- a novas possibilidades de
\emph{con"-córdia.}

Feitas essas considerações, estamos agora em condições de entender a
crítica de Bakhtin à dialética (idealista) e dela extrair as próprias
limitações da ontologia dialógica com vistas à sua superação pela
dialética adorniana que a transformará em um momento determinado da
constituição da poética de Dostoiévski.

\section{1.4. A polifonia como um momento da dialética negativa}

\begin{quote}
Como artista, Dostoiévski adivinhava frequentemente como uma determinada
ideia iria desenvolver"-se e atuar em condições modificadas, que direções
inesperadas tomaria seu sucessivo desenvolvimento e sua transformação.
Para tanto, colocava a idéia no limite das consciências dialogicamente
cruzadas (\versal{BAKHTIN}, 2008, p. 102.
\end{quote}

Já mencionamos como entendemos a noção de consciências dialogicamente
cruzadas: personagens duplicadas como os pares homem do subsolo/Liza e
Raskólnikov/Sônia deparam com o devir do tempo transfigurado em espaço
como uma imagem especular de seu possível futuro: o ressentimento do
homem do subsolo e os sofrimentos de Raskólnikov deparam com possíveis
cicatrizações em meio às relações com Liza e Sônia como vozes
redentoras, como se elas fossem seu devir através do tempo que é
apresentado como um \emph{hic et nunc}, como um espaço radical. O
cruzamento dialógico das consciências cinde a noção de imiscibilidade
das vozes e insere a contradição no seio da teoria de Bakhtin:

\begin{quote}
A dialética e a antinomia existem de fato no mundo de Dostoiévski. Às
vezes, o pensamento dos seus heróis é realmente dialético ou antinômico.
Mas todos os vínculos \emph{lógicos} permanecem nos limites de
consciências isoladas e não orientam as interrelações de acontecimentos
entre elas (2008, pp.8-9).
\end{quote}

Todos os vínculos lógicos permanecem nos limites de consciências
isoladas? Ora, quando o sujeito cartesiano enuncia o \emph{penso, logo
existo}, é preciso que a lógica seja de conhecimento comum, isto é, é
preciso que a lógica seja socializada para que a existência seja
inteligida\footnote{Uma obra paradigmática nesse sentido é o filme
  alemão \emph{O enigma de Kaspar Hauser} (1974), direção de Werner
  Herzog. Kaspar é o filho bastardo de um nobre que, para evitar o
  escândalo cortesão, enclausura o filho em uma torre e priva"-lhe da
  possibilidade de socialização. O~pai apenas visita o filho para lhe
  dar comida. Os~anos vão passando, e o pai de Kaspar, um belo dia, tem
  a ideia de ensinar o filho a escrever. Como todas as mediações
  lógico"-sociais faltam às vivências de Kaspar, o rapaz mal sabe o que é
  postar"-se de pé e articular palavras -- que dirá, então, idealizar
  conceitos para exprimi"-los concretamente sobre o papel.
  Simbolicamente, o pai ordena que o filho escreva o verbo
  \emph{schreiben} (\emph{escrever})\emph{.} Como a metalinguagem
  pressupõe a vivência social da lógica linguística anterior à palavra
  escrita e para além do âmbito propriamente gráfico, Kaspar mal
  consegue deslizar o lápis sobre o papel e, como um papagaio, só faz
  remedar o verbo"-ação que o pai tenta inculcar no filho:
  ``sch"-rei"-ben''. Como seria possível, então, afirmar que os vínculos
  lógicos permanecem nos limites de consciências isoladas? A~própria
  noção de isolamento inexiste para a consciência socializada -- a não
  ser que, como Kaspar Hauser, o isolamento signifique a completa
  não"-socialização.}. Se os vínculos lógicos permanecessem nos limites
de consciências isoladas, não poderia haver diálogo, tudo se reduziria
ao mais ególatra solipsismo. Não poderia, inclusive, haver diálogos
velados, pois a possibilidade de o outro tornar"-se uma projeção do homem
do subsolo reside no fato de que nossa socialização permite que uma
série de ações e reações sejam previstas com base em olhares, gestos
faciais e corporais e prenúncios linguísticos. Os~vínculos lógicos
apenas tendem a permanecer nos limites de consciências isoladas em meio
à loucura. Nesse polo oposto à não"-socialização de Kaspar Hauser, o
mundo torna"-se efetiva representação solipsista, de modo que o tempo e o
espaço lógicos assumem características esgarçadas em relação à sociedade
posta em comum, isto é, comunicada. No mais, os vínculos lógicos
(des)orientam profundamente as interrelações das personagens, conforme
vimos demonstrando com os pares dialógico"-dialéticos homem do
subsolo/Liza e Raskólnikov/Sônia. Assim, é preciso entender \emph{pari
passu} a recusa bakhtiniana à dialética para transformá"-la em negação
determinada e impulsionar as contradições do todo polifônico como
fluidos para a dinâmica poética de Dostoiévski. Vejamos, por exemplo, o
que representaria a sucessão dialético"-temporal para Bakhtin:

\begin{quote}
Se as ideias em cada romance -- os planos do romance são determinados
pelas idéias que lhe servem de base -- fossem realmente distribuídas
como elos de uma série dialética una, cada romance seria um todo
filosófico acabado, construído segundo o método dialético. No melhor dos
casos teríamos diante de nós um romance filosófico, um romance de ideias
(ainda que dialético); no pior, uma filosofia em forma de romance.
\emph{O último elo da série dialética seria fatalmente uma síntese do
autor, que eliminaria os elos anteriores como abstratos e totalmente
superados}. (\ldots) Em nenhum romance de Dostoiévski há formação dialética
de um espírito uno, \emph{geralmente não há formação, não há
crescimento} (\ldots). Nos limites do próprio romance não se desenvolve,
não se forma tampouco o espírito do autor; este (\ldots) contempla ou se
torna um dos participantes. Nos limites do romance, os universos das
personagens estabelecem entre si interrelações de acontecimentos, embora
estas, como já dissemos, sejam as que menos se podem reduzir às relações
de tese, antítese e síntese {[}grifos meus{]} (\versal{IDEM}, p. 28; p. 29).
\end{quote}

Primeiramente, é preciso enfatizar que a síntese em termos dialéticos
jamais elimina os elos anteriores como momentos ``abstratos e totalmente
superados''. O~conceito de \emph{Aufhebung} pressupõe a negação, a
elevação (afirmação) e a superação como momentos recíprocos e
determinados para a constituição de uma síntese que, por sua vez, será
um novo elo do devir dialético em novo patamar. Pensemos, por exemplo,
na trajetória de Raskólnikov. A~princípio, sua negação se constituía
como a afirmação unívoca do próprio eu como se a personagem fosse uma
mônada e não vivesse em sociedade. Isso o levou ao duplo homicídio
(negação da existência do outro) que, por sua vez, impulsionou uma
profunda crise que fez com que a personalidade de Raskólnikov se
cindisse -- \emph{raskol}, cisão -- para que o protagonista pudesse ver
a escatologia de si mesmo em um turbilhão de tempo. Poderia Raskólnikov
superar o fardo futuro do assassínio? Continuaria o duplo homicida a
negar o outro? Então Dostoiévski traz uma nova voz -- um novo vínculo
lógico"-póetico -- para insuflar em Raskólnikov o ímpeto da negação da
negação. Sônia Marmieládova é a antítese do assassínio; ela encarna a
negação de si mesma, a abnegação, para que o outro por quem ela se
compadece possa afirmar não mais a si mesmo contra os demais e para que
Raskólnikov possa afirmar a compaixão. Assim, a dialogia socialmente
constitutiva ao protagonista de \emph{Crime e castigo} tem o potencial
de alçar o outrora criminoso como ser social para além do utilitarismo.
A dialética redentora de Raskólnikov e Sônia processa em ambos um novo
patamar de consciência: em geral, observa"-se apenas a mudança
qualitativa de Raskólnikov, mas Sônia -- e Dostoiévski -- foi (foram) se
descobrindo ao longo da abnegação, em estreita correlação lógico"-poética
com os desdobramentos de seu par dialógico que, agora, já fala dentro
dela (deles) mesma (mesmos). Assim, como poderíamos falar em supressão
dialética dos polos tese e antítese em função da síntese que só se
configura como total imbricação entre presente, passado e futuro?
Raskólnikov não deixará de ser um duplo homicida perante as vítimas
Alióna Ivánovna e sua irmã, mas sua consciência relacionalmente redimida
diante de Sônia o levará a um novo patamar de arrependimento e vivência
-- o castigo como purificação diante do crime\footnote{Ao desdobrarem o
  conceito de teologia teleológica em Dostoiévski, os capítulos 3, 4 e 5
  deste livro argumentarão que Raskólnikov poderá se arrepender de seu
  passado não apenas com vistas ao futuro, mas também como possibilidade
  de reencontro com as vítimas Alióna Ivánovna e sua irmã. A
  possibilidade do perdão desdobrará o tempo como imagem movente da
  eternidade e tentará sondar a (teleo)lógica do além"-mundo
  dostoievskianamente configurado.}.

Em \emph{Problemas da poética de Dostoiévski}, Bakhtin se distancia da
dialética idealista -- a dialética de Marx, reificada pelo Partido,
apenas poderia ser revista pelo revisionismo dos líderes soviéticos. Tal
impossibilidade de crítica à crítica, isto é, de crítica à dialética,
reforçaria a tese de que a obra de Bakhtin sobre Dostoiévski poderia ser
lida como um diatribe \emph{velada} aos comandantes soviéticos -- no
caso, ao Guia Genial dos Povos, o comandante"-mor que adotara o nome de
homem de aço (Stálin) e que, efetivamente, transformava a síntese em
supressão absoluta das teses e antíteses que lhe eram contrárias. Mas, a
meu ver, é importante arremessar tal percepção estético"-política ao
encontro dos e de encontro aos problemas efetivos da crítica bakhtiniana
à dialética. O~crítico segue afirmando que, ``no terreno do idealismo
monástico não pode desabrochar a multiplicidade de consciências
imiscíveis'' (\versal{IDEM}, p. 29). Então, é possível dizer que

\begin{quote}
essas contradições e esses desdobramentos {[}das personagens{]} não se
tornaram dialéticos, não foram postos em movimento numa via temporal,
numa série em formação, mas se desenvolveram em um plano como contíguos
e contrários, consonantes mas imiscíveis, ou como irremediavelmente
contraditórios, como harmonia eterna de vozes imiscíveis ou como
discussão interminável e insolúvel entre elas (\versal{IDEM}, p. 34).
\end{quote}

Primeiramente, Bakhtin entrevê no horizonte de constituição do todo
polifônico a questão da totalidade. Na sequência da argumentação, no
entanto, o crítico se aferra à micrologia das vozes (i) como se elas
fossem imiscíveis, (ii) como se a multiplicidade não fosse passível de
articulação e (iii) como se o caráter insolúvel dos conflitos não
pudesse ser movimentado pela própria contradição -- a dinâmica da
resolução irresoluta. Neste momento, se arremessarmos Adorno contra
Bakhtin, perceberemos que a tentativa de construção do todo polifônico
que veda a dialética não está em condição de apreender que ``a
articulação é a salvação da multiplicidade no uno'' e que ``a
totalidade, contextura sem falhas das obras de arte, não é uma categoria
de fechamento'' (\versal{ADORNO}, 2012, p. 289; p. 285). Ao tentar estruturar"-se
como uma catedral sistêmica que acolhe as contradições como se elas lá
pudessem permanecer estanques, a integralidade impede a reconfiguração
histórica da forma dostoievskiana. A~lógica poética requer o movimento
da contradição, isto é, a dialética, e Adorno pode superar as aporias
bakhtinianas justamente no momento em que elas se tornam estáticas. Se
``a arte tem o seu conceito na constelação de momentos que se
transformam historicamente'' e se ``a arte só é interpretável pela lei
do seu movimento, não por invariantes -- a sua lei de movimento
constitui a sua própria lei formal'' (\versal{IDEM}, p.13; p. 14) --, como é que
a paralisia dialógica de Bakhtin como eternidade insustentável da
tolerância pode acompanhar a importante constatação de que ``Dostoiévski
ainda não se tornou Dostoiévski, está apenas se tornando o próprio?''
(\versal{BAKHTIN}, 2008, p. 327).

Se Bakhtin não houvesse sido paralisado pelas contradições de
Dostoiévski; se Bakhtin houvesse acompanhado o movimento das
contradições dostoievskianas, de modo a questionar as categorias de
sistema, integralidade e sistema integral como conceitos à revelia da
concretude poética, o crítico teria entrevisto a totalidade aberta e
contraditória como o reverso da catedral a"-histórica. Em diálogo com
Adorno, a forma de Dostoiévski procuraria ``fazer falar o pormenor
através do todo'' (\versal{ADORNO}, 2012, p. 221) e, reciprocamente, a forma
procuraria fazer falar o todo a ser desvelado através das imbricações
dos pormenores dialógicos.

Se o processo imanente às obras de arte, algo que ultrapassa o sentido
de todos os momentos singulares, constituiu o enigma, então
simultaneamente ele atenua"-o logo que a obra de arte não é percebida
como alguma coisa de fixo e, por conseguinte, em vão interpretada, mas é
recriada na sua própria constituição objetiva (\versal{IDEM}, p. 194).

Se a totalidade polifônica não é a negação do todo como dialogia
tolerante que se encontra em uma catedral subitamente erigida -- sem que
Bakhtin possa apresentar os momentos de mediação que articulam as partes
ao todo; se Dostoiévski ``ainda não se tornou Dostoiévski'', já que os
conflitos dialógicos não são simplesmente insolúveis, mas dinamicamente
insolúveis, podemos dizer, de forma rente à lógica de Adorno, que o
caráter enigmático das obras não é o seu ponto último, mas que toda obra
autêntica tende a propor igualmente a solução do seu enigma insolúvel.
Assim, passamos a conceber uma totalidade dinâmica -- uma totalidade
aberta e processual, uma totalidade em devir.

\begin{quote}
Se a obra de arte não é em si algo de estável, de definitivo, mas algo
em movimento, a sua temporalidade imanente comunica"-se então às partes e
ao todo ao desdobrar"-se no tempo a sua relação e ao serem capazes de
denunciar essa relação. Se as obras de arte vivem na história em virtude
do seu próprio caráter processual, então podem nela dissipar"-se (\versal{IDEM},
p. 271).
\end{quote}

As obras de arte podem dissipar"-se na história e, dialeticamente, podem
encontrar"-se consigo mesmas pela maturação histórica de sua forma que
passa a colidir com o conteúdo das vozes de modo a iluminar o movimento
de suas contradições. Novamente nos inspirando na lógica estética
adorniana, podemos dizer que a dificuldade em isolar a forma é
condicionada pelo entrelaçamento de toda a forma estética com o
conteúdo; assim, a forma deve ser concebida não só contra o conteúdo,
mas através dele. Ocorre que Bakhtin, ao sintetizar abruptamente o todo
dostoievskiano sem \emph{superar} a contradição de suas partes
constitutivas, deriva precisamente de sua catedral polifônica o caráter
de profunda crítica social presente na poética dostoievskiana.

\begin{quote}
A ênfase principal de toda a obra de Dostoiévski, quer no aspecto da
forma, quer no aspecto do \emph{conteúdo}, é uma luta contra a
\emph{coisificação} do homem, das relações e de todos os valores humanos
no capitalismo. (\ldots) Dostoiévski conseguiu perceber a penetração da
\emph{desvalorização coisificante} do homem em todos os poros da vida de
sua época e nos próprios fundamentos do espírito humano. (\ldots) {[}Eis{]}
\emph{o sentido de sua forma artística}, o qual liberta e descoisifica o
homem.
\end{quote}

(\ldots) As~personagens de Dostoiévski são movidas por um sonho utópico de
fundação de alguma comunidade de seres humanos fora das formas sociais
existentes. Fundar uma comunidade na terra, unificar algumas pessoas
fora do âmbito das formas sociais vigentes -- a isso aspiram o Príncipe
Míchkin, Aliócha {[}Karamázov{]}, aspiram em formas menos conscientes e
menos nítidas todas as demais personagens de Dostoiévski (2008, p. 71;
p. 317).

Bakhtin incorre no mesmo problema ideológico que o crítico atribui ao
idealismo filosófico: o pensamento e a arte como hipostasias para a
sociedade reconciliada. Ainda uma vez, é possível apreender as
insuficiências da polifonia bakhtiniana à luz das aporias que a
\emph{Teoria do romance} desvela para projetar o problema poético em
meio a um mundo crescentemente fragmentado:

\begin{quote}
Lukács, na \emph{Teoria do romance}, rejeita a alternativa de se apelar
ao mundo da forma -- a estética -- como um modelo de coerência que
poderia ser projetado sobre o mundo fragmentário. Tal estratégia
conferiria à arte uma prioridade que ela não pode arrogar para si mesma,
já que a arte é uma entre as várias esferas sociais a competirem entre
si. Tal prioridade estabeleceria uma imposição da racionalidade estética
sobre domínios heterogêneos, tais como a política, a economia, a
religião etc. (\versal{HOLQUIST}, 1986, p. 134)\footnote{A colocação do eslavista
  norte"-americano Michael Holquist sobre o caráter de falsa
  reconciliação do todo através da particularidade da esfera artística
  dialoga com Adorno e Walter Benjamin. Na \emph{Teoria estética},
  Adorno apreende o caráter objetivo da fragmentação artística, quando
  ele diz que ``a liberdade absoluta na arte, que é sempre a liberdade
  num domínio particular, entra em contradição com o estado perene de
  não"-liberdade do todo'' (2012, p. 11). De forma mais rente ao
  desespero subjetivo do leitor -- desespero mediado e fomentado pela
  crise objetiva --, Benjamin bem sabe que ``o romance não pode
  alimentar a esperança de dar o mínimo passo além daquele limite em
  que, convidando o leitor a captar intuitivamente o sentido da vida,
  convida"-o também a escrever um `Finis' embaixo da última página. (\ldots)
  O~que arrasta o leitor para o romance é a esperança de aquecer sua
  vida enregelada numa morte que ele vivencia através da leitura''
  (1980, p. 68; p. 69).}.
\end{quote}

Nesse sentido, enquanto o particular e o universal divergirem, as obras
de arte ``não devem suprimir a universalidade dominante do mundo
administrado através da sua particularização. (\ldots) O~que nelas aparece
já não é ideal ou harmonia; o seu caráter de resolução só tem lugar no
contraditório e no dissonante'' (\versal{ADORNO}, 2012, p. 72; p. 70; p. 133). A
arte, então, tem o potencial de exprimir poderosas tensões armazenadas,
de trazê"-las à tona, de ressignificá"-las, de dar"-lhes um sentido que
aumente o poder de apreensão do real pela Teoria Crítica. Assim, ``pela
recusa intransigente da aparência de reconciliação, a arte mantém a
utopia no seio do irreconciliado'', e ``os antagonismos não resolvidos
da realidade retornam às obras de arte como os problemas imanentes de
sua forma'' (\versal{IDEM}, p. 58; p. 18). A~dialética a percorrer o negativo da
poética que se constitui e se transforma historicamente apreende os
elementos que Bakhtin considera socialmente críticos em Dostoiévski como
apologia e revela o caráter parcial e datado da tentativa de hipostasiar
o todo a partir das partes fragmentárias sem compreender a obra do
escritor de modo a submeter"-se à sua disciplina imanente. Assim, sem
caminhar pelas mediações poéticas devidas, Bakhtin não apenas deriva a
totalidade de modo imediato como entrevê suas raízes, também
imediatamente, no contexto histórico, social e politicamente polifônico
em que Dostoiévski viveu:

\begin{quote}
A própria época tornou possível o romance polifônico. Dostoiévski foi
\emph{subjetivamente} um partícipe dessa contraditória multiplicidade de
planos do seu tempo, mudou de estância, passou de uma a outra e neste
sentido os planos que existiam na vida social objetiva eram para ele
etapas de sua trajetória vital e sua formação espiritual (2008, p. 30).
\end{quote}

No capítulo ``O prenúncio da teologia como teleologia -- e da teleologia
como teologia'', discutiremos de modo mais pormenorizado a mudança de
estância de Dostoiévski do plano radicalmente esquerdista -- o escritor
fez parte do chamado Círculo de Petrachévski, um grupo de
revolucionários que reivindicava a abolição da servidão, base de
reprodução econômica do tsarismo, e, no limite, o estabelecimento de uma
república socialista na Rússia -- para o plano reacionário de defesa da
monarquia contra o niilismo daqueles que propunham fazer tábula rasa da
história para fundar uma nova sociedade. Veremos que a virada
fundamental na cosmovisão dostoievskiana se deu, fundamentalmente,
durante o período em que o escritor esteve em um presídio na Sibéria --
a \emph{casa dos mortos} -- condenado a trabalhos forçados por conta de
sua participação no Círculo de Petrachévski. Na casa dos mortos\emph{,}
Dostoiévski esteve em contato com os oprimidos que a
\emph{intelligentsia} revolucionária queria redimir e lá ele descobriu
com profundidade as aporias, clivagens e contradições entre a revolução
livresca e a tradição histórica do povo. Tais tensões são realmente
importantes para a compreensão da obra de Dostoiévski, mas Bakhtin não
poderia vincular imediatamente a experiência do autor à constituição de
sua obra sem atravessar os momentos imanentes, sem analisar a mimese
interna. Conforme a \emph{Teoria estética} de Adorno enfatiza, ``a
racionalidade é, na obra de arte, o momento criador de unidade e
organizador, não sem relação com a que impera no exterior, mas não copia
a sua ordem categorial'' (2012, p. 91). Para que Bakhtin pudesse
apreender os momentos (negativos) de crítica dostoievskiana ao
capitalismo e à reificação, seria fundamental entrever a forma dialógica
(e dialética) como conteúdo sedimentado, como leitura imanente da
história, de modo que ``o caráter fetichista da mercadoria'' fosse
criticado ``na própria coisa, não a partir de fora e unicamente porque
se transforma em fetiche'' (\versal{IDEM}, p. 43). Tal hipostasia bakhtiniana que
transforma a realidade russa em escatologia dostoievskiana -- e,
reciprocamente, o tenso século \versal{XIX} já revolucionário e
pré"-revolucionário em um romance dostoievskiano -- não consegue
demonstrar, justamente, como as personagens desenvolvem trajetórias
vitais e se (trans)formam espiritualmente.

Agora é possível ver que Bakhtin incorre em uma nova contradição que a
polifonia que busca a integralidade sistêmica não pode desdobrar: as
categorias de coexistência e interação não eram essenciais para a
poética de Dostoiévski? Por que, então, o crítico volta a lançar mão da
categoria de formação? Como Bakhtin poderia apreender o devir em
Dostoiévski se as relações dialógicas estancam o todo como uma catedral
tolerante? Adorno diria que, afinal, em Dostoiévski, ``realmente existe
psicologia {[}e sociologia{]}, ela é uma psicologia do caráter
inteligível, da essência, e não do ser empírico, dos homens como ele
circulam por aí. É~exatamente nisso que Dostoievski é avançado'' (1980d,
p. 270). A~citação adorniana (e antibakhtiniana) desponta como um
hieróglifo profundamente sintético a ser desdobrado pela análise do
próximo capítulo. Adorno fala em essência, e logo a entendemos como a
configuração histórico"-social -- e, no caso de Dostoiévski, escatológica
-- da psicologia do caráter inteligível, da desagregação das relações,
de seu devir limítrofe. A~não"-coincidência das personagens de
Dostoiévski com os seres empíricos arremessa Adorno contra Bakhtin, mas
não deixa de enraizar o escritor na história como se suas obras nada
dissessem para os homens em transformação. O~devir desdobra a essência
de um caráter monástico e interior para, agora por meio de Bakhtin,
descobrir a voz interior como alteridade e altercação, como
inconclusibilidade histórica dos homens não apenas por conta de seu
caráter inequivocamente dialógico, mas porque Dostoiévski entrevê que a
empostação da voz no mundo e a convivência social dos homens constituem
não apenas a sua possibilidade de formação, mas também a reiteração da
deformação que os reifica e exacerba suas angústias e desencontros.
Assim, também é preciso arremessar Adorno e Horkheimer contra si mesmos,
agora por intermédio de Bakhtin, para que Dostoiévski continue a se
tornar Dostoiévski, para que as obras falem mais sobre si mesmas não
apenas pelo caráter dialógico e positivo, mas pela negação da síntese
polifônica em meio ao todo fraturado. Em uma passagem sobre o escritor
russo em meio à \emph{Dialética do Esclarecimento}, Adorno e Horkheimer
dizem que

\begin{quote}
contrariamente ao que imaginavam Dostoiévski e os apóstolos alemães da
interioridade, a consciência moral consistia para o ego em devotar"-se ao
substancial no mundo exterior, na capacidade de fazer seu o verdadeiro
interesse dos outros (1985, pp. 184-185).
\end{quote}

Enquanto Adorno e Horkheimer, em um sobrevoo algo distanciado da
constituição propriamente poética de Dostoiévski (\emph{distant
reading}) o aproximam do idealismo alemão, o \emph{close reading} de
Bakhtin só faz distanciar o escritor dos princípios do monologismo
ideológico, que teriam encontrado

\begin{quote}
na filosofia idealista a expressão mais nítida e teoricamente precisa. O
princípio monístico, isto é, a afirmação da unidade do \emph{ser},
transforma"-se, na filosofia idealista, em princípio da unidade da
\emph{consciência.} (\ldots) Em essência, o idealismo conhece apenas uma
modalidade de interação cognitiva entre as consciências, ou seja, o
sujeito que é cognoscente e domina a verdade, ensinando"-a ao que não é
cognoscente e comete erros, vale dizer, conhece a interrelação entre o
mestre e o discípulo e, consequentemente, apenas o diálogo pedagógico
(2008, p. 89; p. 91).
\end{quote}

Já sabemos que, para Bakhtin, a consciência é a diversidade do ser, a
interação dialógica com o outro -- ainda que a noção de imiscibilidade
das vozes, conforme já averiguamos, não leve às últimas consequências o
próprio princípio de movimento com que Bakhtin se imbricou ao analisar o
material dostoievskiano. Como o crítico russo apenas pensa em termos
coextensivos; como Bakhtin veda a coexistência das personagens como um
momento da formação, do tempo e da formação temporal que move o espaço
coextensivo como um devir, Bakhtin não acompanha a dialética encarniçada
que contrapõe o senhor e o escravo, dialética que transforma a verdade
em algo relacional e fluido a imiscuir o mestre e o discípulo, de modo
que as posições não sejam estanques, mas se movimentem em meio ao
diálogo que se confunde com o duelo -- a alteridade como uma possível e
recorrente altercação. Senhor e escravo são contingências, sobretudo no
capitalismo que Bakhtin apreende como a base da poética dostoievskiana,
capitalismo que desconstrói o privilégio aristocrático e passa, então, a
arremessar o escravo contra o senhor, de modo que o senhor também começa
a temer a escravidão. Ora, que mais fazem as personagens dostoievskianas
senão duelar de maneira intestina de modo a trazer à luz o caráter
objetivo das aporias?

O penúltimo capítulo deste livro, ``O Sermão da Estepe: Ivan Karamázov e
a filosofia dostoievskiana da história'', se desdobra, justamente, sobre
as antíteses espiritualmente ateias de Ivan -- a contradição será
movimentada a contento -- e as teses ortodoxas do monge Aliócha
Karamázov, irmão de Ivan. Dostoiévski não apenas apreende a simbologia
que \emph{irmana} o ateísmo como negação determinada da fé ortodoxa e
como \emph{Aufhebung} espiritual -- eis a ideia central do capítulo em
questão --, como arremessa Ivan contra Aliócha e Aliócha contra Ivan
para levá"-los a novos patamares de contradição, para dinamizar a
pedagogia e tensionar as posições inequívocas de mestre e discípulo.
Mas, ainda uma vez, vale dizer que, como Bakhtin prescinde da dialética
para encontrar a polifonia como movimento da totalidade aberta e
contraditória em Dostoiévski, o crítico russo não consegue entender que
a tese do \emph{atual} mestre encontra o discípulo como uma possível
\emph{antítese} a ser desdobrada até que este último traga ao primeiro
uma nova síntese como tese subsequente que tensione e subverta as
posições pedagógicas inicialmente configuradas. Se Bakhtin deixasse de
apreender a poética de Dostoiévski como fotos estáticas da catedral
dialógica e a pusesse em movimento em uma série cinética, isto é,
cinematográfica, a narrativa animaria a contradição em meio ao enredo
que se volta contra si mesmo, e o tempo, congregado pelo espaço, ao
invés de se reproduzir linearmente, buscaria novos patamares
qualitativos de contradição como o devir da dialética, já que ``o
momento do espírito não é, em qualquer obra de arte, um ente; em cada
uma é algo que está em devir, que se constitui'' (\versal{ADORNO}, 2012, p. 145).
Em meio ao movimento dialético, ``cada coisa só é o que ela é
tornando"-se aquilo que ela não é'' (\versal{ADORNO} e \versal{HORKHEIMER}, 1985, p. 29).
Ao invés da catedral acabada como a utopia já posta, como o positivo
tangível, o negativo nos levaria a caminhar entre os escombros do
sofrimento. ``É mais tangível retratar figuras reais de sofredores
atônitos. Por sinal, eu passei a vida toda fazendo isso'' (\versal{DOSTOIÉVSKI}
apud \versal{KELLY}, 1988, p. 254). Tal afirmação de Dostoiévski incorpora pela
mediação sujetiva a objetividade histórica e fraturada e acompanha a
dialética de Adorno e Horkheimer:

\begin{quote}
Os grandes artistas jamais foram aqueles que encarnaram o estilo de
maneira \emph{mais íntegra e mais perfeita} {[}grifo meu{]}, mas aqueles
que acolheram o estilo em sua obra como uma atitude dura contra a
expansão caótica do sofrimento, como verdade negativa. (\ldots) Se o
esclarecimento {[}e a poética{]} não acolhe{[}m{]} dentro de si a
reflexão sobre esse elemento regressivo, ele {[}e ela{]} está
{[}estão{]} selando seu próprio destino (\versal{ADORNO} e \versal{HORKHEIMER}, 1985, p.
122; p. 13).
\end{quote}

Após a configuração da contradição dialética como \emph{conteúdo
temático} fundamental, devemos passar a um último aspecto teórico que,
desta vez, aproxima o materialismo de Bakhtin da Teoria Crítica.

Já vimos que a Teoria Crítica consegue dinamizar o \emph{close reading}
bakhtiniano, de modo a vislumbrar os conceitos que podem desdobrar as
contradições dialógicas rumo ao todo polifônico, contradições que o
crítico russo considera insolúveis. Em um momento posterior, a dialética
imanente à poética de Dostoiévski se transferiu às (contra)posições de
Bakhtin e Adorno. O~sobrevoo da dialética negativa a partir de Adorno e
Horkheimer discerniu Dostoiévski como um apóstolo da interioridade
idealista, ao passo que a dialogia só fazia afastar o escritor russo do
monologismo próprio ao idealismo filosófico. Arremessamos, então, Adorno
contra si mesmo por intermédio de Bakhtin, para que a história,
substrato da poética, pudesse ser apreendida de forma imanente, mediante
a disciplina própria à constituição da obra.

Retomemos, agora, um elemento de constituição da poética de Dostoiévski
em um novo patamar de questionamento: a dialogia não explica como a
liberdade \emph{determinada} -- conceito que lhe é estranho -- das
personagens dostoievskianas pode conviver com categorias como plano
autoral e totalidade que não amordacem as vozes, isto é, os momentos
particulares, as personagens, em função de uma síntese feita pelo
próprio Dostoiévski. Por outro lado, Bakhtin ainda entrevê em seu
horizonte teórico a \emph{necessidade} de desvelar o todo polifônico
\emph{integral --} todo sistêmico, hierárquico e antidialógico, a
negação absoluta de \emph{Problemas da poética de Dostoiévski.} É~como
se Bakhtin somente conseguisse tatear entre totalidades fechadas,
sistêmicas e, portanto, antidialógicas. Assim, insuflemos ar dialético
na ossificação da catedral dialógica para colocar a totalidade
polifônica e aporética em movimento. Consideramos que a poética de
Dostoiévski apresenta a totalidade a partir de suas fissuras e ruínas,
pelo movimento da contradição. Neste momento, mediados por esta
trajetória, podemos acompanhar a confluência entre Bakhtin e a Teoria
Crítica que encontrará a voz autoral de Dostoiévski transformada em
objetividade histórica no ápice de sua subjetividade.

Bakhtin, a princípio, nos diz que a autonomia do herói parece contrariar

\begin{quote}
o fato de ele ser representado inteiramente como um momento da obra de
arte e, consequentemente, ser, do começo ao fim, totalmente criado pelo
autor. {[}Mas,{]} Em realidade, tal contradição não existe. Afirmamos a
liberdade dos heróis nos limites do plano artístico e neste sentido ela
é criada do mesmo modo que a não"-liberdade do herói objetificado (2008,
p. 73).
\end{quote}

Neste momento, ao expor as bases materiais de sua apreensão estética,
Bakhtin transforma liberdade e não"-liberdade em \emph{instâncias
possíveis} em meio aos limites do plano artístico. Ainda uma vez, no
entanto, não sabemos \emph{como} o plano artístico se constitui --
apenas sabemos que, para Bakhtin, tal categoria é fundamental para a
estruturação da obra. O~crítico russo não aventa a possibilidade de o
plano artístico ser \emph{uma categoria aberta e aporética, uma
categoria em devir}. Se tivesse levado às últimas consequências o
diálogo entre arte e história, Bakhtin poderia ter acompanhado o
movimento da contradição que enreda a liberdade determinada do herói à
não"-liberdade do todo, de modo a fraturar a reconciliação polifônica e
entrever as fissuras entre a parte e o todo como algo próprio ao
material dostoievskiano historicamente configurado. Mostraremos, no
próximo capítulo deste livro, \emph{como} tal imbricação contraditória
ocorre, quando analisarmos a forma fetichista de \emph{Memórias do
subsolo.} Por ora, continuemos a acompanhar Bakhtin, de modo a
entendermos como o crítico correlaciona a liberdade determinada do herói
às categorias de plano artístico e criação e como o embasamento
materialista de Bakhtin o aproxima de Adorno:

\begin{quote}
Criar não significa inventar. Toda criação é concatenada tanto por suas
leis próprias quanto pelas leis do material sobre o qual ela trabalha.
Toda criação é determinada por seu objeto e sua estrutura e por isso não
admite o arbítrio e, em essência, nada inventa, mas apenas descobre
aquilo que é dado no próprio objeto. Pode"-se chegar a uma ideia
verdadeira, mas esta tem a sua lógica, daí não poder ser inventada, ou
melhor, produzida do começo ao fim. Do mesmo modo não se inventa uma
imagem artística seja ela qual for, pois ela também tem a sua lógica
artística, as suas leis. Quando nos propomos uma determinada tarefa,
temos de nos submeter às suas leis (\versal{BAKHTIN}, 2008, pp. 73-74)\footnote{``O
  `autor"-artista' não inventa a personagem, ele a pré"-encontra já dada
  independentemente do seu ato puramente artístico, não pode gerar de si
  mesmo a personagem'' (\versal{BAKHTIN}, 2003, pp. 184-185).}.
\end{quote}

\emph{Quando nos propomos uma determinada tarefa, temos de nos submeter
às suas leis:} a negação de Bakhtin em relação à criação subjetiva
\emph{ex nihilo} diz respeito à apreensão do crítico de que a logicidade
do material se impõe ao criador, de modo que a criação se transforme no
\emph{processo criativo que imbrica sujeito e objeto contínua e
reciprocamente.} Ocorre que as leis que enformam o material artístico
não são estáticas, elas vão sendo plasmadas historicamente, de maneira
que as relações entre as partes e o todo -- e, conforme vimos discutindo
ao longo deste capítulo, a própria noção de todo -- se tornam mais
contraditórias e modulam continuamente as vozes das personagens.

Theodor Adorno reverbera a logicidade material de Bakhtin ao afirmar que

\begin{quote}
a parte da obra que ``pertence'' ao compositor, como a qualquer artista
produtivo, é incomparavelmente menor do que supõe a opinião vulgar,
orientada ainda pela noção do gênio. Quanto mais alta uma estrutura
musical, tanto mais o compositor se relacionará com ela como o seu
simples órgão de realização, como alguém que obedece à exigência do
objeto. (\ldots) ``O compositor propõe a regra a si mesmo, para depois lhe
obedecer'' (Hans Sachs) (1980a, p. 264).
\end{quote}

O caráter necessário da crítica imanente faz Bakhtin e Adorno
apreenderem a completa imbricação de sujeito e objeto para a
constituição da obra de arte. Já não estamos falando de entes distintos,
mas de uma relação em que a autoria, ``a expressão das obras de arte, é
o não"-subjetivo no sujeito'', e o ``êxito subjetivo da obra de arte''
desponta no momento em que ``o sujeito nela desaparece'' (\versal{ADORNO}, 2012,
pp. 175-176; p. 95). O~êxito e as fraturas subjetivas da obra de arte,
nesse momento, apontam para a história, suas completudes e, em relação
ao sujeito"-objeto Dostoiévski, suas aporias. Se Bakhtin acolhesse o
movimento da contradição para superar a paralisia de sua catedral, o
crítico mimetizaria, assim como o autor"-objeto Dostoiévski, a disciplina
da obra para acompanhar suas metamorfoses históricas. Pois, se ``os
materiais e os objetos são histórica e socialmente pré"-formados como os
seus procedimentos técnicos e modificam"-se de um modo decisivo em
virtude do que lhes acontece nas obras'' (\versal{IDEM}, pp. 141-142), a
polifonia dialética estaria em condições de realizar a crítica do
``esclarecimento {[}e da poética{]} que só reconhece{[}m{]} como ser e
acontecer o que se deixa captar pela unidade. Seu ideal é o sistema do
qual se pode deduzir toda e cada coisa'' (\versal{ADORNO} e \versal{HORKHEIMER}, 1985, p.
22). A~autoria subjetiva se objetivaria ao longo da constituição da
obra, de modo que a síntese sistêmica do autor como subsunção da
alteridade das personagens se veria contradita pelo movimento que,
mediante o soerguimento da poética contraditória --
edifício"-para"-a"-ruína, afirmação que pressupõe a negação e a consequente
negação da negação --, transformaria a unidade em implosão de si mesma,
em unidade dos múltiplos escombros. Não apenas a voz do autor
Dostoiévski seria contradita ao lado das demais personagens, mas a
instância autoral, \emph{em meio à constituição da obra e no devir de
suas configurações}, superaria a subjetividade e a contingência do
artista para mediar a ressignificação da história como mimese imanente
da arte. Como artista -- e, sobretudo, como um artista profundamente
vinculado ao movimento da história --, Dostoiévski supera suas posições
pessoais para, em meio à criação, ser recriado não apenas como criador,
mas como sujeito"-objeto histórico, relação por meio da qual o espírito
de sua época é (re)configurado artisticamente. E, segundo as premissas
deste capítulo, se a obra de arte não coincide consigo mesma, se ela
pressupõe a contradição entre seus momentos e movimentos constitutivos,
a inconclusividade da obra de Dostoiévski pode nos falar não apenas
sobre sua época, mas, fundamentalmente, sobre a escatologia
poético"-histórica que a aproxima de nossos tempos.

Se, com Bakhtin e a Teoria Crítica, com a Teoria Crítica contra Bakhtin
e com Bakhtin em contraposição à Teoria Crítica, Dostoiévski ainda está
se tornando Dostoiévski, é hora de nos submetermos à (in)disciplina
contraditória da obra dostoievskiana, em estreito diálogo com o
fetichismo da mercadoria analisado por Karl Marx, para configurarmos a
polifonia como um momento da dialética negativa. Tentaremos, então,
descobrir -- ou melhor, continuaremos a perguntar -- como pode a arte
``transcender a circulação do meramente vigente, à qual, por outro lado,
ela deve a possibilidade da própria existência?'' (\versal{ADORNO}, 1980a, p.
268) Como pode a possibilidade da própria existência transcender a
circulação do meramente vigente, à qual, por outro lado, ela deve a
arte? Como pode a arte transcender a possibilidade da própria
existência, à qual, por outro lado, ela deve a circulação do meramente
vigente? Como pode a possibilidade da própria existência transcender a
arte, à qual, por outro lado, ela deve a circulação do meramente
vigente? O~homem do subsolo e Liza (\emph{Memórias do subsolo}),
Raskólnikov, Sônia e Svidrigáilov (\emph{Crime e castigo}), o Príncipe
Míchkin, Rogójin e Nastácia Filíppovna (\emph{O idiota}), Nikólai
Stavróguin e o clérigo Tíkhon (\emph{Os demônios}), os irmãos Aliócha e
Ivan Karamázov, Cristo e o grande inquisidor (\emph{Os irmãos
Karamázov}) insinuam que, no labirinto dostoievskiano, a saída tende a
se configurar como uma nova entrada.

\chapter*{Capítulo 2\\
\bigskip
\emph{A dialética polifônica de Fiódor Dostoiévski}}

\addcontentsline{toc}{chapter}{Capítulo 2\\\scriptsize{\emph{A dialética polifônica de Fiódor Dostoiévski}}}
\hedramarkboth{Capítulo 2}{}

\section{2.1. No princípio eram a dicção -- e a contradição}

Comecemos a análise, \emph{in media res}, \emph{ouvindo} o homem do
subsolo a discorrer sobre si mesmo (e sobre \emph{nós}) em seu púlpito
narrativo:

Sou um homem doente\ldots Um homem mau. Um homem desagradável. Creio que
sofro do fígado. Aliás, não entendo níquel da minha doença e não sei, ao
certo, do que estou sofrendo. Não me trato e nunca me tratei, embora
respeite a medicina e os médicos. Ademais, sou supersticioso ao extremo;
bem, ao menos o bastante para respeitar a medicina. (Sou suficientemente
instruído para não ter nenhuma superstição, mas sou supersticioso.) Não,
se não quero me tratar, é apenas de raiva. Certamente não compreendeis
isto. Ora, eu compreendo. (\ldots) Se me dói o fígado, que doa ainda mais
(\versal{DOSTOIÉVSKI}, 2004, p. 15).

A menção ao verbo \emph{ouvir} dá o tom para um narrador"-personagem
cujas inflexões discursivas precisamos escavar para descobrirmos as
fundações do subsolo. O~homem do subsolo fala com a alteridade espectral
como o orador que articula seu discurso diante da plateia. Ainda assim,
logo veremos que os membros da plateia são mais do que meras abstrações.
A alteridade espectral -- em cujo bojo \emph{nós}, os leitores
pressupostos, também estamos incluídos -- é parte fundamental das
engrenagens da forma dostoievskiana.

O homem do subsolo tem 40 anos no início da narrativa e vive se
arrastando em meio à decrepitude há 20 anos. ``Já estive empregado,
atualmente não'' (\versal{IBIDEM}). Tão logo recebe uma herança de 6 mil rublos,
nosso (anti"-)herói, outrora um assessor"-colegial\footnote{O professor
  Boris Schnaiderman, tradutor de \emph{Memórias do subsolo}, nos revela
  que o cargo de assessor"-colegial equivale a um ``posto mediano da
  administração civil, no regime czarista'' (\versal{IBIDEM}, p. 17).},
aposenta"-se sem mais e passa a viver em seu cantinho -- o subsolo. A
personagem diz padecer do corpo (``doente'' e ``sofro do fígado'') e da
alma (``mau'' e ``desagradável''). Neste primeiríssimo momento, nós
ainda não sabemos que estamos sendo cotejados -- ainda não fomos
avisados sobre a queda de braço. Podemos pensar, a princípio, que se
trata de um pobre diabo que só faz flagelar"-se. A~herança, ao invés de
remediá"-lo, parece tê"-lo feito chafurdar no lodo. ``Se me dói o fígado,
que doa ainda mais''. Mas, ao oferecermos a mão ao náufrago para que ele
tente emergir da areia movediça, nos damos conta de que o homem do
subsolo nos quer tragar rumo à cirrose de seu subsolo. \emph{Se me dói o
fígado, que lhes doa ainda mais.} Senão, vejamos: ele nada entende sobre
sua doença e não se trata, ``embora respeite a medicina e os médicos''.
Até aqui, nada de novo no \emph{front} masoquista. Ocorre que o homem do
subsolo se diz supersticioso ao extremo -- ``ao menos o bastante para
respeitar a medicina''. Ora, nós não lemos o termo errado: trata"-se da
medicina, e não de bruxaria. Reza o desencantamento do mundo que a razão
proscreve a magia. Sendo assim, a personagem nos apresenta o primeiro
descompasso de sua voz. Dicção e contradição se mostram irmanadas desde
o princípio. E~mais: nosso (anti"-)herói é suficientemente instruído para
não ter nenhuma superstição, mas, ainda assim, ``sou supersticioso''.
Sombras se esgueiram de forma rente às luzes racionais. (Como veremos, à
racionalidade discursiva -- que, desde o conteúdo mais epidérmico, já se
mostra deveras irracional -- subjaz uma dinâmica que só faz alimentar"-se
de si mesma \emph{por meio do homem do subsolo e de seus
interlocutores/contrapontos}.)

Mas, ``se não quero me tratar, é apenas de raiva''. Logo, podemos (nos)
perguntar: ``Raiva do quê? Raiva de quem?'' Ao que a dialogia subliminar
do homem do subsolo se antecipa -- e começa a nos desancar: ``Certamente
não compreendeis isto''. Assim, o homem doente, mau, desagradável e
potencialmente cirrótico, o homem instruído porém supersticioso, o homem
que se flagela agora passa a nos aviltar. E~não se trata de um único
outro, trata"-se de uma plateia espectral composta por nós outros (vós).
Se o detentor do fluxo narrativo sentencia que nós não compreendemos por
que ele não quer se tratar, o homem do subsolo faz questão de dizer que
``eu compreendo''.

Se procedermos a uma primeira abstração do conteúdo modulado pelo (e
através do) homem do subsolo, a que \emph{lógica} \emph{subterrânea}
chegaremos? O~protagonista se avilta e se eleva, diz e se desdiz, vai e
volta. Movimento de gangorra, queda de braço, cabo de guerra. Desde o
primeiro parágrafo de \emph{Memórias do subsolo}, a cinética da forma se
insinua.

A personagem fala -- ou melhor, discursa -- à exaustão. ``Um momento!
Deixai"-me tomar fôlego\ldots'' (\versal{IDEM}, p. 17). Em seu monólogo dialogado, o
homem do subsolo fala conosco, fala por nós, fala contra si e contra
nós. Se os leitores nos fiarmos no canto de Circe do (anti"-)herói, se
tentarmos definir o homem do subsolo a partir do \emph{conteúdo} volúvel
e volátil de seu discurso, seremos solapados de tempos em tempos e não
acompanharemos a cadência motriz da narrativa. Identifiquemos, então,
mais alguns marcos de inflexão para auscultarmos como a banda toca no
subsolo:

Não aceitava gratificações; no entanto, devia premiar"-me ao menos desse
modo. (É um mau gracejo; mas não vou riscá"-lo. Escrevi"-o pensando que
sairia muito espirituoso; mas agora, percebendo que apenas pretendi
assumir uma atitude arrogante e ignóbil, não o riscarei, de propósito!)
(\versal{IDEM}, p. 15)

Menti a respeito de mim mesmo quando disse, ainda há pouco, que era um
funcionário maldoso (\versal{IDEM}, p. 16).

Não vos parece que eu, agora, me arrependo de algo perante vós, que vos
peço perdão?\ldots Estou certo de que é esta a vossa impressão\ldots Pois
asseguro"-vos que me é indiferente o fato de que assim vos pareça\ldots
(\versal{IDEM}, pp. 16-17)

Propositalmente, não aglutinei as três citações em único parágrafo para
que os espaços em brancos entre elas mimetizem os transcursos narrativos
cujas balizas e pontos de virada são as próprias citações. Quando os
leitores pensamos que já sabemos quem é o homem do subsolo a ponto de,
como Pôncio Pilatos, dizermos \emph{ecce homo}, \emph{eis o homem}, a
definição se nos escapa e se volta contra si mesma e contra nós mesmos.
O homem do subsolo se premia ao não aceitar gratificações, ele se premia
ao se privar das locupletações -- o gracejo é ruim, os leitores (que a
personagem teme e desanca ao mesmo tempo) escarnecem do narrador, mas é
precisamente por isso que o chiste deve permanecer no discurso/texto.
Assim, quando o homem do subsolo diz que mentira para os leitores e que
também, a seu ver, nos parecera que ele estava a pedir perdão {[}e que,
bom, isso (supostamente) lhe era indiferente\ldots{}{]}, os leitores de
ouvidos atentos, aqueles que acompanhamos o compasso e a cadência do
subsolo, podemos \emph{prever os solavancos e contradi(c)ções do
conteúdo a partir da volubilidade irônica própria ao movimento da
forma.} A~ironia do subsolo não é apenas um matiz que colore a narrativa
de vez em quando. A~ironia estrutura a narrativa e a faz mover"-se sobre
e contra si mesma, como que a forjar os alicerces do subsolo. Assim, a
melodia irônica, quando apreendida em seu devir contraditório, nos
permite antecipar os sobressaltos da canção. Em \emph{Memórias do
subsolo}, a forma como subsolo alicerça e enseja o conteúdo das
memórias.

Isso posto, podemos elencar algumas perguntas fundamentais: \emph{como}
se dá o duelo entre o homem do subsolo e as vozes que ele projeta e que
o perseguem? Há algum \emph{sentido} para o movimento contraditório da
forma dostoievskiana? Como o homem do subsolo se indispõe contra a
\emph{intelligentsia} revolucionária para espreitar os (des)caminhos da
história e suas (im)possibilidades emancipatórias\footnote{A análise
  propriamente dita do \emph{conteúdo} das contraposições do homem do
  subsolo em relação às posições filosóficas, ideológicas e políticas da
  \emph{intelligentsia} revolucionária russa será realizada tanto no
  capítulo 3, ``O prenúncio da teologia como teleologia -- e da
  teleologia como teologia'', quanto no capítulo 4, ``Sermão da Estepe:
  Ivan Karamázov e a filosofia dostoievskiana da história''\emph{,} ao
  longo dos quais discutirei a negação dialeticamente determinada de
  Dostoiévski em relação ao socialismo e as afinidades eletivas da
  utopia com o cristianismo. Neste segundo capítulo, demonstrarei como o
  movimento tautológico da forma se transforma em conteúdo decantado
  para apreender a reificação das vozes como o sentido -- ou melhor,
  como a falta de sentido -- do movimento da dialética polifònica.},
como é que o conteúdo de seu discurso se comunica com a forma do
subsolo?

As respostas para tais perguntam estruturam o diálogo original que este
livro pretende trazer à tona, isto é, a analogia dialética entre
\emph{Memórias do subsolo} e o devir fetichista da mercadoria
desenvolvido por Karl Marx ao longo dos primeiros capítulos do volume 1
de \emph{O capital. }

\section{2.2. Marx e a arquitetura do subsolo}

Chegamos ao momento de desvelar \emph{como} se dá o duelo entre o homem
do subsolo e as vozes que ele projeta e que o perseguem. Como já
sabemos, Bakhtin analisou com minúcia a articulação relacional das vozes
das personagens dostoievskianas. No entanto, como o crítico pressupôs o
concerto polifônico como uma catedral estática de tolerância e simples
coexistência da diferença, Bakhtin não pôde apreender o \emph{movimento}
da dialogia, sua vinculação necessária com a dialética. Ademais, como
Bakhtin postulou, \emph{a priori}, a equipolência das vozes, ele não
pôde extrair as implicações \emph{negativas} do caráter relacional das
vozes, isto é, o crítico não explicou como a equipolência dostoievskiana
flerta de maneira rente com o caráter relacional que torna as vozes
fungíveis -- e as reifica. A~partir de agora, Bakhtin passará pelo crivo
de Marx para que a lógica subterrânea de Dostoiévski venha à tona.

Já sabemos que o homem do subsolo sempre \emph{se mede com (e contra) o
outro.} Vejamos mais de perto as diferentes modulações de tal duelo:

\textbf{(i)} Vou dizer"-vos solenemente que, muitas vezes, quis tornar"-me
um inseto (\ldots) (\versal{DOSTOIÉVSKI}, 2004, p. 18).

\textbf{(ii)} Pensais, sou capaz de jurar, que escrevo tudo isso para
causar efeito, para gracejar sobre os homens de ação, e também por mau
gosto (\ldots) (\versal{IDEM}, p. 18).

\textbf{(iii)} Considerei"-me, continuamente, mais inteligente que todos
à minha volta, e às vezes -- acreditam? -- tinha até vergonha disso.
Pelo menos, a vida toda olhei de certo modo para o lado e nunca pude
fitar as pessoas nos olhos (\versal{IDEM}, p. 21).

\textbf{(iv)} Para nós, homens de pensamento (\ldots) (\versal{IDEM}, p. 22).

\textbf{(v)} Eu vos inquieto, faço"-vos mal ao coração, não deixo ninguém
dormir. Pois não durmais, senti vós também, a todo instante, que estou
com dor de dentes. (\ldots) Senti"-vos mal, ouvindo os meus gemidos
ignobeizinhos? Pois que vos sintais mal (\ldots) (\versal{IDEM}, pp. 27-28).

A dialogia bakhtiniana que apenas constata o caráter relacional das
vozes -- a noção ético"-ontológica de que o outro é parte indissolúvel e,
paradoxalmente, imiscível do eu -- não analisa os papéis diferenciais
que o eu e o(s) outro(s) desempenham em seus duelos. O~postulado
dialógico desconhece os papéis distintos das \emph{vozes/polos
relacionais.} Analisemo"-los, então, com a devida minúcia.

Na primeira frase -- ``Vou dizer"-vos solenemente que, muitas vezes, quis
tornar"-me um inseto'' --, a relação se desdobra sobre si mesma e nos
mostra um caráter tridimensional. Solenemente, o homem do subsolo
(posição 1) \emph{nos} diz (posição 2) que quis se tornar um inseto
(posição 3). Na primeira posição, a personagem se mede conosco com
\emph{status} sobrelevado. Na posição final, a dinâmica do subsolo
procedeu a um rebaixamento de sua condição, de tal maneira que o outrora
altivo desceu do altar -- e, consequentemente, \emph{nós}, que
primeiramente fôramos rebaixados, \emph{a posteriori} somos alçados em
relação ao inseto solene.

Na segunda frase -- ``Pensais, sou capaz de jurar, que escrevo tudo isso
para causar efeito, para gracejar sobre os homens de ação, e também por
mau gosto'' --, há uma projeção sobremaneira dostoievskiana.
Identificamos, novamente, três posições na relação -- (1) o homem do
subsolo que causa efeito e graceja sobre os homens de ação, (2) os
outros (``Pensais'') e (3) o homem do subsolo que tem mau gosto. O~homem
do subsolo, ainda uma vez, passa de escritor altivo e galhofeiro a
escrevinhador de mau gosto -- algo próximo ao que ocorrera na frase
anterior. No entanto, o polo da alteridade, polo com o qual o homem do
subsolo se mede, não existe a partir de si mesmo -- trata"-se de uma
projeção da personagem, isto é, a alterexistência emana, paradoxalmente,
do solipsismo relacional do homem do subsolo. Assim, se, de (1) a (3), a
personagem decai, ela ao menos existe a partir de si mesma, ao passo que
nós, (2) os outros, não temos existência relacional autônoma e,
portanto, somos (existimos) rebaixados na mesma medida em que a relação,
momentaneamente, nos eleva.

Não nos deve surpreender que, na terceira frase -- ``Considerei"-me,
continuamente, mais inteligente que todos à minha volta, e às vezes --
acreditam? -- tinha até vergonha disso. Pelo menos, a vida toda olhei de
certo modo para o lado e nunca pude fitar as pessoas nos olhos'' --, o
homem do subsolo, mais inteligente que todos, duvide dos demais (e,
consequentemente, de si mesmo, já que ele é o criador dos demais) e,
como consequência de sua genialidade altiva, não consiga sequer fitar as
pessoas nos olhos, as mesmas pessoas que, há algumas palavras, lhe eram
inferiores.

A frase quatro -- ``Para nós, homens de pensamento'' -- nos traz uma
equiparação positiva (ainda que o homem do subsolo não nos conheça), ao
passo que a última frase nos apresenta o homem do subsolo com o chicote
em riste diante da alteridade: ``Eu vos inquieto, faço"-vos mal ao
coração, não deixo ninguém dormir. Pois não durmais, senti vós também, a
todo instante, que estou com dor de dentes. (\ldots) Senti"-vos mal, ouvindo
os meus gemidos ignobeizinhos? Pois que vos sintais mal''.

É importante frisar que as cinco frases ocorrem em progressão, elas
aparecem em páginas que se sucedem. Assim, acompanhamos,
\emph{progressivamente}, o caráter desidêntico tanto do homem do subsolo
quanto de seus polos relacionais, \emph{a depender da relação específica
de que estejamos a tratar.} Como Bakhtin inviabiliza o desenvolvimento
da dialogia \emph{ao longo do tempo} e procura cristalizá"-la no
\emph{espaço coextensivo}, o crítico não consegue acompanhar o devir (e
o sentido) das contraposições. Precisamos, agora, entender o que
acontece dentro das relações para além das constatações epidérmicas das
idas e vindas e dos \emph{ups and downs}.

Marx começa a nos auxiliar com o desdobramento das aporias bakhtinianas
quando o autor apreende, \emph{em meio às relações de troca de
mercadorias}\footnote{Para Marx, a moderna divisão social do trabalho em
  termos capitalistas é a ``condição de existência para a produção de
  mercadorias, embora, inversamente, a produção de mercadorias não seja
  a condição de existência para a divisão social do trabalho. (\ldots)
  Apenas produtos de trabalhos privados autônomos e independentes entre
  si confrontam"-se como mercadorias'' (1998, p. 50). Não estou a
  transformar o homem do subsolo em proprietário de mercadorias, mas a
  protoforma capitalista do século \versal{XIX} é o contexto basilar de
  \emph{Memórias do subsolo.} O~fato de a personagem ter saído da esfera
  de venda de sua força de trabalho por conta da herança recebida não a
  retira, de forma alguma, da reificação social e psicológica das
  relações intersubjetivas -- esse é o fetiche por excelência de
  \emph{Memórias do subsolo}, fetiche que sua forma a mimetizar a
  reificação enseja e reproduz, conforme demonstrarei. Assim, a dialogia
  bakhtiniana, ao tornar ontológica e positiva a interdependência das
  vozes, não apreende o caráter reificado da vinculação social das
  pessoas, reificação que nos \emph{aparece} como a (suposta)
  independência da mônada burguesa e a necessidade de confronto com os
  demais.}, a tensão entre valor de uso e valor de troca. ``Os valores
de uso constituem o conteúdo material da riqueza, qualquer que seja a
forma social desta. Na forma de sociedade a ser por nós examinada {[}a
sociedade capitalista{]}, eles constituem, ao mesmo tempo, os portadores
materiais do -- valor de troca'' (1998, pp. 45-46). Em nossa análise, a
noção de valor de uso é problemática por excelência, uma vez que, no
universo dostoievskiano, não há um \emph{locus} de subjetividade que não
seja, em si e para si, relacional. Veremos, na seção final deste
capítulo, como um valor dissonante do homem do subsolo -- a tentativa de
vivenciar a bondade -- se vê silenciado pelo concerto dialético
reificado. Por ora, no entanto, continuemos a acompanhar a argumentação
de Marx:

\begin{quote}
É precisamente a abstração de seus valores de uso que caracteriza
evidentemente a relação de troca das mercadorias. Dentro da mesma um
valor de uso vale exatamente tanto como outro qualquer, desde que esteja
disponível em proporção adequada. (\ldots) {[}Em \emph{1 quarter de trigo =
a quintais de ferro, em 20 varas de linho = 1 casaco}{]} Ambas {[}as
mercadorias{]} são (\ldots) iguais a uma terceira {[}coisa{]}, que em si e
para si não é nem uma nem outra. Cada uma das duas, como valor de troca,
deve, portanto, ser redutível a essa terceira. (\ldots) {[}Na relação de
troca de mercadorias{]} Ao desaparecer o caráter útil dos produtos do
trabalho, desaparece o caráter útil dos trabalhos neles representados, e
desaparecem também, portanto, as diferentes formas concretas desses
trabalhos, que deixam de diferenciar"-se um do outro para reduzir"-se em
sua totalidade a igual trabalho humano, a trabalho humano abstrato.
(\ldots) O~que há de comum, que se revela na relação de troca ou valor da
mercadoria, é, portanto, seu valor {[}igual trabalho humano, trabalho
humano abstrato{]} (\versal{IDEM}, pp. 46-47; p. 46; p. 47).
\end{quote}

Devemos estabelecer as mediações para caminharmos do universo de Marx
para o subsolo de Dostoiévski. Em primeiro lugar, é importante frisar
que, para Marx, ``os personagens econômicos encarnados pelas pessoas
nada mais são que as personificações das relações econômicas, como
portadores das quais elas se defrontam''. Ademais, trata"-se de ``um
processo social {[}que passa{]} por trás das costas dos produtores''
(\versal{IDEM}, p. 80; p. 52). Sintomaticamente, o homem do subsolo afirma que
``a vida toda algo me arrastava a fazer esses trejeitos, a tal ponto que
acabei perdendo poder sobre mim mesmo'' (\versal{DOSTOIÉVSKI}, 2004, p. 29). Não
quero afirmar com isso que as personagens de Dostoiévski se resumam à
dinâmica dialógico"-dialética que lhes passa por trás das costas. Cada
personagem dostoievskiana, sempre em correlação com seus pares, encarna
universos complexos e repletos de vivências e questionamentos. (Nesse
sentido, os três capítulos que compõem a segunda parte deste livro, a
configurar uma antítese em relação aos dois capítulos que compõem esta
primeira parte, poderão elucidar \emph{conteúdos} fundamentais próprios
às vozes das personagens dostoievskianas.) Neste momento, interessa"-nos
dizer que a \emph{lógica} subjacente aos portadores das relações
econômicas a que Marx se refere nos pode ajudar sobremaneira a
esclarecer como a forma dostoievskiana enseja relações reificadas entre
as personagens que, analogamente, seriam portadoras -- ou, para
adaptarmos o termo de Bakhtin ao universo lógico deste livro, as
personagens seriam \emph{porta"-vozes} -- de uma forma que elas, em
determinados momentos, pensam articular como sujeitos, mas que, na
verdade, lança mão das personagens para sujeitá"-las a (não-)relações
danificadas.

Voltemos à citação de Marx acima mencionada e a analisemos, em termos
dostoievskianos, com mais minúcia. Marx nos diz que ``é precisamente a
abstração de seus valores de uso que caracteriza evidentemente a relação
de troca das mercadorias''. A~relação de troca de mercadorias pressupõe
uma equiparação, isto é, uma igualdade. Quando dizemos que 20 varas de
linho equivalem (ou são iguais) a 1 casaco, estamos dizendo que\emph{,
em meio à relação de troca}, duas coisas diferentes possuem algo em
comum. Por outro lado, não pode haver relação de igualdade onde não há
diferença, uma vez que não trocamos 20 varas de linho por 20 varas de
linho e/ou 1 casaco por um mesmo casaco. Como as 20 varas de linho são
distintas, em si e para si, do casaco, a relação de equiparação das
mercadorias precisa proceder à abstração dos valores de uso -- as
propriedades físicas e de satisfação de necessidades fornecidas pelo
linho e pelo casaco -- para encontrar aquilo que faz com que duas coisas
distintas tenham algo em comum. Esse algo em comum -- essa terceira
coisa que desponta em meio à relação de troca -- é, Marx nos informa,
igual trabalho humano, trabalho humano abstrato.

Como bem sabemos, o homem do subsolo \emph{se equipara} aos demais (e a
si mesmo) sucessiva e tresloucadamente. O~homem do subsolo, em si e para
si, \emph{não é idêntico} à alteridade espectral -- ainda que a
alteridade espectral, da qual, queiramos ou não, acabamos por fazer
parte, seja a projeção fantasmagórica da paranoia dialógica da
personagem. Como já vimos, a relação de \emph{equiparação} pressupõe,
dialeticamente, diferença e igualdade. Para Bakhtin, a diferença diz
respeito à imiscibilidade das vozes -- e já vimos que tal característica
colide com a noção de igualdade que pressupõe dialogia interna entre as
personagens e equipolência das vozes. Ao apreendermos a dialogia sob o
prisma da dialética, no entanto, conseguimos escavar o subsolo dialético
que torna fungíveis diferença e igualdade. Senão, vejamos: a logicidade
social da troca de mercadorias, que faz com que igual trabalho humano,
isto é, trabalho humano abstrato, seja a substância de equiparação de
coisas distintas entre si nos permite pensar que, em meio às relações
dialógicas de Dostoiévski, a forma articula as personagens de modo a
equipará"-las pelo substrato \emph{do duelo e da ironia.} A~trajetória
até aqui esmiuçada já nos permite dizer que duelos encarniçados subjazem
a cada colocação do homem do subsolo. A~ironia, por sua vez, apresenta
um matiz mais sofisticado: aquele que vive para duelar mutila,
quintessencialmente, o caráter social e necessariamente relacional que
constitui os homens. Aviltar os demais, em termos da dialogia ontológica
apreendida por Bakhtin, também significa aviltar a si mesmo -- tornar"-se
inseguro, desconfiado (não se pode fiar com o outro), irritadiço,
briguento, individualista, solitário. Características contumazes do
homem do subsolo. No entanto, como nosso (anti-)herói leva ``até o
extremo, em minha vida, aquilo que não ousastes levar até a metade
sequer'' (\versal{DOSTOIÉVSKI}, 2004, p. 146), as flagelações contra os demais
acabam lacerando suas próprias costas. Ao danificar as relações com os
demais -- relações de que o homem do subsolo não consegue e não pode
prescindir --, o sadismo acaba se revertendo em masoquismo, de tal
maneira que a sociedade alienada de si mesma, a sociedade socializada
pela reificação, encontra no homem do subsolo uma mônada burguesa
fortíssima, o articulador da (auto)ironia como sadomasoquismo. A~partir
de agora, começaremos a resvalar as fundações do subsolo formal de
Dostoiévski.

Precisamos investigar, agora, se os papéis desempenhados pelos polos
dialógicos se equivalem. Já sabemos que o sentido da relação de
\emph{troca dialógica} é a equiparação -- e/ou a mutilação dos
intercâmbios humanos. No entanto, Bakhtin não nos diz como os polos
dialógico"-fungíveis participam concretamente das relações de troca/duelo
para além da noção de equipolência que já questionamos frontalmente. Já
sabemos também que o homem do subsolo ora é altivo, ora é um anão; ora é
Golias, ora é Davi; ora é Jesus, ora é Judas; ora é Cristo, ora é
Barrabás. Ao descobrirmos a lógica (auto)contraditória do subsolo,
podemos prever, ao lado de Jesus Cristo, o que o homem do subsolo,
vestido com a pele de cordeiro do apóstolo Pedro, fará após idolatrar o
Messias. Assim clamou Pedro (tese): ``Mesmo que {[}tu, Jesus,{]} sejas
para todos uma ocasião de queda, para mim jamais o serás''. Assim
redarguiu Jesus (antítese): ``Em verdade te digo: nesta noite mesma,
antes que o galo cante, três vezes me negarás''. Assim falou Jesus
Cristo sobre Pedro, a síntese do mundo dos homens: ``Tu és Pedro, e
sobre esta pedra edificarei a minha Igreja'' (\versal{MATEUS}, 26, 33-34; 16,
18). Nós somos Pedro, e sob sua pedra edificamos o nosso subsolo.

Não é difícil perceber que o homem do subsolo que empunha a clava assume
um papel distinto da personagem que flagela as próprias costas. Ocorre
que, como somos socializados para observar as pessoas e as coisas (as
pessoas instrumentalizadas como coisas) como entes monadológicos, como
existências em si e por si mesmas, não nos damos conta de que o real,
mesmo como aquilo que se reproduz sob a égide da reificação, é
relacional. Assim, ao (anti-)herói que empunha a clava se equipara a
alteridade trêmula a ranger os dentes, ao passo que ao inseto
subterrâneo a se flagelar se equipara o outro colossal. Tanto o homem do
subsolo quanto a alteridade não coincidem consigo mesmos \emph{ao longo
de suas relações metamórficas.} Bakhtin não pôde analisar tal aspecto
pelo fato de distanciar a dialogia do transcurso dialético que se
desenvolve \emph{ao longo do tempo.} Assim, continuemos a escavar a
relação dialógica de equiparação para descobrirmos mais matizes da
arquitetura do subsolo.

Marx (1998) nos diz que

\begin{quote}
a relação mais simples de valor {[}por exemplo, \emph{20 varas de linho
= 1 casaco}, ou, de forma mais geral, \emph{x mercadoria A~= y
mercadoria B}{]} é evidentemente a relação de valor de uma mercadoria
com uma única mercadoria de tipo diferente, não importa qual ela seja. A
relação de valor entre duas mercadorias fornece, por isso, a expressão
mais simples de valor para uma mercadoria (p. 54).
\end{quote}

Na expressão de valor, segundo Marx, há dois polos: o polo da forma
relativa de valor e o polo da forma equivalente. Se tomarmos como
exemplo a expressão \emph{20 varas de linho = 1 casaco},

\begin{quote}
o linho expressa seu valor no casaco, o casaco serve de material para
essa expressão de valor. A~primeira mercadoria representa um papel
ativo, a segunda um papel passivo. O~valor da primeira mercadoria é
apresentado como valor relativo ou ela encontra"-se sob a forma relativa
de valor. A~segunda mercadoria funciona como equivalente ou encontra"-se
em forma equivalente (\versal{IDEM}, p. 54).
\end{quote}

O que, na leitura bakhtiniana de Dostoiévski, ficava sempre latente,
isto é, pululava como que a ser \emph{des"-coberto}, agora pode vir à
luz: o homem do subsolo participa, simultânea e sucessivamente, de uma
série de relações/duelos de equiparação. Em cada uma delas, ele (e nós)
ocupa(mos) um papel determinado. Ora ele expressa seu ego sobre (e
contra nós), de tal maneira que sua voz seja relativa à pequenez e à
passividade de nossa posição equivalente, ora a relação se inverte, e o
homem do subsolo, devidamente acocorado como o equivalente da relação,
passa a se flagelar enquanto (pensa que) o insultamos. Como se trata de
uma série vertiginosa de relações, as metamorfoses do subsolo implicam
inversões de papéis num piscar de olhos, de uma frase para a outra, de
uma oração para a outra. Ainda assim, como Marx bem observa,

\begin{quote}
é verdade que a expressão 20 varas de linho = 1 casaco (\ldots) encerra
também a relação contrária: 1 casaco = 20 varas de linho. Porém, assim
preciso inverter a equação para poder expressar o valor relativo do
casaco e, tão logo eu faço isso, torna"-se o linho equivalente ao invés
do casaco. A~mesma mercadoria não pode, portanto, aparecer, ao mesmo
tempo, sob ambas as formas na mesma expressão de valor. Essas formas
excluem"-se polarmente, {[}isso porque a{]} outra mercadoria, que figura
como equivalente, não pode ao mesmo tempo encontrar"-se em forma relativa
de valor. Não é ela que expressa seu valor. Ela fornece apenas o
material à expressão do valor de outra mercadoria'' (\versal{IDEM}, p. 55; pp.
54-55).
\end{quote}

Com a linguagem histórico"-social das mercadorias, Marx nos mostra que o
indivíduo monadológico até pode se imaginar independente em relação aos
demais em termos de falseamento ideológico da realidade. Mas, como sua
existência é intrinsecamente mediada pela alteridade social, o homem do
subsolo não pode monologar sem o diálogo, ainda que, no limite, sua
sanha solipsista mergulhe na esquizofrenia e transforme o estilhaço de
seu alterego em um outro que se lhe equipare. É~assim que, quando
analisamos as vozes dialógicas dialeticamente, percebemos que

\begin{quote}
forma relativa de valor e forma equivalente pertencem uma à outra, se
determinam reciprocamente, são momentos inseparáveis, porém, ao mesmo
tempo, são extremos que se excluem mutuamente ou se opõem, isto é, polos
da mesma expressão de valor; elas se repartem sempre entre as diversas
mercadorias relacionadas entre si pela expressão de valor (\versal{IDEM}, p. 54).
\end{quote}

A forma equivalente apresenta o polo de permutabilidade da relação de
troca, de tal maneira que, em meio às relações danificadas das
personagens dostoievskianas, ela representa, \emph{momentaneamente}, o
polo de instrumentalização. Como, em Dostoiévski, a dialética hegeliana
que contrapõe o senhor ao escravo e, necessariamente, o escravo ao
senhor é levada às últimas consequências, ninguém pode estar seguro
diante das contingências do poder. A~personagem que grita e expressa seu
ego sobre (e contra) a outra logo participará de uma nova relação, em
meio à qual o lobo pode ser reduzido a um trêmulo cordeiro. No capítulo
anterior, quando falamos sobre a dialética envolvendo crime e castigo em
Raskólnikov, vimos que o jovem assassino passa por um processo
contraditório de vilania e culpa na medida em que \emph{se defronta (e
se equipara)} com a memória de seus assassinatos. A~princípio, Amália
Ivánovna bem merece machadadas na têmpora, já que o papel passivo de
equivalente fungível, aos olhos utilitários de Raskólnikov, serve muito
bem à velha usurária. A~tese do crime parece se impor sem mais, até que
a antítese do fardo de ter aspergido sangue alheio faz com que
Raskólnikov passe pela cisão (\emph{raskol}) de se tornar um outro para
si mesmo e de fazer parte de uma nova relação: posteriormente, é a
memória dos corpos inertes e inocentes de Amália Ivánovna e sua irmã que
expressa o sentido histórico de infração da norma sobre (e contra) o
cálculo utilitário do homicida. A~tese pressupõe o movimento rumo à
antítese. Mas, como já sabemos, a antítese reconciliada, por sua vez,
não finda o duelo das estórias. A~antítese se volta sobre (e contra) si
mesma e se transforma em uma nova tese a ser denegada (e superada) por
uma tese outra. Assim, a dialética já nos permite iluminar, de forma
mais totalizada, as galerias constitutivas do subsolo dostoievskiano.
Quando, no capítulo anterior, nos referimos ao movimento da forma como a
dinâmica dos escombros, pensávamos, justamente, em construções que se
erigem não para a calmaria da trégua e da reconciliação, mas,
contraditoriamente, para a própria ruína. É~como se a colocação de
pilastras no subsolo para impedir o desabamento das galerias fosse
apenas uma trégua momentânea para o movimento que há de vir, para os
entrechoques que continuarão a se impor. Ainda não estamos em condições
de entender por que, em termos processuais e totais, o movimento da
forma se alimenta de si mesmo por meio de seus entrechoques dialéticos e
por meio das personagens, mas a análise se encaminha para a resolução
(irresoluta) de tal imbróglio.

Voltemos à análise da relação de valor -- a forma lógica do duelo
dostoievskiano. Marx já nos disse que ``a forma equivalente de uma
mercadoria é (\ldots) a forma de sua permutabilidade direta com outra
mercadoria''. Assim, uma peculiaridade chama a atenção quando observamos
a forma equivalente, já que, em tal polo de expressão, ``o valor de uso
torna"-se forma de manifestação de seu contrário, do valor'' (\versal{IDEM}, p.
59). Na expressão \emph{20 varas de linho = 1 casaco}, as 20 varas de
linho são um valor de uso distinto do casaco, mas, para que expressem
seu valor como produto de igual trabalho humano, de trabalho humano
abstrato, as 20 varas de linho, \emph{em meio à relação de equiparação,}
utilizam o corpo do casaco, em proporção determinada socialmente -- no
caso, 1 casaco. Assim, quando observamos o polo passivo, isto é, o polo
de permutabilidade, o polo equivalente, vemos que a relação fez com que
o casaco se tornasse, além de casaco, algo distinto de si mesmo, isto é,
valor de uso e valor. O~casaco, como valor de uso, continua a ser um
casaco, mas, como polo equivalente da relação de valor, seu corpo físico
expressa uma relação fisicamente intangível, ou seja, a equiparação
abstrata do trabalho de produção do casaco com o trabalho de produção do
linho, em determinada proporção, como igual trabalho humano, como
trabalho humano abstrato. Assim, é possível dizer que a antítese que
distancia e aproxima os polos da expressão de valor da mercadoria também
impõe uma cisão em um dos polos -- o polo passivo, o polo equivalente
--, já que, \emph{em meio à relação de equiparação}, o casaco, que se
distingue, em si e para si, do linho, acaba por lhe ser equivalente, em
determinada proporção, como igual trabalho humano. O~polo equivalente,
\emph{em meio à relação de equiparação}, é um duplo para si mesmo: seu
corpo físico pressupõe a abstração de si mesmo para que a forma relativa
de mercadoria expresse seu valor.

Em termos dostoievskianos, o polo equivalente -- em nossos exemplos, ora
a alteridade espectral (vós; \emph{nós}), ora o próprio homem do subsolo
-- verga seu corpo para receber as chibatadas (reais e/ou imaginárias)
daquele que exerce a tirania do polo relativo. Ademais, devemos observar
o polo equivalente sincrônica e diacronicamente, de modo a acompanharmos
o processo de sua metamorfose. Momentaneamente e em determinada relação,
a voz equivalente é amordaçada -- quando tiramos uma foto deste quadro
narrativo, notamos uma antítese externa entre os dois polos. No entanto,
quando a nossa câmera analítica tira fotos da sequência narrativa, ou,
por outra, quando colocamos os quadros narrativos em movimento, como em
um filme, vemos que há, ao lado da antítese externa entre as duas vozes
contrapostas, uma antítese interna em meio ao polo/voz que desempenha o
papel equivalente. No universo de Marx, a antíse interna do polo
equivalente diz respeito ao fato de que, em meio à relação de troca de
mercadorias, a forma equivalente da mercadoria expressa \emph{valor} --
igual trabalho humano, trabalho humano abstrato -- por meio de seu valor
de uso, que, como já sabemos, não contém valor social em si e para si.
Em Dostoiévski, a voz equivalente é instrumentalizada para que a voz
relativa se gabe e a desanque, ela expressa, em (e contra) sua
subjetividade relacional, o oposto de si mesma, mas também enseja, a
partir de sua antítese interna a seu papel na relação, a movimentação da
antítese externa rumo a uma nova (e contraditória) antítese externa,
isto é, rumo a uma nova relação de equiparação dialógica. Como nós
estamos analisando, especificamente, \emph{Memórias do subsolo}, a
miríade de relações contraditórias desponta de tal obra. Mas, desde o
capítulo anterior -- e tal abordagem também vai se dar ao longo do
desenvolvimento das discussões da segunda parte deste livro --, podemos
romper as fronteiras das obras de Dostoiévski e entrever que a dialogia
tende a um concerto antitético infinito, como se, conforme afirmou
Sophie Ollivier (1994), Dostoiévski houvesse composto uma ``dialética
sem síntese'' (p. 54), ou, para adaptarmos a boa expressão original de
Boris Schnaiderman (1983), estivéssemos diante de um \emph{turbilhão sem
sementes}. Assim, ao longo de seu movimento, a poética dostoievskiana
dissolve no ar tudo o que é sólido e tenta fincar raízes. O~turbilhão
autofágico do homem do subsolo -- turbilhão que, como veremos mais de
perto na última seção deste capítulo, acossa a prostituta Liza -- é
transmitido ao duplo homicida Raskólnikov, até que a prostituta Sônia
Marmieládova, irmã literária de Liza, se apieda do protagonista de
\emph{Crime e castigo} e o acompanha até o presídio siberiano. Quando a
reconciliação parece ter estancado o movimento antitético com a paz da
síntese, o espectro de um velho adágio romano ronda a Eurásia de
Dostoiévski: \emph{Se queres paz, prepara"-te para a guerra.} Rogójin,
irmão literário de Raskólnikov e assassino de Nastácia Filíppovna,
requer o perdão cristão do Príncipe Míchkin, a síntese dostoievskiana de
Jesus Cristo e Dom Quixote, em \emph{O idiota.} Quando Míchkin pensa em
estender a mão para o homicida, o jovem tuberculoso Hippolit o agarra
pelo colarinho, já que a Providência não teve piedade de sua condição ao
condená"-lo a um cadafalso do qual, semana a semana, ele tem cada vez
mais consciência. O~náufrago Hippolit acompanhará, até o último momento,
seu próprio naufrágio, sem que a piedade cristã do Príncipe Míchkin ou
de Aliócha Karamázov possam fazer algo a respeito. É~assim que, diante
do universo abandonado por Deus, em meio ao niilismo mais contumaz,
Kiríllov (\emph{Os demônios}) quer proceder à imitação de Cristo: se a
seita cristã foi fundada a partir do holocausto de seu fundador -- se o
sangue confere legitimidade à dinastia --, Kiríllov pretende fundar a
seita nihil com o cadáver de seu suicídio. Agora, tudo parece resolvido,
não parece haver nada mais malévolo a acontecer, mas Ivan Karamázov,
irmão do monge Aliócha Karamázov, e Nikólai Stavróguin, o super"-homem de
\emph{Os demônios}, se apresentam como os seguidores da seita
kiríllovica. Ivan, como grande intelectual, fará terra arrasada da
teologia cristã\footnote{No capítulo 4, ``Sermão da Estepe: Ivan
  Karamázov e a filosofia dostoievskiana da história'', veremos que a
  crítica arrasadora de Ivan em relação à teologia cristã se estabelece,
  dialeticamente, como uma negação determinada que lança as bases para
  novas perquirições espirituais em e a partir de Dostoiévski, de modo
  que Ivan (e esta é a ideia fundamental do capítulo) pode ser tido como
  um ateu espiritual. Por sinal, o clérigo Tíkhon, personagem de
  \emph{Os demônios}, assim sentencia: ``O ateísmo completo está no
  penúltimo degrau da fé mais perfeita (se subirá esse degrau já é outra
  história)'' (\versal{DOSTOIÉVSKI}, 2004, p. 662).}, ao passo que Stavróguin,
mais preocupado com a prática \emph{nihil}, passará a violar meninas em
tenra idade, uma vez que, \emph{se Deus não existe, tudo é permitido.} O
clérigo Tíkhon, ao ouvir a confissão pedófila de Stavróguin, entrevê que
a personagem sente regozijo na mesma medida em que se lacera moralmente.
Já não há discernimento entre prazer e dor, ambos estão umbilicalmente
atrelados até as raias do incesto. É~assim que, ainda uma vez (e
sempre), a protoforma dialética do homem do subsolo se insinua como a
\emph{lógica subterrânea} da forma (e do conteúdo) em Dostoiévski:

\begin{quote}
Quanto mais consciência eu tinha do bem e de tudo o que é ``belo e
sublime'', tanto mais me afundava em meu lodo e tanto mais capaz me
tornava de imergir nele por completo. Porém, o traço principal estava em
que \emph{tudo isso parecia ocorrer"-me não como que por acaso, mas como
algo que tinha de ser} {[}grifo do movimento poético coercitivo que
passa às costas da personagem{]}. (\ldots) Chegava a ponto de sentir certo
prazerzinho secreto, anormal, ignobilzinho quando às vezes, em alguma
horrível noite de Petersburgo, regressava ao meu cantinho e me punha a
lembrar com esforço que, naquele dia, tornara a cometer uma ignomínia e
que era impossível voltar atrás. Remordia"-me então em segredo,
dilacerava"-me, rasgava"-me e sugava"-me, até que o amargor se
transformasse, finalmente, em certa doçura vil, maldita e, depois, num
prazer sério, decisivo! Sim, num prazer, num prazer! (\ldots) O~prazer
provinha justamente da consciência demasiado viva que eu tinha da minha
própria degradação; vinha da sensação que experimentava de ter chegado
ao derradeiro limite; de sentir que, embora isso seja ruim, não pode ser
de outro modo; de que não há outra saída; de que a pessoa nunca mais
será diferente, pois, ainda que nos sobrasse tempo e fé para isto,
certamente não teríamos vontade de fazê"-lo e, mesmo se quiséssemos, nada
faríamos neste sentido, mesmo porque em que nos transformaríamos?
(\versal{DOSTOIÉVSKI}, 2004, p. 19; pp. 19-20; p. 20)
\end{quote}

Ocorre que o clérigo Tíkhon ainda acredita que a transformação é
possível -- e tal possibilidade se exacerba quando o nada empareda seu
interlocutor. Assim, munido do perdão e do paradoxo dostoievskianos, ele
recomenda a Stavróguin a cicatrização por meio do monastério. {[}Por
sinal -- e em tempo: como o homem do subsolo é deveras (des)idêntico em
relação a si mesmo, a personagem, em uma relação de equiparação com seu
alterego, assim sentencia a respeito de seu estado mórbido que funde
sadismo a masoquismo: ``Finalmente, quase acreditei (e talvez tenha
acreditado realmente) que o meu estado normal era esse'' (\versal{IDEM}, p.
19).{]} Aliás, como veremos no quinto e último capítulo deste livro, ``A
utopia como cicatrização do espírito'', Stavróguin e até mesmo Kiríllov
não foram suficientemente radicais em seu niilismo. O~homem ridículo,
protagonista de ``O sonho de um homem ridículo'', leva às últimas
consequências a frieza e a indiferença de um universo desprovido de
qualquer sentido. Para ele, não se trata de matar e estuprar -- não
necessariamente nesta ordem -- as outras, não se trata de fundar o que
quer que seja como se ainda quiséssemos permanecer em um mundo
transpassado pela idiotia, como se quiséssemos transmitir algum legado
mesmo após o suicídio. Trata"-se, pura e simplesmente, de cometer
suicídio. Aqui, o niilismo parece ter alcançado uma tese escatológica
que não pode movimentar uma nova antítese. Mas, como já estamos
familiarizados com a retroalimentação do universo dostoievskiano, é
possível prever -- e isso será desdobrado a contento no capítulo 5 deste
livro -- que o homem ridículo, ao tocar o fundo do umbral, passará por
uma experiência quintessencial -- justamente, o sonho transcendental --
que o fará tensionar a descrença absoluta.

Nossa análise buscou dar um salto dialético da parte para o todo, de tal
maneira que alcançamos uma visão panorâmica da dinâmica do subsolo.
Procedamos a um novo salto dialético que nos levará do princípio de
\emph{Memórias do subsolo} até a aterrissagem sobre o último parágrafo
da narrativa: ``Aliás, ainda não terminam aqui as `memórias' deste
paradoxalista. Ele não se conteve e as continuou. Mas parece"-nos que se
pode fazer ponto final aqui mesmo'' (\versal{IDEM}, p. 147). Percebamos que uma
voz autoral se transforma no polo relativo para sentenciar o término da
narrativa por detrás das costas do homem do subsolo como polo
equivalente, isto é, por detrás das costas do outrora protagonista que,
ao fim e cabo, perde a sua condição de \emph{narrador-}personagem. E~por
que isso teria acontecido? Ora, nosso sobrevoo dostoievskiano a partir
das equiparações dialógico"-dialéticas já nos pode fornecer uma resposta
a tal questão. De forma coerente com o movimento potencialmente infinito
da forma (e do conteúdo) em Dostoiévski, o fim de \emph{Memórias do
subsolo} é um não"-término, ou, para sermos mais precisos, um
antitérmino, um término antitético em si e para além de si mesmo. Qual
um \emph{deus ex machina}, a voz autoral susta a narrativa sem nos
explicar por que lhe parece ``que se pode fazer ponto final aqui
mesmo''. E~tudo isso apesar de, uma frase antes, a voz autoral haver
dito que o homem do subsolo não se contivera e acabara continuando suas
memórias. Isso nos mostra, a partir da imanência da dinâmica poética,
que a noção de movimento infinito se imbrica ao movimento da forma. Na
segunda parte deste livro, veremos como o conteúdo infinito se converte
em \emph{movimento da eternidade,} mas já estamos nos aproximando do
momento em que poderemos mostrar como e explicar por que o movimento
infinito da forma é análogo ao fetiche da mercadoria. Por ora, no
entanto, tenhamos em mente que o término inconcluso (a resolução
irresoluta) tanto de \emph{Memórias do subsolo} quanto dos duelos
dostoievskianos que se estendem através das outras obras pressupõe as
colisões vertiginosas e infindas de teses e antíteses.

\section{2.3. A forma dinheiro escava as galerias subterrâneas}

Já estamos em condições de entrelaçar as diversas equiparações
dostoievskianas, uma vez que já analisamos a dinâmica da célula e já
sobrevoamos a tessitura que vincula os órgãos. Marx continuará a nos
ajudar na escavação do subsolo. É~chegada a hora de sabermos o que advém
do desdobramento da forma simples de valor (\emph{x mercadoria A~= y
mercadoria B}) em direção à forma dinheiro (por exemplo, \emph{1 casaco
= 2 onças de ouro}), em meio à qual o polo de valor equivalente é
ocupado, enfim, pela figura social que representa o não"-valor de uso por
excelência, a coisa externa que, (supostamente) em termos passivos, faz
equivaler todos os valores de uso contra ela expressos, isto é, o
dinheiro. Quando desvelarmos a ciranda tautológica que transforma o
dinheiro em capital, estaremos em condições de entender qual é o
(não-)sentido da lógica subterrânea da forma em Dostoiévski.

Marx (1998) nos informa que

\begin{quote}
o segredo da expansão de valor, a igualdade e a equivalência de todos os
trabalhos, porque e na medida em que são trabalho humano em geral,
somente pode ser decifrado quando o conceito da igualdade humana já
possui a consciência de um preconceito popular. Mas isso só é possível
numa sociedade na qual a forma mercadoria é a forma geral do produto de
trabalho e, por conseguinte, também a relação das pessoas umas com as
outras como possuidoras de mercadorias é a relação social dominante (p.
62).
\end{quote}

Quando, na segunda parte deste livro, discorrermos sobre a filosofia
dialética da história em Dostoiévski e em Hegel, veremos como a
construção do ``conceito da igualdade humana'' como um ``preconceito
popular'', isto é, como a segunda natureza do imaginário coletivo, veio
à tona por meio de enormes sofrimentos e tragédias históricos. Por ora,
devemos salientar o que a dinâmica de Dostoiévski já tornou patente:
estamos diante da igualdade humana \emph{negativa}, a igualdade
reificada que nos coage ao ápice do utilitarismo -- eis a igualdade da
equiparação, a igualdade da fungibilidade. A~relação social dominante
que coisifica a relação das pessoas umas com as outras como possuidoras
-- ou pior, como não"-possuidoras -- de mercadorias alicerça a dinâmica
do subsolo como mimese poética a levar às últimas consequências o
sentido e o ressentimento históricos. As~masmorras do subsolo -- os
\emph{loci} dos polos das relações de equiparação -- transformam"-se em
conteúdo histórico precipitado; a forma, em seu movimento dialético,
enforma e deforma o conteúdo das vozes. Para empregarmos uma imagem que
desponta do \emph{ethos} -- ou melhor, do \emph{pathos} --
dostoievskiano, devemos dizer que as vozes das personagens (o conteúdo)
são enforcadas pelas cordais vocais (a forma como patíbulo).

A superação (\emph{Aufhebung}) da igualdade reificada pela igualdade
emancipatória é um teor de verdade que, em Dostoiévski, agoniza como
antítese e sobrevive como ímpeto de síntese em meio ao subsolo.
Dostoiévski nos revela o caráter escatologicamente negativo da noção
aristotélica de que os homens somos \emph{animais sociais.} A~sociedade
socializada pela reificação aproxima os homens reduzindo"-os a portadores
de relações econômicas -- ou pior, a porta"-vozes de relações
intersubjetivas deformadas por interesses utilitários -- e apenas muito
contingencialmente possibilita que as trocas entre as pessoas resultem
em comunhão. Como a (re)produção da realidade se impõe \emph{erga
omnes}, a percepção e a vivência das relações humanas para além da
reificação se veem deformadas pelo existente tal como ele se apresenta.
A obra de Dostoiévski, então, disseca a (tauto)lógica que faz com que o
condenado se afeiçoe ao verdugo\footnote{Nesse sentido, o poeta
  brasileiro Augusto dos Anjos se torna um exímio leitor de Dostoiévski
  com a visceralidade \emph{histórico"-subterrânea} de seus ``Versos
  Íntimos'' (In: \emph{Eu e Outras Poesias.} Rio de Janeiro: Civilização
  Brasileiro, 1998, p. 69): ``Vês! Ninguém assistiu ao formidável/
  Enterro de tua última quimera./ Somente a Ingratidão -- esta pantera
  --/ Foi tua companheira inseparável!// Acostuma"-te à lama que te
  espera! / O~Homem, que, nesta terra miserável,/ Mora, entre feras,
  sente inevitável / Necessidade de também ser fera.// Toma um fósforo.
  / Acende teu cigarro! / O~beijo, amigo, é a véspera do escarro,/ A~mão
  que afaga é a mesma que apedreja.// Se a alguém causa inda pena a tua
  chaga, / Apedreja essa mão vil que te afaga,/ Escarra nessa boca que
  te beija!''

  Na segunda parte deste livro, procurarei arremessar Dostoiévski (e a
  verve demoníaca de Augusto dos Anjos) contra si mesmo(s), de modo a
  descobrirmos que a mão que fere é a mesma mão que pode curar. Assim,
  \emph{Ouves? Alguém já ausculta a formidável/ Súplica de tua Fênix./
  Não só a Gratidão -- esta Pandora --/ É~tua companheira inseparável!//
  Levanta"-te do charco que te prostra!/ O~Homem, que, nesta terra ainda
  purulenta,/ Mora, entre feras, sente inextirpável/ Vontade de já não
  ser fera.// Toma um fósforo. Reconcilia a tua noite com o dia!/ O
  beijo, bela, é o prenúncio do amparo,/ A~mão que fere é a mesma que
  pode curar.// Se a ninguém causa ainda pena a tua chaga/ Cura essas
  mãos vis que não afagam,/ Ampara essa boca que ainda não beija. }}.

Ao lado de Marx, acompanhemos o processo da metamorfose da forma simples
de valor até sua figuração mais exuberante, totalizante -- e
totalitária.

\begin{quote}
A forma dinheiro da mercadoria é apenas a figura mais desenvolvida da
forma simples de valor, isto é, da expressão do valor de uma mercadoria
em outra mercadoria qualquer. (\ldots) A~forma individual de valor passa
por si mesma a uma forma mais completa. (\ldots) Assim, conforme ela entre
numa relação de valor com esta ou aquela outra espécie de mercadoria,
surgem diferentes expressões simples de valor, de uma mesma mercadoria.
(\ldots) Sua expressão individualizada de valor converte"-se, portanto, em
uma série constantemente ampliável de suas diferentes expressões simples
de valor (1998, p. 61; p. 64).
\end{quote}

Quando Marx nos apresenta a \emph{via crucis} que culmina na forma de
valor total ou desdobrada -- isto é, \emph{z mercadoria A = u mercadoria
B} ou \emph{v mercadoria C = w mercadoria D} ou \emph{= x mercadoria E}
etc. --, o concerto dialético"-polifônico a equiparar e a entrelaçar o
homem do subsolo, Liza, Raskólnikov, Sônia Marmieládova, o Príncipe
Míchkin, Rogójin, Kiríllov, Stavróguin, o clérigo Tíkhon, os irmãos
Aliócha e Ivan Karamázov e o homem ridículo já nos aparece em movimento
para além das fronteiras de suas respectivas obras. Cada (contra)posição
das vozes pressupõe sua(s) respectiva(s) antítese(s). A~voz do homem do
subsolo, por exemplo, é agora expressa em (e contra) inumeráveis outras
vozes do subsolo poético de Dostoiévski. Qualquer outra personagem
torna"-se eco da voz subterrânea. Assim, desponta essa voz mesma pela
primeira vez verdadeiramente como gelatina de ironia e contradição
indiferenciadas. Pois a ironia e a contradição que geram o movimento
poético são agora expressamente representadas como vozes equiparadas a
quaisquer outras vozes, sejam quais forem a forma e o conteúdo
peculiares que elas possuam, e, portanto, se objetivam em (e contra)
Liza ou Raskólnikov, Sônia ou Míchkin etc. Por meio de sua forma
irônico"-contraditória, o homem do subsolo se encontra agora também em
relação dialético"-dialógica não mais apenas com a alteridade espectral,
mas sim com o mundo da polifonia. Como voz relacional, o homem do
subsolo é o cidadão por excelência deste mundo. Ao mesmo tempo,
depreende"-se da interminável série de suas expressões de equiparação que
são apenas momentâneos à ironia contraditória do subsolo a forma e o
conteúdo específicos da voz na qual (e contra a qual) eles se
manifestam.

Agora, estamos bem próximos de entender que, quando dizemos que o homem
do subsolo desempenha o papel de equivalente universal, uma antítese
interna ao seu polo de equiparação se estabelece ao lado da antítese
externa que contrapõe toda a miríade de personagens ao nosso
(anti-)herói. O~homem do subsolo, como narrador"-personagem, como a
protoforma do devir formal de Dostoiévski, como a chave para a mediação
entre polifonia e dialética, articula as equiparações do subsolo na
mesma medida em que por elas é articulado. O~homem do subsolo realiza a
dialética histórico"-etimológica da palavra \emph{sujet}: o
\emph{sujeito} também é \emph{súdito}, o sujeito também é sujeitado.
Isso não nos deve surpreender, na medida em que já analisamos os papéis
diferenciais, contrapostos e vertiginosamente sucessivos dos polos de
valor relativo e equivalente. ``No mesmo grau em que se desenvolve a
forma valor em geral desenvolve"-se também a antítese entre ambos os
polos, a forma valor relativa e a forma equivalente'' (\versal{MARX}, 1998, p.
68). Ocorre que, na posição de equivalente universal, o homem do subsolo
acaba por sofrer a abstração por excelência -- \emph{todas} as vozes,
por peculiares e específicas que sejam, expressam seu valor \emph{por
meio da dinâmica do subsolo.} É~como se todas as demais personagens
dostoievskianas houvessem surgido a partir dos estilhaços do \emph{Big
Bang} subterrâneo. Assim, analisemos ainda uma vez, em diálogo com a
concretude de \emph{Memórias do subsolo}, o \emph{modus operandi} de tal
abstração por meio de uma sucessão de fragmentos:

\begin{quote}
\textbf{(i)} Eis o que seria melhor mesmo: que eu próprio acreditasse,
um pouco que fosse, no que acabo de escrever. Juro"-vos, meus senhores,
que não creio numa só palavrinha de tudo quanto rabisquei aqui! Isto é,
talvez eu creia, mas, ao mesmo tempo, sem saber por quê, sinto e
suspeito estar mentindo como um desalmado (\versal{DOSTOIÉVSKI}, 2004, p. 51).

\textbf{(ii)} {[}O homem do subsolo assume a voz da alteridade
espectral{]}: E~como são importunas, como são insolentes as suas saídas,
e, ao mesmo tempo, como o senhor tem medo! Afirma absurdos e se satisfaz
com eles; diz insolências, mas sempre se assusta com elas e pede
desculpas. Assegura não temer nada e, ao mesmo tempo, busca o nosso
aplauso. Garante estar rangendo os dentes e, simultaneamente, graceja,
para nos fazer rir. Sabe que os seus gracejos não têm espírito, mas, ao
que parece, está muito satisfeito com a sua qualidade literária. (\ldots)
Vangloria"-se da sua consciência, mas, na realidade, apenas vacila, pois,
embora o seu cérebro funcione, o seu coração está obscurecido pela
perversão, e, sem um coração puro, não pode haver consciência plena,
correta (\versal{IDEM}, pp. 51-52).

\textbf{(iii)} Está claro que eu mesmo inventei agora todas estas vossas
palavras. Isto provém igualmente do subsolo (\versal{IBIDEM}).

\textbf{(iv)} E eis mais um problema para mim: para que, realmente, vos
chamo de `senhores', para que me dirijo a vós como leitores de verdade?
(\versal{IBIDEM}).

\textbf{(v)} Eu escrevo unicamente para mim e declaro de uma vez por
todas que, embora escreva como se me dirigisse a leitores, faço"-o apenas
por exibição, pois assim me é mais fácil escrever. Trata"-se de forma,
unicamente de forma vazia, e eu nunca hei de ter leitores. Já declarei
isto uma vez\ldots (\versal{IDEM}, p. 53).

\textbf{(vi)} Alguém poderia implicar com essas palavras e me perguntar:
se de fato não conta com leitores, para que faz tais contratos consigo
mesmo? (\versal{IBIDEM})

\textbf{(vii)} {[}Eu{]} talvez também imagine, de propósito, diante de
mim um público para que me comporte de modo mais decente, quando estiver
escrevendo (\versal{IBIDEM}).
\end{quote}

Já estamos em condições de dizer que, no primeiro trecho, o homem do
subsolo, em termos dialéticos, mente ao dizer a verdade e diz a verdade
ao mentir. (E, sem que ele saiba por quê, o movimento contraditório da
mentira o toma de assalto.) A~depender do \emph{locus} de equiparação de
que faz parte -- se o polo relativo ou o polo equivalente --, nosso
(anti-)herói ora expressa (e deforma) determinado conteúdo, ora é
equiparado à voz alheia que também o deforma. A~conjuminação entre
mentira poética e teor de verdade -- e teor de mentira e verdade poética
-- diz respeito a um processo infindo que imbrica uma série de relações
e que arremessa o homem do subsolo, contínua e vertiginosamente, para
diversos pontos nodais de equiparação ao longo do devir subterrâneo de
Dostoiévski.

No segundo trecho, a antítese interna ao homem do subsolo como
equivalente universal funde forma e conteúdo e nos apresenta a
quintessência hermenêutica da poética de Dostoiévski. Nosso (anti-)herói
não apenas assume, ostensivamente, a voz da alteridade espectral, mas
também se torna um outro para si mesmo. Ao elencar a contumácia de suas
equiparações -- ``afirma absurdos e se satisfaz com eles; diz
insolências, mas sempre se assusta com elas e pede desculpas; assegura
não temer nada e, ao mesmo tempo, busca o nosso aplauso; garante estar
rangendo os dentes e, simultaneamente, graceja, para nos fazer rir'' --,
conseguimos escavar as galerias do subsolo como se estivéssemos
procedendo a um raio"-x esquelético. O~processo da metamorfose do homem
do subsolo não faz com que ele retorne ao \emph{locus} irônico de origem
como se fosse incólume às contradições e (auto)flagelações. O
protagonista se distancia continuamente do ponto de origem -- o
sadomasoquismo desvelado desde o parágrafo inicial de suas memórias --,
mas, dialeticamente, seu devir mantém um paralelismo com a fratura
original. Ao invés de uma forma poética circular -- ao invés, portanto,
de um eterno retorno --, estamos falando de um movimento helicoidal em
que a contradição, ao invés de se resolver, contradiz a si mesma para
continuar sendo contraditória, isto é, para continuar a se mover
contraditoriamente. Assim, o processo de equiparação e fungibilidade das
vozes encerra relações contraditórias e mutuamente exclusivas. O~devir
da dialética polifônica não suprime essas contradições, mas gera a forma
dentro da qual elas podem mover"-se. Essa é, em termos gerais, a dinâmica
da poética dostoievskiana com a qual as contradições ficcionais se
resolvem. É~uma contradição, por exemplo, que um corpo (uma voz) caia
constantemente em outro(a) e, com a mesma constância, fuja dele(a). A
elipse é uma das formas de movimento em que essa contradição tanto se
realiza como se resolve. Assim, já podemos entrever que o subsolo das
vozes dostoievskianas é escavado por meio de túneis helicoidais formados
por elipses sucessivas, equiparadas e entrelaçadas. Daqui a pouco,
quando Marx nos apresentar ainda mais elementos para entendermos a
imbricação das galerias subterrâneas, isto é, a imbricação das
sucessivas relações de equiparação, tendo em vista que já analisamos a
dialética de seus polos constitutivos, conseguiremos desvelar, \emph{in
toto}, a noção de (falta de) sentido da poética de Dostoiévski.

Voltemos, agora, aos fragmentos de \emph{Memórias do subsolo}, pois
``está claro que eu mesmo inventei agora todas estas vossas palavras.
Isto provém igualmente do subsolo'' (\versal{IDEM}, p. 52). O~conteúdo desvela a
forma, que, por sua vez, enforma e deforma o conteúdo. Já sabemos que a
desidentidade objetiva do homem do subsolo faz com que ele acredite em
si mesmo enquanto mente. Ocorre que, em um ato falho poeticamente
estrutural, nosso (anti-)herói nos revela que \emph{isto provém
igualmente do subsolo}. Ora, o protagonista cínico não está apenas se
desincumbindo de sua sordidez, ele nos está dizendo que o próprio
\emph{locus} que o enforma e que conforma o subsolo enseja tais
contradições. Quando outra masmorra do subsolo é aberta com mais esta
chave hermenêutica, o paradoxalista nos pergunta ``para que me dirijo a
vós como leitores de verdade?'' (\versal{IBIDEM}). Ao que bem podemos responder:
o sentido de sua forma (sentido que estamos para desvelar) pressupõe a
dialogia em relação a leitores pressupostos, a leitores de \emph{mentira
(e de mentiras).} Em termos do conteúdo volúvel, trata"-se da
impossibilidade reificada que o homem do subsolo tem para conviver,
efetivamente, com as pessoas que (não) o cercam. Mas, dialeticamente,
também se trata da impossibilidade de o homem do subsolo se bastar a si
mesmo -- o solipsismo povoado pela alteridade ausente diz respeito a um
ímpeto de convivência que, mesmo sendo ficcional, sequer a reificação
mais encarniçada pôde silenciar de todo. Então ficamos sabendo que o
homem do subsolo escreve unicamente para si mesmo e que, embora escreva
como se estivesse se dirigindo a leitores, ``faço"-o apenas por exibição,
pois assim me é mais fácil escrever. \emph{Trata"-se de forma, unicamente
de forma vazia}, e eu nunca hei de ter leitores. Já declarei isto uma
vez\ldots'' {[}grifo meu{]} (\versal{IDEM}, p. 53). Para a dialética de Dostoiévski
-- e suas personagens patológicas que o digam --, o incesto lança mão do
cordão umbilical para entrelaçar verdade e mentira. O~homem do subsolo,
intérprete por excelência da forma dostoievskiana, anuncia que o subsolo
erige uma forma vazia. Mas agora sabemos que, em seu devir, trata"-se de
uma forma que, ao equiparar os mais diversos conteúdos, dá vazão às e
esvazia (e silencia) as mais diversas vozes. Trata"-se de uma forma
esvaziada pela reificação em termos poéticos e históricos, uma forma que
se esvazia como conteúdo precipitado pelo niilismo. Nada pode perdurar,
a não ser o processo de nadificação da realidade (ficcional). Nenhuma
ideia sobrevive ao corredor polonês do subsolo, a não ser o processo
ideológico de reificação, a ideia social por excelência, o conteúdo por
excelência da forma de reprodução social. Ao fim e ao cabo, quando o
homem do subsolo projeta a nossa voz para se perguntar ``para que faz
tais contratos consigo mesmo, se de fato não conta com leitores'', a
resposta nos leva, de forma helicoidal e dialética, bem ao início deste
capítulo, quando mencionamos que nosso paradoxalista narra -- ou melhor,
discursa -- junto a um púlpito e diante de uma plateia: ``{[}Eu{]}
talvez também imagine, de propósito, diante de mim um público para que
me comporte de modo mais {[}in{]}decente, quando estiver escrevendo''
(\versal{IBIDEM}).

Ora, para poder ser equiparado às vozes díspares de seu público -- para
proceder à e passar pela abstração mais geral --, o homem do subsolo,
\emph{sujeito e súdito}, alcança a universalidade \emph{negativa} da
forma dinheiro, que, a bem dizer, é o desenvolvimento ulterior da forma
equivalente geral das mercadorias/vozes cuja apresentação fizemos mais
acima. A~forma equivalente geral pode ser recebida por qualquer voz. Por
outro lado, uma única voz quintessencial encontra"-se na forma
equivalente geral, porque e na medida em que é excluída por todas as
demais vozes como equivalentes. E~só a partir do momento em que essa
exclusão se limita definitivamente a uma personagem determinada a forma
valor relativa unitária do concerto polifônico adquire consistência (e
consciência) objetiva(s) e validade poética geral. Então, a personagem e
sua(s) voz(es) específica(s), com cuja voz a forma equivalente se funde
poeticamente, torna"-se mercadoria dinheiro ou funciona como dinheiro.
Torna"-se sua função especificamente poética e, portanto, seu monopólio
poético, desempenhar o papel de equivalente geral em meio ao concerto
dialético"-polifônico.

``A forma dinheiro é apenas o reflexo aderente a uma única mercadoria
das relações de todas as outras mercadorias'' (\versal{MARX}, 1998, p. 83). Vale
frisar que a abstração que fez com que o homem do subsolo se fundisse à
sua herança de 6 mil rublos é a cristalização dialética do processo que
nos pôde demonstrar como a dialogia das personagens se estrutura. A
forma dinheiro, como equivalente universal, congrega em si e para si a
abstração de sua própria voz na mesma medida em que recebe das demais
personagens, nos polos relativos infindos, o mote de suas expressões.
Tal fungibilidade, ao invés de alicerçar a equipolência bakhtiniana das
vozes, implica a equiparação/reificação dos sentidos, ressentimentos e
ideias que as personagens tentam fazer valer contra as demais. É~assim
que a forma dinheiro é a fotografia mais bem acabada de um processo que,
como sabemos, tivera início desde o primeiro parágrafo de \emph{Memórias
do subsolo.} Mas, como já mencionamos neste capítulo, a poética de
Dostoiévski não é fotográfica, ela não se coagula como uma estátua.
Nosso processo de análise se fundiu ao movimento narrativo precisamente
porque a primazia do objeto requer o acompanhamento cinético, a sucessão
fotográfica -- uma poética, a bem dizer, fílmica. Assim, se já
conseguimos entender como a dialogia vincula as personagens entre si,
precisamos, por fim, desvelar como as relações são articuladas tendo em
vista a mediação da forma dinheiro, isto é, como (e por que) a poética
de Dostoiévski percorre as galerias do subsolo.

\section{2.4. O Evangelho segundo o Capital: Mitgefangen, mitgehangen; presos juntos, juntos enforcados}

A análise do solipsismo relacional do homem do subsolo -- a forma e a
forja para as relações danificadas entre as personagens dostoievskianas
-- segue sobretudo um caminho oposto ao devir poético. Ela começa
\emph{post festum} e, por isso, com os resultados ainda mais reificados
do processo dialógico em termos histórico"-sociais. À~época de
Dostoiévski -- e de Marx e, posteriormente, de Bakhtin --, a coruja de
Minerva ainda não levantara seu voo crepuscular para nos revelar a
transformação da reificação e do fetichismo em segunda natureza da
sociedade capitalista socializada como a (suposta) ontologia dos homens
e mulheres reduzidos a (não-)possuidores de mercadorias. É~assim que,
hoje, de maneira radicalmente encarniçada -- e, sobretudo, após o
colapso da antítese soviética (ainda que a reboque das profundas
contradições do socialismo realmente existente) --, deparamos com uma
``formação social em que o processo de produção domina os homens, e
ainda não o homem o processo de produção'' (\versal{MARX}, 1998, p. 76). Ao
traduzirmos Marx para o universo dostoievskiano, escavamos uma poética
que faz as personagens falarem à rouquidão e que as lança em relações de
contraposição como se elas estivessem realmente \emph{convivendo e
coexistindo}. Mas, como bem sabemos, trata"-se da mutilação fundamental
da convivência humana -- é como se todas as vezes em que o homem do
subsolo e suas vozes poéticas a lhe serem equiparadas quisessem superar
a solidão da (não-)coexistência hedonista, atomística e coercitiva, algo
turvasse o ímpeto por comunhão, de modo a que o caráter socialmente
gregário dos homens e mulheres tivesse que passar por um longo e
redivivo período de quarentena. Como a maior parte das personagens
dostoievskianas não tem a sorte (reificada) do homem do subsolo --
lembremo"-nos de que as diatribes solipsistas de nosso (anti-)herói
amparam"-se sobre a herança de 6 mil rublos que lhe permite o isolamento
povoado de vozes --, o resultado e a resultante dos duelos narrativos
tendem a expressar, por meio dos conteúdos em duelo e fraturados, aquilo
que a forma fetichista já reificara. Eis a redução do processo de
socialização a um velho provérbio alemão: \emph{Mitgefangen,
mitgehangen}, isto é, \emph{presos juntos, juntos enforcados.} Assim,
nossas personagens descobrem que a mesma poética do subsolo, que as
torna vozes atomísticas e (supostamente) independentes, torna
independente delas mesmas o processo poético e suas relações dentro
desse processo, e também descobrem que a (suposta) independência
recíproca das personagens se complementa num sistema de dependência
reificada universal.

Analisemos, então, a lógica de totalização do processo que Marx alcunhou
de vampirismo -- o Dr. Frankenstein se ajoelha em face de seu
monstrengo, o filho de Saturno passa a devorá"-lo, os criadores se
prostram diante da criatura --, uma vez que o fetichismo das mercadorias
se alimenta do sangue dos produtores para continuar a se reproduzir
contra todos aqueles e aquelas que o soerguem e não veem suas vidas
serem drenadas pelo processo que lhes passa às costas.

Os homens relacionam entre si seus produtos de trabalho como valores não
porque consideram essas coisas como meros envoltórios materiais de
trabalho humano da mesma espécie. Ao contrário. Ao equiparar seus
produtos de diferentes espécies na troca, como valores, equiparam seus
diferentes trabalhos como trabalho humano. Não o sabem, mas fazem. (\ldots)
Todas as mercadorias são não"-valores de uso para seus possuidores e
valores de uso para seus não"-possuidores. Elas precisam, portanto,
universalmente, mudar de mãos. Mas essa mudança de mãos constitui sua
troca, e essa troca as refere como valores entre si e as realiza como
valores. As~mercadorias têm que se realizar, portanto, como valores,
antes de poderem realizar"-se como valores de uso (\versal{MARX}, 1998, p. 72; p.
80).

As personagens dostoievskianas relacionam entre si suas vozes em termos
dialógicos não porque consideram o concerto polifônico como a resultante
do fato de que todos somos animais sociais. Ao contrário. Ao equiparar
suas vozes de diferentes matizes na troca dialógica, como valores
fungíveis, elas equiparam suas diferentes subjetividades como igualdade
poética \emph{negativa} -- ainda que possa haver potencial emancipatório
encalacrado em busca de comunhão. As~personagens não o sabem, mas fazem.
Como nem mesmo o meu nome é algo em si e para si meu, uma vez que o
signo de identidade que nos é mais próprio tem um caráter fortemente
alheio por nos ter sido dado por nossos pais, isto é, pela
alteridade\footnote{Assim, a pergunta ``Como você se chama?'', em termos
  lógico"-sociais, seria mais precisa se assim fosse formulada: ``Como
  primeiramente chamaram você?''}, todos os ímpetos das personagens
dostoievskianas são não"-valores de uso para seu porta"-vozes e valores de
uso para seus não"-porta"-vozes. Elas precisam, portanto, universalmente
entrar em contato e se comunicar -- ainda que as relações de equiparação
só façam silenciar os demais. Mas essa comunicação constitui sua troca,
sua equiparação, e essa troca as refere como valores entre si e as
realiza como valores. As~personagens têm que se realizar, portanto, como
valores (como súditos), antes de poderem realizar"-se como valores de uso
(como sujeitos).

Notemos que, tanto em Marx quanto em Dostoiévski, o processo fetichista
nos apresenta uma racionalidade virada de ponta"-cabeça. A~princípio,
imaginamos que os possuidores de mercadoria, (supostamente)
independentes entre si, estabelecem o intercâmbio de seus produtos para
a realização, em termos racionais, do metabolismo social. Eu produzo
linho, você vende casacos. O~linho, como valor de uso, não me interessa.
O casaco, como valor de uso, não lhe interessa. Ocorre que, para que
nossas respectivas demandas individuais sejam satisfeitas -- para que o
sentido do valor de uso se realize --, é preciso que consigamos
percorrer o corredor polonês da troca de mercadorias. O~fetichismo,
fundamental e primeiramente, realiza suas próprias demandas -- é preciso
que as trocas, em termos tautológicos, se alimentem de si mesmas. Em
seguida, e apenas em seguida, as demandas dos possuidores de mercadorias
podem ser satisfeitas. Nesse sentido, já conseguimos entrever que, ao
longo da ciranda fetichista, os meios drenam os fins, a ponto de os
meios se tornarem fins em si mesmos. Tal apreciação já nos dá um
panorama geral do labirinto com o qual o subsolo se funde e se confunde.
Mas, para entendermos como o fetichismo poético se movimenta,
precisamos, a partir de agora, acompanhar o entrelaçamento das trocas de
mercadorias, isto é, a conexão entre as diferentes equiparações do
concerto polifônico de Dostoiévski.

Nesse sentido, Marx nos diz que

\begin{quote}
o processo de intercâmbio da mercadoria opera"-se por meio de duas
metamorfoses opostas e reciprocamente complementares -- transformação da
mercadoria em dinheiro e sua retransformação de dinheiro em mercadoria.
Os momentos da metamorfose da mercadoria são, ao mesmo tempo, transações
do possuidor de mercadoria -- venda, intercâmbio da mercadoria por
dinheiro; compra, intercâmbio do dinheiro por mercadoria, e unidade de
ambos os atos: vender para comprar. (\ldots) O~processo de intercâmbio da
mercadoria se completa, portanto, na seguinte mudança de forma:
Mercadoria -- Dinheiro -- Mercadoria, M -- D -- M (1998, p. 93; p. 94).
\end{quote}

Vender para comprar: tal movimento nos parece racional, uma vez que,
quando conseguimos vender nossa força de trabalho, fazemo"-lo para
sobreviver, isto é, para comprar a sobrevivência. Em termos
dostoievskianos, dialogar para comunicar: tal movimento nos parece
racional, uma vez que, quando dizemos algo para alguém, fazemo"-lo para
pô"-lo em comum, para existir -- ou melhor, para \emph{co"-existir.}
Ocorre que a racionalidade do movimento M -- D -- M se imbrica à
tautologia do fetiche D -- M -- D, ou, por outra, comprar para vender,
comprar para vender mais caro, comprar para acumular, acumular sobre o
acúmulo. É~quando o movimento da forma dinheiro se transforma em
capital. Em termos dostoievskianos, dialogar para monologar, dialogar
para silenciar, monologar sobre o monólogo dialógico. É~quando o
movimento da forma dostoievskiana se transforma em fetiche. Analisemos
mais de perto essa metamorfose que enreda M -- D -- M a D -- M -- D -- o
processo da metamorfose que enforca as vozes dostoievskianas com suas
cordas vocais.

Marx quebra o processo M -- D -- M em seus momentos constitutivos --
respectivamente, M -- D, primeira metamorfose da mercadoria ou venda, e
D -- M, metamorfose segunda ou final da mercadoria, isto é, compra --
para que entendamos suas imbricações.

\begin{quote}
O processo uno é um processo bilateral, do polo do possuidor de
mercadorias, venda, do polo contrário, do possuidor de dinheiro, compra.
Ou venda é compra, M -- D ao mesmo tempo D -- M. (\ldots) D -- M, a compra,
é ao mesmo tempo venda, M -- D; a última metamorfose de uma mercadoria
é, por isso, simultaneamente, a primeira metamorfose de outra mercadoria
(\versal{IDEM}, pp. 95-96; p. 97).
\end{quote}

Quando, anteriormente, tiramos uma fotografia da relação de equiparação
geral que fez com que todas as personagens se contrapusessem em relação
ao homem do subsolo, fizemos menção ao fato de que se tratava de um
quadro pertencente a uma sequência fílmica. Nós analisamos as funções
dos polos relativo e equivalente em meio às relações, mas não havíamos
mostrado efetivamente como as diferentes relações das personagens se
imbricavam. Agora, ao traduzirmos Marx para o universo de Dostoiévski,
entendemos que o transcurso de M -- D até D -- M nos fornece os
alicerces para acompanharmos a ciranda do subsolo. A~contraposição de
uma personagem em relação às outras é mediada, conforme podemos ver,
pela forma dinheiro. Quando a primeira mercadoria é vendida (M -- D), o
comprador pode satisfazer sua necessidade mediante a troca solvente, ele
retira a mercadoria de circulação e, com isso, a realiza como valor de
uso, isto é, como não"-mercadoria, ao passo que o vendedor, após a troca,
recebe seu quinhão. O~segundo movimento (D -- M) transforma o vendedor
em comprador, e, para tanto, um segundo proprietário de mercadorias
entra na cirada. Assim, em termos da nossa analogia entre Dostoiévski e
a lógica da mercadoria em Marx, quando o primeiro diálogo é articulado,
o falastrão pode satisfazer sua eloquência mediante a equiparação, ele
projeta suas ideias em meio ao concerto polifônico e, com isso, supõe
que está se realizando como sujeito, isto é, como voz em si e para si,
ao passo que o interlocutor, durante e após a equiparação, se vê,
momentaneamente, silenciado. O~segundo diálogo transforma o interlocutor
em falastrão, e, para tanto, uma nova personagem entra na ciranda.
Notemos que a forma dinheiro não precisa estar objetivamente presente em
meio ao universo dostoievskiano -- nas cenas que analisaremos na próxima
(e última) seção deste capítulo, a figura monetária estará presente, mas
essa não é uma condição \emph{sine qua non}, uma vez que a (tauto)lógica
de equiparação e abstração da forma dinheiro movimenta a poética de
Dostoiévski de modo a alcançar, por meio e para além das personagens, um
\emph{status} impessoal e onipresente.

No capítulo anterior, Bakhtin bem apontou que os conflitos
dostoievskianos se dão de maneira radicalmente extensiva -- isto é,
\emph{co"-extensiva}, em meio ao espaço, e não através do tempo. Dissemos
que o crítico russo, ao denegar a dialética de maneira estanque, não
pôde entender que Dostoiévski lança mão de um tempo radical que se funde
ao espaço, como se os diálogos fossem como que relações especulares do
devir das personagens que se equiparam -- no exemplo que utilizei, Liza
e Sônia eram, respectivamente, as imagens potencialmente emancipadas do
homem do subsolo e de Raskólnikov. Agora, podemos aprofundar a colocação
que fizéramos anteriormente. Já sabemos que o homem do subsolo, por
exemplo, participa de uma série vertiginosa de relações \emph{sucessivas
e simultâneas.} Já sabemos também que, a depender de cada uma das
relações de que faz parte, nosso (anti-)herói desempenha papéis
distintos. Ao invés, então, de pensarmos em uma série relacional linear
-- algo como a linha de produção de \emph{Tempos Modernos}, de Chaplin
--, pensemos na imbricação e na sobreposição vertiginosas das mais
diversas relações com os mais variados sentidos. Falamos de uma expansão
não apenas horizontal e vertical, mas de um movimento tridimensional --
chegamos ao espectro da geometria analítica e seus eixos x, y e z. Em
tais relações, há temporalidades escorreitas -- a sequência cronológica
estrita --, mas também há rememorações, ruminações, ressentimentos. O
homem do subsolo, por exemplo, oscila continuamente entre o tempo
presente e a (im)possibilidade de perdoar ao e (não) se enraivecer com o
tempo passado. Interessa"-nos dizer, então, que as relações se imbricam
umas às outras como uma sucessão vertiginosa de teses e antíteses que se
pressupõem e se repelem, de modo que o movimento, para prosseguir,
precisa denegar continuamente a si mesmo e, conforme já sabemos, precisa
sujeitar os sujeitos que pensam articular suas vozes -- ou melhor, seus
discursos -- como se eles fossem os marcos inicial e final (o alfa e o
ômega) de uma ciranda tautológica que se alimenta incessantemente de si
mesma.

Mas, como disséramos anteriormente, o ciclo M -- D -- M parece racional,
pois a venda para a compra diz respeito à satisfação objetiva das
necessidades humanas, sejam elas quais forem. Ocorre que, conforme Marx
nos diz,

\begin{quote}
o dinheiro afasta as mercadorias constantemente da esfera da circulação,
ao colocar"-se continuamente em seus lugares na circulação e, com isso,
distanciando"-se de seu próprio ponto de partida. Embora o movimento do
dinheiro seja portanto apenas a expressão da circulação de mercadorias,
a circulação de mercadorias aparece, ao contrário, apenas como resultado
do movimento do dinheiro (1998, p. 100).
\end{quote}

O quiproquó está armado. Movem"-se as mercadorias, como bem sabemos --
aliás, tal frase já nos torna vítimas do canto fetichista de Circe (o
feitiço se volta contra o feiticeiro), já que as mercadorias são levadas
ao mercado, elas não se movem sozinhas. Mas, ora, Marx afirma que

\begin{quote}
o mistério da forma mercadoria consiste (\ldots) no fato de que ela reflete
aos homens as características sociais do seu próprio trabalho como
características objetivas dos próprios produtos de trabalho, como
propriedades naturais sociais dessas coisas e, por isso, também reflete
a relação social dos produtores com o trabalho total como uma relação
social existente fora deles, entre objetos. (\ldots) Não é nada mais que
determinada relação social entre os próprios homens que para eles aqui
assume a forma fantasmagórica de uma relação entre coisas (\versal{IDEM}, p. 71).
\end{quote}

Vendedores levam suas mercadorias ao mercado, compradores as subtraem da
esfera da circulação para realizá"-las como valores de uso -- isto é,
como não"-mercadorias --, mas tal movimento, sempre mediado pela forma
dinheiro, parece projetar esta forma sempre mais além, sempre para um
ponto mais distante em relação ao ponto de partida, como se o movimento
do dinheiro sobre si mesmo fosse o objetivo da circulação, e não a
consecução das demandas dos homens e mulheres de boa vontade. Nesse
sentido, os mais diversos ramos da divisão social do trabalho e as mais
diversas vozes do subsolo dostoievskiano se rendem diante da abstração
monetária, pois, conforme já sabemos, ``no dinheiro é apagada toda
diferença qualitativa entre as mercadorias {[}e, assim{]}, ele apaga,
por sua vez, como \emph{leveller} radical, todas as diferenças'' (\versal{IDEM},
p. 111). Resta saber, então, como M -- D -- M, vender para comprar, se
torna um momento da ciranda fetichista D -- M -- D, comprar para vender
mais, comprar para vender mais caro -- ou, em termos dostoievskianos,
como o diálogo como comunicação e coexistência se transforma na ciranda
dialético"-tautológica do silêncio loquaz, o concerto polifônico que
permite a expressão das vozes à rouquidão na mesma medida em que torna
fungíveis (e assépticos) seus mais diversos conteúdos.

Já sabemos que a forma direta de circulação de mercadorias é M -- D --
M, isto é, transformação de mercadoria em dinheiro e retransformação de
dinheiro em mercadoria, vender para comprar. Ocorre que, conforme nos
diz Marx, ao lado dessa forma (aparentemente) racional, encontramos, no
entanto,

\begin{quote}
uma segunda, especificamente diferenciada, a forma D -- M -- D,
transformação de dinheiro em mercadoria e retransformação de mercadoria
em dinheiro, comprar para vender. Dinheiro que em seu movimento descreve
essa última circulação transforma"-se em capital, torna"-se capital e, de
acordo com sua determinação, já é capital (\versal{IDEM}, pp. 121-122).
\end{quote}

Marx nos pede para que vejamos mais de perto a circulação D -- M -- D.
Assim, logo percebemos que ela percorre, como circulação simples de
mercadorias, duas fases antitéticas:

\begin{quote}
Na primeira fase, D -- M, compra, o dinheiro é transformado em
mercadoria. Na segunda fase, M -- D, venda, a mercadoria é
retransformada em dinheiro. A~unidade de ambas as fases é, porém, o
movimento global, que troca dinheiro por mercadoria e, novamente, a
mesma mercadoria por dinheiro, compra mercadoria para vendê"-la, ou, se
não se consideram as diferenças formais entre compra e venda, compra
mercadoria com o dinheiro e dinheiro com a mercadoria. O~resultado, em
que todo o processo se apaga, é troca de dinheiro por dinheiro, D -- D
(\versal{IBIDEM}).
\end{quote}

Geralmente, dizemos que o camarada que rasga dinheiro, em uma sociedade
reproduzida segundo a lógica dos recursos escassos, só pode ser louco.
Mas, ora, como é possível que o metabolismo social, alicerçado sobre a
dinâmica do capital, proceda à tautologia do dinheiro que se desdobra
sobre si mesmo? Por que o entesourador, em sã consciência, arriscaria
seu tesouro como capital se se trata de trocar dinheiro por dinheiro?
Afinal, quem quer comprar precisa, necessariamente, encontrar alguém
disposto a vender. No entanto, quem precisa vender, ora, não
necessariamente consegue encontrar alguém disposto a comprar. Assim, ``é
agora evidente que o processo de circulação D -- M -- D seria insosso e
sem conteúdo caso se quisesse, por intermédio de seu rodeio, permutar o
mesmo valor em dinheiro por igual valor em dinheiro'' (\versal{IBIDEM}). Marx
está para nos mostrar, então, que a ciranda do capital, na verdade, deve
ser expressa por D -- D', sendo que

\begin{quote}
D' = D + ΔD, ou seja, igual à soma de dinheiro originalmente adiantado
mais um incremento. Esse incremento, ou o excedente sobre o valor
original, chamo de -- mais"-valia (\ldots{}). O~valor originalmente
adiantado não só se mantém na circulação, mas altera nela sua grandeza
de valor, acrescenta mais"-valia ou se valoriza. E~esse movimento
transforma"-o em capital (\versal{IDEM}, p. 124).
\end{quote}

É muito importante frisar que, no transcurso de \emph{O capital}, Marx
demonstra que a formação da mais"-valia, dialeticamente, se dá e não se
dá na esfera de circulação das mercadorias. Na verdade, a consumação do
processo de extração da mais"-valia só se realiza efetivamente com a
ciranda D -- M -- D', mas é na esfera da produção, isto é, em meio à
exploração da mercadoria por excelência -- a força de trabalho humano
--, que a mais"-valia é efetivamente usurpada, na medida em que é o
trabalho humano que valoriza o valor, é o trabalho que, além de
transformar os valores de uso -- 20 varas de linha em 1 casaco, por
exemplo --, agrega a eles valor social real a ser explorado (e não
devidamente remunerado aos trabalhadores) pela ciranda dos proprietários
dos meios de produção. A~construção do conteúdo social da mais"-valia,
fundamental para as análises de Marx, não será acompanhada \emph{pari
passu} neste nosso capítulo\footnote{Como Marx e a Teoria Crítica bem o
  sabiam, a historicização das categorias de análise é fundamental para
  o acompanhamento do movimento de reprodução social. Assim, uma leitura
  bastante peculiar da obra de Marx a arremessar o autor de \emph{O
  capital} contra si mesmo e para além de si mesmo pode ser encontrada
  ao longo da obra do teórico social e ensaísta Robert Kurz (1943-2012).
  Ao analisar o devir da valorização do valor, Kurz apreende uma
  profunda transformação na maneira pela qual o capital passa a se
  reproduzir com o advento da revolução microeletrônica ocorrida nos
  processos produtivos após a Segunda Guerra Mundial. Para Kurz, a
  extração real de mais"-valia se torna cada vez mais
  (auto)contraditória, na medida em que o trabalho humano vem sendo
  preterido em massa da esfera produtiva. Sem valorização real do valor
  pela mediação do trabalho humano -- e com a consequente contração do
  poder de consumo por parte da legião de desempregados e subempregados
  --, o capital, para Kurz, morde cada vez mais a própria a cauda e,
  além de desmantelar setores produtivos em escala global e legar
  inanição e miséria aos \emph{loci} proscritos pela exploração, realiza
  a tautologia de sua falta de sentido ao se financeirizar radicalmente,
  isto é, ao se realizar virtualmente como D -- D'. Assim, o mercado
  financeiro, o \emph{locus} da circulação do capital -- esfera em que,
  para Marx, não há real valorização do valor --, assume a vanguarda da
  crise e, por intermédio de um sistema de empréstimos sem lastro no
  processo real de produção de mercadorias, reproduz dinheiro como se
  ele pudesse se valorizar indefinidamente como capital, isto é, como se
  a antecipação de dinheiro por meio de empréstimos pudesse se amparar
  na produção real de mais"-valia futura. Quando a bancarrota, as
  falências e a inadimplência se sucedem de forma vertiginosa e
  contumaz, a valorização do valor, para Kurz, apresenta suas
  contradições fundamentais por meio do \emph{capitalismo de cassino} e
  dos \emph{sujeitos monetários sem dinheiro.} É~assim que, sempre para
  o teórico alemão, formam"-se as bolhas financeiras que, à iminência da
  ruptura, sustentam a reprodução real da riqueza com a ficção de seus
  créditos virtuais. Para o devido aprofundamento da análise de tal
  crise histórica que acaba por realizar a ciranda do capital como D --
  D' -- algo como o apogeu do tumor que, com o imperialismo de suas
  metástases, extermina o corpo que alimenta sua própria rapina --,
  conferir as seguintes obras de Robert Kurz: \emph{O colapso da
  modernização: da derrocada do socialismo de caserna à crise da
  economia mundial.} Rio de Janeiro: Paz e Terra, 2004; \emph{Com todo o
  valor ao colapso.} Juiz de Fora: \versal{UFJF}, Pazulin, 2004; \emph{Os últimos
  combates.} Petrópolis: Vozes, 2004. Conferir também o \emph{Manifesto
  contra o trabalho}, do qual Kurz, como então integrante do \emph{Grupo
  Krisis}, foi um dos coautores. Eis o link para a obra:
  \emph{http://www.krisis.org/1999/manifesto-contra-o-trabalho}.
  Por fim, aqui está um link para uma página que compila os textos de
  Kurz e as entrevistas que o autor concedeu:
  \emph{http://obeco.planetaclix.pt/robertkurz.htm}.
  Acesso em 01/03/18.}. Para as discussões que ora estabelecemos,
interessa"-nos o resultado tautológico da ciranda do capital, a noção de
que a razão, ao ser instrumentalizada pela valorização do valor,
transforma"-se em um movimento global irracional, uma criação que
subordina as criaturas que a erigem. Nesse sentido, vale a pena
reiterar, agora com mais minúcia, as diferenças fundamentais entre o
processo de aquisição de mercadorias (M -- D -- M) e o giro tautológico
do capital, a valorização do valor (D -- M -- D').

Assim, na circulação simples de mercadorias, M -- D -- M,

\begin{quote}
ambos os extremos têm a mesma forma econômica. Eles são ambos
mercadoria. Eles são também mercadorias de mesma grandeza de valor. Mas
eles são qualitativamente valores de uso diferentes, por exemplo, grãos
e roupas. O~intercâmbio de produtos, a mudança dos diferentes materiais
em que o trabalho social se representa, constitui aqui o conteúdo do
movimento. (\ldots) Consumo, satisfação de necessidades, em uma palavra,
valor de uso, é, por conseguinte, seu objetivo final (\versal{IDEM}, p. 123).
\end{quote}

Se quisermos adaptar o universo lógico de Bakhtin às categorias de nossa
análise, será possível dizer que o crítico russo entreviu na poética de
Dostoiévski a equipolência das relações de equiparação como a consecução
racional M -- D -- M. Nesse caso, a troca apresenta um teor de verdade e
universalização, uma vez que se trata de um intercâmbio voltado para a
satisfação efetiva das necessidades sociais. Em M -- D -- M, o dinheiro
é não um fim em si mesmo, mas a mediação efetiva para que as mercadorias
sejam equiparadas e, por fim, trocadas. Assim, em M -- D -- M, o
dinheiro não é capital. Se o movimento da poética de Dostoiévski se
pautasse pela circulação simples das vozes, seu sentido seria, de fato,
a \emph{comunicação} do conteúdo, e não a reificação que se volta e se
desdobra sobre si mesma. Estaríamos, então, em meio à catedral dialógica
na qual os intercâmbios implicariam a equipolência das diferenças. E,
conforme havíamos antecipado no capítulo anterior e ao longo deste
capítulo procuramos demonstrar, se M -- D -- M fosse a corda vocal a
modular as vozes, o concerto polifônico ouviria a tolerância como a
melodia e o sentido de sua totalidade.

Ocorre que a escavação dialética do subsolo de Dostoiévski nos revela
não o teor de verdade da troca M -- D -- M, mas a tautologia reificada
da ciranda D -- M -- D. O~diálogo visa ao silêncio -- ou pior, à
loquacidade que, como já sabemos, projeta a voz do outro sem que a
alteridade sequer tenha sido interpelada, sem que, muitas vezes, ela
sequer esteja presente. Quando o diálogo realmente acontece -- e
trouxemos uma pletora de exemplos ao longo destes dois primeiros
capítulos, assim como o faremos na segunda parte deste livro --, a
equiparação que visa ao próprio movimento torna as vozes impotentes
diante da impossibilidade de comunhão e verdadeira troca. É~assim que,
já à primeira vista, a ciranda D -- M -- D parece sem conteúdo, porque
tautológica. Marx, então, prossegue:

\begin{quote}
Ambos os extremos têm a mesma forma econômica. Eles ambos são dinheiro,
portanto não"-valores de uso qualitativamente diferenciados, pois
dinheiro é a figura metamorfoseada das mercadorias, em que seus valores
de uso específicos estão apagados. (\ldots) Uma soma de dinheiro pode
diferenciar"-se de outra soma de dinheiro tão somente mediante sua
grandeza. (\ldots) Seu motivo indutor e sua finalidade determinante {[}da
ciranda D -- M -- D{]} são, portanto, o próprio valor de troca, (\ldots) e
já por isso o movimento é sem fim. (\ldots{}) O~fim de cada ciclo
individual, em que a compra se realiza para a venda, constitui,
portanto, por si mesmo, o início de novo ciclo. A~circulação simples de
mercadoria -- a venda para a compra -- serve de meio para um objetivo
final que está fora da circulação, a apropriação de valores de uso, a
satisfação de necessidades. A~circulação do dinheiro como capital é,
pelo contrário, uma finalidade em si mesma, pois a valorização do valor
só existe dentro desse movimento sempre renovado. Por isso o movimento
do capital é insaciável (\versal{IDEM}, p. 124; p. 125).
\end{quote}

O diálogo monológico e monologado, além de insaciável, discursa por
discursar, e não para pôr as ideias em comum e para compartilhá"-las com
os outros. Assim, a resolução para a aporia da totalidade em meio ao
concerto polifônico se resolve ao permanecer irresoluta, isto é, ao
permanecer contraditória. A~catedral desaba. Suas ruínas -- a
imagem"-síntese das personagens danificadas de Dostoiévski -- são
movimentadas pelo turbilhão poético que, ao invés de erigir abrigos ou
novas moradas (esse seria o sentido racional das trocas M -- D -- M), só
faz arrastar os escombros tautologicamente sem lhes permitir a
reconciliação de suas fraturas. Chegamos, então, ao \emph{bunker} da
poética de Dostoiévski, o subsolo. As~personagens soerguem pilastras e
aríetes para não serem esmagadas -- se elas conseguissem vislumbrar a
totalidade insana do movimento, descobririam que a energia de suas vozes
é drenada para que a poética que as enforma e deforma constitua um
sujeito automático. A~dialética polifônica se torna o sujeito de um
processo em que ela, por meio da modulação constante das vozes e dos
papéis diferenciais das personagens, amplifica a sua própria ressonância
-- como diálogo monologado ela se repele de si mesma e, como voz
original, se torna um discurso a projetar o e a prescindir do público. A
polifonia se torna o processo reificado da dialética, ela orquestra as
vozes sujeitadas e, assim, se transforma em um sujeito automático. A
polifonia articula os e se articula pelos diálogos das personagens,
sustenta"-se e se multipla por meio deles e, então, recomeça a mesmidade
de seus ciclos contraditórios ainda uma vez e sempre. Diálogos
monologados, diálogos reificados, diálogos sem reconciliação -- diálogos
que geram ruídos, diálogos que gerem o silêncio. Eis o concerto
dissonante da polifonia de Dostoiévski.

\section{2.5. Do leito do subsolo, a extração de mais"-valia se reverte em
menos"-valia}

Até agora, transitamos majoritariamente pelo subsolo. Passemos a
acompanhar, então, nosso (anti-)herói em meio às suas (in)ações
reificadas pela superfície do mundo. O~tempo retroage, o homem do
subsolo tem apenas 24 anos quando a segunda parte de \emph{Memórias do
subsolo} -- ``A propósito da neve molhada'' -- vem à tona. O~subsolo há
muito o apartara do contato com as pessoas. Nesse sentido, quando
ouvimos o concerto polifônico \emph{como um todo} munidos de nossos
resultados analíticos, apreendemos camadas e mais camadas de solidão,
ressentimento, (auto)comiseração -- e, dialeticamente, saudade. O
turbilhão dialógico"-reflexivo do homem do subsolo foi contrapondo
barricadas às possibilidades de o protagonista estabelecer relações com
as pessoas. Ocorre que, a despeito de seus libelos contra os demais --
e, fundamentalmente, contra si mesmo --, o homem do subsolo não
conseguia extrair de si mesmo o teor de verdade da reconciliação:

Às vezes, era muito penoso ir à repartição: isto chegou a tal extremo
que, muitas vezes, voltei doente para casa. Mas, de súbito, sem mais nem
menos, vinha uma fase de ceticismo e indiferença (tudo me acontecia por
fases) e eu mesmo passava a rir da minha intolerância e das minhas
repugnâncias (\ldots). Ora não queria falar com ninguém, ora não só
iniciava uma conversa, mas tentava até tornar"-me amigo deles. Toda a
repugnância desaparecia num repente. Quem sabe? Talvez ela nunca
existisse em mim (\versal{DOSTOIÉVSKI}, 2004, p. 54)\footnote{Já não nos
  surpreende, é claro, que tal ciranda tenha tido início logo nas
  primeiras páginas de \emph{Memórias do subsolo.} Vejamos ainda uma vez
  como a cacofonia transforma a dialética entre a maldade e a
  reconciliação, \emph{em termos fetichistas, nas duas faces da mesma
  moeda:} ``Mas sabeis, senhores, em que consistia o ponto principal da
  minha raiva? O~caso todo, a maior ignomínia, consistia justamente em
  que, a todo momento, mesmo no instante do meu mais intenso rancor, eu
  tinha consciência, e de modo vergonhoso, de que não era uma pessoa má,
  nem mesmo enraivecida (\ldots). Minha boca espumava, mas, se alguém me
  trouxesse alguma bonequinha e me desse chazinho com açúcar, é possível
  que me acalmasse. Ficaria até comovido do fundo da alma, embora,
  certamente, depois rangesse os dentes para mim mesmo e, de vergonha,
  sofresse de insônia por alguns meses. É~hábito meu ser assim. (\ldots)
  Nunca pude tornar"-me mau. A~todo momento constatava em mim a
  existência de muitos e muitos elementos contrários a isso. Sentia que
  esses elementos contraditórios realmente fervilhavam em mim. Sabia que
  eles haviam fervilhado a vida toda e que pediam para sair, mas eu não
  deixava. Não deixava, de propósito não os deixava extravasar'' (\versal{IDEM},
  p. 16).}.

Já sabemos que confiar no homem do subsolo em termos unívocos significa
fiar a corda que logo nos enforcará. Já sabemos, ademais, por que tudo
lhe acontecia por fases/equiparações antípodas -- e entrelaçadas. Mas,
ainda assim, por que o homem do subsolo não extirpa de si, de uma vez
por todas, o ímpeto por beleza e comunhão que Stendhal certa vez
entreviu como a promessa de felicidade? (O homem do subsolo bem poderia
nos dizer que a promessa, se é o que há de vir, também desponta como o
que ainda não veio.) O~ímpeto por fraternidade permanece cativo no
subsolo como Pandora em sua caixa"-cárcere. Ainda assim, ele lá
permanece.

Embora eu tenha dito realmente que invejo o homem normal até a
derradeira gota da minha bílis, não quero ser ele, nas condições em que
o vejo (embora não cesse de invejá"-lo. Não, não, em todo caso, o subsolo
é mais vantajoso!) Ali, pelo menos, se pode\ldots Eh! Mas estou mentindo
agora também. Minto porque eu mesmo sei, como dois e dois, que o melhor
não é o subsolo, mas algo diverso, absolutamente diverso, pelo qual
anseio, mas que de modo nenhum hei de encontrar! Ao diabo o subsolo!
(\versal{IDEM}, pp. 50-51)

Bem vemos que o homem do subsolo, a quintessência do niilismo e do
ceticismo sobre os quais discorreremos ao longo dos três últimos
capítulos deste livro, faz a dúvida girar sobre e contra si mesma. Nosso
escorpião encalacrado quer abandonar sua cela, mas o ímpeto de superação
se vê ilhado e emparedado pela contumácia da reificação. ``Está claro
que não vos descreverei o que me sucedeu três dias mais tarde; se lestes
o meu primeiro capítulo, `O Subsolo', podeis adivinhar sozinhos'' (\versal{IDEM},
p. 70). O~mesmo renegado que clama, com o cerne de si, por algo outro,
acaba por modular, ainda uma vez (e sempre), o sujeito automático que o
sujeita ao subsolo, já que

\begin{quote}
{[}eu{]} não podia compreender sequer um papel secundário e justamente
por isso desempenhava bem tranquilamente, na realidade, o último dos
papéis. Herói ou imundície, não havia meio"-termo. Foi exatamente isto
que me perdeu, porque na imundície eu me consolava com o fato de ser
herói em outra hora, e o herói disfarçava consigo a imundície (\versal{IDEM}, p.
71).
\end{quote}

Afinal, ``pode porventura um homem consciente respeitar"-se um pouco
sequer?'' Eis, em suma, ``a consciência de que não tendes um inimigo,
mas a dor existe'' (\versal{IDEM}, p. 26; p. 28). Como já escavamos as galerias
do subsolo, sabemos que o eu narcísico acaba sendo o mais refinado
inimigo não apenas dos demais, mas, sobretudo, de si mesmo. Quando nosso
(anti-)herói nos diz que ``o sofrimento (\ldots) constitui a causa única da
consciência'' (\versal{IDEM}, p. 48), sabemos que a consciência que não consegue
se transformar em práxis e solidariedade também se imbrica às causas
múltiplas do sofrimento.

Mas é chegada a hora de acompanharmos nosso (anti-)herói para além da
masmorra de seu subsolo. A~analogia entre Marx e Dostoiévski nos
apresenta a \emph{ciranda do capital} como uma imagem"-síntese para a
\emph{arquitetura formal do subsolo.} A~partir de agora, então, veremos
como a ciranda do capital funda, dostoievskianamente, a
\emph{menos"-valia}, isto é, a desvalorização do ser humano na mesma
medida em que os coadjuvantes (o homem do subsolo entre eles) se
transformam em porta"-vozes para a autovalorização do dinheiro como
capital.

Encontremos, então, nosso homem do subsolo às voltas com mais uma de
suas tentativas ``de abraçar toda a humanidade'' (\versal{IDEM}, p. 74). Chega a
hora de visitar Símonov, seu ex"-colega de escola. É~bem verdade que

\begin{quote}
havia muitos outros colegas meus de escola em São Petersburgo, mas não
me dava com eles e até deixara de cumprimentá"-los na rua. É~possível que
eu me tenha transferido para outra repartição justamente para não ficar
junto deles e romper de vez com toda a minha odiosa infância. A~maldição
cubra aquela escola e aqueles terríveis anos de forçado! (\ldots) Restavam,
é verdade, dois ou três, que eu ainda cumprimentava ao encontrar. Entre
estes, Símonov, que em nada se distinguira na nossa escola, que era
quieto e de hábitos regulares, mas em quem eu reconhecia certa
independência de caráter e, mesmo, honestidade. Não creio, até, que
fosse de inteligência muito limitada. Passei com ele, outrora, certos
momentos bastante luminosos, mas não duraram muito e, de repente, como
que se cobriram de névoa. (\ldots) Eu suspeitava que ele tinha grande
repugnância por mim, mas, mesmo assim, frequentava a sua casa, pois não
tinha certeza disso.

Pois bem, de uma feita, numa quinta"-feira, não suportando mais a minha
solidão (\ldots), lembrei"-me de Símonov. Enquanto subia a escada para o
quarto andar, onde ele morava, ia justamente pensando que esse
cavalheiro já se cansava da minha companhia e que eu ia em vão à sua
casa. Mas, como sempre ocorria, tais considerações pareciam impelir"-me,
ainda mais, para uma situação dúbia e, por isto, entrei. Havia quase um
ano que eu vira Símonov pela última vez (\versal{IDEM}, pp. 74-75).
\end{quote}

O homem do subsolo nos traz um juízo algo dúbio sobre Símonov. Elogios
envoltos por ressentimento. O~nome de Símonov, tanto em russo quanto em
português, nos remete à mercantilização das práticas espirituais, isto
é, à simonia
(симония). Como o
\emph{valor} de Símonov é expresso pelo mecanismo de equiparação
universal, bem sabemos que o homem do subsolo, inominado para sempre
como a abstração monetária que contém todos os nossos nomes, logo fará
parte do comércio de (auto)indulgências.

Parecia haver um conclave no apartamento de Símonov, pois dois outros
colegas de escola do homem do subsolo lá estavam presentes. O~primeiro é
Trudoliubov, cujo nome é composto pelo substantivo
труд (\emph{trud},
labuta) e любовь
(\emph{liubov}, amor), ou seja, aquele que ``ama o trabalho'', conforme
nos ensina Boris Schnaiderman (\versal{IDEM}, p. 78). O~homem do subsolo lhe
reputa uma personalidade pouco digna de nota: trata"-se de um ``militar
de estatura elevada e fisionomia frígida, bastante honesto, mas que se
inclinava diante de todo êxito e era capaz de conversar unicamente sobre
produção''. (Veremos dentro em breve a que labuta o amor de Trudoliubov
e seus amigos se dedica.) O~segundo é Fierfítchkin, cujo nome algo
aliterante lembra o disse"-me"-disse da maledicência (e do tráfico de
influências). O~homem do subsolo não deixa por menos:

\begin{quote}
Fierfítchkin, descendente de alemães, de pequena estatura e cara de
macaco, {[}era{]} um imbecil que zombava de todos, meu acirrado inimigo
desde os primeiros anos da escola -- ignóbil e insolente fanfarrãozinho
que fingia a máxima delicadeza de espírito, não obstante fosse, no
fundo, um covarde (\versal{IDEM}, p. 77).
\end{quote}

Ora, ``quem de vós estiver sem pecado, seja o primeiro a lhe atirar uma
pedra'' (\versal{JOÃO}, 8, 7). Na esteira de nosso fetichismo polifônico, o velho
dito \emph{quem não conhece que te compre} está para o homem do subsolo
assim como a cabeça está para a carapuça. Já sabemos que as equiparações
vertiginosas fundem todas as diferenças. Assim, a definição que o homem
do subsolo faz sobre Fierfítchkin reflete nosso Narciso subterrâneo
diante do espelho.

Símonov, Trudoliubov e Fierfítchkin desprezam o recém"-chegado como a
``mais ordinária das moscas''. Segundo a leitura do homem do subsolo que
há muito já se confunde com o látego contra as próprias costas, o
desprezo se devia ao ``fracasso da minha carreira de funcionário e pelo
fato de eu ter decaído muito, de andar mal trajado etc., o que, aos seus
olhos, era um sinal evidente da minha incapacidade e insignificância''
(\versal{IDEM}, p. 75). Mas, ora, que faziam tão notáveis cavalheiros ali
reunidos?

\begin{quote}
Estava em curso uma conversa séria e até animada sobre o jantar de
despedida que aqueles cavalheiros pretendiam organizar para o dia
seguinte, em homenagem ao amigo deles, Zvierkóv, que era oficial e
estava de partida para uma província distante (\versal{IBIDEM})
\end{quote}

Quando Boris Schnaiderman (\versal{IBIDEM}) nos informa que o nome Zvierkóv tenta
enjaular uma \emph{fera}
(зверь,
\emph{zvier}), derivamos, rapidamente, as características do homenageado
que tanto causam repulsa (e inveja) ao homem do subsolo. A~boa aparência
e o farisaísmo; o carisma e o tato político; a liderança e a riqueza --
isso sem mencionar o sucesso com as mulheres. Zvierkóv partia para uma
província distante como um pedágio para, no futuro, poder voltar a São
Petersburgo como um tarimbado conquistador. Assim, as colocações do
homem do subsolo só fazem imiscuir o joio ao trigo, a apologia à
crítica. Amizade e tráfico de influência se cumprimentam, verdade e
reificação já não se distinguem, de tal maneira que o jantar de
despedida pressupõe a vantagem vindoura para os amigos/investidores.
\emph{Friends with benefits.}

Zvierkóv desponta para o homem do subsolo como o homem em quem ele não
pôde se transformar. Nosso (anti-)herói se sabe mais inteligente, mas
ele também sabe que a sociedade das gratificações e do compadrio só faz
relegar o mérito. Assim, o ostracismo revolta nosso protagonista pela
profunda consciência que ele tem de ser um coadjuvante. A~crítica se
confunde com a apologia, na medida em que o homem do subsolo não
consegue desconstruir, \emph{in toto}, a atração que as relações
reificadas exercem sobre seu imaginário. É~como se o homem do subsolo,
municiado por Oscar Wilde (2006), assim quisesse nos dizer: ``Fazer
parte da alta sociedade não é mais do que uma maçada, não fazer parte
dela é trágico'' (p. 81). É~nesse sentido que a tautologia do subsolo,
com suas idas e vindas, nos apresenta a dinâmica do ressentimento, ou
melhor, do \emph{re"-sentimento,} pois a vontade de poder volta à tona
com cada vez mais força quando a presa trêmula ousa sair de sua toca
subterrânea. E, quando a presa depara com a altivez da fera, a
consciência não a impulsiona para que ela deixe de invejá"-la -- se as
relações danificadas socialmente não apresentam veredas para além do
duelo, o mero ímpeto subjetivo de transformação nos mostra sua
insuficiência para modificar o mundo em si e para si.

A contabilidade dos \emph{friends with benefits} começa a inventariar os
custos -- ou melhor, os investimentos -- para o jantar de despedida. 7
rublos por cabeça. ``Já que somos três'' -- Símonov, Trudoliubov e
Fierfítchkin ignoravam solenemente o homem do subsolo --, ``serão vinte
e uma pratas; pode"-se jantar bem. Zvierkóv, naturalmente, não vai
pagar'' (\versal{IDEM}, p. 78). Quando nosso protagonista se mostra contrafeito e
diz que quer participar do jantar, Trudoliubov sentencia com o cenho
franzido: ``Você nunca esteve em boas relações com Zvierkóv''. Ao que o
homem do subsolo redargue com (suposta) nobreza: ``Creio que ninguém tem
o direito de julgar isto (\ldots). Talvez eu o queira agora, justamente
pelo fato de não ter estado em boas relações com ele'' (\versal{IDEM}, p. 79). A
tréplica de Trudoliubov -- a fiarmos, ainda que momentaneamente, em
nosso narrador"-personagem -- desvela, de fato, que o ressentimento não é
unívoco, isto é, que os duelos/equiparações apresentam as feridas
purulentas de ambos os polos relacionais: ``Bem, quem é que pode
compreendê"-lo?\ldots Todas essas alturas\ldots -- disse (\ldots) sorrindo com
ironia'' (\versal{IBIDEM}). Ainda que de maneira reificada, eis uma exígua
concessão do vício à virtude: os fariseus bem adaptados se sentem mais
rentáveis do que os pensadores, mas, ainda assim, a menção irônica de
Trudoliubov às alturas inefáveis do homem do subsolo se volta contra o
próprio piadista, uma vez que, no cume do pensamento, o ar para pessoas
como Zvierkóv \emph{et caterva} é deveras rarefeito.

Mas chega o momento em que é preciso separar credores de devedores:
Símonov sentencia que o homem do subsolo será incluído no jantar a ser
realizado no \emph{Hôtel de Paris} -- uma homenagem a um
peters\emph{burguês} célebre como Zvierkóv merece uma corte parisiense,
isto é, um reduto de primeira grandeza, algo para além da subordinação
eslava --, mas, quando em Roma, é preciso proceder como os romanos.
Assim, quando Trudoliubov e Fierfítchkin saem do apartamento de Símonov
sem as devidas despedidas em relação ao homem do subsolo, Símonov
interpela nosso (anti-)herói: ``Hum\ldots sim\ldots então, é amanhã. Vai dar o
dinheiro agora? É~para saber com certeza -- balbuciou confuso''
(\versal{IBIDEM}). Logo ficamos sabendo que a hesitação entre polida e cética de
Símonov dizia respeito a uma dívida de 15 rublos que o devedor do
subsolo nunca quitara, apesar de jamais a ter esquecido. {[}Ora, ora: a
solidão do subsolo também diz respeito a (ir)responsabilidades que
enfrentam a dialética da altivez e da autoflagelação: quando se sente um
rei, o homem do subsolo passa por cima de suas dívidas e as considera
mero preconceito a ser superado pelas alturas do reino inefável de suas
ideias; quando se sente um súdito, a vergonha por ser um devedor o
achincalha ainda mais. Assim, os polos opostos se repelem e se atraem
para condenar nosso (anti-)herói à solidão e ao cinismo -- a bem dizer,
à solidão cínica de quem se exime.{]}

Não prossigamos sem fazer as contas: 7 rublos para o jantar + 15 rublos
de dívida = 22 rublos. A~despeito dos juízos ácidos de nosso
narrador"-personagem, Símonov se mostra suficientemente educado e
generoso para aceitar que o devedor faça parte da claque em prol de
Zvierkóv, contanto que o homem do subsolo pague ``amanhã, durante o
jantar. Eu disse apenas para se saber\ldots Por favor\ldots'' (\versal{IDEM}, p. 80).

A polidez dúbia de Símonov inventa uma desculpa para enxotar o homem do
subsolo com elegância. Como é de práxis no subsolo, nosso protagonista,
tão logo fora do apartamento de Símonov, começa seus contorcionismos
mentais para justificar para si (e para nós) o fato de que, na verdade,
ele não quer ir ao jantar -- mas, como bem sabemos, é precisamente por
isso que o homem do subsolo será o primeiro a chegar ao \emph{Hôtel de
Paris.} Ocorre que o dinheiro é uma boa desculpa para que ele não vá.
``Ao todo, tinha nove rublos guardados. Mas, destes, era preciso dar
sete no dia seguinte, como ordenado mensal, a meu criado Apolón, a quem
eu pagava sete rublos sem comida. (\ldots) Não os pagar era impossível,
tendo em vista o gênio de Apolón'' (\versal{IDEM}, pp. 80-81). Ora, ora, se o
homem do subsolo pagasse o salário a Apolón, ainda lhe restariam 2
rublos para amortizar a dívida junto a Símonov -- ou para pendurar mais
5 rublos junto ao credor, já que os 2 rublos viabilizariam parte do
jantar. ``Bem que eu sabia, porém, que não lhe pagaria o dinheiro {[}a
Apolón{]} e iria, sem falta, ao jantar'' (\versal{IBIDEM}).

Em Dostoiévski, os humilhados e ofendidos, justamente pelo achaque
contumaz ao amor próprio, costumam desenvolver um senso de si que
contrasta fortemente com a ausência de quaisquer bens. É~como se os
despossuídos, munidos apenas de sua força de trabalho, afiassem ainda
mais o orgulho com a tentativa errante de se apossarem da própria
personalidade, de não se deixarem subjugar pela contumácia da injustiça.
O homem do subsolo, o baluarte da crítica apologista, mostra sua cara de
feitor sob a máscara abolicionista. ``Mas deixarei para alguma outra
ocasião falar deste canalha, desta minha úlcera'' (\versal{IBIDEM}). Logo veremos
que Apolón, longe de ser um canalha, é um trabalhador muito cônscio de
sua própria dignidade e de seus direitos, a despeito de todos os ardis
que nosso (anti-)herói lhe possa atribuir.

Sem mais, passemos às 17 horas do dia seguinte, no \emph{Hôtel de
Paris.} Como prevíramos, o homem do subsolo é o primeiro a chegar ao
jantar. ``Mas não se tratava mais de ser o primeiro''. Não apenas
ninguém ainda chegara, como também a mesa ainda sequer fora posta. ``O
que significaria aquilo? Depois de muito interrogatório, consegui saber
finalmente, \emph{por meio dos criados}, que o jantar estava encomendado
para as seis horas e não cinco. \emph{Isto me foi confirmado no bufê}''
{[}grifos meus{]} (\versal{IDEM}, p. 85). ``Assim, pois, os últimos serão os
primeiros, e os primeiros serão os últimos'' (\versal{MATEUS}, 20, 16). Não foram
os outrora colegas de escola do homem do subsolo -- supostamente, seus
iguais -- que o avisaram sobre o novo horário do jantar. Foram os
criados do \emph{Hôtel de Paris} -- criados como Apolón -- que, no bufê,
e não junto à mesa posta como sói acontecer com um cliente distinto, o
avisaram sobre a canalhice de Símonov, Trudoliubov e Fierfítchkin --
isso para não mencionarmos o prazer sádico de Zvierkóv, o porta"-voz do
jantar, a moeda de troca da vez. Era preciso, então, enfrentar tal
humilhação, paulatinamente, ao lado dos garçons que ignoravam a presença
de nosso protagonista coadjuvante enquanto traziam os candelabros e
dispunham os pratos e talheres. Súbito, após longos 60 minutos de
autocomiseração diante dos serviçais, a camarilha de Zvierkóv adentra o
\emph{Hôtel de Paris} ao longo do tapete vermelho que não havia
recepcionado nosso (anti-)herói.

\begin{quote}
Zvierkóv entrou na frente do grupo, evidentemente como chefe. Tanto ele
como os demais estavam rindo, mas, vendo"-me, empertigou"-se, acercou"-se
de mim sem se apressar, inclinou um pouco o busto com alguma faceirice e
me deu a mão, carinhosamente; não muito, mas com certa delicadeza
cautelosa, quase de general, como se, dando"-me a mão, ele se protegesse
de algo (\versal{IDEM}, p. 86).
\end{quote}

Para o homem do subsolo -- e para a sociedade tsarista amplamente
acostumada com a distinção hierárquico"-estamental --, todos e quaisquer
elementos que estabelecem mediações (ou, de modo mais preciso,
\emph{equiparações}) entre as pessoas, ou pior, entre os porta"-vozes de
suas respectivas posições sociais, reforçam as clivagens e distinções.
Em face do general Zvierkóv, nosso soldado raso do subsolo recebe o
cumprimento cinicamente afetuoso quase como uma necessidade de bater
continência. Ele esperava pilhérias de pronto por conta do completo
desrespeito de seus \emph{muy amigos}, mas a galhofa não tarda a chegar.
Primeiramente, os convivas discutem a imprudência -- jamais denominada
como a mais rematada canalhice -- de ter feito nosso paradoxalista do
subsolo esperar durante uma hora. Em seguida, o general Zvierkóv, de
modo ainda mais condescendente, passa a esmiuçar o trabalho e o soldo do
soldado do subsolo, como se o oficial estivesse a examinar se os
coturnos de seu subordinado estão devidamente engraxados.

\begin{quote}
{[}Zvierkóv{]}: -- Be"-e"-m, e quanto à manutenção?

{[}Homem do subsolo{]}: -- Que manutenção?

{[}Zvierkóv{]}: -- Quero dizer, o o"-ordenado?

{[}Homem do subsolo{]}: -- Mas, está"-me arguindo? Por quê?

Aliás, no mesmo instante, eu disse quanto ganhava. Estava ficando
terrivelmente vermelho.

-- Não é muito -- observou Zvierkóv com superioridade.

-- Sim, não dá para jantar em cafés e restaurantes -- acrescentou com
impertinência Fierfítchkin.

-- A meu ver, é simplesmente uma miséria -- observou Trudoliubov, sério
(\versal{IDEM}, p. 87).
\end{quote}

Ainda que saiba apreciar o prazer do \emph{harakiri}, nem mesmo o homem
do subsolo consegue suportar tamanhos desaforos. E~qual \emph{não é} a
nossa surpresa ao descobrirmos que o revanchista"-mor começa a disparar,
primeiramente, contra seu arqui"-inimigo dos tempos escolares, aquele que
já havíamos descoberto como um possível duplo do homem do subsolo --
Fierfítchkin. Pois, para começar, {[}meu caro{]}, ``estou jantando aqui,
no `café e restaurante', à minha própria custa, e não de alguém mais,
observe isto muito bem, \emph{Monsieur} Fierfítchkin'' (\versal{IDEM}, p. 89).
(Logo veremos que, cínica e tautologicamente, o homem do subsolo
procederá à imitação do gigolô Fierfítchkin.) Sem mais, o ressentimento
encalacrado sobrevoa o campo minado dos \emph{muy amigos} e passa a
bombardeá"-los com as pechas de medíocres, ignorantes e comezinhos.
Zvierkóv tenta restabelecer a ordem hierárquica entre seus subordinados,
mas o pileque do homem do subsolo -- ``bebia \emph{Laffitte} e xerez aos
copos'' (\versal{IDEM}, p. 91) -- vai gestando a ira até que a crítica e a inveja
se fundem e entram em erupção:

\begin{quote}
− Sr. Tenente Zvierkóv -- comecei --, saiba que detesto as frases, os
fraseadores e as cinturas apertadas\ldots Este é o primeiro ponto, e agora
vem o segundo. -- Todos ficaram muito agitados. -- Ponto número dois:
detesto a bajulação e os bajuladores. Sobretudo os bajuladores! Ponto
número três: amo a verdade, a franqueza e a honradez -- prossegui quase
maquinalmente, porque eu mesmo estava ficando gelado de horror, não
compreendendo como ousava falar daquele modo\ldots -- Amo o pensamento,
\emph{Monsieur} Zvierkóv; amo a camaradagem de verdade, de igual para
igual (\ldots) (\versal{IDEM}, p. 92).
\end{quote}

O orador do subsolo diz que detesta as frases e os fraseadores -- quanto
às cinturas apertadas dos altos oficiais de uniforme e das mulheres de
espartilho, nosso (anti-)herói só as odeia porque não pode
conquistá"-las, e não porque sua crítica é radical a ponto de
ressignificar a vontade de poder. Ele detesta os bajuladores acima de
tudo, sobretudo porque a bajulação bem articulada e reificada lhe mostra
sua incapacidade para o jogo político -- jogo que o homem do subsolo
tanto queria liderar. Ademais, o fanfarrão do subsolo, o mesmo que
atrasa o salário de Apolón e em quem mal podemos nos fiar, ama a
verdade, a franqueza e a honradez -- a concessão do vício à virtude diz
respeito ao fato de o homem do subsolo ter articulado seu cinismo
\emph{maquinalmente}, isto é, sem dar efetivo conteúdo à forma reificada
de suas diatribes. Ainda assim, suas críticas alcançam sentidos
socialmente estruturais, na medida em que desmascaram o caráter
classista da camaradagem que jamais se dá de igual para igual. (É claro
que, se o homem do subsolo pudesse envergar a farda de Zvierkóv, a
desigualdade repleta de galões, medalhas e gratificações polpudas --
para muito além das propinazinhas que o protagonista dissera não aceitar
-- o vestiria à perfeição.) Mas, como já era totalmente previsível, não
há qualquer fixidez nas posições dialéticas do homem do subsolo. Ele não
consegue manter a postura crítica, e logo as invectivas não apenas
começam a duvidar de si mesmas como passam a temer os e até mesmo a
pedir desculpa aos \emph{muy amigos} insultados. Ocorre que um insulto
pressupõe que o ofendido considere o ofensor digno de insultá"-lo. Quando
o homem do subsolo ``peço"-lhe a sua amizade, Zvierkóv, eu o ofendi'', o
general escorraça o pracinha de volta para a caserna do subsolo:
``Ofendeu? O~senho"-or! A~mi"-im! Saiba, prezado senhor, que nunca nem em
circunstância alguma o senhor \emph{me} pode ofender!'' (\versal{IDEM}, p. 96)

Se não estivéssemos em uma obra de Dostoiévski, seria possível dizer que
muito dificilmente alguém conseguiria chafurdar ainda mais no próprio
charco. Mas, como se trata do concerto polifônico reificado, sempre é
possível encontrar camadas mais profundas de degradação subsolo adentro.
Nosso (não-)herói estava ali, coberto de escarros. Súbito, enquanto
Símonov dava gorjeta aos garçons -- os mesmos serviçais que, a
princípio, lhe haviam avisado, a contragosto e na cozinha, sobre o novo
horário do jantar --, nosso herói se aproxima para lhe pedir 6 rublos.
Símonov tenta se desvencilhar do intelectual mendigo, mas a insistência
que lhe agarra pelo capote -- ``Por que me recusa? Sou algum canalha?
Disso depende tudo, todo o meu futuro, todos os meus projetos'' (\versal{IBIDEM})
-- faz com que o credor, para se desvencilhar de uma vez por todas de
seu devedor, quase atire o dinheiro contra o pedinte. Agora, os pajens
Símonov, Trudoliubov e Fierfítchkin estão prontos a escoltar Zvierkóv
rumo ao encontro com Olímpia, a cortesã cujos predicados olímpicos
apenas um digníssimo general pode custear.

Não percamos as contas: o homem do subsolo acaba de acrescentar 6 rublos
à antiga de dívida de 15 rublos junto a Símonov -- dívida que, é claro,
jamais será quitada. E~não seria nada despropositado perguntarmos se o
homem do subsolo de fato chegou a pagar os 7 rublos pelo jantar, tamanho
o alvoroço com que toda aquela farsa terminou. Considerando"-se, ademais,
que o serviçal do subsolo ainda não pagara os 7 rublos mensais -- comida
não inclusa -- a seu criado Apolón e que ele tinha 9 rublos guardados,
chegamos ao seguinte acúmulo de capital: 30 rublos, para o caso
muitíssimo improvável de o homem do subsolo ter quitado o jantar, ou 37
rublos, caso Símonov, já prevendo o calote do subsolo, tenha coberto a
parte do convidado indesejado e indesejável. (Caso isso tenha ocorrido,
a mais"-valia do subsolo se escora ainda mais sobre a menos"-valia das
relações interpessoais, na medida em que a dívida junto a Símonov passa
de 21 para 27 rublos.)

O mesmo homem do subsolo que acusara Fierfítchkin de não comer às
próprias custas agora se dirige ao prostíbulo de Olímpia para tirar a
desforra com seus oponentes e, eventualmente, fornicar às custas do
financiamento alheio. \emph{Enemies with benefits. }

Nosso (anti-)herói não encontra a camarilha de Zvierkóv no lupanar. Que
fazer, então, com o ódio a ser ejaculado? Ora, à falta de um duelo
(ficcional, é claro) contra os opressores, resta subjugar os pobres
diabos que suportam o subsolo sobre as costas -- do humilde cocheiro que
o levou do \emph{Hôtel de Paris} à casa de (muita) tolerância até a
jovem prostituta letã chamada Liza. Nesse momento, o simbolismo da
cadeia de humilhação é por demais tangível: os peters\emph{burgueses}
escolhem o \emph{Hôtel de Paris} como o local mais apropriado para o
jantar em homenagem a Zvierkóv, uma vez que o primitivismo russo parece
lacaiesco aos próprios russos. Nosso intelectual do subsolo, burocrata
proletário entre os pequenos"-burgueses, agora poderá ejacular a
humilhação sofrida contra a lumpemproletária de Riga -- como dissemos no
capítulo anterior, Letônia, Lituânia e Estônia, isto é, os Países
Bálticos, bem sabem o que significa o amor imperial e incestuoso da Mãe
Rússia. Já sabemos, ademais, que o homem do subsolo, para se elevar
diante daquela que já não poderia estar mais rebaixada, lança mão de
passagens belas e sublimes, nosso galanteador articula uma oratória
livresca para lembrar a Liza que a vida de prostituta a vai drenar
rapidamente -- e que ela precisa se persignar e depois se arrepender de
tudo aquilo. Assim, a vingança dostoievskiana é oblíqua e dissimulada:
ao invés de enxovalhar de uma só vez a mulher jovem, prostituta e letã,
o homem algo mais velho, burocrata e russo pretende passar por quem não
é, ou seja, o homem do subsolo, ao se alçar como galanteador
paternalista e herói romântico para salvar Liza daquela vida impura,
tenta proceder à imitação do conquistador Zvierkóv em face do pódium que
o congratula com Olímpia. Ocorre que, ao fim e ao cabo, a incauta Liza,
verdadeiramente tocada pelo discurso (supostamente) edificante, resolve
mostrar ao pregador do subsolo uma carta que um estudante de medicina de
boa família lhe havia enviado. Ele não sabia de nada, o rapaz lhe tinha
amor verdadeiro, então ainda havia possibilidades de que Liza fosse
acolhida por uma boa casa.

A despeito de todo o ardil e de todas as admoestações farsescas, aquela
esperança pura e trêmula acaba por trazer uma verdadeira comiseração ao
homem do subsolo. Já não nos é nada difícil prever que ele irá maculá"-la
quando voltar a se encontrar com Liza, mas, por ora, o ímpeto de
comunhão, mutilado e frágil, faz com que ele dê seu endereço à jovem
para que ela o visite em um local mais humano, um local em que as
relações possam se dar de igual para igual -- como fôra a reivindicação
que o paradoxalista do subsolo fizera em face de Zvierkóv.

Antes de prosseguirmos, uma pergunta propícia e contábil se impõe: teria
Liza recebido seus honorários ainda que o intercurso entre ela e o homem
do subsolo só tenha envolvido palavras? Se nosso (anti-)herói apenas
insuflou o idílio da moça sem remunerar suas esperanças, o processo de
extração da mais"-valia se torna ainda mais contumaz em seu esvaziamento
das relações humanas como menos"-valia. Em suma, remunerar Liza é algo
reificado; não a remunerar, no entanto, é algo sórdido -- e sumamente
dostoievskiano. A~mais"-valia se vale da menos"-valia para se
autovalorizar, ainda que, em termos tangíveis e objetivos, não tenha
havido acréscimo de valor.

Mas eis que os momentos terminais de nossa análise nos devolvem à toca
de nosso protagonista. ``Eu não podia morar em \emph{chambres"-garnies}:
meu apartamento era meu palacete, minha casca, o estojo em que me
escondia de toda a humanidade'' (\versal{IDEM}, p. 129). É~no subsolo que
acompanharemos as duas últimas equiparações dialógico"-dialéticas de
nosso paradoxalista: Apolón e Liza farão as vezes dos polos equivalente
e relativo para que totalizemos o concerto polifônico como o movimento
da contradição.

O homem do subsolo já nos dissera que o salário de Apolón -- 7 rublos
sem comida -- ainda não fora pago. Se não se tratasse de nossa
personagem volúvel, bem poderíamos imaginar, de pronto, a trágica e
rotineira relação de exploração entre patrões e empregados domésticos.
Ocorre que ``Apolón, o diabo sabe por quê, parecia"-me pertencer àquele
apartamento, e durante sete anos eu não consegui enxotá"-lo'' (\versal{IBIDEM}).
Ainda que a dialética envolvendo senhor e escravo também aguilhoe o
senhor à relação senhorial, não é difícil entrever que, de forma
corriqueira, é o subordinado que teme ser enxotado pelo patrão, e não o
mandatário que sente dificuldade em se desvencilhar de seu serviçal.
Mas, como se trata do homem do subsolo, já entendemos que senhor e
escravo são polos de constituição do próprio duelo identitário da
personagem -- e, de modo mais \emph{subterrâneo}, da \emph{forma} que o
articula. Assim, ``era impossível, por exemplo, reter o seu ordenado por
dois ou três dias que fosse. Ele teria criado um caso tal que eu não
saberia onde me esconder'' (\versal{IBIDEM}). Pobre homem do subsolo: de fato, se
Apolón não receber seus 7 rublos sem comida, o patrão é que não poderá
se alimentar. Só que, desta vez, o duelo envolvendo o senhor subordinado
e Apolón já se estende, indevida e inusitadamente, por duas semanas.
Tratava"-se do ímpeto arrivista de fazer valer a vontade de poder do
patrão. Diante disso, quais foram, segundo o homem do subsolo, as
atitudes de Apolón?

\begin{quote}
{[}Apolón{]} começava por fixar em mim um olhar extremamente severo, não
o desviava por alguns minutos seguidos, sobretudo ao receber"-me ou à
despedida, quando eu saía. Se eu, por exemplo, conseguia suportá"-lo e
fingia não notar aqueles olhares, ele, calado como antes, passava aos
suplícios seguintes. Entrava sem mais nem menos, suave e tranquilamente,
em meu quarto, quando eu estava caminhando ou lendo, parava à porta,
levava a mão às costas, afastava um pé e fixava em mim o olhar, que já
não era apenas severo, mas decididamente desdenhoso. Se eu lhe
perguntava de chofre o que queria, não respondia nada, continuava a
olhar"-me firmemente por mais alguns segundos e, depois, apertando os
lábios de certo modo peculiar, virava"-se lentamente no mesmo lugar, com
um ar significativo, e ia, também lentamente, para o seu quarto. Umas
duas horas mais tarde, tornava a sair de lá, aparecendo de novo daquele
modo na minha frente (\versal{IDEM}, p. 130).
\end{quote}

Em que Apolón age de maneira despropositada? Ora, em países como a
Rússia e o Brasil, ao longo de cujas histórias as esferas pública e
privada foram conformadas (e deformadas) pelas terríveis relações de
assimetria legadas pela servidão e pela escravidão e, consequentemente,
por enraizamentos muito frágeis de tradições democráticas, a postura de
Apolón só pode passar pela mais rematada insolência. Afinal, como é que
um serviçal ousa instigar o senhor a cumprir suas obrigações? Pois
Apolón, munido de um profundo senso de si e de sua posição -- uma
consciência reformista \emph{avant la lettre} que seria inflamada em
termos revolucionários pela geração de Lênin --, apenas cruza a
fronteira da privacidade de seu patrão para fazê"-lo lembrar"-se de que
sua própria privacidade, como criado, havia sido violada. Apolón é
sumamente educado -- em termos políticos mais aguerridos, ele não
precisaria sê"-lo diante do abuso contumaz por parte do patrão --, ele
impõe sua presença para sentenciar que \emph{eu existo e quero
subsistir.} Mas mesmo o contrato social mais usurpador -- 7 rublos
mensais sem comida -- parece demasiado para a mesquinhez do subsolo.
Assim, em mais um de seus rompantes -- isto é, em mais uma de suas
bravatas --, nosso burguês proletário vocifera a Apolón: ``-- Espere! --
rugi, correndo para junto dele. -- Não se mova! Assim. Responda agora:
para que veio olhar?'' (\versal{IDEM}, p. 131) Ao que Apolón, de forma educada e
verdadeiramente senhorial -- isto é, munido do teor de verdade que
pressupõe trocas humanas baseadas em respeito e igualdade, ainda que os
desníveis sociais as deformem --, assim redargue:

\begin{quote}
-- Se agora o senhor tem alguma coisa para me mandar fazer, a minha
tarefa é executar -- respondeu, depois de um novo silêncio, ciciando
baixo e espaçadamente, as sobrancelhas erguidas, e tendo girado
calmamente a cabeça de cima de um ombro para o outro, e tudo isto com
uma tranquilidade aterradora (\versal{IBIDEM}).
\end{quote}

Democrata \emph{avant la lettre}, Apolón sentencia que a autoridade bem
pode ser exercida -- ``minha tarefa é executar'' --, desde que não haja
autocracia. A~despeito (e precisamente por conta) das deformações
próprias ao narrador"-personagem, vemos como um veio de crítica social
efetiva e emancipatória se insinua pelas frestas do subsolo. Mas é
justamente essa postura segura de si e de sua posição por parte de
Apolón -- ainda que ele esteja subordinado ao patrão --, é justamente
tal personalidade senhorial que o homem do subsolo, o senhor, não
consegue suportar. Cada personagem que é equiparada ao homem do subsolo
lhe mostra uma fração de dignidade que lhe é ausente, como se, em cada
cotejamento, um estilhaço de sua identidade já esgarçada se perdesse.
Como nosso herói não existe em si e para si -- e aqui já não se trata
apenas de sua ontologia relacional --, mas como um \emph{sujet} (sujeito
e súdito) do processo de equiparação tautológica da forma
dostoievskiana, cada extração de mais"-valia parece dele subtrair novas
frações de humanidade. Mais"-valia, ainda uma vez, como menos"-valia.
Assim, tresvairado por ter sido \emph{rebaixado à condição de patrão} no
preciso momento em que lança mão de sua fúria senhorial, o homem do
subsolo explode: ``Ouça! -- gritei"-lhe. -- Aqui está o dinheiro, você
vê: está aqui! (\ldots) Todos os setes rublos, mas você não os receberá,
não rece"-e"-eberá até que venha respeitosamente, de cabeça baixa,
pedir"-me perdão. Está ouvindo?!'' (\versal{IDEM}, p. 132) Notemos que o homem do
subsolo prolonga a vogal \emph{e} da mesma forma como antes Zvierkóv o
fizera: ``-- D\emph{i"-i-}ga"-me, você\ldots trabalha num departamento? (\ldots)
D\emph{i"-i-}ga"-me: o que foi que o obr\emph{i"-i-}gou a deixar o emprego
anterior? (\ldots) B\emph{e"-e"-e}m, e quanto à sua manutenção? (\ldots) Quero
dizer, o \emph{o"-o}rdenado?'' (\versal{IDEM}, pp. 88-89) A~voz opressora de
Zvierkóv rasga o homem do subsolo, então nosso (anti-)herói procura
repassar a humilhação que sofrera a seu criado Apolón. No entanto, como
o empregado não lhe serve como bucha de canhão, o homem do subsolo fica
premido entre o cume e a base da pirâmide. Apolón não se curva diante do
patrão como o homem do subsolo fizera em face de Zvierkóv. Não haverá
pedido de desculpa por parte do empregado, uma vez que, a bem dizer,
quem deve explicações é o patrão. Se o homem do subsolo fosse senhor de
si como Apolón o é, ele teria esconjurado a camarilha de Zvierkóv como
uma corja de bajuladores e, antes de abandonar o \emph{Hôtel de Paris},
a dívida junto a Símonov teria sido finalmente quitada. Mas como a
ciranda de nossa(s) personagem(ns) apenas permite que a liberdade
desponte por meio de espasmos -- a consciência da liberdade não permite
sua vivência para além das frestas do subsolo --, nosso (anti-)herói
começa a espumar quando Apolón diz que pode denunciá"-lo para a polícia,
uma vez que o empregado fora chamado, de forma sintomática (e
subterrânea), de ``carrasco'' por seu patrão. Quando o homem do subsolo
enfim ordena a Apolón que vá à polícia, ou então ``você nem imagina o
que acontecerá'', o criado, há sete anos acostumado com o fogo fátuo do
patrão, atesta para os devidos fins: ``E onde já se viu uma pessoa ir
procurar uma autoridade contra si mesma? E, quanto ao medo, o senhor
está gritando inutilmente, porque não vai acontecer nada'' (\versal{IDEM}, p.
133). Apolón bem sabe que o cão ladra, mas a caravana passa. Assim,
quando o ridículo do duelo se mostra em toda a sua extensão, \emph{o
acaso poeticamente objetivo} de Dostoiévski faz com que o homem do
subsolo volte a se tornar moeda de troca para uma personagem que retorna
à ciranda de equiparações no preciso momento em que Apolón e nosso
(anti-)herói já não têm mais nada a dizer um para o outro. ``Ali está
\emph{uma qualquer} e pergunta pelo senhor'' (\versal{IBIDEM}) -- eis como Apolón
avisa a seu patrão que a jovem Liza quer vê"-lo. Neste momento, o
empregado ironiza o patrão descarregando seu desdém contra uma mulher
que, ele pressupõe, só pode ser \emph{uma qualquer} -- afinal, a
sociedade dos títulos, das honrarias e das gratificações reificadas
destinaria que tipo de dama a nosso peão do subsolo? Neste momento,
Apolón também mostra sua face de tsar sob sua máscara republicana.

Pobre Liza: a letã, jovem e prostituta terá que \emph{pagar} pela
menos"-valia que esmaga o homem do subsolo entre as equiparações com a
camarilha de Zvierkóv e Apolón. E, através de uma inversão sumamente
dostoievskiana, a bondade de Liza será retribuída com o \emph{pagamento
pelos serviços prestados. }

Ao ver Liza, o homem do subsolo, inusitadamente convertido em anfitrião,
se desespera diante de seu criado:

\begin{quote}
− Apolón -- murmurei numa fala apressada e febril, atirando na sua
frente os sete rublos que estiveram o tempo todo em meu punho. -- Aí
está o seu salário. Como vê, entrego"-lhe. Mas, em compensação, deve
salvar"-me. Traga"-me imediatamente da taverna chá e dez torradas. Se você
não quiser ir, fará uma pessoa infeliz! Você não sabe que mulher ela
é\ldots Ela é tudo! Você talvez esteja pensando alguma coisa\ldots Mas não
sabe que mulher ela é!\ldots (\versal{IDEM}, p. 134)
\end{quote}

Ainda uma vez (e sempre), não percamos as contas: caso tenha quitado sua
parte do jantar junto ao \emph{Hôtel de Paris} -- coisa de que bem
devemos duvidar --, o homem do subsolo acumula, após o pagamento de 7
rublos (sem comida) a Apolón, o montante de 23 rublos. Para o caso bem
mais do que provável de o homem do subsolo, como Fierfítchkin, ter
jantado às custas dos outros, restam"-lhe 30 rublos. O~dinheiro -- ou
melhor, o movimento tautológico do dinheiro, o capital -- impõe"-se como
o grande mediador narrativo, como se, ao fazer com que as personagens
dele dispusessem como se elas girassem ao redor do capital, o dinheiro
dissesse, à imitação de Deus: ``Eu sou aquele que sou'' (\versal{ÊXODO}, 3, 14).
Para Apolón, o homem do subsolo tece os elogios mais pródigos sobre
Liza: ``Ela é tudo!'' Quando um elogio é tecido às costas do elogiado,
geralmente o reconhecimento também se volta para aquele que está tecendo
o elogio. Afinal, Liza é tudo -- e ela acaba de chegar à casa do homem
do subsolo, aquele que, além de querer desanuviar o juízo maldoso de
Apolón sobre a moça, ainda tenta compensar a derrota no duelo com o
criado com uma possível elevação de sua virilidade. Como Apolón, com o
ímpeto legítimo da revanche por parte daquele que tivera o salário
atrasado, não vai à taverna imediatamente para comprar as iguarias com
as quais o homem do subsolo pretende agradar Liza, o contato entre o
anfitrião e a visitante já começa mediado pela humilhação -- apenas um
dos conteúdos da forma ubíqua que o capital assume para nos submeter.

\begin{quote}
− Você não sabe, Liza, o que este carrasco é para mim. É~o meu
carrasco\ldots Foi agora comprar torradas; ele\ldots

E, de repente, eu me desfiz em lágrimas. Era uma crise. Tinha tanta
vergonha, em meio aos soluços, mas não podia mais contê"-los (\versal{IDEM}, p.
135).
\end{quote}

A vergonha, verdadeira sombra de nosso protagonista, também depara com
seus próprios limites. Vá lá que a corja pequeno"-burguesa de Zvierkóv o
humilhe. É~dificílimo aceitar que um criado como Apolón assuma as rédeas
da situação. Mas a humilhação diante de Liza assume o ápice do arrivismo
para nosso (anti-)herói. Ocorre que, paradoxal e dostoievskianamente,
como logo veremos, Liza é a única personagem de \emph{Memórias do
subsolo} que, a bem dizer, estende o punho para o afago, e não para o
golpe, para além do horizonte do cálculo utilitário.

``− Liza, você me despreza? -- disse eu, olhando"-a fixamente,
\emph{tremendo de impaciência por saber o que ela pensava.} (\ldots) Eu
estava enraivecido contra mim mesmo, mas, naturalmente, ela é que devia
sofrer as consequências'' (\versal{IDEM}, p. 136).

O homem do subsolo, corroído pela humilhação, promete a si mesmo que não
dirá mais nada. Liza, sem saber o que fazer, tenta retomar o fio da
meada que a levara do prostíbulo à casa de nosso (anti-)herói. Ela diz
que quer abandonar tudo aquilo, mas nosso protagonista permanece
impassível. ``− Eu não vim estorvá"-lo? -- insinuou ela com timidez,
quase imperceptivelmente, e começou a levantar"-se'' (\versal{IDEM}, p. 137). É
demais para o nosso arrivista do subsolo: já não basta o criado ser
senhor de si, agora é preciso suportar mais uma ``explosão de dignidade
ofendida'' (\versal{IBIDEM})? O~homem do subsolo perde as estribeiras, ele diz
que Liza só viera à sua casa para ouvir mais palavras piedosas e
conselhos, ela provavelmente queria receber mais uma injeção de ânimo.
``Pois saiba, saiba de uma vez, que eu então estava rindo de você''
(\versal{IBIDEM}). Ao mesmo tempo em que passa a flagelar Liza, o homem do
subsolo, capaz de franqueza e sinceridade apenas nos momentos mais
escatológicos -- como se a verdade apenas pudesse ser vivenciada como
algo intersticial e fugidio --, o protagonista nos fornece mais uma de
suas várias chaves hermenêuticas que, a partir do conteúdo de sua voz,
delineia a ciranda tautológica de equiparações/duelos como a forma do
subsolo, o anfiteatro em que é orquestrado o concerto
dialético"-polifônico:

\begin{quote}
Eu tinha sido ofendido, {[}Liza,{]} ao jantar, pelos que estiveram
naquela casa antes de mim. Fui até lá para espancar um deles, um
oficial; mas não deu certo, não o encontrei; tinha que desabafar sobre
alguém o meu despeito, tomar o que era meu; apareceu você, e eu
descarreguei sobre você todo o meu rancor, zombei de você.
Humilharam"-me, e eu também queria humilhar; amassaram"-me como um trapo,
e eu também quis mostrar que podia mandar\ldots Eis o que aconteceu; e você
pensou que eu fui para lá de propósito para salvá"-la, não? (\ldots) Salvar!
(\ldots) Salvar do quê? Mas eu mesmo talvez seja pior que você. Por que não
me lançou então às fuças, quando eu lhe fui pregar sermão: ``E você
mesmo, por que veio à nossa casa? Veio pregar moral?'' Eu precisava
então ter poder, precisava de um jogo, precisava conseguir as suas
lágrimas, a sua humilhação, a sua histeria. (\ldots) \emph{Mas eu próprio
não suportei isto, porque sou um crápula; assustei"-me e, o diabo sabe
para quê, dei de bobo o meu endereço a você.} Em seguida, ainda antes de
chegar em casa, já eu a xingava a mais não poder. Justamente por causa
desse endereço {[}grifos meus{]} (\versal{IDEM}, pp. 137-138).
\end{quote}

Não temos quaisquer reparações a fazer em relação à lucidez (momentânea)
do homem do subsolo. O~paroxismo da humilhação e do revanchismo faz com
que a personagem narre o trecho sem qualquer desfaçatez -- ou melhor,
quase sem qualquer desfaçatez. O~trecho acima grifado diz mais sobre o
que omite do que sobre aquilo que explicita. É~apenas uma meia verdade
-- meia verdade a esconder uma verdade e meia -- o fato de o homem do
subsolo não ter suportado toda aquela humilhação e o posterior jogo com
Liza porque, em suas palavras, ``eu sou um crápula''. Na verdade,
crápula rematado nosso (anti-)herói seria se já não se sentisse
incomodado com a reificação, se a deformação das relações
intersubjetivas já se lhe impusesse como a segunda natureza da
realidade. Não foi por mero susto que o homem do subsolo deu seu
endereço para Liza, não -- foi por compaixão e solidariedade, porque ela
também é uma humilhada e ofendida, porque ela ainda resguarda um ímpeto
de superação em relação ao atual estado de coisas, ímpeto que a
reificação do subsolo faz questão de deformar, mas cujo teor de verdade
não pôde ser de todo extirpado. Só que, como já dissemos em nossa
análise, o homem do subsolo, apesar de sentir os laivos da verdade,
apesar de nutrir o ímpeto por comunhão, já não consegue acreditar na
humanidade da amizade. Ele já não se sente apto para a vida pulsante, a
vida viva, a vida que o subsolo aguilhoou. Assim, a sinceridade e a
franqueza equivalem a uma despedida do convívio da pessoa que não pode
ser sua companheira, mas apenas uma cúmplice -- ou pior, uma potencial
delatora. É~por isso que o homem do subsolo não perdoa Liza por tê"-lo
visto chorar, por ter conhecido a pobreza em meio à qual ele sobrevive,
por tê"-lo descoberto como um lacaio diante de Apolón, por tê"-lo visto
humilhado e ofendido. Ora, para alguém acostumado às relações humanas
como duelos inequívocos, nosso narrador"-personagem só poderia esperar
que Liza, ao ser humilhada e ao deparar \emph{in toto} com a condição
lamentável do subsolo, fosse embora sem mais. Afinal, se tudo é ditado
por amor próprio e utilitarismo, \emph{para que} Liza suportaria tudo
aquilo? Ocorre que, como um movimento em falso na dinâmica reificada do
subsolo -- como um atrito em demasia a emperrar suas engrenagens --, eis
que algo de inusitado despontou:

\begin{quote}
Mas aí deu"-se um fato estranho. (\ldots) Ofendida e esmagada por mim, Liza
compreendera muito mais do que eu imaginara. Ela compreendera de tudo
aquilo justamente o que a mulher sempre compreende em primeiro lugar,
quando ama sinceramente, isto é, compreendera que eu mesmo era infeliz.
(\ldots) E, quando eu comecei a chamar"-me de canalha e crápula, quando me
correram lágrimas (\ldots), todo o seu rosto se contorceu em não sei que
convulsão. Quis levantar"-se, deter"-me; e, quando acabei de falar, não
foi para os meus gritos que ela dirigiu atenção (\ldots), mas para o fato
de que, provavelmente, era muito difícil a mim mesmo dizer tudo aquilo.
(\ldots) Súbito, pulou da cadeira num repente insopitável e, querendo
atirar"-se toda para mim, mas ainda tímida e não ousando sair do lugar,
estendeu"-me os braços\ldots Nesse ponto, o meu coração também se
confrangeu. E~ela se lançou subitamente a mim, rodeou"-me o pescoço com
os braços e chorou. Eu também não resisti e chorei aos soluços, de modo
como nunca ainda me acontecera\ldots

− Não me deixam\ldots Eu não posso ser\ldots bondoso! -- mal proferi; em
seguida fui até o divã, caí nele de bruços e passei um quarto de hora
soluçando, presa de um verdadeiro acesso de histeria. Ela deixou"-se cair
junto a mim, abraçou"-me e pareceu petrificar"-se naquele abraço (\versal{IDEM},
pp. 139-140).
\end{quote}

Eis o momento mais lírico e emancipatório de \emph{Memórias do subsolo},
momento que dá vazão, literal e literariamente, à noção de que o leitor
procura aquecer sua vida enregelada por meio do sopro de Pandora que a
obra de arte lhe traz. Liza não é adepta do evangelho segundo Talião, a
jovem não retribui as humilhações sofridas com novos insultos, já que,
efetivamente, ela se dispôs a oferecer a outra face para o homem do
subsolo. Ela se apieda de nosso (anti-)herói com tanto mais ímpeto de
comunhão quanto mais ele se flagela e vai erigindo e desconstruindo as
muralhas de sua solidão. Quando os dois se abraçam e choram
conjuntamente, vem à tona o sentido mais pungente e utópico da
\emph{compaixão}, isto é, do \emph{pathos conjunto}, do \emph{sofrimento
irmanado.} ``Porque onde dois ou três estão reunidos em meu nome, aí
estou eu no meio deles'' (\versal{MATEUS}, 18, 20). Depois disso, quando o homem
do subsolo afrouxa os aguilhões de Pandora e exala que não o deixam e
que ele também não consegue ser bondoso, chegamos à fronteira da
expressão de um ímpeto por síntese e reconciliação, um ímpeto por um
mundo inteiramente outro. Narrá"-lo \emph{hic et nunc}, quando a
reificação relega a bondade a um momento inusitadíssimo de ruptura e
comunhão, significa se resignar diante da contingência da caridade. E~é
por isso que, a despeito da luz cicatrizante que chega a banhar o
subsolo, precisamos extrair a decorrência dialeticamente trágica da
colocação de Walter Benjamin a respeito do leitor e das leituras que
buscam a hipostasia da reconciliação histórico"-pessoal por meio da
Utópolis ficcional: ``O romance não pode alimentar a esperança de dar o
mínimo passo além daquele limite em que, convidando o leitor a captar
intuitivamente o sentido da vida, convida"-o também a escrever um `Finis'
embaixo da última página'' (1980b, p. 68).

Quando, em meio ao choro e ao ranger de dentes, o homem do subsolo se
pergunta ``Meu Deus! Será possível que a tenha invejado?'' (\versal{DOSTOIÉVSKI},
2004, p. 141) -- isto é, será possível que a pobre Liza \emph{lhe seja
superior?} --, a deformação da poética dostoievskiana arrasta ainda uma
vez nosso (anti-)herói para o cárcere de seu subsolo. Afinal, como
entender, ou melhor, como vivenciar um amor que apresenta laivos de
incondicionalidade? Ora, em verdade, em verdade o homem do subsolo nos
diz que

\begin{quote}
o amor consiste justamente no direito que o objeto amado voluntariamente
nos concede de exercer tirania sobre ele. Mesmo nos meus devaneios
subterrâneos, nunca pude conceber o amor senão como uma luta: começava
sempre pelo ódio e terminava pela subjugação moral; depois não podia
sequer imaginar o que fazer com o objeto subjugado (\versal{IDEM}, p. 142).
\end{quote}

Sócrates já nos dissera que os cativos da caverna acabam por se afeiçoar
às sombras e aos aguilhões. Se algo além da penumbra irreconciliada
desponta, o homem do subsolo empunha as correntes como se elas já
houvessem se tornado suas grandes companheiras. Assim, como arremate de
nossa análise -- como término irresoluto porque não reconciliado --, o
homem do subsolo estanca as lágrimas e volta a ficar em silêncio. O
silêncio que prenuncia o pus.

\begin{quote}
Um quarto de hora depois, eu estava andando a passos largos, numa
impaciência furiosa, de um canto a outro do quarto, e a cada instante
acercava"-me do biombo e espiava Liza por uma pequena fresta. Ela estava
sentada no chão, a cabeça reclinada sobre a cama e, provavelmente,
chorava. (\ldots) Ela adivinhara que o arroubo da minha paixão fora
justamente uma vingança, uma nova humilhação, e que ao meu ódio de
antes, quase sem objeto, se acrescentara já um ódio \emph{pessoal,
invejoso,} um ódio por ela\ldots (\versal{IDEM}, p. 141).
\end{quote}

Desde o princípio de nossa análise acompanhamos o solipsismo relacional
daquele que só faz julgar os outros por si mesmo -- daquele que só faz
equiparar os outros a si mesmo. Com a subsunção de Liza pela projeção
solipsista -- nosso narrador cínico e cáustico sequer se incomoda em
dizer que ``não afirmo que ela compreendesse tudo isto com nitidez''
(\versal{IBIDEM}) --, o tiro de misericórdia está a um passo de ser desferido.

\begin{quote}
Passaram"-se mais alguns minutos, e ela ainda não se levantara, como se
estivesse esquecida de si mesma. \emph{Tive o descaramento de bater
devagarinho no biombo, para lembrar\ldots} De repente, sobressaltou"-se,
ergueu"-se e começou a procurar o seu lenço, o chapeuzinho, a peliça,
\emph{como se quisesse escapar de mim\ldots} Dois minutos depois, saiu
vagarosamente de trás do biombo e me dirigiu um olhar penoso. Sorri com
maldade, aliás à força, por uma questão de decência, e virei"-me,
evitando"-lhe o olhar.

− Adeus -- disse ela, encaminhando"-se para a porta.

De repente, corri até ela, agarrei"-lhe a mão, abri"-a, \emph{coloquei
ali\ldots} e tornei a fechá"-la. E, no mesmo instante, me virei e corri o
quanto antes para o outro canto a fim de não ver, pelo menos\ldots
{[}grifos meus{]} (\versal{IDEM}, p. 143)
\end{quote}

``Tive o descaramento de bater devagarinho no biombo, para lembrar'':
nosso protagonista canalha, o polo relativo da relação de equiparação,
força Liza como polo equivalente a sentir o quanto a permanência de sua
invasão ao subsolo é indesejada e indesejável. Liza começa a juntar suas
coisas, já a ocupar o polo relativo da relação subsequente, ``como se
quisesse escapar de mim'' -- nosso lobo volta a vestir a pele de
cordeiro. Mas, se o cume da beleza em \emph{Memórias do subsolo}
despontou há pouco, é preciso que a mais sórdida das ações lhe faça
frente como uma antítese inescapável. É~assim que, diante da retomada do
amor próprio de Liza a dizer ``adeus'', o homem do subsolo vai até ela
e, com a lâmina da reificação a rasgar a jovem e a si mesmo -- e, a bem
dizer, a todos nós --, a personagem reproduz a quintessência da
\emph{mais"-valia como menos"-valia} ao enxotar Liza com a mediação de 5
rublos: eis a amortização da dívida pelo serviço da compaixão que lhe
fora prestado.

O homem do subsolo não quis ver o desenlace de sua brutalidade. Ele não
viu que Liza, com toda a altivez e dignidade, amassou a nota azul de 5
rublos e a jogou sobre o tampo da mesa como que a afirmar, ainda que
contingencialmente, a mulher para muito além da prostituta.

Se somarmos a menos"-valia do ódio, dos ressentimentos e das humilhações
como extrações imateriais de mais"-valia, a figura monetária dos 30
rublos passa a girar ao redor de si mesma e a nos fazer girar a seu
redor. Os~criadores se prostram em oração diante da criatura que lhes
drena as energias. O~Sol volta ao girar ao redor da Terra, Ptolomeu
volta a suplantar Copérnico.

Chegamos, então, ao término da escavação de \emph{Memórias do subsolo.}
``Aliás, ainda não terminam aqui as `memórias' deste paradoxalista. Ele
não se conteve e as continuou. Mas parece"-nos que se pode fazer ponto
final aqui mesmo'' (\versal{IDEM}, p. 147). Já havíamos citado o trecho final da
obra, e agora, até mais do que em momentos anteriores, não nos deve
causar surpresa que, além de terminar de maneira irresoluta,
\emph{Memórias do subsolo}, em seu desfecho aberto, nos forneça mais uma
chave hermenêutica a imbricar o movimento tautológico da forma à
sujeição do conteúdo.

Se Dostoiévski fosse, de forma unívoca, um rematado niilista, a tarefa
deste livro estaria terminada. Em diálogo com Bakhtin, restabelecemos as
mediações entre dialogia e dialética com vistas à totalização
contraditória do concerto polifônico. Ao longo de tal processo,
descobrimos que a forma dostoievskiana, por meio da imanência de seu
movimento, guarda um profundo e \emph{subterrâneo diálogo} com o devir
fetichista da forma mercadoria até sua constituição como D -- M -- D',
isto é, até sua movimentação tautológica como capital. Ocorre que,
dialeticamente, as vozes dostoievskianas se rebelam contra o patíbulo
das cordas vocais. O~conteúdo como utopia irrompe contra o fetichismo da
forma.

Nesse sentido, como já disséramos no preâmbulo deste livro, os três
próximos capítulos, isto é, a segunda parte de nossa análise a se
configurar como uma antítese em relação à tese desta primeira parte,
passarão a investigar os sentidos (e os ressentimentos) da filosofia da
história em Dostoiévski, de modo que nossa investigação, como
totalidade, também se estruture contraditoriamente, isto é, para que as
tomadas de posição de nossas análises reverberem a dialética polifônica
fundamental à obra de Dostoiévski e às suas imbricações com o movimento
da história.

\part[Parte \versal{II} -- Antítese]
{Antítese\\[\bigskipamount] 
      \large O conteúdo em Dostoiévski como a cicatrização do espírito rumo à utopia?}

\vspace*{\fill}

\, \ 
\begin{minipage}{0.84\textwidth}
\scriptsize\emph{E clamavam em alta voz, dizendo: ``Até quando tu, que és o Senhor, o
Santo, o Verdadeiro, ficarás sem fazer justiça e sem vingar o nosso
sangue contra os habitantes da terra?}

\smallskip
\hspace*{\fill}-- \emph{Apocalipse}, 6, 10
\end{minipage}

\bigskip

\, \ 
\begin{minipage}{0.84\textwidth}
\scriptsize\emph{O ateísmo completo está no penúltimo degrau da fé mais perfeita (se
subirá esse degrau já é outra história).}

\smallskip
\hspace*{\fill}-- Clérigo Tíkhon, \emph{Os demônios}\footnotemark
\end{minipage}
\footnotetext{Tradução de Paulo Bezerra.
  São Paulo: Editora 34, 2004, p. 662.}

\bigskip

\, \ 
\begin{minipage}{0.84\textwidth}
\scriptsize\emph{Eu vi a verdade, eu vi e sei que as pessoas podem ser belas e felizes,
sem perder a capacidade de viver na terra. Não quero e não posso
acreditar que o mal seja o estado normal dos homens.}

\smallskip
\hspace*{\fill}-- Homem ridículo, ``O sonho de um homem ridículo''\footnotemark
\end{minipage}
\footnotetext{Tradução
de Vadim Nikitin. São Paulo: Editora 34, 2003, pp. 121-122.}

\thispagestyle{empty}

\chapter*{Capítulo 3\\
\bigskip
\emph{O prenúncio da teologia como teleologia -- e da teleologia como teologia}}

\addcontentsline{toc}{chapter}{Capítulo 3\\\scriptsize{\emph{O prenúncio da teologia como teleologia -- e da teleologia como teologia}}}
\hedramarkboth{Capítulo 3}{}

\section{3.1. Preâmbulo}

Genebra, 29 de setembro de 1867: em uma carta para a sobrinha Sofia
Aleksandrovna, Dostoiévski pergunta se ``você lê alguns jornais? Por
favor, faça isso! Hoje em dia eles precisam ser lidos, não apenas por
ser moda, mas \emph{para que se possam traçar as conexões evidentes
entre os pequenos e os grandes eventos} {[}grifo meu{]}'' (2011, p.
128). Serguei Hackel (1982) argumenta que ``a finalidade da arte de
Dostoiévski não é retratar os eventos da vida cotidiana, mas sua ideia
geral, a qual é entrevista de forma aguda e extraída corretamente a
partir da multiplicidade total dos fenômenos interrelacionados da vida''
(p. 7). Os \emph{pequenos e os} \emph{grandes eventos}, assim, estariam
vinculados ao movimento da totalidade, ao espírito de época:
perspectivas de que Dostoiévski jamais abriria mão; eis um dos sentidos
de sua obra literária.

O episódico e o contingente, como podemos depreender, não eram
desprovidos de mediações para revelar ao escritor uma possível filosofia
da história, suas prováveis leis de desenvolvimento, as contradições em
que o todo se enreda. Seria possível dizer que a totalidade em
Dostoiévski deve ser inicialmente entrevista como a quintessência dos
momentos parciais que apontam para além de si mesmos\footnote{A frase em
  questão estabelece um diálogo dialético com a seguinte colocação de
  Vladimir Safatle no prefácio para a obra \emph{Três Estudos sobre
  Hegel} (São Paulo: Editora Unesp, 2013), de Theodor Adorno: ``Adorno
  insiste que a totalidade em Hegel deve ser inicialmente vista como a
  quintessência dos momentos parciais que apontam para além de si
  mesmos'' (pp. 25-26). Como ficará claro ao longo deste capítulo e,
  sobretudo, nos capítulos 4 e 5, a mediação dialética entre parte e
  todo, fundamental na filosofia de Hegel, também se faz essencial na
  filosofia da história de Dostoiévski.}. Nesse sentido, Kate Holland
(2000) afirma que

\begin{quote}
nos escritos de Dostoiévski sobre a expressão artística da realidade,
nós deparamos recorrentemente com uma situação ambígua: os fragmentos
dissonantes parecem resistir ao poder unificador e totalizante da arte,
mas, ao mesmo tempo, a arte atua como uma espécie de filtro, por meio do
qual os fragmentos podem ser transfigurados e conduzidos a um novo tipo
de toalidade (p. 95).
\end{quote}

Quando Holland menciona a arte como um possível filtro para a apreensão
dos fragmentos que, mediados esteticamente, se interrelacionariam em um
novo tipo de totalidade, poderíamos pensar sobre a noção adorniana do
artista como mediação subjetivo"-objetiva para a (res)significação do
contexto histórico por meio da imanência da constituição da obra de
arte\footnote{Adorno desenvolve tal ideia tanto na \emph{Teoria
  estética} quanto no ensaio ``O artista como representante'' (In:
  \emph{Notas de Literatura I}. Tradução de Jorge de Almeida. São Paulo:
  Editora 34, 2003), em diálogo com a obra de Paul Valéry. Para Adorno,
  a postura de Valéry, além de apreender, pela imanência da obra, um
  possível sentido de época, despontaria como um \emph{locus} de
  reconciliação em meio à sociedade não reconciliada, o negativo (a
  negação da negação) em meio à reificação, já que a feitura artística
  de Valéry conteria um princípio de ação social racionalmente
  concatenada a apontar para a utopia, para uma sociedade outra e
  emancipada.}. Dostoiévski não apenas se defronta com o material -- o
episódico jornalístico --, como se propõe a estabelecer as mediações
para que o supostamente contingente e disperso possibilite uma leitura
do espírito de sua época. O~sujeito se objetiva ao se confundir com a
lógica do material, ao mesmo tempo em que a lógica do material é
subjetivamente mediada pelo artista que se torna um sujeito social. A
história e sua movimentação como que passam através de sua consciência.
Mas, aqui, é importante frisar que o motor artístico não é a Ideia, mas
o caráter relacional da consciência do artista que deixa de ser uma
mônada subjetiva para se objetivar em meio à realidade histórica por
conta de sua imbricação com o material.

O contingente, então, jamais eclodiria como algo fragmentado e
desprovido de sentido -- algo em si e a esmo que tanto poderia ter
ocorrido aqui quanto ali. Dostoiévski aponta um caráter de
\emph{necessidade} lógico"-histórica que, mediatamente, conectaria o todo
às partes.

Florença, 26 de fevereiro de 1869: em uma carta para o amigo Nikólai
Nikoláievitch Strákhov, Dostoiévski afirma que

\begin{quote}
tenho minha própria ideia de arte: \emph{o que a maioria das pessoas
entende como fantástico ou como falta de universalidade}, eu tomo por
algo próximo à \emph{suprema essência da verdade. Há muito deixei de ver
as áridas observações de trivialidades cotidianas como realismo -- é bem
o oposto.} Em qualquer jornal encontramos reportagens sobre fatos
totalmente autênticos, mas que alguns veem como fora do comum. Nossos
escritores os veem como fantásticos e, por isso, não prestam nenhuma
atenção a eles. \emph{E eles são verdadeiros, pois são fatos. Mas quem
se preocupa em observá"-los, gravá"-los, descrevê"-los? Acontecem todos os
dias e a todo o momento, logo não são nem um pouco excepcionais}
{[}grifos meus{]} (2011, p. 156).
\end{quote}

Discutiremos, neste e no próximo capítulo, o que poderia ser entendido
como um dos matizes da \emph{suprema essência da verdade} para
Dostoiévski. De qualquer modo, observamos que o escritor parte da
empiria factual para desvelar e desdobrar sentidos para além da
micrologia cotidiana. A~factualidade faria parte de uma rede
histórico"-conceitual que vincularia a parte ao todo. O~inusitado, então,
não seria meramente contingente, mas poderia representar o desdobramento
\emph{necessário}, uma vez que factual, de determinado sentido histórico
até então recôndito. Dostoiévski toma tal revelação como um importante
mister de sua arte.

A finalidade do método se confunde com o método como finalidade quando
pensamos em Dostoiévski como um leitor do devir histórico de sua época.
Ora, se falamos em devir, transportamos os desdobramentos das ideias do
escritor para a nossa própria época, de modo que eles sejam lógica e
historicamente reconfigurados. A~compilação do episódico e contingente
faria parte do projeto artístico de Dostoiévski que via a parte como a
fenomenologia do todo, a aparição concreta da lógica histórica. Como
artista, como sujeito que se funde ao devir do objeto e se torna
mediação social individualmente configurada, Dostoiévski criava

\begin{quote}
através da dramatização imaginativa dos limites absolutos de uma ideia e
mediante a percepção intuitiva do comportamento humano concreto e
apropriado para viver nessas situações ``fantásticas'' (embora
perfeitamente verossímeis). Essa tendência crescente àquilo que se pode
chamar de escatologia ideológica, conjugada com o gênio psicológico que
Dostoiévski já demonstrara de forma ampla, abrir"-lhe"-á o caminho para as
grandes obras (\versal{FRANK}, 2002, p. 345)\footnote{Victor Terras (1985), em
  uma linha argumentativa próxima a Joseph Frank, também discorreu sobre
  o sentido artístico da escatologia em Dostoiévki: ``A acusação de que
  os romances de Dostoiévski têm traços góticos, retratando paixões
  viscerais ou perversas, intrigas, assassinatos, suicídios etc. é, de
  fato, válida. A~resposta de Dostoiévski a tais críticas era que os
  extremos seriam mais reveladores da essência da condição humana do que
  a chamada `média'. Esta é uma questão fundamental que fazia
  Dostoiévski discordar da maioria de seus contemporâneos. (\ldots) A~força
  de Dostoiévski reside precisamente nas crises da vida humana, que,
  mesmo sendo raras, ainda assim são reais; o ímpeto do escritor não se
  voltava para a vida cotidiana: namoros e casamentos, pessoas ganhando
  o pão e formando famílias etc. Aquilo a que se atribui maior
  importância depende da \emph{Weltanschaaung} {[}cosmovisão{]} do
  autor'' (p. 166).}.
\end{quote}

Imaginação, aqui, é o encontro de Bakhtin e Adorno, da dialogia com a
dialética, do sujeito e do objeto não mais como entes, mas como polos,
como o diálogo dialético, como o movimento da relação. A~imaginação
torna"-se a imbricação com o movimento da história, com suas
(im)possibilidades lógicas. Como sujeito social, Dostoiévski já não
pensa a contradição de fora; ao mergulhar sua consciência criativa no
turbilhão da Ideia como devir histórico, a contradição é o motor de sua
causalidade artística, da constituição de suas obras, do duelo que
imbrica forma e conteúdo. Joseph Frank, munido de laivos idealistas,
emprega o termo \emph{percepção intuitiva} para designar o mergulho
objetivo de Dostoiévski na configuração histórica. Os pressupostos deste
livro tornam \emph{necessário} o emprego dos termos \emph{projeção} e
\emph{prospecção}; se as ideias geridas pela história têm uma logicidade
imanente, a razão subjetiva (e socializada) pode captar o sentido de seu
desenvolvimento ao buscar as pistas da racionalidade objetiva (o devir
da história) a partir de seus vestígios -- o contingente como aparência
e motor para a essência da história, seu movimento, seus desdobramentos.

Ao invés de utilizarmos a noção bastante questionável de que Dostoiévski
era um ``profeta'' da história -- noção que poderia, \emph{a priori},
ser derivada das noções de \emph{projeção} e \emph{prospecção} que ora
utilizamos --, podemos falar, ao lado de Theodor Adorno, sobre o
conceito de \emph{possibilidade objetiva}, isto é, sobre as derivações
históricas prováveis, mas ainda não aparentes e concretas, algo que
\emph{pode} vir à tona, ou seja, que apresenta o \emph{potencial de}
\emph{necessidade} em meio à fenomenologia da história. Adorno (2013)
rechaça a profecia que paira sobre a ou que rompe com a logicidade da
história e discorre a respeito da noção de possibilidade objetiva, em
Hegel, por meio de termos que nos ajudam a compreender a
\emph{racionalidade histórico-``profética''} de Dostoiévski:

\begin{quote}
Conforme sua própria ideologia, e como um empregado tolerado por
interesses mais poderosos, Hitler tentou erradicar o bolchevismo,
enquanto sua guerra trouxe a gigantesca sombra do mundo eslavo sobre a
Europa, aquele mundo eslavo do qual Hegel já dizia de forma premonitória
que ainda não havia entrado na história. No entanto, \emph{Hegel não foi
habilitado para isso por meio de um olhar histórico profético, pelo qual
ele nutria apenas desprezo, mas por meio daquela força construtiva que
penetra inteiramente no que é, sem sacrificar a si mesma como razão,
crítica e consciência da possibilidade} {[}grifos meus{]} (p. 81).
\end{quote}

É bem verdade que o próprio Dostoiévski fez coro à deificação do
profeta, mas podemos entender o seguinte fragmento do escritor, citado
por Sergei Hackel (1982), como uma versão dostoievskiana da noção de
possibilidade objetiva aberta pela história: ``a realidade como um todo
está longe de ser exaurida pelo cotidiano, porque ela existe, em grande
medida, na forma de uma palavra latente e a ser expressa. De tempos em
tempos, surgem profetas que adivinham e expressam a palavra integral''
(p. 10). A~parte latente aponta para além de si mesma -- eis a mediação
dostoievskiana da parte que deixa entrever o todo.

É nesse sentido que Robert"-Louis Jackson (1981), cem anos após a morte
do escritor russo, cita um fragmento do \emph{Diário de um Escritor}, de
Dostoiévski, que data de 1877:

A meu ver, o presente período também terminará na velha Europa com algo
colossal, i.e., talvez não literalmente idêntico aos eventos que deram
cabo do século \versal{XVIII}, mas, ainda assim, será algo gigantesco,
fundamental e terrível -- e também algo que trará uma mudança para o
mundo todo, ou, ao menos, no Ocidente, na velha Europa.

Logo em seguida, Jackson comenta que

\begin{quote}
as palavras de Dostoiévski são impressionantes; foram necessários cem
anos para que conseguíssemos apreendê"-las integralmente. {[}Jackson
sobrevoa o processo de eclosão da Revolução Russa, das duas guerras
mundiais e da Guerra Fria, que ainda despontava com vigor em 1981.{]}
(\ldots) É~claro que nós não vamos ressuscitar a imagem desgastada de
Dostoiévski como um ``profeta'': muito do que ele anteviu de fato
aconteceu; muito do que ele esperava não veio à tona; e o que ocorreu
nem sempre aconteceu quando ou onde ele esperava. A~Rússia, a Europa e a
Ásia não seguiram uma trajetória precisa ou prescrita, e nós faríamos
mal em buscar na obra de Dostoiévski ou de Marx esquemas e previsões
exatos, isto é, apreender suas colocações como algo além de
impressionantes metáforas que prefiguram insurreições iminentes, caos e
transformações -- muito daquilo que Nietzsche, mais tarde, chamaria de
``a crise do niilismo europeu'' (1985, p. 4).
\end{quote}

O próprio Dostoiévski, em breve preâmbulo às \emph{Memórias do subsolo},
procurou ilustrar a maneira pela qual sua arte (tese) se enredava à
história (antítese) para que a \emph{realidade ficcional} despontasse
como síntese.

\begin{quote}
Tanto o autor como o texto destas memórias são, naturalmente,
imaginários. Todavia, \emph{pessoas como o seu autor não só podem, mas
devem até existir em nossa sociedade, desde que consideremos as
circunstâncias em que, de um modo geral, ela se formou} {[}grifo meu{]}
(2000a, p. 14).
\end{quote}

Partimos do pressuposto -- e procuraremos prová"-lo neste e nos próximos
capítulos por meio da imbricação com a logicidade do material
dostoievskiano -- de que, para o escritor, a história tinha uma
finalidade, um sentido -- e uma série de ressentimentos, caso seu
movimento fosse arregimentado por cálculos e construções que procurassem
se contrapor à \emph{natureza viva} do homem, de suas capacidades e
limitações. A~discussão fundamental de Dostoiévski sobre a filosofia da
história se dá em proximidade com os e em contraposição aos socialistas,
a \emph{intelligentsia} russa que antecedeu a geração de Lênin, Trótski
e Stálin. O~próprio conceito teológico de \emph{natureza viva} se
confunde com a teleologia histórica para Dostoiévski. Assim, cabe a este
capítulo desenvolver as bases da discussão mundana de Dostoiévski, isto
é, suas argumentações que não espreitaram em si e por si mesmas o
movimento da história como um momento da eternidade, mas que se
enredaram aos objetos e objetivos contingentes daquele determinado
período histórico para que o escritor pudesse apresentar as aporias da
utopia socialista.

Neste momento, lançaremos mão de fragmentos das correspondências do
escritor\footnote{\emph{Dostoiévski: correspondências, 1838-1880}.
  Tradução de Robertson Frizero. Porto Alegre: 8Inverso, 2011.}, de
\emph{Recordações da casa dos mortos}, \emph{Memórias do subsolo},
\emph{Crime e castigo}, \emph{O idiota} e \emph{Os demônios} para
acompanharmos o devir das ideias através das obras como progressão e
projeção. No capítulo seguinte, iremos a dois capítulos fundamentais de
\emph{Os irmãos Karamázov -- ``}A revolta'' e ``O grande inquisidor''
--, para alcançarmos as decorrências da teleologia teológica de
Dostoiévski no além"-mundo -- e, então, suas possíveis repercussões para
este mundo.

Construiremos, então, a partir da dialogia dialética de Dostoiévski, um
novo diálogo entre as obras, de modo que os objetos, ao falarem entre
si, nos digam mais sobre si mesmos e sobre os conceitos que desenvolvem.

\section{3.2. O cálculo utilitário à luz e às sombras da história}

Partimos da morte de Deus, o cataclismo histórico que Nietzsche chamou
de crepúsculo dos ídolos. Escatologicamente, o desespero epiléptico de
Dostoiévski diante da possível inexistência de Deus o levou a imaginar o
devir de um mundo que derrogasse o decálogo de Moisés\footnote{Neste
  momento, não estamos apenas diante da lúgubre poética de Dostoiévski a
  exacerbar os desdobramentos possíveis de determinada ideia. Joseph
  Frank nos conta como o escritor foi tomado por profunda angústia ao
  deparar com o quadro \emph{Cristo morto} (1521-1522), de Hans Holbein,
  o Moço, exposto no Museu da Basileia. (Há uma réplica em forma de
  escultura do quadro de Holbein na igreja do Hotel dos Inválidos, em
  Paris, igreja que antecede a câmara onde se encontra o sarcófago de
  Napoleão I.) Quando deparei com o \emph{Cristo morto}, me lembrei
  imediatamente dos relatos de Frank sobre as agruras de Dostoiévski ao
  ver representada a derrocada de Cristo após ser retirado da cruz --
  seu corpo esquálido, as marcas dos suplícios que não mais
  cicatrizariam, o crepúsculo da transcendência diante da imanência que
  assassinara Deus. Dostoiévski sofreria uma série de ataques
  epilépticos após essa experiência, fato que nos mostra como o escritor
  se imbricava subjetivamente à objetividade de seu tempo a ponto de
  sofrer, \emph{escatológica} e somaticamente, as decorrências das
  aporias históricas. Para mais detalhes, conferir Joseph Frank,
  \emph{Dostoiévski: os anos milagrosos, 1865-1871}. São Paulo: Edusp,
  2003, pp. 298-300; p. 352; p. 433; e p. 438.}. Assim, a máxima
atribuída a Ivan Karamázov se transformaria em 11º Mandamento: \emph{se
Deus não existe, tudo é permitido}. Serguei Hackel (1982) cita o
Príncipe Míchkin, protagonista de \emph{O idiota:} ``A ideia que liga os
homens entre si já não existe'' (p. 10). Para Dostoiévski, então, o
\emph{não matarás} se tornaria uma mera contingência. Um mundo sem Deus
não apresentaria quaisquer bases para as ações morais. O~torvelinho do
cálculo utilitário e do relativismo ético se instauraria. Então, o
niilismo sentenciaria que tudo o que é o sagrado deve ser profanado, já
que tudo o que é sólido se desmancha no ar\footnote{Há uma verdadeira
  pletora de fontes para a polissêmica discussão sobre o surgimento e os
  desdobramentos do niilismo no século \versal{XIX}. Além de mencionar os tomos
  de Joseph Frank -- verdadeiro \emph{subsolo} histórico"-ideológico para
  este livro \emph{--}, gostaria de citar três outras fontes que parecem
  desdobrar o nada ético dostoievskiano em direções centrífugas. A
  primeira, rente à movimentação de Dostoiévski, é o livro panorâmico --
  mas nem por isso menos preciso e denso -- do historiador polonês
  Andrzej Walicki, \emph{A History of Russian Thought: From
  Enlightenment to Marxism.} Stanford: Stanford University Press, 1979,
  pp. 309-326. A~segunda fonte acompanha o niilismo ético de modo que a
  negatividade possa voltar"-se contra si mesma para que o sentido de
  época das ideias revolucionárias seja apreendido em seu ímpeto de
  (des)construção. Dialogamos, agora, com Marshall Berman e seu
  formidável \emph{Tudo o que é sólido desmancha no ar}. São Paulo:
  Companhia das Letras, 1986, sobretudo com o capítulo I {[}sobre o
  \emph{Fausto}, de Goethe, como o primórdio do niilismo capitalista
  (pp. 37-84){]}, o capítulo \versal{II} {[}sobre o ímpeto construtivo"-destrutivo
  do \emph{Manifesto Comunista}, de Marx e Engels (pp. 85-125){]} e o
  capítulo \versal{III} {[}sobre a modernização e a modernidade de São
  Petersburgo como espaço para confrontos ético"-classistas a reboque,
  sobretudo, mas não só, do homem do subsolo (pp. 167-269){]}. A
  terceira fonte a ser mencionada é o escritor francês Honoré de Balzac,
  cujo imaginário histórico"-escatológico contribuiu sobremaneira para as
  reflexões sobre as rupturas éticas a partir do advento da modernidade
  capitalista e ateia. Cito, aqui, um trecho de \emph{Eugênia Grandet}
  (São Paulo: Abril Cultural, 1971, pp. 106-107) que provavelmente
  tornou ainda mais febril o jovem Dostoiévski que então cursava a
  Academia Militar de São Petersburgo e já aspirava a uma efetiva
  trajetória literária: ``Os avarentos não creem numa vida futura, o
  presente é tudo para eles. Essa reflexão lança uma luz horrível sobre
  a época atual, na qual, mais que em qualquer outro tempo, o dinheiro
  domina as leis, a política e os costumes. Instituições, livros, homens
  e doutrinas, tudo conspira para minar a crença numa vida futura,
  \emph{sobre a qual se apoia o edifício social há 1800 anos} {[}grifo
  meu{]}. Hoje em dia, o esquife é uma transição pouco temida. O~futuro,
  que nos esperava para além do réquiem, transportou"-se para o presente.
  Chegar \emph{per fas et nefas} {[}\emph{pelo lícito e pelo ilícito}{]}
  ao paraíso terrestre do luxo e dos prazeres vãos, petrificar o coração
  e macerar o corpo em busca de posses passageiras, \emph{como outrora
  se sofria o martírio da vida em busca de bens eternos}, eis a ideia
  geral! Ideia aliás inscrita por toda parte, até nas leis, que
  perguntam ao legislador: `Que pagar?', ao invés de: `Que pensas?'
  \emph{Quando essa doutrina tiver passado da burguesia ao povo, que
  será do país?}'' {[}grifos meus{]}. Para uma análise detida das
  relações entre Balzac e Dostoiévski, conferir Joseph Frank,
  \emph{Dostoiévski: as sementes da revolta, 1821-1849}. São Paulo:
  Edusp, 1999, pp. 130-156. (Frank menciona a obra \emph{Balzac and
  Dostoevsky}, do crítico russo Leonid Grossman. London: Ardis, 1995.)}.

Eis a ambiência histórico"-ideológica que nos leva, primeiramente, à
festa de aniversário do Príncipe Míchkin, comemorada em sua
\emph{datcha} -- figuras da nobreza de São Petersburgo se fazem
presentes. Ao longo de toda a obra dostoievskiana, as personagens estão
à iminência da convulsão -- as ideias as extravasam de modo a serem
articuladas junto ao outro, pelo outro, contra o outro. A~dialogia de
Bakhtin bem pôde analisar o caráter de coexistência necessária da
plêiade ideológica das personagens dostoievskianas. No entanto, a
pressuposição da equipolência das vozes, a meu ver, escamoteia o caráter
aporético das contínuas discussões. É~como se as ideias fossem
inexoravelmente arremessadas entre si e contra si mesmas, de modo que
sua logicidade fosse questionada de forma imanente, tendo em vista seus
possíveis desdobramentos e rupturas lógico"-práticos, até que uma
determinada temática via de regra subjacente se apresentasse como a
obsessão dostoievskiana por excelência.

Em meio a tragos de vodca, peixes fritos já nodosos, pepinos em conserva
e conversas laterais, as personagens acabam chegando ao ápice do diálogo
como duelo. Em disputa, o destino da humanidade. O~aniversário do
Príncipe Míchkin, então, transforma"-se em um mote para que o coral de
vozes analise (e contradiga) as (im)possibilidades de haver algum
substrato propriamente humano para a (re)produção da sociedade diante da
derrocada de Deus. O~bufão Liébediev, bem mais do que ébrio, desafia
todos os senhores ateus ali presentes:

Com que os senhores vão salvar o mundo e onde descobriram o seu caminho
normal -- os senhores são homens de ciência, da indústria, das
associações, de pagar salário e tudo o mais? Com quê? Com crédito? O~que
é o crédito? Aonde o crédito vai levá"-los? (2002, p. 417)

Liébediev observa as fundações do cotidiano, seu \emph{subsolo.} Nesse
sentido, as reflexões de Liébediev dialogam com o próprio Dostoiévski,
em uma carta mencionada por Robert Louis"-Jackson (2002): ``Não há
\emph{fundações} para a nossa sociedade, não há princípios estruturantes
para a vida. Uma colossal erosão -- tudo está ruindo, despencando, sendo
negado, como se nada houvesse existido (\ldots)'' (p. 13). Se Deus não
existe e tudo é permitido, a modernidade dá lugar a uma nova hierarquia
no Olimpo. Novos deuses vêm à tona, todos eles regidos pela batuta do
crédito, as rédeas do capital. Ora, que tipo de relações intersubjetivas
se reproduzem em uma sociedade fundada pela maximização das satisfações
pessoais? Em uma sociedade (des)estruturada pela concentração de
riquezas que estimula a escassez de recursos e a consequente luta de
seus agentes pelo quinhão da sobrevivência? Em uma sociedade em meio à
qual a adição para si pressupõe a subtração alheia? Tudo isso, prossegue
Liébediev, ``sem aceitar nenhum fundamento moral além da satisfação do
egoísmo individual e da necessidade material? A~paz universal, a
felicidade universal derivam da necessidade!'' (2002, p. 417) Como a
tese dostoievskiana pressupõe o movimento antitético, o exaltado Gánia
faz as vezes de defensor do utilitarismo:

Sim, porque a necessidade universal de viver, de beber e comer -- enfim,
a mais plena convicção científica de que o senhor não irá satisfazer
essa necessidade sem uma associação universal e nem a solidariedade dos
interesses -- é, parece, um pensamento bastante forte para servir de
ponto de apoio e ``fonte da vida'' para os futuros séculos da
humanidade. (\ldots) E~por acaso é pouco (\ldots) o sentimento da
autopreservação? Porque o sentimento da autopreservação é a lei normal
da humanidade\ldots (2002, p. 417)

Como sabemos que o torvelinho não é estático, que ele não se configura
como um edifício coeso, mas se constrói para em seguida ser destruído,
de modo que os escombros sejam a matéria"-prima para novas
(des)construções, a dialética de Dostoiévski volta a arremessar
Liébediev contra Gánia -- a sequência das obras nos apresenta as
personagens como o devir das ideias contrapostas.

Gánia apresenta Charles Darwin como um fundamento mundano para a
sociedade que prescinde da transcendência e transforma Deus em mais um
de seus nichos de mercado. (Os leitores do século \versal{XXI}, no entanto, já
conhecemos a antítese histórica que transformou o darwinismo social em
lei do mais forte como mote para o neocolonialismo e o genocídio
industriais.) Assim, a escatologia de Dostoiévski, ainda uma vez, leva
às últimas consequências a ideia de que a nova versão guerra de todos
contra todos possa refundar o contrato social. Uma nova voz deve se
enredar a Gánia como desdobramento antípoda de sua lógica: ``Quem lhe
disse isso? -- gritou subitamente Ievguiêni Pávlovitch. -- Essa lei é a
verdade, mas é tão normal quanto a lei da destruição, e talvez da
autodestruição. Porventura só na autopreservação está toda a lei normal
da humanidade?'' (2002, p. 417) Fiódor Dostoiévski desvela a dialética
do esclarecimento. Progresso e regresso como momentos recíprocos (e
concomitantes) da história. Para além do princípio utilitário do prazer:
historicamente acostumados ao discurso da servidão voluntária, os homens
deixariam de ser súditos (\emph{sujets}, \emph{subjects}) para se
tornarem sujeitos (\emph{sujets}, \emph{subjects}) da noite para as
Luzes? Ora, o homem do subsolo, o autonomeado paradoxalista das
\emph{Memórias do subsolo}, bem suspeita de que súdito e sujeito são
duas faces da mesma moeda histórica que torna necessárias tanto a
transformação do mais desenfreado hedonismo em sadismo -- o outro como
instrumento -- quanto a regressão do eu de sujeito a objeto (súdito) da
própria violência socialmente gerida -- o masoquismo como substrato do
sadismo. A~síntese regressiva de tal dialética apresenta a modernidade
capitalista como a gestora do sadomasoquismo\footnote{Vale frisar que
  Dostoiévski entrevê uma coincidência ideológica essencial entre o
  \emph{ethos} da modernidade capitalista e a \emph{intelligentsia}
  russa socialista. O~filósofo inglês Jeremy Bentham (1748-1832)
  estabelece as bases para as ações dos indivíduos a partir do duelo
  encarniçado entre os princípios da dor e do prazer -- a busca pela
  felicidade e a supressão do sofrimento. A~maximização das satisfações
  pessoais eleva"-se como doutrina modernizante, de modo que o
  capitalismo então nascente não mais precisasse se ocupar com
  anacronismos feudais como honra e lealdade e de modo que a burguesia
  não mais se envergonhasse por hastear a bandeira do egoísmo como
  condição humana universal, atemporal e supra"-histórica. Em pleno
  século \versal{XIX}, a Rússia ainda reproduzia sua sociedade sobre bases
  monárquico"-feudais, então os republicanos socialistas se apropriam do
  individualismo metodológico ocidental tanto para questionar as bases
  (supostamente) orgânicas do tsarismo calcado sobre a desigualdade
  estamental quanto para propor uma nova sociedade socialista em que a
  felicidade de um fosse a base para a felicidade de todos. Tendo em
  vista tal síntese antitética, a obra de Dostoiévski também pode ser
  lida como a apreensão e o questionamento das profundas contradições
  que envolvem o transplante de modelos centrais para a periferia do
  capitalismo. O~ornitorrinco russo nos apresenta o mamífero capitalista
  que bota ovos socialistas -- do utilitarismo competitivo seria feita
  tábula rasa do passado para alicerçar a nova sociedade da
  solidariedade. Para discussões mais prolongadas sobre os debates que
  forjaram a \emph{intelligentsia} russa no século \versal{XIX}, conferir Joseph
  Frank, \emph{Dostoiévski: os efeitos da libertação, 1860-1865}. São
  Paulo: Edusp, 2002; Andrzej Walicki, \emph{A History of Russian
  Thought: From Enlightenment to Marxism} (conferir nota 4); e Isaiah
  Berlin, \emph{Pensadores russos}. São Paulo: Companhia das Letras,
  1988.}. Vejamos o que o homem do subsolo tem a dizer a esse respeito:

\begin{quote}
Oh, dizei"-me, quem foi o primeiro a declarar, a proclamar que o homem
comete ignomínias unicamente por desconhecer os seus reais interesses, e
que bastaria instruí"-lo, abrir"-lhe os olhos para os seus verdadeiros e
normais interesses, para que ele imediatamente deixasse de cometer essas
ignomínias e se tornasse, no mesmo instante, bondoso e nobre, porque,
sendo instruído e compreendendo as suas reais vantagens, veria no bem o
seu próprio interesse, e sabe"-se que ninguém é capaz de agir
conscientemente contra ele e, por conseguinte, por assim dizer, por
necessidade, ele passaria a praticar o bem? (\ldots) {[}Mas{]} quando foi
que aconteceu ao homem, em todos esses milênios, agir unicamente em prol
da sua vantagem? (\ldots) E~se porventura acontecer que a vantagem humana,
\emph{alguma vez}, não apenas pode, mas deve até consistir justamente em
que, em certos casos, desejamos para nós mesmos o prejuízo e não a
vantagem? E, se é assim, se pelo menos pode existir tal possibilidade,
toda a regra fica reduzida a nada (2000a, pp. 32-33).
\end{quote}

Desde o capítulo anterior, já sabemos que, em termos ideológicos, o
homem do subsolo ocupa, necessária, alternada e contraditoriamente, as
posições de promotor e advogado de defesa. No entanto, o entrechoque das
ideias nos apresenta as impossibilidades emancipatórias do utilitarismo
socialista à luz das contradições históricas. Não à toa os
revolucionários franceses -- que tanto influenciaram a
\emph{intelligentsia} russa -- pretendiam fazer tábula rasa da história
para fundar a nova era com a égide da Declaração dos Direitos do Homem e
do Cidadão. A~lógica contraditória do homem do subsolo nos demonstra que
a antítese histórica arremete contra a própria tese abstrata dos
\emph{citoyens} franceses: ficaríamos surpresos ao descobrirmos que a
mesma revolução que procurou fazer tábula rasa da história acabou por
dar vazão ao Terror da guilhotina e ao imperialismo de Napoleão
Bonaparte? Nesse sentido, Robert Louis"-Jackson (2002) cita um fragmento
de Dostoiévski que aparece nos rascunhos para o romance \emph{Os
demônios}:

\begin{quote}
``Mas, então, qual será a responsabilidade de vocês pelos rios de sangue
que vocês querem derramar?'' -- seguida de uma resposta obtusa:
``Nenhuma responsabilidade, nós simplesmente ofereceremos as nossas
cabeças {[}em sacrifício{]}. A~futura sociedade será criada após a
destruição universal, e, quanto antes, melhor (p. 14) \footnote{Tanto a
  guilhotina quanto Napoleão e seus desdobramentos históricos rendem
  bons motes dostoievskianos. Ora, não seria a guilhotina a síntese mais
  rematada das relações incestuosas que o homem do subsolo apreende
  entre o utilitarismo e o sadismo? Se compararmos a guilhotina com a
  morte medieval a pauladas e a machadadas, estaremos diante da piedade
  racional: o condenado moderno tem uma morte clínica e indolor.
  Instantânea. A~compaixão minimiza a agonia -- e anula as chances de
  sobrevivência. A~razão (instrumental) não nos livrou do cadafalso. Ela
  aprimorou a decapitação. A~maximização das satisfações pessoais -- as
  \emph{minhas} satisfações, as satisfações da \emph{minha} família, da
  \emph{minha} classe, da \emph{minha} raça, do \emph{meu} país -- faz
  com que o inegável e emancipatório progresso técnico dê guarida,
  reproduza e refine o atavismo histórico. Eis o cálculo utilitário à
  luz e às sombras da história.}.
\end{quote}

\emph{Eu sou aquele que sempre nega.} A~máxima de Mefistófeles, o
espírito malévolo que assola o \emph{Fausto}, de Goethe, bem poderia ser
atribuída ao~\textbf{(}e incorporada pelo) jovem estudante de Direito
Ródion Raskólnikov, protagonista de~\emph{Crime e castigo}. Raskólnikov
é um pobretão que mal consegue custear os estudos -- motivo que, ao fim
e ao cabo, faz com que ele tenha que abandonar a faculdade. O~teto
abaulado de seu quarto, um sótão, verga suas muitas ideias. Mas, para
pensar, é preciso pagar o aluguel há muito em atraso. Ainda assim,
Raskólnikov caminha em meio à febre de suas reflexões. Sim, porque
Napoleão sequer hesitava diante da carnificina de seus soldados nos
campos de batalha. É~como se a história precisasse de cadáveres como
força motriz. Os~gritos lancinantes de dor lubrificam as engrenagens. Os
mesmos gritos que o generalíssimo francês não ouve. Afinal, um soldado
deve lutar. Assim pensam os homens extraordinários -- prossegue a
logicidade de Raskólnikov \textbf{--}, aqueles que vieram ao mundo para
legislar, aqueles diante de quem a massa ordinária deve se curvar. Se
Deus já não existe, o deus terreno, Augusto César Napoleão, não deve se
importar com o \emph{não matarás}. Quem tem algo a dizer para a
humanidade não pode estacar diante de escrúpulos comezinhos\footnote{Se
  Raskólnikov tivesse vivido para conhecer Ióssif Vissariónovitch
  Djugachvili, também conhecido como Stálin, o líder soviético lhe teria
  ensinado que uma única morte, de fato, é uma tragédia; um milhão de
  vítimas tornam"-se material estatístico.}.

Raskólnikov pretende se autoproclamar imperador como o fez Napoleão. Não
falta muito, o plano já foi esboçado, o jovem está a um passo da
realização. Um~único detalhe -- bem pequeno, na verdade -- o distancia
do trono: ocorre que Raskólnikov, sem ter onde cair vivo, precisa vender
o almoço para poder jantar. (O café da manhã também é negociável.) Logo,
o ex"-estudante de Direito precisa empenhar seus derradeiros objetos de
valor para Alióna Ivánovna, a velha usurária que o explora, o piolho
cuja existência, sempre segundo a logicidade de Raskólnikov, só faz
emperrar seus planos para proceder à imitação de Alexandre, o Grande,
isto é, para fazer com que a humanidade efetivamente evolua. Mas e se o
piolho usurário se transformar no primeiro grande teste para saber se
Raskólnikov está além do populacho? Uma vil exploradora não pode fazê"-lo
tremer. Em face dos soldados que morriam em pé congelados pelo general
inverno russo, Napoleão fazia um trocadilho espirituoso, dava de ombros
e, quando de sua volta a Paris, era ovacionado pela multidão pronta a
fornecer mais buchas de canhão para as guerras do imperador. Eis que o
dostoievskiano não hesita e prepara sua machada. Golpes secos rompem a
têmpora de Alióna Ivánovna. Como imprevistos não apenas acontecem, mas,
sobretudo, despencam, a irmã da velha usurária aparece no apartamento de
Alióna Ivánovna bem no momento em que o carrasco Raskólnikov está diante
do cadáver endinheirado. Uma única morte é uma tragédia; duas, uma
decorrência. De um momento para o outro, Ródion Romanovitch Bonaparte
transforma"-se em um duplo homicida.

Que lógica subterrânea alicerçou as ações de Raskólnikov? Quais os
princípios de seu \emph{cálculo utilitário?~}Tudo o que é sagrado é
profanado: ao invés de apreender que é um indivíduo que vive e convive
em sociedade, que estabelece relações com os demais, que os outros fazem
parte de sua (de)formação, Raskólnikov só faz observar a alteridade como
massa de manobra, isto é, como instrumento para seus próprios fins
utilitários. Seu cálculo maximiza os próprios interesses e transforma o
outro em alavanca. Se, no limite, for preciso prescindir do outro, que
assim seja. Conforme argumenta Lyudmila Parts (2009),

\begin{quote}
o exemplo de Raskolnikov mostra suspeição em relação à compaixão sem
Deus: sua piedade inicial pelos outros -- sua família, a família de
Marmieládov e todo o mundo de pobreza a seu redor -- leva"-o a um
protesto, a uma demanda por mudanças imediatas. Mas, uma vez que ele
rejeita Deus e a interpretação religiosa do sofrimento e da compaixão,
não há nada que possa evitar que Raskólnikov tente se tornar um deus --
chame"-o de Napoleão ou Maomé -- com o poder de decidir o valor e o
destino das outras pessoas. A~compaixão sem fé conduz Raskólnikov ao
homicídio (pp. 70-71).
\end{quote}

A usurária Alióna Ivánovna de fato pôde atestar que tudo o que é sólido
se desmancha no ar. Ora, não estamos diante dos primórdios da lógica
concorrencial que estrutura o capitalismo e pisa sobre o graveto frágil
da compaixão? Raskólnikov se quer um legislador, um revolucionário, ele
assassina uma das agentes da burguesia. Mas, curiosa e
contraditoriamente, a lógica que estrutura sua ação só pode ser
considerada revolucionária diante da completa vacuidade ética que
estabelece em relação à tradição do \emph{não matarás} oriunda do
decálogo de Moisés. No mais, Raskólnikov lança mão de um princípio
relativista que transforma o eu, o ego, em princípio único de todas as
coisas\footnote{Em ``Dostoevsky versus Max Stirner'', Nadine Natov
  (2002) discorre sobre a influência do individualismo radical do
  filósofo alemão Max Stirner sobre a \emph{intelligentsia} russa da
  década de 1840 e, notadamente, sobre a obra de Dostoiévski. O~livro de
  Stirner, \emph{Der Enzige und sein Eigentum} (1844) -- publicado, no
  Brasil, como \emph{O único e sua propriedade} (São Paulo: Martins
  Fontes, 2009) --, foi fundamental para que a escatologia criativa de
  Dostoiévski derivasse as aporias e rupturas éticas do individualismo
  radical. Nesse sentido, Natov cita alguns trechos de Stirner que
  prenunciam a lógica utilitária e homicida de Raskólnikov: ``Em uma
  breve introdução a seu livro {[}\emph{Der Einzige und sein
  Eigentum}{]}, intitulada `Todas as coisas não significam nada para
  mim'', Stirner escrever: ``Que Deus se preocupe com o que é divino;
  que o homem se preocupe com o que é humano. Minha preocupação não é
  nem com o divino, nem com o humano, nem com a verdade, com o que é
  bom, justo, livre etc., mas somente com o que é meu (\emph{das
  Meinige}), e não se trata de algo geral, mas único (\emph{einzig}),
  como eu sou único. Nada significa mais para mim do que eu mesmo!'
  (\ldots) Falando sobre `O proprietário' (\emph{Der Einziger}), Max
  Stirner definiu seu conceito de egoísmo da seguinte maneira: `O
  egoísmo não pensa em sacrificar nada, em desistir de nada que ele
  quer; ele simplesmente decide: `o que eu quero, eu tenho que obter e
  vou buscar'. Em uma passagem anterior, Stirner escrevera: `(\ldots) o que
  o homem conseguir obter lhe pertence: o mundo pertence a \emph{mim}
  (\ldots); o certo é o que lhe convém'" (p. 28; p. 32; p. 35).}. Se o
indivíduo vivesse em uma bolha autogerida, não haveria grandes dilemas.
Ocorre que o homem, a despeito do hedonismo de Raskólnikov, é um animal
social -- desde as características mais tenras até a complexidade da
intelecção, o homem se forma mediado pelo diálogo socialmente
configurado. Assim, o cálculo utilitário do protagonista de \emph{Crime
e castigo}, a princípio tido como emancipatório,~traz à tona o princípio
regressivo que passará a estruturar a modernidade. Para vencer a guerra
cotidiana de todos contra todos, Raskólnikov precisa amordaçar sua
identidade social e aniquilar o outro que é parte primordial de sua
formação dialógica.

O transcurso posterior do romance, cuja primeira parte termina com o
duplo homicídio cometido por Raskólnikov, narrará uma dolorosa dialética
entre crime e castigo. Raskólnikov precisará caminhar com o fardo de ter
aspergido sangue alheio. O~jovem terá que se submeter ao exame da
própria consciência -- e de sua inescapável vaidade. Se sofre por conta
de suas vítimas, será mesmo Raskólnikov um Napoleão? A~despeito de sua
megalomania, não fará o ex"-estudante de Direito parte da massa ordinária
que tanto despreza? Ainda que Raskólnikov estivesse caminhando entre as
ruínas do cristianismo como cosmovisão socialmente estabelecida, o
Decálogo de Moisés, nos primórdios da modernidade, continuava a calar
fundo em meio ao imaginário coletivo. A~profunda dor moral que o jovem
homicida sente pode ser relacionada ao peso da tradição que o
socializou. O~cálculo utilitário ainda era embrionário. Vale frisar que
Raskólnikov foi o mentor e o executor da ideia. A~sociedade
contemporânea, cada vez mais distante do \emph{ethos} religioso, tornou
mais complexo e mediado, cínico e introjetado o cálculo utilitário de
que Raskólnikov lançou mão.

\section{3.3. A utopia como degredo siberiano}

Genebra, 29 de setembro de 1867: em uma carta para a sobrinha Sofia
Aleksandrovna, Dostoiévski descreve qual seria, segundo sua perspectiva,
um dos problemas essenciais do ideário socialista:

\begin{quote}
Para obter a paz na Terra, {[}os socialistas{]} querem extirpar a fé
cristã, aniquilar os grandes Estados e dividi"-los em países menores,
abolir o capital, declarar que toda propriedade é de uso comum etc. E
tudo isso é dito \emph{sem qualquer demonstração lógica}; o que eles
aprenderam há trinta anos é o que ainda estão a balbuciar hoje.
\emph{Apenas quando o fogo e a espada tiverem exterminado tudo será
possível}, segundo eles, que a paz se estabeleça {[}grifos meus{]}
(2011, p. 129).
\end{quote}

O trecho em questão enumera uma série de pontos fundamentais para a
contraposição dostoievskiana em relação à utopia socialista. Mas, antes
de mais nada, vale frisar que, no fim da década de 1840, o escritor fora
condenado a longos anos de trabalhos forçados na Sibéria por ter
participado do Círculo de Petrachévski, um grupo revolucionário cuja
principal reivindicação voltava"-se contra a servidão da gleba e,
consequentemente, contra a base econômico"-social do tsarismo. O~degredo
siberiano, na verdade, despontou como decorrência da comutação da pena
de morte a que todos os integrantes de Petrachévski haviam sido
condenados\footnote{Para uma descrição minuciosa da participação de
  Dostoiévski no Círculo de Petrachévski, sua detenção na fortaleza
  Pedro e Paulo durante os inquéritos e todo o cerimonial de execução e
  comutação da pena de morte na Praça Semiônov, em São Petersburgo,
  conferir Joseph Frank, \emph{Dostoiévski: os anos de provação,
  1850-1859.} São Paulo: Edusp, 1999, pp. 23-105.}.

Tal colocação tem o sentido de afirmar que Dostoiévski dificilmente
poderia ser definido como um reacionário típico. É~bem verdade que a
estada siberiana na \emph{casa dos mortos} transfiguraria sobremaneira a
cosmovisão do escritor. Mas Dostoiévski não denegava a utopia socialista
por si mesma, como o faria um reacionário que quisesse refrear a
movimentação progressista (e contraditória) da história. Como veremos
neste e no próximo capítulo, Dostoiévski entrevia profundas
contiguidades entre as utopias cristã e socialista. ``Eu vos repito: é
mais fácil um camelo passar pelo fundo de uma agulha do que um rico
entrar no Reino dos Céus. (\ldots) Assim, pois, os últimos serão os
primeiros e os primeiros serão os últimos'' (\versal{MATEUS}, 19, 24; 20, 16). O
questionamento em relação ao \emph{status quo} e à reprodução da
desigualdade social; o ímpeto de construção de uma sociedade solidária,
em meio à qual o lema de Marx -- \emph{de cada um, conforme sua
capacidade; a cada um, conforme sua necessidade} -- trouxesse a máxima
do Sermão da Montanha para a estepe da história: ``Amarás o teu próximo
como a ti mesmo'' (\versal{MARCOS}, 12, 31). O~conflito de Dostoiévski com a
\emph{intelligentsia} revolucionária se exacerba diante da perspectiva
de um mundo sem Deus -- isto é, sem o sentido da eternidade, sem a
salvaguarda para os valores éticos -- que só faria coroar o ego como
princípio único de todas as coisas. Desenvolvemos este ponto mais acima
e o desenvolveremos mais extensamente na sequência da argumentação. A
partir de agora, no entanto, será importante analisarmos outros quatro
aspectos mencionados por Dostoiévski como problemas intrínsecos do
socialismo: (i) a desconexão das ideias com o movimento real da
história; (ii) a abolição do capital e da propriedade comum; (iii) a
aniquilação e a partilha dos grandes Estados; (iv) a guerra total como
mote para a paz futura e perpétua. Tais aspectos esquematicamente
listados se imbricam profundamente em meio à crítica dostoievskiana à
utopia socialista. Apenas os articulamos separadamente para mostrarmos
os momentos de metamorfose das reflexões de Dostoiévski.

\subsection{3.3.1. História de proveta, socialismo de estufa}

Após sua primeira viagem para a Europa -- a Rússia, diante do Ocidente,
sente os complexos de ser considerada e de se considerar um país
eurasiano --, Dostoiévski e o narrador de \emph{Notas de inverno sobre
impressões de verão} refletem sobre o caráter livresco das ideias
progressistas que pretendem transformar o real passando por cima de suas
contradições históricas:

\begin{quote}
Ouvi dizer recentemente que um proprietário rural dos nossos dias
{[}referência irônica a Liev Tolstói{]}, para se fundir com o povo,
também começou a usar traje russo e passou até a frequentar, assim
vestido, as assembleias de aldeia; mas os camponeses, ao vê"-lo, diziam
entre si: ``Por que este fantasiado se arrasta atrás de nós?'' E~o tal
proprietário não conseguiu fundir"-se com o povo. (\ldots) E~até hoje, em
nosso meio, quantos destes progressistas de estufa não existem que
formam entre os nossos mais avançados homens de ação, que estão
extremamente satisfeitos com a sua condição de criaturas de estufa e
nada mais exigem? (\ldots) Atualmente, o povo já nos considera {[}a nós,
nobres{]} de todo estrangeiros e não compreende uma palavra, um livro,
um pensamento nosso, e isto, digam o que quiserem, é progresso. (\ldots)
Quão convencidos estamos agora da nossa vocação civilizadora, quão do
alto resolvemos os problemas, e que problemas: não há solo, não há povo,
a nacionalidade é apenas um determinado sistema de impostos, a alma, uma
\emph{tabula rasa}, uma cerinha com a qual se pode imediatamente moldar
um homem verdadeiro, um homem geral, universal, um homúnculo: basta para
isto aplicar os frutos da civilização europeia e ler dois ou três livros
(2000b, pp. 94-95).
\end{quote}

A pilhéria dostoievskiana em relação ao aristocrata Tolstói,
proprietário da belíssima herdade de Iasnaia Poliana, tem raízes mais
profundas do que a ironia sugere. Na \emph{casa dos mortos}, Dostoiévski
pôde entrar em contato com boa parte do lumpesinato russo que, segundo a
\emph{intelligentsia}, possuía consciência de classe e se tornaria a
vanguarda da revolução após a tomada do poder. Ocorre que a efetiva
consciência de classe que Dostoiévski encontrou entre os prisioneiros
era bastante contraditória em relação à emancipação: os presos se
incomodavam com o fato de que um nobre como Dostoiévski quisesse agir
como se não houvesse distinções sociais. O~escritor pôde verificar que
havia, de fato, um profundo ressentimento social entre os detentos, mas
isso poderia estar ligado mais à usurpação do poder por parte da nobreza
do que a uma demanda efetivamente democrática e socialista por parte da
esmagadora maioria de prisioneiros pobres. Assim, não se trataria,
\emph{a priori} e apenas, de uma demanda por uma sociedade radicalmente
outra -- conforme o socialismo livresco poderia depreender --, mas de
uma lógica de corroboração da tradição -- que se respeitasse a
hierarquia social, de modo que cada estamento cumprisse o seu papel.
Ora, bem podemos ver, então, quão distinto era esse impulso da teoria
messiânica que atribuía aos oprimidos o germe \emph{inequívoco} da
revolução.

Assim, Dostoiévski passou a apreender um problema \emph{congênito} na
utopia socialista: tratava"-se de transplantar modelos ocidentais
\emph{diretamente} para a Rússia, como se a peculiaridade periférica do
país pudesse ser esquadrinhada e modificada por cálculos
matemático"-administrativos que passariam por cima da historicidade das
relações sociais efetivas. E~mais: quando tais tentativas sistemáticas
da utopia de gabinete entrassem em choque com as contradições que a
realidade lhes contraporia, que garantias haveria de que a revolução não
se reverteria no mais trágico autoritarismo para continuar a enquadrar a
história em seus moldes abstratos e pré"-concebidos? Susan McReynolds
(2003) sintetiza tal tensão com a impossibilidads de os revolucionários
``(\ldots) transformarem a sociedade por decreto'' (p. 83). Kate Holland
(2000), por sua vez, procura responder a essa pergunta ao acompanhar as
abstrações dos revolucionários que, para Dostoiévski, geravam os
paradoxos para a transformação da sociedade e dos homens.

O pensamento linear, na famosa frase de {[}Aleksandr{]} Herzen
{[}revolucionário russo contemporâneo de Dostoiévski e exilado em
Londres{]}, partia do pressuposto de que a história tinha um enredo, um
plano. Os~pensadores lineares concebiam sistemas racionais para provar a
infalibilidade de suas visões de mundo, mas a acabavam baseando em seus
sistemas. Quando os acontecimentos e as pessoas reais não correspondiam
a essas teorias concebidas \emph{a priori}, tais pensadores, ao invés de
questioná"-las, negavam"-se a aceitar a realidade e subtraíam os desvios
de suas visões lineares. Visões de mundo lineares, então, frequentemente
levavam à negação das dúvidas, perguntas e paradoxos, elementos que,
para Dostoiévski, embasavam toda a psicologia humana (p. 97).

O homem do subsolo, então, volta a se insinuar para que vislumbremos em
que poderia redundar o cálculo utilitário da classe redentora\footnote{Qual
  o pressuposto da teoria messiânica do proletariado senão o cálculo
  econômico"-utilitário de que os oprimidos não têm nada a perder a não
  ser os próprios aguilhões? Tal raciocínio subestima as resistências
  identitárias das classes e indivíduos socialmente mediados pela
  distinção hierárquica -- resistências que viriam a ser estudadas com
  muita profundidade pela psicologia social e pela Teoria Crítica, após
  o apoio maciço da classe trabalhadora alemã ao nacional"-socialismo. Ao
  não se distanciar do homem histórico de carne e osso que eventualmente
  compreende as próprias vantagens, mas que carrega consigo as
  contradições de sua socialização, Dostoiévski pôde acompanhar com
  dinamismo e profundidade os movimentos que aproximam e confundem
  prazer e dor, revolta e submissão, questionamento e resignação. Para
  uma discussão sobre o enrijecimento do caráter identitário que se
  contrapõe aos interesses econômicos das classes (a vantagem
  utilitária), conferir a importante obra de Wilhelm Reich,
  \emph{Psicologia de massas do fascismo.} São Paulo: Martins Fontes,
  1988.}, caso ele quisesse arregimentar a história com a farda (e a
camisa"-de"-força) de sua revolução:

\begin{quote}
Proclamo com insolência que todos esses belos sistemas, todas essas
teorias para explicar à humanidade os seus interesses verdadeiros,
normais -- a fim de que ela, ansiando inexoravelmente por atingir essas
vantagens, se torne de imediato bondosa e nobre --, por enquanto tudo
isso não passa, a meu ver, de pura logística! Sim, logística! Sem
dúvida, afirmar essa teoria da renovação de toda a espécie humana por
meio do sistema das suas próprias vantagens é, a meu ver, quase o
mesmo\ldots bem, que afirmar, por exemplo (\ldots) que o homem é suavizado
pela civilização, tornando"-se por conseguinte, pouco a pouco, menos
sanguinário e menos dado à guerra. (\ldots) Mas o homem é a tal ponto
afeiçoado ao seu sistema e à dedução abstrata que está pronto a deturpar
intencionalmente a verdade, a descrer de seus olhos e seus ouvidos
apenas para justificar a sua lógica. (\ldots) \emph{A civilização elabora
no homem apenas a multiplicidade de sensações e\ldots absolutamente nada
mais. E, através do desenvolvimento dessa multiplicidade, o homem talvez
chegue ao ponto de encontrar prazer em derramar sangue} {[}grifo meu{]}
(2000a, pp. 34-35).
\end{quote}

O bufão Liébediev, então, aproveita a deixa do homem do subsolo para
aprofundar as contradições do sistema utilitário de vantagens em
correlação com a história do poder:

\begin{quote}
Eu não acredito (\ldots) nas carroças que transportam comida para toda a
humanidade! Porquanto as carroças que transportam comida para toda a
humanidade, \emph{sem o fundamento moral do ato, podem excluir com o
maior sangue frio uma parte considerável da humanidade do prazer com o
transportado, o que já aconteceu\ldots} {[}grifo meu{]} (2002, p. 419)
\end{quote}

Eis que Chigáliov, teórico revolucionário de \emph{Os demônios},
desponta em meio às diatribes que esgarçam as (im)possibilidades da
utopia. Chigáliov não é um idealista desmesurado que imagina que a
história dará um salto qualitativo absoluto após a revolução -- ele tem
um conhecimento empírico devido para correlacionar momentos de fratura
histórica com as ações possíveis das vanguardas do poder, vale dizer, os
grupos revolucionários que realmente têm consciência sobre a
possibilidade de transformação político"-social. Chigáliov esboça sua
teoria de administração do falanstério levando em consideração o
discurso da servidão voluntária e a continuação da dialética que só faz
senhores e escravos se engalfinhar. Ele propõe

\begin{quote}
como solução final do problema, dividir os homens em duas partes
desiguais. Um décimo ganha liberdade de indivíduo e o direito ilimitado
sobre os outros nove décimos. Estes devem perder a personalidade e
transformar"-se numa espécie de manada e, numa submissão ilimitada,
atingir uma série de transformações da inocência primitiva, uma espécie
de paraíso primitivo, embora, não obstante, continuem trabalhando. As
medidas que o autor propõe para privar de vontade os nove décimos dos
homens e transformá"-los em manada através da reeducação de gerações
inteiras são excelentes, baseiam"-se em dados naturais e são muito
lógicas (2004, p. 392).
\end{quote}

O historiador e politólogo Chigáliov entende a importância
simbólico"-pragmática do culto ao líder. Afinal, quando foi que a
história deixou de ditar ao homem que ele deve buscar alguém diante de
quem se inclinar? Na \emph{casa dos mortos}, Dostoiévski bem pôde
descobrir que

\begin{quote}
as gerações não esquecem de uma hora para a outra o que herdaram; não se
desprendem com facilidade do que receberam no sangue, daquilo que
sugaram com o leite materno; não existem transformações súbitas. Seus
erros, seus pecados originais, não basta reconhecê"-los. Cumpre
erradicá"-los bem, e não é empreendimento fácil (1982, p. 178).
\end{quote}

Ainda uma vez, o \emph{demônio} dostoievskiano de Chigáliov pode
reivindicar o espírito capitalista do socialismo como sua projeção
histórico"-escatológica:

\begin{quote}
Chigalióv é um homem genial! (\ldots) Ele inventou a ``igualdade''! (\ldots)
Todos são escravos e iguais na escravidão. (\ldots) A~primeira coisa que
fazem é rebaixar o nível da educação, das ciências e dos talentos. O
nível elevado das ciências e das aptidões só é acessível aos talentos
superiores, e os talentos superiores são dispensáveis! \emph{Os talentos
superiores sempre tomaram o poder e foram déspotas} Os~talentos
superiores não podem deixar de ser déspotas e sempre trouxeram mais
depravação do que utilidade; eles serão expulsos ou executados. A~um
Cícero corta"-se a língua, a um Copérnico furam"-se os olhos, um
Shakespeare mata"-se a pedradas -- eis o chigaliovismo. Os~escravos devem
ser iguais: sem despotismo ainda não houve nem liberdade nem igualdade,
\emph{mas na manada deve haver igualdade, e eis aí o chigaliovismo!}
(\ldots) Não precisamos de educação, chega de ciência! Já sem a ciência há
material suficiente para mil anos, mas precisamos organizar a
obediência. No mundo só falta uma coisa: obediência. A~sede de educação
já é uma sede aristocrática. (\ldots) Tudo será reduzido a um denominador
comum, é a plena igualdade. (\ldots) Mas precisamos também da convulsão;
disso cuidaremos nós, os governantes. Os~escravos devem ter governantes.
Plena obediência, ausência total de personalidade, mas uma vez a cada
trinta anos Chigáliov lançará mão também da convulsão, e de repente
todos começam a devorar uns aos outros, até um certo limite, unicamente
para não se cair no tédio. O~tédio é uma sensação aristocrática; no
chigaliovismo não haverá desejos. Desejo e sofrimento para nós, para os
escravos o chigaliovismo {[}grifos meus{]} (2004, pp. 407-408).
\end{quote}

Fiódor Dostoiévski, profícuo intérprete dos sentidos e ressentimentos da
modernidade, entrevê o filisteísmo democrático -- a igualdade negativa,
a igualdade coercitiva, a igualdade dos escravos -- como o prenúncio da
indústria cultural. Assim, uma colocação fundamental do filósofo Günther
Anders sobre o caráter classista e elitista da arte a impulsioná"-la para
a contínua superação de si mesma nos estimula a pensar sobre um possível
aparte de Chigáliov tendo em vista o igualitarismo socialista que
paralisaria o confronto e a distinção sociais em prol do rebaixamento
antiaristocrático e padronizado:

\begin{quote}
Mas aquilo a que damos o nome de arte provém sem dúvida de épocas em que
existiam relações de domínio, ou seja, diferenças de posição social,
portanto, também diferenças sociais de linguagem. De fato, a distância
da beleza ou da obra de arte é a tal ponto reflexo da distância e do
desnível sociais (\ldots) que, por mais desagradável que a tese possa soar,
\emph{a neutralização de classes paralisaria a arte} (\ldots)\emph{; se não
expressamente em relação à arte, pelo menos em relação ao ``espírito''
em geral} {[}grifo meu{]} (2007, pp. 88-89).
\end{quote}

A resposta de Chigáliov não pressuporia o recrudescimento das relações
de domínio e o rebaixamento da arte a mera propaganda? Ou pior: o
recrudescimento das relações de domínio por meio do rebaixamento da arte
a mera propaganda. Arte imediatamente inteligível, consumível -- e
laudatória. Se a neutralização das classes paralisa a arte, a propaganda
a democratiza. Afinal, a modernidade burguesa se vinga dos séculos de
jugo aristocrático apresentando aos antigos carrascos o orgulho pelos
aguilhões que por tanto tempo a rebaixaram. A~vingança não é a superação
dialética do privilégio aristocrático pela manutenção/negação
determinada de seu momento de verdade -- o socialismo identificaria a
finalidade não"-utilitária do caráter aristocrático como o privilégio que
poderia ser elevado a valor universal com o desenvolvimento tecnológico
e a justiça social que a revolução fundaria\footnote{Para uma discussão
  sobre o socialismo como um meio para a libertação dos seres humanos
  com vistas à realização e à elevação de suas potencialidades éticas e
  estéticas, conferir Oscar Wilde, \emph{A alma do homem sob o
  socialismo.} Porto Alegre: \versal{L\&PM} Pocket, 2003.}; a vingança é a
negação absoluta da superação (\emph{Aufhebung}), o ressentimento
filisteu contra a bela vida estética tachada como vagabundagem. O
burguês é tão filisteu quanto o proletário; apenas a \emph{quantidade},
isto é, a \emph{contabilidade}, separa as almas/classes com o cálculo
atuarial que destinará a cobertura ao endinheirado e as favelas para o
miseralato.

Ao apreender o filisteísmo socialista, Dostoiévski não engatilha sua
crítica apenas contra a utopia. O~escritor já encontrara o princípio do
falanstério -- o formigueiro humano -- na capital da burguesia, a
Londres da revolução \emph{industrial}:

\begin{quote}
A civilização há muito já está condenada no próprio Ocidente (\ldots) e é
defendida apenas pelos proprietários {[}embora ali todos (burgueses e
proletários) sejam proprietários ou queiram sê"-los{]}, a fim de salvar o
seu dinheiro. (\ldots) \emph{Toda nitidez, toda contradição, se acomoda ao
lado da sua antítese e com ela avança teimosa, de braço dado,
contradizendo"-se mutuamente mas sem se excluir, é claro} {[}grifo
meu.{]} (\ldots) E, no entanto, também ali se processa a mesma luta tenaz,
surda e já antiga, a luta de morte do princípio pessoal, comum a todo o
Ocidente, com a necessidade de acomodar"-se de algum modo ao menos,
formar de algum modo uma comunidade e instalar"-se num formigueiro comum;
transformar"-se nem que seja num formigueiro, mas organizar"-se sem que
uns devorem os outros, senão todos se tornarão antropófagos! (2000b, p.
98; p. 112)
\end{quote}

Dostoiévski nos fala sobre a aparente desordem da moderna sociedade
produtora de mercadorias, desordem que, ``em essência, é a ordem
burguesa no mais alto grau'' (2000b, p. 113): eis a conclusão do
viajante Dostoiévski que, em Londres, apreende a profunda contiguidade
entre a pretensa liberdade moderna e a coação siberiana na \emph{casa
dos mortos}; a profunda contiguidade entre a modernidade que baniu Deus
e o lastro dos valores morais e a fundação da utopia sobre o cálculo
utilitário que leva à lei do mais forte; a profunda contiguidade entre o
sonho de liberdade e elevação do homem e o pesadelo da caserna que
democratiza as servidões material e espiritual. A~\emph{casa dos mortos}
poderia ser entendida como uma metáfora para o formigueiro humano
socialista e/ou capitalista.

Nesse preciso sentido, Robert Louis"-Jackson (1981) apreende a crítica
dostoievskiana ao materialismo burguês -- o \emph{ethos} da modernidade
apartada da transcendência que se espraia tanto pelos revolucionários
socialistas quanto pelos empreendedores burgueses.

\begin{quote}
Nós já refletimos sobre quão avessos a Dostoiévski eram os valores
burgueses -- o materialismo desenfreado, o egoísmo e o individualismo
vorazes, a idolatria das coisas e a obsessão pelos ``bens'' --, valores
que se tornaram dominantes em boa parte da vida nos Estados Unidos e na
Europa Ocidental e que, agora, parecem fadados a se espraiar por todo o
mundo capitalista? Em \emph{Notas de inverno sobre impressões de verão},
Dostoiévski ataca não apenas a pobreza e a iniquidades produzidas pelo
capitalismo do século \versal{XIX} e sua idolatria do comércio e do mercado, mas
``os próprios trabalhadores que, em seus corações, também são
proprietários''. Em \emph{O jogador}, Dostoiévski explorou o
entrelaçamento fatal entre economia e psicologia no homem social.
\emph{Roulettenburg} -- nome da cidade em que a ação acontece e título
original da novela -- é, em grande medida, a cidade clássica do
capitalismo, e a análise de Dostoiévski da cidade e seus habitantes é,
entre outras coisas, uma crítica clássica ao dinheiro e à sociedade
burguesa de mercado. Aqui, Marx e Dostoiévski se encontram. Em \emph{Os
irmãos Karamázov}, Dostoiévski parece ter antecipado algo profundamente
característico da vida da classe média quando, por meio de Zósima, ele
retrata o homem contemporâneo ``atado às inumeráveis necessidades que
ele criou para si mesmo''. As~palavras de Dostoiévski alcançam o coração
do chamado \emph{American way of life.} ``Quando interpreta a
liberdade'', prossegue Zósima, ``como a multiplicação e a rápida
satisfação dos bens, os homens distorcem sua própria natureza -- já que
muitos desejos e hábitos inócuos e estúpidos passam a ser estimulados.
Será que isso é liberdade?'', pergunta Zósima (pp. 7-8).
\end{quote}

Jackson nos mostra como a crítica dostoievskiana à modernidade apresenta
matizes complexos e pouco lineares. A~crítica se refere ``à profunda
antipatia de Dostoiévski ao foco materialista tanto da sociedade
burguesa quanto da socialista'' (\versal{JACKSON}, 2002, p. 15). Assim, as
diatribes dostoievskianas contra os revolucionários não se referem, por
si sós, à transformação profunda da sociedade -- transformação pela qual
Dostoiévski sempre nutriu simpatias, dadas as afinidades entre os
\emph{ethos} cristão e socialista --, o que não nos permite arregimentar
o escritor, unilateralmente, nas fileiras dos reacionários. Jackson
entreviu os momentos de contiguidade entre Dostoiévski e Marx na crítica
às relações sociais sob a égide do capital. No entanto, Marx se voltou
para os elementos construtivos e progressistas da revolução, ao passo
que Dostoiévski, ao acompanhar a ruptura dos revolucionários em relação
à transcendência e à eternidade, pôde entrever as sombras projetadas
pelo Iluminismo, o regresso umbilicalmente vinculado ao progresso. O
materialismo filisteu, que aproximava os polos a princípio antípodas de
socialismo e capitalismo, poderia enredar os homens e as mulheres na
busca incessante por propriedade e bens, de tal maneira que a
modernidade erigisse um novo deus: a mercadoria -- e seu fetichismo.
Nesse sentido, os agentes da revolução material poderiam ser
transformados em engrenagens. Assim, Jackson (2002) prossegue afirmando
que

\begin{quote}
Dostoiévski desafia a legitimidade do processo histórico que subvertera
o estado do homem -- um processo histórico, notemos, que, para
Dostoiévski, enreda tanto burgueses quanto socialistas em sua adoção
comum do materialismo. Supostamente, o homem é o centro desse processo
histórico e o objetivo final desse ``progresso'' (\ldots). Mas isso é
verdade apenas como ``ideia''. Na realidade, o homem é um meio, já que o
fim são as instituições, a complexidade das relações sociais, o poder da
indústria e o comércio (p. 17).
\end{quote}

Vemos, assim, que a dialética dostoievskiana, ao acompanhar as
\emph{possibilidades objetivas} em devir, apreendeu as afinidades
eletivas que, após a Segunda Guerra Mundial, \emph{pareceram} dividir o
mundo em dois blocos radicalmente antípodas. É~nesse contexto que
Jackson (1981) se refere à ``contemporaneidade de Dostoiévski em nosso
mundo'' (p. 4). Dostoiévski bem poderia dizer que, por um lado, se de
fato havia contraposições bélicas entre \versal{EUA} e \versal{URSS}, por outro, a disputa
envolvia mais contiguidades do que norte"-americanos e soviéticos
gostariam de admitir para suas populações em pânico pela hecatombe
iminente da Guerra Fria.

\subsection{3.3.2. O socialismo siberiano da casa dos mortos}

Omsk, fim de janeiro, início de fevereiro de 1854: em uma carta para
Natália Dmitrievna Fonvisin, esposa do dezembrista M.A. Fonvisin,
Dostoiévski discorre sobre as agruras do \emph{comunismo} da \emph{casa
dos mortos}: ``Estar só é um estado natural, como comer e dormir; nessa
forma concentrada de comunismo, qualquer um se torna um inimigo ferrenho
da humanidade. A~constante companhia de outros funciona como um veneno
ou uma praga'' (2011, p. 78). A~experiência da coação coletivista seria
expandida em \emph{Recordações da casa dos mortos}:

\begin{quote}
A falta de liberdade não consiste jamais em estar segregado, e sim em
estar em promiscuidade, pois o suplício inenarrável é não se poder estar
sozinho. A~vida comum é fenômeno social escolhido e voluntário, ao passo
que os companheiros de presídio são impostos pela sorte aziaga e
niveladora de intentos, e não pela vontade selecionadora de inclinações.
Inconscientemente, todos os detentos sofrem quando em promiscuidade, bem
mais do que com seus devaneios ilimitados (\versal{DOSTOIÉVSKI}, 1982, p. 22).
\end{quote}

Ora, se a tabela logarítmica da utopia já calcula e projeta todos os
desejos humanos; se o todo provê detenção, comida e roupa vez por outra
lavada; se a nostalgia da liberdade é compensada por um companheirismo
coercitivo em que a amizade \emph{deve} florescer contra qualquer
espontaneidade, a ironia do homem do subsolo só faz perguntar: que mais
o homem pode desejar? Assim, eis que o paradoxalista das \emph{Memórias
do subsolo} irrompe ainda uma vez para espreitar entre as fissuras do
socialismo de caserna:

\begin{quote}
As vossas vantagens são o bem"-estar, a riqueza, a liberdade, a
tranquilidade etc. etc.; de modo que o homem que se declarasse, por
exemplo, consciente e claramente, contra todo esse cadastro, seria, na
vossa opinião (\ldots), um obscurantista ou um demente completo, não é
verdade? Mais eis o que é surpreendente: por que sucede que todos esses
estatísticos, mestres da sabedoria e amantes da humanidade, ao computar
as vantagens humanas, deixam de mencionar uma delas? (\ldots) Não existirá,
de fato (e eu digo isto para não transgredir a lógica), algo que seja a
quase todos mais caro que as maiores vantagens (justamente a vantagem
omitida, aquela de que se falou ainda há pouco), mais importante e
preciosa que todas as demais e pela qual o homem, se necessário, esteja
pronto a ir contra todas as leis, isto é, contra a razão, a honra, a
tranquilidade, o bem"-estar, numa palavra, contra todas estas coisas
belas e úteis, só para atingir aquela vantagem primeira, a mais
preciosa, e que lhe é mais cara que tudo? (\ldots) Essa vantagem é
admirável justamente por destruir continuamente todas as nossas
classificações e sistemas elaborados pelos amantes da espécie humana,
para a felicidade desta (2000a, pp. 34-35).
\end{quote}

Mas que vantagem seria essa? Uma vantagem não"-utilitária, uma vantagem
desvantajosa, mas, ainda assim, a mais importante entre todas as demais?
Até aqui, o homem do subsolo a apreendeu \emph{negativamente}, pelo
movimento indômito da contradição. Seria esse não"-ser a quintessência do
homem, o \emph{sentido} de sua existência? O~paradoxalista do subsolo
chega a aventar a possibilidade de o homem abrir mão de todas as suas
(supostas) regalias no presídio socialista para tatear em direção a essa
vantagem desvantajosa que seria a expressão do inominável -- quiçá a
pura vivência que vai se configurando enquanto é vivida, com a única
condição de que seja autóctone e autônoma, de que possa vir da expressão
mais telúrica do homem. Como o homem do subsolo nos lega como método
reflexivo o pensamento a contrapelo das próprias aspirações, uma
pergunta se insinua com efetiva objetividade contraditória: conseguiria
essa (des)vantagem das vantagens burlar as estruturas de coação e
padronização? A~\emph{casa dos mortos} pôde apresentar a Dostoiévski a
irrupção da (des)vantagem em estreita correlação com a escatologia
criativa que procura imaginar uma forma de organização social que
pudesse dar vazão a seu ímpeto.

\begin{quote}
Não raro as autoridades raciocinam: como é que um determinado detento
passa anos e anos suportando sua sorte, quieto, acomodado, chegando
mesmo devido ao seu comportamento a ser designado ``veterano'' (ou
monitor) e de súbito, sem um motivo lógico, como que possuído pelo
demônio, se põe a beber, a fazer estardalhaço, a brigar, indo às vezes
até a cometer as piores faltas, matando ou ferindo alguém, desacatando
abertamente um alto funcionário, ou coisas assim? Toda a administração
fica admirada. Ora, razão há e de sobra para explosões dessa repentina
natureza em tal gente; \emph{provêm, por mais inesperadas que sejam, de
uma ânsia pela posse outra vez da personalidade, de uma instintiva
angústia em busca do próprio eu, o desejo de recuperação de sua entidade
humilhada, tudo isso se desencadeando subitamente até o furor, até o
paroxismo ou mesmo até a insânia em seus aspectos assim incontidos e
convulsos} {[}grifo meu{]}. (\ldots) Aqui não se trata mais de saber se há
ou não nexo. A~razão não intervém nesse paroxismo. (\ldots) Se a vontade é
beber, então que se beba à farta; se o risco em que incorre é um fato,
então que se arrisque logo tudo de uma vez! Em face mesmo do assassínio
por que se deter?! Basta começar para então advir a embriaguez que já
agora nada mais sofrerá! O~melhor é deixar esse homem dar expansão um
pouco à sua potencialidade. Assim haverá mais conveniência social. Está
bem. Mas como? (1982, pp. 72-73)
\end{quote}

A (des)vantagem quintessencial é a expressão mais cabal da
\emph{vontade.} A~``ânsia pela posse outra vez da personalidade'' se
confunde com uma liberdade incriada. Nesse preciso sentido, Zsuzsanna
Bjørn Andersen (2000) cita um trecho de um folhetim escrito pelo então
jovem Dostoiévski em 1847:

\begin{quote}
Quando um homem está insatisfeito, quando ele não consegue se expressar
e revelar o que há de melhor nele (não por vaidade, mas por causa da
mais natural necessidade de realizar seu ego na vida real), ele se vê
acossado, de uma só vez, por uma situação incrível: Fulano, se me for
permitido dizer, se torna alcoólatra; Beltrano passa a jogar e vira um
trapaceiro; Sicrano só faz arranjar brigas; por fim, outro perde a
cabeça por causa da ambição, ao mesmo tempo em que a despreza e até
mesmo sofre por seu ímpeto desvairado (pp. 53-54).
\end{quote}

A ``instintiva angústia em busca do próprio eu'' começa a fazer com que
o não"-ser, contraditoriamente, passe pela negação de si mesmo.
Dostoiévski contrapõe de modo claro o \emph{sujet} do socialismo de
caserna e da vindoura indústria cultural ao eu que busca a expressão de
si, no limite, contra a própria vantagem estatística, utilitária e
economicamente pré"-calculada. ``Aqui não se trata de saber se há ou não
nexo. A~razão não intervém nesse paroxismo''. Eis um mote que faz com
que parte da fortuna crítica de Dostoiévski o considere um
irracionalista que proscreve a razão -- tal caminho, a meu ver datado e
bastante questionável, pode levar ao Dostoiévski defensor do mistério
ortodoxo e da Mãe Rússia como antídotos entrelaçados para a corrosão do
Ocidente. Trataremos de tais questões mais adiante. Por ora, vale frisar
que a racionalidade contra a qual Dostoiévski se coloca diz respeito ao
cálculo utilitário"-instrumental que projeta vivências atuariais por
sobre o e à revelia do devir efetivo da história. Se o eu se estrutura
como uma busca da própria personalidade, como um ímpeto de expressão da
própria vontade, a compressão de tais \emph{sentidos} -- seja em meio à
pobreza extrema, seja em meio à abonança industrial -- configura uma
humilhação que transformará a alma humana em um gêiser à iminência da
irrupção. Para o narrador dostoievskiano da \emph{casa dos mortos}, uma
organização social libertária deveria incorporar tal característica
essencialmente contraditória da alma humana, de modo que fosse possível
a esse todo permitir que o indivíduo se expressasse de forma própria e
autóctone. Mas o que a experiência siberiana lhe apresenta? Uma metáfora
para os modelos de transformação social que simplesmente fazem tábula
rasa das contradições históricas e projetam um homem de proveta como a
média estatística das aspirações humanas. A~vontade -- ou, para usarmos
um termo ainda mais indômito, o \emph{desejo} --, quando planificada e
padronizada, se esteriliza, reverte"-se no contrário de si mesma:
embotamento ao invés de ímpeto, reclusão ao invés de liberdade. Por
outro lado, como conceber uma organização social que possa dar vazão à
expressão indômita da personalidade? A~noção de \emph{posse da
personalidade}, para Dostoiévski, é um equivalente para a expressão
efetivamente telúrica da vontade, seja ela qual for. Estamos falando de
um escopo ilimitado de ações que vai do ímpeto por uma sobremesa ao
assassínio -- a obra de Dostoiévski também pode ser lida como um
inventário polissêmico da fenomenologia da vontade, de sua expressão
factual a partir dos movimentos ctônicos do desejo. Assim, a questão que
esgarça as contradições das transformações revolucionárias se insinua
ainda uma vez: como conceber uma organização social que possa dar vazão
à expressão indômita da personalidade? Outra síntese para a obra
dostoievskiana desponta: uma resposta aos ímpetos livrescos da
\emph{intelligentsia} russa em sua tentativa revolucionária de
esquadrinhar o homem e fazê"-lo viver sobre bases radicalmente outras sem
que haja raízes histórica e socialmente configuradas para tanto -- mais
adiante falaremos sobre as aporias da construção matemática da
\emph{fraternidade.} Quando Dostoiévski apreende, escatologicamente, a
reversão da utopia em distopia, a contiguidade entre revolução e reação,
totalidade e totalitarismo, seu devir lógico"-narrativo tem em mente tais
reflexões forjadas pela experiência da \emph{casa dos mortos. }

Neste momento, extraiamos da obra de Dostoiévski um exemplo analítico da
expressão escatológica da vontade como sentido -- e como profundo
ressentimento. Eis que desponta o Príncipe Míchkin, o protagonista que
vive segundo a máxima de que \emph{a beleza salvará o mundo}.
Dostoiévski o concebeu como uma fusão entre Jesus Cristo e Dom Quixote.
Cristo, o Sermão da Montanha e o oferecimento da outra face a partir do
amor mútuo. Quixote, o cavaleiro de La Mancha e o sonho de que os
valores nobres não se arrefecessem; assim, para driblar a realidade,
para tornar o sonho menos perecível, Quixote concebe um segundo sonho
ainda mais onírico, uma fantasia ainda mais distante da realidade --~o
Príncipe Míchkin poderia chamá"-la de utopia, seu norte de fraternidade.
Mas a missão cristã de Míchkin não será fácil. De um lado, temos
Rogójin, profundo niilista, de quem Míchkin se aproxima desde a primeira
cena do romance, quando ambos voltam a São Petersburgo na mesma cabine
do trem. Míchkin sentirá por Rogójin compaixão e amizade. Rogójin, por
sua vez, anuncia desde os primórdios de suas conversas com Míchkin que
seria capaz de matar a belíssima Nastácia Filíppovna, por quem se sente
profundamente apaixonado. E~eis que o imbróglio dostoievskiano acaba
transformando Nastácia primeiramente em esposa de Míchkin e, depois, em
amante de Rogójin. Míchkin se apieda pelo passado tétrico de Nastácia,
que fora explorada desde cedo por Totski, aristocrata lascivo que,
entrevendo a beleza vindoura da então adolescente, passou a mantê"-la
como concubina no ``chalé das delícias'', a casa de campo de suas
orgias. A~paixão doentia de Rogójin de fato leva Nastácia ao patíbulo.
Rogójin assassina aquela que também havia sido a bem"-amada de Míchkin,
aquela que o Cristo quixotesco de Dostoiévski tanto queria redimir. E
agora, Príncipe, que fazer? Se Míchkin julgar Rogójin sem mais --
``assassino impiedoso!'' --, o que acontecerá com a lógica piedosa do
Sermão da Montanha? Mas, ora, quem fere os dez mandamentos não deve ser
apedrejado? Que diz Jesus Míchkin a esse respeito? Assim narrou o
Príncipe Quixote: os fariseus levam a Cristo uma adúltera. Segundo a lei
consuetudinária, a mulher deve ser apedrejada fora dos muros da cidade.
Se Jesus corroborasse tal lei, obedeceria à tradição, mas renegaria o
Sermão da Montanha e a lógica da compaixão. Se, por outro lado, Cristo
abraçasse a adúltera, a lei de Moisés seria enxovalhada. Que fazer? Eis
uma dicotomia inescapável entre a justiça e o amor, a clava e o perdão.
Que fazer?

\begin{quote}
Jesus se inclinou para frente e escrevia com o dedo na terra. Como todos
insistissem, ergueu"-se e disse"-lhes: ``Quem de vós estiver sem pecado,
seja o primeiro a lhe atirar uma pedra''. Inclinando"-se novamente,
escrevia na terra. A~essas palavras, sentindo"-se acusados pela sua
própria consciência, eles se foram retirando um por um, até o último, a
começar pelos mais idosos (\versal{JOÃO}, 8, 6-9).
\end{quote}

Se a compaixão de Cristo abre os braços para afagar o e perdoar ao
assassino, que dizer do corpo inerte e esfaqueado de Nastácia a clamar
por justiça? Rogójin, o homicida, deve voltar a conviver em meio à
sociedade que ele ultrajou? Que fazer se houver uma nova falta, um novo
assassínio? Será mesmo possível abrir mão da retaliação, do evangelho
segundo Talião, para oferecer a outra face? A~seguinte passagem do
evangelho não nos deixa dúvidas sobre o ímpeto de perdão de Cristo:
``Então Pedro se aproximou dele e disse: `Senhor, quantas vezes devo
perdoar a meu irmão, quando ele pecar contra mim? Até sete vezes?'
Respondeu Jesus: `Não digo até sete vezes, mas até setenta vezes sete'"
(\versal{MATEUS}, 18, 21-22). Para a humanidade acostumada à lógica do bode
expiatório, à necessidade de encontrar alguém a quem culpar, o perdão
caridoso que só faz oferecer a outra face pode se confundir com
sucessivas notas promissórias para voltar a infringir. Ademais, o Sermão
da Montanha de Jesus Míchkin também não tem efetivo enraizamento
histórico. Se o protagonista de \emph{O idiota} quiser conciliar o
perdão ao assassino Rogójin com as condolências por sua amada Nastácia,
terá que passar por uma cisão extremamente destrutiva. Parece impossível
haver uma síntese. Rogójin e Nastácia se repelem com tanta força quanto
dois polos que possuem a mesma carga eletromagnética. Para permanecer
cristão, isto é, para abraçar a ambos, que poderá Míchkin fazer? A
solução que Dostoiévski oferece para o impasse de Míchkin torna a
resolução ainda mais irresoluta -- eis o movimento da vontade que se
volta contra si mesma a despeito da égide histórica do cálculo
utilitário.

Como Míchkin não pode conciliar o perdão a Rogójin com a piedade por
Nastácia neste mundo, o Quixote de Dostoiévski acaba realizando a
imitação de Cristo: o Príncipe Míchkin se oferece em holocausto, sua
razão se cinde,~\emph{o idiota} de fato fica louco. Com o estilhaçamento
de si mesmo, com a crucificação de sua sanidade, Míchkin permanece tão
cristão quanto Dom Quixote continua a ser um cavaleiro medieval em meio
à modernidade. Somente o sonho pôde resguardá"-lo. Ainda assim, não se
pode dizer que Míchkin optou entre Rogójin ou Nastácia. Diante da
impossibilidade de viver em comunidade com ambos, Dostoiévski narra uma
resolução que permanece historicamente irresoluta para que seu novo
Cristo prolongue o dilema que vitimou o Messias original.

Rogójin, Nastácia e o Príncipe Míchkin levaram suas contradições
constitutivas às últimas consequências. Desde o início, Rogójin anunciou
que poderia vir a assassinar Nastácia, ainda que ela fosse seu grande
amor. Nastácia, por sua vez, não conseguiu superar a humilhação que a
enredou no ``chalé das delícias'' do lascivo Totski. Míchkin lhe
estendeu a mão para um amor elevado, um amor que lhe pudesse trazer o
esquecimento em relação aos abusos e ignomínias do passado. Mas
Nastácia, tomada pelo ressentimento dostoievskiano, já não conseguia
projetar sua vida para além do charco da humilhação. Para Nastácia, o
cálculo utilitário se confunde com o mais rematado masoquismo -- o ego
se transforma em automutilação. Assim, torna"-se contraditoriamente
inteligível o fato de Nastácia ter abandonado o Príncipe Míchkin para se
entregar a seu assassino -- vale frisar que Nastácia bem pressentia as
intenções homicidas de Rogójin. A~humilhação tomou de todo sua vontade,
de modo que ela acabou se oferecendo em holocausto. Ora, já não dissemos
que esta foi justamente a solução irresoluta (e cristã) do Príncipe
Míchkin? Quem, então, pôde sair ileso? O~assassino Rogójin? Mas Rogójin
amava Nastácia, e seu amor humilhado pela dubiedade da moça que oscila
de sua órbita para a redenção ao lado de Míchkin o fere sem solução.
Assim, para sair redimido, o assassino Rogójin também precisa (se)
mutilar.

Após acompanharmos a expressão (autofágica) da vontade por meio das
trajetórias de Rogójin, Nastácia e Míchkin, o \emph{ethos} da pergunta
dostoievskiana mencionada mais acima volta a se insinuar: seria possível
imaginar alguma forma de organização social que pudesse dar vazão à
(des)vantagem quintessencial do homem?

\begin{quote}
Em uma tentativa de explicar o bolchevismo russo à senhora Ottoline
Morrell, Bertrand Russell certa vez notou que, ainda que se tratasse de
algo terrível, parecia o tipo certo de governo para a Rússia: ``Se você
se perguntar como as personagens de Dostoiévski deveriam ser governadas,
você acabará entendendo o bolchevismo'' (\versal{BERLIN}, 1994, p. xiii).
\end{quote}

A anedota (tragi)cômica de \emph{Sir} Bertrand Russell reverbera a
tensão dostoievskiana com propriedade. No entanto, seriam os extremos da
automutilação o caminho inexorável para os \emph{humilhados e
ofendidos?} Não haveria um sentido ulterior que pudesse não apenas
refrear, como também superar (\emph{aufheben}) o ressentimento que tende
a estilhaçar a expressão da personalidade e a faz voltar"-se contra si
mesma? Se esse sentido existir em meio à cosmovisão dostoievskiana, ele
poderá transcender a trajetória individual para ser partilhado como uma
teleologia histórica? Tais questões nos preparam para a totalização da
trajetória deste capítulo, cujas perguntas se irradiarão para ``O Sermão
da Estepe: Ivan Karamázov e a filosofia dostoievskiana da
história''\emph{.} Por ora, ainda precisamos estruturar mais mediações
para podermos arremessar a humilhação contra si mesma, a (des)vantagem
quintessencial contra a própria mutilação; em suma, Dostoiévski contra
Dostoiévski.

Voltemos à \emph{casa dos mortos.} Que outra figura social poderia
expressar a \emph{posse da personalidade} como sinônimo para aquilo que
estamos chamando de (des)vantagem quintessencial? Dostoiévski narrou uma
série de arroubos que constituiriam uma fenomenologia da \emph{busca do
próprio eu} -- a bebedeira à farta, uma nova transgressão por parte do
condenado, o assassínio de Nastácia, o holocausto de Míchkin. Mas que
figura social aparece como a mediação inequívoca para que os detentos --
dentro ou fora das muralhas da \emph{casa dos mortos --} possam exprimir
a \emph{vontade?}

\begin{quote}
O dinheiro tinha no presídio um valor enorme, essencial: era uma
potência. Posso afirmar de forma absoluta que o presidiário que tivesse
algum dinheiro, por menor que fosse a quantia, sofria dez vezes menos do
que o que não tivesse nenhum. \emph{A administração possuía o seguinte
ponto de vista: para que dinheiro, se o governo fornece tudo?!} {[}grifo
meu{]}. Repito, porém: se tirassem dos detentos a possibilidade de terem
dinheiro consigo, ou perderiam a razão ou morreriam como moscas (embora
providos de tudo); ou acabariam cometendo as piores faltas, estes por
desespero, aqueles para forçar novo julgamento e variar fosse lá como
fosse seu destino, ``já que estamos mesmo tralhados'', como acentuavam
(1982, p. 71).
\end{quote}

Teria Dostoiévski, crítico ferrenho da autofagia capitalista, se
transformado, neste momento, em um burguês? A~dialética sugere que a
negação do capitalismo em Dostoiévski é determinada. Em face do
formigueiro humano que subsume a personalidade e a despe de todas as
suas expressões fenomenológicas -- entre as quais o dinheiro e sua
materialização como propriedade privada --, Dostoiévski cerraria
fileiras com a burguesia \emph{liberal} (os primórdios do capitalismo)
para apreender as profundas contradições em que o socialismo se
enredaria. Partindo da liberdade abstrata, matemática e absoluta, a
utopia construiria o socialismo de caserna expresso pelo autoritarismo
da \emph{casa dos mortos}\footnote{Neste momento, é importante empregar
  a escatologia criativa de Dostoiévski contra suas próprias premissas.
  O~capitalismo tardio já não é liberal -- o ideário de pequenos
  \emph{shopkeepers} perfilados sob a batuta da livre concorrência há
  muito foi solapado pelos trustes e cartéis multinacionais que
  amputaram a mão invisível. Assim, seria importante pensar na
  reconfiguração histórica do papel do dinheiro que, à época de
  Dostoiévski, apresentava um caráter mais liberal do que em nosso
  contexto de recrudescimento da indústria (cultural). Já entrevimos que
  a crítica dostoievskiana à utopia socialista de forma alguma excluía o
  Ocidente capitalista -- o autor bem apreendeu uma contiguidade
  profunda entre o sentido da civilização e o ímpeto do formigueiro.
  Assim, o Dostoiévski escatológico, neste momento, mostra"-se mais
  dialético do que o Dostoiévski liberal que se aferra ao dinheiro como
  momento de expressão da personalidade. A~escatologia negativa de
  Dostoiévski prenunciou o refinamento estatístico da indústria cultural
  que vai mapeando, classificando e hipnotizando os sujeitos reduzidos a
  consumidores para que o dinheiro, em sua \emph{aparência} de livre
  realização do desejo, acabe referendando o autoritarismo das posições
  ideologicamente construídas \emph{a priori}. Se a lógica
  artístico"-histórica de Dostoiévski prenunciou a indústria cultural, é
  importante arremessar o escritor contra si mesmo para descobrir os
  limites de suas posições mais parciais e vinculadas a determinadas
  premissas de sua época. Ainda assim, o colapso do socialismo real -- o
  socialismo de migalhas para a esmagadora maioria daqueles que não eram
  e não são admitidos na camarilha de Chigáliov -- nos mostra que a
  Teoria Crítica também precisa pensar contra si mesma para apreender,
  em termos relacionais, os momentos de maior liberdade propiciados
  pelas relações de mercado -- seus limites, contradições e
  possibilidades progressistas em face do socialismo de caserna.}\emph{.}
A personalidade, então, teria a propriedade privada como uma de suas
expressões e extensões.

Mas isso não faria de Dostoiévski um defensor da cultura burguesa? Na
verdade, como se pode depreender a partir da (contra-)argumentação
dostoievskiana, o dinheiro e a propriedade privada não esgotam a
expressão da personalidade -- eles estabelecem mediações fundamentais
para que os desejos do homem venham à tona. Dostoiévski não pretende
estruturar a história em função do crédito -- o que o preocupa
profundamente é a supressão de tais categorias como maneira de
restringir as possibilidades de expressão do homem. O~escritor bem
entrevia as limitações do capitalismo para dar vazão ao amplo leque de
fenomenologia da vontade. Mas tais limitações seriam levadas às últimas
consequências se o chigaliovismo executasse o seu projeto utópico de
escravidão\footnote{Em estreito diálogo com o chigaliovismo apreendido
  por Dostoiévski, podemos mencionar a seguinte passagem
  histórico"-escatológica de \emph{A alma do homem sob o socialismo}, de
  Oscar Wilde: ``Se o socialismo for autoritário; se houver governos
  armados de poder econômico como estão agora armados de poder político;
  se, numa palavra, houver tiranias industriais, então o derradeiro
  estado do homem será ainda pior que o primeiro'' (2003, pp. 18-19).}.

Neste momento, ao imbricarmos o devir concreto e contraditório da
história, a noção espectral e ainda mais contraditória da busca pelo
próprio eu e a negação (determinada) da razão (utilitária), deparamos
com uma nova personagem dostoievskiana: Razumíkhin (\emph{Crime e
castigo}), nome derivado de
разум (\emph{razum};
razão), aquele que nos teria a apresentar uma possível síntese entre a
potência da razão e sua vinculação telúrica com a realidade \emph{viva}:

\begin{quote}
Para eles não é só a humanidade -- que se desenvolveu pela via histórica
e \emph{viva} até o fim -- que vai finalmente converter"-se numa
sociedade normal, mas, ao contrário, é o sistema social que, saindo de
alguma cabeça de matemático, vai imediatamente organizar toda a
sociedade e num abrir e fechar de olhos a tornará justa e pura antes de
qualquer processo vivo, sem qualquer via histórica e viva! É~por isso
que eles detestam tão instintivamente a história: nela veem ``só
deformidades e tolices'', e tudo se explica exclusivamente pela tolice!
É por isso que detestam o processo \emph{vivo} da vida: a \emph{alma
viva} é dispensável! A~alma viva exige vida, a alma viva não obedece à
mecânica, a alma viva é desconfiada, a alma viva é retrógrada! E~mesmo
que cheire a carniça, pode ser feita de borracha, mas aí não é viva, aí
não tem vontade, aí é escrava, incapaz de rebelar"-se! E~daí resulta que
no falanstério reduziram tudo a uma simples alvenaria de tijolos e à
disposição de corredores e quartos! O~falanstério está pronto, mas a
natureza dos senhores ainda não está pronta para o falanstério, ela quer
a vida, ainda não concluiu o processo vital, é cedo para ir para o
cemitério! (2001, pp. 265-266)
\end{quote}

O fragmento de Razumíkhin nos traz uma série de (contra)pontos. Em
primeiro lugar, Razumíkhin fala sobre a possibilidade do desenvolvimento
da humanidade ``pela via histórica e \emph{viva} até o fim''. Vemos,
então, que à história de proveta da \emph{intelligentsia} contrapõe"-se
um movimento histórico efetivo e \emph{vivo} que tem uma finalidade.
Ora, que Razumíkhin quer dizer com tal \emph{vivacidade?} E~qual seria a
teleologia da história?

Ademais, o aparte de Razumíkhin nos diz que o processo \emph{vivo} da
história se imbrica, necessariamente, às contradições que se foram
confundindo com a condição humana. Mas, se há uma finalidade, poderíamos
pensar que a teleologia procurará superar (\emph{aufheben}) o holocausto
de Nastácia Filíppovna para que ela possa encontrar a si mesma em um
novo patamar de desenvolvimento -- o perdão que transforma o devir em
esquecimento (negação determinada) para não mais estacar em meio ao
charco do ressentimento. E~há, também, uma nova colocação em relação ao
falanstério: ``O falanstério está pronto, mas a natureza dos senhores
ainda não está pronta para o falanstério, ela quer a vida, ainda não
concluiu o processo vital, é cedo para ir para o cemitério!'' É~como se
a utopia estivesse saltando por sobre o movimento da história para
projetar o seu fim, quando na verdade se trata de desarraigar velhos
hábitos para fazer com que a natureza humana \emph{viva} possa continuar
a se contradizer em novos patamares qualitativos. Só que, neste momento,
a voz dostoievskiana já não prescinde \emph{in toto} do falanstério. Por
sinal, a dialética faz com que o falanstério, antes muito próximo da
\emph{casa dos mortos,} agora se transforme em fim ulterior do
``processo vital''. Como o homem não estaria pronto para o falanstério,
viver sob sua égide, \emph{agora}, seria estancar a história. Razumíkhin
fala em \emph{cemitério}, em paralisia do movimento. Mas, antes de
prosseguirmos, analisemos com mais minúcia o novo imbróglio em que se
imbricou a teleologia dostoievskiana.

Razumíkhin é o antípoda de Raskólnikov e, dialeticamente, seu melhor
amigo. Enquanto Raskólnikov projeta uma nova humanidade a partir dos
escombros de Deus; enquanto o protagonista de \emph{Crime e castigo}
pretende fundar a nova era sobre e contra o decálogo de Moisés por meio
do \emph{matarás}; enquanto a história, para Raskólnikov, deveria ser
impulsionada pelo cálculo do grande legislador, Razumíkhin utiliza sua
\emph{razum} (razão) para se imbricar com a vida. Estudante de Direito
como Raskólnikov, Razumíkhin se põe a dar aulas particulares, escreve
ensaios para revistas e periódicos, projeta planos de empreendimentos
editoriais e literários. Seu desenvolvimento se quer telúrico, atrelado
às possibilidades concretas sem se resignar diante dos freios e da
inércia sociais. Enquanto Raskólnikov impulsiona seu cálculo utilitário
com o hedonismo mais rematado, Razumíkhin trabalha -- e espera. De um
lado, o duplo homicídio como terra arrasada para forjar uma nova era
abstratamente configurada; de outro, o trabalho vinculado à vocação, o
cálculo voltado para a vida, para o contraste efetivo com a realidade, a
\emph{razum} de Razumíkhin. Não à toa, antes de ser condenado à Sibéria,
Raskólnikov felicita Dúnietchka, sua irmã, pelo fato de ela pretender se
casar com Razumíkhin. Eis um dos momentos simbólicos de sugestão
dostoievskiana para além do lodo em que Raskólnikov se afundara -- um
marco para um recomeço. Mas parte considerável da fortuna crítica se
perde na floresta negra de Dostoiévski -- ou melhor, na taiga russa -- e
não apreende os sutis momentos de síntese que o escritor apresenta. É
bem verdade que tais momentos mais se assemelham a estilhaços diante da
vasta estepe de dor, rancor e transgressão que Dostoiévski descortina.
Precisamente neste sentido me parece importante perceber a dialética
polifônica do escritor que faz com que determinadas vozes, ainda que
combalidas e frágeis, se apresentem como pergaminhos a serem
arremessados dentro de garrafas de Pandora em meio ao mar contraditório,
\emph{vivo} e teleológico da história.

Se voltarmos agora ao \emph{subsolo} de nosso paradoxalista, a crítica à
razão, mediada por Razumíkhin, passará a se movimentar pela negação
determinada que recupera os momentos de verdade da racionalidade
arremessada contra si mesma:

\begin{quote}
Embora o homem já tenha aprendido por vezes a ver tudo com mais clareza
do que na época bárbara, ainda está longe de ter"-se \emph{acostumado} a
agir do modo que lhe é indicado pela razão e pelas ciências. (\ldots) Que
prazer se pode ter em desejar segundo uma tabela? Mais ainda: no mesmo
instante, o homem se transformará num pedal de órgão ou algo semelhante;
pois, que é um homem sem desejos, sem vontades nem caprichos, senão um
pedal de órgão? Que pensais disso? Calculemos as probabilidades: pode
tal coisa acontecer ou não? (\ldots) A~razão, meus senhores, é coisa boa,
não há dúvida, mas razão é só razão e satisfaz apenas a capacidade
racional do homem, enquanto o ato de querer constitui a manifestação de
toda a vida, isto é, de toda a vida humana, com a razão e com todo o
coçar"-se. E, embora a nossa vida, nessa manifestação, resulte muitas
vezes em algo bem ignóbil, é sempre a vida e não apenas a extração de
uma raiz quadrada. (\ldots) Que sabe a razão? Somente aquilo que teve tempo
de conhecer (\ldots), enquanto a natureza humana age em sua totalidade, com
tudo o que nela existe de consciente e inconsciente, e, embora minta,
continua vivendo (2000a, p. 37; p. 40; e p. 41).
\end{quote}

Para o homem do subsolo, o real pode vir a se tornar racional, porque o
racional ainda não é de todo real. O~trecho anterior prenuncia as
descobertas da psicanálise. A~razão dá conta de um vasto espectro de
experiências humanas, mas pouca ingerência tem (e ainda pouco conhece)
sobre as regiões nebulosas da alma -- zona que o século \versal{XX} viria a
chamar de psique. O~homem está para além de sua capacidade racional, e
volto a frisar que tal perspectiva orientou correntes da fortuna crítica
a considerar Dostoiévski um escritor que defende, inexoravelmente, as
contradições insolúveis da condição humana. Ora, neste momento já
podemos perceber que sua crítica à razão é dialeticamente determinada.
Diante da \emph{intelligentsia} revolucionária que pretende arregimentar
os homens em meio ao falanstério da \emph{casa dos mortos}, o escritor
se mostra um importante defensor da expressão centrífuga, não"-utilitária
e contraditória do homem.

\begin{quote}
O homem, seja ele quem for, sempre e em toda parte gostou de agir a seu
bel"-prazer e \emph{nunca} segundo lhe ordenam a razão e o interesse;
pode"-se desejar ir contra a própria vantagem e, às vezes,
\emph{decididamente se deve} (\ldots). Uma vontade que seja nossa, livre,
um capricho nosso, ainda que dos mais absurdos (\ldots) -- tudo isto
constitui aquela vantagem das vantagens (\ldots) que não se enquadra em
nenhuma classificação e devido à qual todos os sistemas e teorias se
desmancham continuamente, com todos os diabos! (\ldots) O~homem precisa
unicamente de uma vontade \emph{independente}, custe o que custar essa
independência e leve aonde levar (2000a, p. 39).
\end{quote}

No entanto, a obra de Dostoiévski também apreende os momentos de verdade
da utopia socialista \emph{contra} o irracionalismo mais rematado. Que
os homens se mutilem e se voltem contra si mesmos não significa que não
possa haver um ímpeto por redenção, uma lógica para além da humilhação.
E Dostoiévski, a reboque de profícuas reflexões sobre a filosofia da
história, tende a nos mostrar que tal transmutação da condição humana
precisa de solo \emph{vivo} e dinâmico para poder transcender a condição
meramente individual e se transformar em um falanstério verdadeiramente
orgânico -- em uma organização social. Caso tal organização, no entanto,
se ponha a padronizar os homens como os pintinhos de um galinheiro, a
infração e o caos se insinuarão como a história real em face das tabelas
logarítmicas da \emph{intelligentsia} para a transformação social.

\begin{quote}
Não estou propriamente defendendo o sofrimento e tampouco a
prosperidade. Defendo\ldots o meu capricho e que ele me seja assegurado,
quando necessário. (\ldots) Se, em lugar do palácio, existir um galinheiro,
e se começar a chover, talvez eu trepe no galinheiro, a fim de não me
molhar; mas, assim mesmo, não tomarei o galinheiro por um palácio, por
gratidão, pelo fato de me ter protegido da chuva. Estais rindo, dizeis
até que, neste caso, galinheiro e palácio são a mesma coisa. Sim,
respondo, se fosse preciso viver unicamente para não me molhar. Mas que
fazer se eu próprio meti na cabeça que não é apenas para isto que se
vive e que, se se trata de viver, deve"-se fazê"-lo num palácio? É~a minha
vontade, o meu desejo. Somente o podereis desarraigar de dentro de mim
quando transformardes os meus desejos. Bem, modificai"-os, seduzi"-me com
algo diverso, dai"-me outro ideal. Mas, por enquanto, não tomarei o
galinheiro por um palácio. (\ldots) Não considerarei como o coroamento dos
meus desejos um prédio de aluguel com apartamentos para inquilinos
pobres e contratos por um prazo de mil anos (2000a, pp. 48-49).
\end{quote}

Bem vemos que Dostoiévski arremessa a utopia contra si mesma para que
ela revele suas aporias em face das tentativas (contraditórias) de
expressão da personalidade -- a busca insone pela própria
individualidade, ainda que o encontro consigo mesma signifique a
autodestruição. Dialeticamente, então, arremessemos também a
automutilação contra si mesma: continuaria a haver movimento quando a
personalidade deparasse com o próprio despenhadeiro? A~fortuna crítica
que transforma Dostoiévski em um apologista da resignação -- o escritor
da condição humana à revelia de suas contingências e transformações --
teria que lidar com a noção de \emph{processo vital} que Razumíkhin traz
à tona. Mas qual seria a imagem dostoievskiana da maldade mais rematada
e do caos universal? Qual seria o ápice do torpor individual que levaria
a personalidade a se fragmentar de forma irredimível? Esse sofrimento
seria inescapável ou, no charco do torpor, no ápice da crise, a
dialética se insinuaria para que o indivíduo tentasse se encontrar em si
e para além de si mesmo? Eis que a \emph{casa dos mortos} apresenta a
Dostoiévski a utopia que se reduziu à mais completa distopia:

\begin{quote}
Certa vez estive a pensar: para se aniquilar um ser humano livre,
castigá"-lo sem nexo ou, em vez de um homem livre, se se quisesse fazer
um facínora virar um covarde com a só ideia de trabalho, \emph{bastaria
que àquele e a este se desse trabalho do caráter mais absurdo e inútil
possível}. Os~trabalhos forçados das organizações penitenciárias ou de
degredo, por menos interesses imediatos e individuais que apresentem
para o detento, pelo menos significam um trabalho que há de beneficiar a
outrem, digamos assim, e \emph{cuja realização tem uma lógica
utilitária}. O~forçado durante o trabalho se considera operário
provisório, é pedreiro, abre alicerce, mistura cal, cimento, terra,
levanta parede, serra, corta; nisso se aplica, tem um plano a cumprir e
ultimar. Não raro se interessa, capricha, colabora. Mas se em tais horas
de tarefa lhe ordenarem levar água dum depósito para outro até enchê"-lo,
depois esvaziá"-lo, indo encher o que antes esvaziou, ou fosse desfiar
areia num crivo, ou transportar terra de um canto para outro, depois
transferi"-la de novo para o local anterior, estou em mim que isso,
aquilo ou aquiloutro, ao cabo de uma semana, se tanto, o irritaria a tal
ponto que preferiria se enforcar ou então cometeria desatinos de
possesso, não aturando tal vilania nem tormento. Essa espécie de castigo
seria insensatez hedionda, tortura macabra e \emph{inutilidade perversa}
afetando não só a vítima como os mandantes {[}grifos meus{]} (1982, p.
21).
\end{quote}

No limite \emph{negativo}, Dostoiévski interpreta o trabalho
completamente inútil -- metáfora para a vida tautológica sem qualquer
sentido que a transcenda -- como o mal por excelência que estilhaçaria
tanto carrascos quanto escravos. Escatologicamente, a inutilidade
completa tem um efeito tão deletério quanto o formigueiro utópico. O
galinheiro ao menos permite que a vida \emph{viva} flua pelas infrações
que irrompem contra a tabela logarítmica dos planos abstratos de
transformação social; a ruptura completa da utilidade, por sua vez,
destrói qualquer possibilidade teleológica para o indivíduo. É~como se a
imortalidade \emph{negativa} fosse concedida ao condenado para que sua
consciência se reduzisse à indiferença da água que ele carrega
tautologicamente de um lado para o outro do pátio, sem que sua atividade
possua qualquer finalidade para além do vaivém inútil. A~morte, nesse
sentido, deixaria de ser uma temeridade. A~morte o redimiria. Padecendo
do mesmo infortúnio, carrasco e detento viveriam o mito de Sísifo:
condenado a rolar uma pesada pedra morro acima sem a possibilidade de
alcançar o ápice, Sísifo vê sua sobrevivência reduzida a mero arremedo
por conta da sentença tautológica.

Eis um importante ponto de inflexão que faz o ateu (e dostoievskiano)
Albert Camus sentenciar: ``É preciso imaginar Sísifo feliz'' (2006,
p. 168). Ao abandonar qualquer possibilidade de transcendência, Camus
transita de Dostoiévski para Nietzsche de modo a buscar o instante"-já e
a redenção tangíveis como fragmentos de felicidade. Camus está
amplamente ciente da contrapartida hedonista e utilitária de Raskólnikov
para a morte de Deus. Ainda assim, sua lógica mundana (e mediterrânea)
busca a reconciliação aquém da transcendência. A~finitude que se vê
emparedada pela morte -- ou pior, pela consciência da morte -- tenta
imaginar o sorriso de Sísifo a despeito de sua condenação irrevogável. A
felicidade talvez se tornasse mais intensa ao ter consciência do
cadafalso que a vai decepar. ``Ou não'' -- poderia insinuar a dúvida
constitutiva do homem do subsolo. Nesse sentido, Nietzsche fala sobre a
importância do \emph{esquecimento} como condição fundamental não para a
\emph{superação} da condição histórico"-consciente, mas para o
enraizamento fugaz do ser em seu instante, para a ruminação do que, por
ora, ainda lhe é tangível.

\begin{quote}
Toda ação exige esquecimento, assim como toda vida orgânica exige não
somente luz, mas também a escuridão. Um homem que quisesse sentir as
coisas de maneira absoluta e exclusivamente histórica seria semelhante
àquele que fosse obrigado a se privar do sono, ou a um animal que só
pudesse viver continuamente dos mesmos alimentos. É~portanto possível
viver, \emph{e mesmo ser feliz} {[}grifo meu{]}, quase sem qualquer
lembrança, como o demonstra o animal; mas é absolutamente impossível
viver sem esquecimento (2005, p. 362).
\end{quote}

Munido de sua volubilidade contumaz, o homem do subsolo talvez
perguntasse a Nietzsche se ele trocaria seu pathos filosófico pela
insciência feliz do animal. Ainda assim, o paradoxalista do subsolo
reconheceria uma convergência entre o esquecimento nietzscheano e o
ímpeto para o perdão em Dostoiévski. Enquanto Nietzsche discorre sobre o
esquecimento como mote fundamental para que o
eu"-que"-se"-sabe"-para"-o"-cadafalso não seja diluído por sua consciência
histórica, Dostoiévski entreveria o esquecimento como o ímpeto para a
superação da crise, como o salto dialético para o perdão. Este será um
aspecto fundamental para a discussão do capítulo 5 a respeito do
niilismo do homem ridículo que, à beira do suicídio e dialeticamente,
reencontra o sentido das coisas. Para o ser que descobre a eternidade --
o devir do ser --, o esquecimento deixa de ser a tentativa de driblar o
tempo para se reconhecer como a ruptura do rancor contumaz. Discutiremos
tais questões de forma pormenorizada também no capítulo ``O Sermão da
Estepe: Ivan Karamázov e a filosofia dostoievskiana da história''. Por
ora, podemos apreender que, se desistisse de rolar a pedra de sua vida,
Sísifo seria imediatamente esmagado; se continuasse, a finitude daria
cabo de sua esperança. Pois essa seria a imagem da existência sem Deus
para Dostoiévski. Aqui, já não se trata tão"-somente da transcendência
como salvaguarda para os valores morais. Trata"-se da eternidade como
possibilidade de cura, como tentativa de libertação da cativa Pandora.

Dostoiévski entreviu laivos e lampejos de superação da finitude
tautológica na própria \emph{casa dos mortos}, em suas imediações, em
meio àquelas

\begin{quote}
pessoas cuja finalidade na vida se reduz a tratar fraternalmente os
``infelizes'' {[}os condenados{]}, compadecendo"-se deles, tratando"-os
como a seus próprios filhos, com um desvelo inefável. (\ldots) Há quem diga
(e também já li em alguma parte) que o mais alto amor sentido pelo
próximo não passa no fundo de mero egoísmo. Onde em tal caso pudesse
haver egoísmo é coisa com que não atino (1982, pp. 74-75).
\end{quote}

Os jogos de palavra sempre podem reverter a abnegação -- vale dizer, a
negação do ego -- em utilitarismo escatológico e masoquista do eu que
sente prazer na renúncia a si mesmo, no mais fundamental altruísmo. Mas
Dostoiévski entreviu a excepcionalidade da ação daqueles e daquelas que
transformaram a própria vida em renúncia ao eu pelo outro. Para o
escritor, veremos que se tratava do sentido da eternidade -- a
teleologia teológica da história -- que impulsionava a caridade de
Sísifo. Ademais, o que poderia motivar um cativo condenado a mais de 20
anos de \emph{casa dos mortos? }

\begin{quote}
Seja o prisioneiro quem for, e seja outrossim curto o seu tempo de
detenção, não se habitua a considerar a sua sorte como algo definitivo,
positivo, como parte da sua existência real. Cada detento acaba por se
considerar na prisão não como morador e sim como hóspede. A~pena de
vinte anos toma a configuração de apenas dois anos e nada lhe modifica a
convicção de que quando deixar o presídio com 55 anos já não seja uma
pessoa válida de 35 anos. Pensa que então a vida voltará e arreda todo e
qualquer pensamento duvidoso e desagradável relativamente a tal assunto.
Mesmo os de tempo indeterminado, da seção especial, não raro esperam que
alguma alteração possa vir de Petersburgo. ``Remeta"-se o detento Fulano
por tanto tempo para os trabalhos de minas em Nertchinsk''. Seria
formidável! Primeiro, dali até Nertchinsk, só a viagem demora quase seis
meses; depois é muito mais agradável marchar no meio de uma leva
correcional do que estacionar num presídio! E~uma vez decorrido o prazo
lá em Nertchinsk, então\ldots E~assim pensam homens já de cabelos brancos!
(\ldots) No fundo, porém, pensam na vez de se verem livres das cadeias. E
para quê? Ora, para quê! Para, deixando aquela masmorra de abóbadas
baixas e de muralhas espessas, irem para um presídio com um pátio onde
possam andar\ldots Só, e nada mais. Liberdade, rua, estrada? Nunca mais!
Cada qual deles sabe muito bem que os acorrentados à barragem ficarão
perpetuamente no presídio, até a morte, com algemas nos pulsos e nos
tornozelos. Sabe disso e todavia sua mais bela esperança, seu mais
ardente desejo é que o tempo de viver chumbado às lajes passe logo. Como
suportaria estar assim acorrentado durante cinco ou mais anos, se não
embalasse essa esperança? Morreria ou endoideceria. Sem tal crença, como
resistir? (1982, pp. 88-89)
\end{quote}

Dostoiévski nos fala, então, sobre a projeção de um outro à revelia da
espada de Dâmocles que há de nos calar. Eis o caráter utópico do
escritor que arremessa o falanstério contra si mesmo para extrair dos
projetos de transformação da humanidade suas aporias e o sentido de sua
superação. Se, diante do cadafalso, Dostoiévski recorre à resistência de
Pandora, precisamos agora discernir quais seriam os momentos
propositivos de sua utopia e quais implicações dela adviriam. Ao
escavarmos esse novo subsolo, chegaremos a uma dualidade antitética que
arremessará Dostoiévski contra si mesmo para separarmos o joio do trigo:
o Dostoiévski datado e afeito à eslavofilia rente ao \emph{ethos}
nacionalista do século \versal{XIX} e o Dostoiévski, a meu ver, transtemporal,
cuja universalidade abre a possibilidade de investigações para além de
suas afinidades eletivas com a Ortodoxia russa e o anacronismo
histórico. Eis a tese da Rússia redentora e a antítese da história como
teleologia teológica -- e como teologia teleológica -- para muito além
do messianismo nacional. Este último Dostoiévski lançará as bases para,
no próximo capítulo, encontrarmos Aliócha e Ivan Karamázov, Jesus Cristo
e o grande inquisidor, como articuladores do Sermão da Estepe. Por ora,
estabeleçamos as mediações que farão a contraposição do Dostoiévski
eslavófilo"-medievalista em relação ao Dostoiévski que se enreda às
implicações dialéticas do \emph{processo vital.}

\section{3.4. O prenúncio do Sermão da Estepe para além da montanha
nacional}

As \emph{Notas de inverno sobre impressões de verão} trazem à tona um
Dostoiévski que reflete sobre as possibilidades emancipatórias da utopia
tomando o Ocidente como contraponto negativo em relação às
características autóctones da Rússia. Mas suas ideias cruzam a fronteira
das obras, e as formulações do Dostoiévski viajante vão tomando forma e
conteúdo nos demais trabalhos. Acompanhemos, então, as mediações que nos
levarão a esta primeira proposição dostoievskiana que erige a Rússia
ortodoxa e tsarista como uma cidadela em relação à corrosão
individualista do Ocidente.

\begin{quote}
A razão revelou"-se inconsistente ante a realidade e, além disso, os
próprios homens de razão, os próprios sábios, começam a ensinar agora
que a razão pura nem existe no mundo, que não existem as conclusões da
razão pura, que a lógica abstrata é inaplicável à humanidade, que existe
a razão de Ivan, de Piotr, de Gustave, mas que a razão pura nunca
existiu; que tudo isto não passa de uma invenção do século \versal{XVIII},
destituída de fundamento (2000b, p. 129).
\end{quote}

A alusão do trecho dostoievskiano a Immanuel Kant é clara. Nesse
sentido, precisamos estabelecer algumas mediações para que possamos
compreender quais são as decorrências da aniquilação da razão pura para
o acirramento capitalista das relações de Ivan com -- ou pior, contra --
Piotr e Gustave. Primeiramente, para continuarmos próximos às reflexões
dostoievskianas que procuram apreender o embasamento ético em correlação
com a reciprocidade do Sermão da Montanha, mencionemos uma passagem de
\emph{A grande transformação,} obra da estudiosa de religiões Karen
Armstrong, em meio à qual a base kantiana para a ação racional, o
imperativo categórico ético, poderia ser tida como uma secularização da
regra de ouro presente nas mais diversas tradições religiosas:

\begin{quote}
Não faças aos outros o que não farias a ti mesmo. (\ldots) Se, por exemplo,
toda vez que nos sentíssemos tentados a tecer um comentário hostil sobre
um colega, um irmão ou um país inimigo, pensássemos em como nos
sentiríamos se o mesmo comentário se referisse a nós -- e desistíssemos
de tecê"-lo --, superaríamos a nós mesmos. Esse seria um momento de
transcendência. Se tal atitude se tornasse habitual, poderíamos viver
num \emph{ekstasis} permanente, não porque entramos num transe exótico,
mas porque ultrapassamos as fronteiras do egocentrismo. (\ldots) O~teste é
simples: se nossas convicções -- seculares ou religiosas -- nos tornam
hostis, intolerantes e maldosos em relação à fé alheia, não são
``profícuas''. Se, porém, nos impelem a agir compassivamente e a honrar
o estranho, são boas, úteis e válidas. Esse é o teste da verdadeira
religiosidade em todas as grandes tradições (2008, p. 14; pp. 413-414).
\end{quote}

Após a apreensão de que a regra de ouro pressupõe a ação do ego para
além de si mesmo, isto é, a ação do ego em (cor)relação com a alteridade
tendo por base a universalização da regra de ouro como imperativo
categórico ético, podemos mencionar duas passagens da \emph{Dialética do
Esclarecimento}, em meio às quais Adorno e Horkheimer apreendem o
potencial emancipatório da razão kantiana para a transformação radical
da sociedade (tese) e a antítese que nela se instaura com a reificação
da razão como o cálculo utilitário do capital, aspecto que já havia
despontado com a tensão em meio ao trecho dostoievskiano envolvendo a
razão universal e os interesses conflitantes de Ivan, Piotr e Gustave.

\begin{quote}
{[}Tese:{]} O princípio kantiano de ``fazer tudo com base na máxima de
sua vontade enquanto tal, de tal modo que essa vontade possa ao mesmo
tempo ter por objeto a si mesma como uma vontade legisladora
universal''. (\ldots) A~emancipação incondicional determinada como a
essência do esclarecimento (1985, p. 108).
\end{quote}

Em seguida, como havíamos dito anteriormente, e a reboque do
\emph{ethos} dostoievskiano que procura pensar a contrapelo das próprias
aspirações emancipatórias, os frankfurtianos trazem à tona a antítese
que explicita as aporias da regra de ouro kantiana para a construção da
utopia:

\begin{quote}
Os conceitos kantianos são ambíguos. A~razão contém como ego
transcendental supraindividual a Ideia de uma convivência baseada na
liberdade, na qual os homens se organizem como um sujeito universal e
superem o conflito entre a razão pura e a empírica na solidariedade
consciente do todo. A~Ideia desse convívio representa a verdadeira
universalidade, a Utopia. \emph{Mas, ao mesmo tempo, a razão constitui a
instância do pensamento calculador que prepara o mundo para os fins da
autoconservação e não conhece nenhuma outra função senão a de preparar o
objeto a partir de um mero material sensorial como material para a
subjugação} {[}grifo meu{]} (1985, p. 83).
\end{quote}

Se a contradição objetiva instalada na razão pura -- e historicamente
configurada -- pressupõe a projeção da utopia como a universalidade
secular da regra de ouro que lança as bases para a convivência de Ivan,
Piotr e Gustave, ao mesmo tempo em que arregimenta o utilitarismo
instrumental para (re)produzir a dominação como instância de projeção do
e pensamento sobre o mundo -- o ego como ferramenta atuarial --, o que
poderia refrear a versão capitalista da guerra de todos contra todos? O
universalismo da revolução proletária? Dostoiévski, em Londres,
arremessa o proletariado contra seu próprio (e suposto) ímpeto
messiânico:

\begin{quote}
A quem {[}os burgueses devem{]} temer então? Aos operários? Mas os
próprios operários são, no íntimo, proprietários: todo o seu ideal
consiste em se tornar proprietários e acumular o maior número possível
de objetos; assim é a natureza. A~natureza não é concedida em vão. Tudo
isso foi cultivado e educado durante séculos. Uma racionalidade não se
transforma facilmente, não é fácil abandonar hábitos seculares,
penetrados na carne e no sangue (2000b, p. 129).
\end{quote}

A dialética dostoievskiana entrevê contiguidades fundamentais nos
movimentos que a aparência pressupõe como antitéticos\footnote{Para uma
  análise historicamente profícua sobre o trabalho e o proletariado como
  momentos intracategoriais fundamentais para a maturação histórica do
  capitalismo -- ao invés da tradicional via da luta de classes que
  contrapõe capital e trabalho como tese e antítese inequívocas --,
  conferir o \emph{Manifesto contra o trabalho}, do \emph{Grupo Krisis}.
  Disponível em:
  \emph{http://o-beco.planetaclix.pt/mctp.htm}.
  Consulta feita no dia 01/03/18. O~teórico social alemão Robert Kurz
  (1943-2012), um dos autores do \emph{Manifesto,} propõe uma leitura
  historicamente instigante do capitalismo de Estado socialista.
  Enquanto a Europa Ocidental teve a seu dispor quatro séculos de
  escravidão para consumar aquilo que Marx chamaria de acumulação
  primitiva de capitais e transfigurar, na carne e no sangue (como diria
  Dostoiévski), os valores feudais de forma que se fossem modificando as
  práticas consuetudinárias em correlação com a nova forma de
  metabolismo social, a Rússia deu o salto mortal do tsarismo para o
  moderno sistema produtor de mercadorias sem mediações paulatinas.
  Assim, Kurz apreende os campos de concentração sino"-soviéticos como o
  \emph{subsolo} da história de modernização compulsória de tais países
  para que eles pudessem competir no sistema"-mundo do capital. O~navio
  negreiro e a \emph{plantation} além"-mar equivaleriam, \emph{mutatis
  mutandi}, ao arquipélago Gulag de Stálin e Mao Tsé"-Tung. Conferir
  \emph{O colapso da modernização: da derrocada do socialismo de caserna
  à crise da economia mundial.} Rio de Janeiro: Paz e Terra, 2004.}.
Assim, chegamos a um cenário de terra arrasada em que a utopia se vê
desprovida de suas bases de constituição. Mas Dostoiévski traz à tona a
noção de \emph{natureza} como o devir dos usos e costumes que se vão
cristalizando como características identitárias de um povo. ``Não é
fácil abandonar hábitos seculares, penetrados na carne e no sangue''. No
entanto, tal colocação pressupõe (i) a possibilidade do tempo como
antítese e transformação e (ii) o questionamento sobre se a corrosão
ocidental dos valores se estenderia irrestritamente por todos os lugares
da mesma forma, ou se haveria alguma nação que teria condições de
resistir à autofagia capitalista -- a derrocada da civilização -- por
conta de suas características autóctones que fazem parte de sua carne e
de seu sangue.

Mais adiante, apresentaremos a possibilidade do tempo (eternidade) como
superação como uma antítese em relação à Rússia idealizada por
Dostoiévski como antídoto para a patologia capitalista. Por ora,
precisamos analisar os momentos de esgarçamento da fraternidade no
Ocidente para apreendermos as características de salvaguarda que a Mãe
Rússia\footnote{Os países bálticos, a Polônia e a Ucrânia,
  historicamente acossados pelo imperialismo russo, bem conhecem as
  contradições que a salvaguarda da Mãe Rússia lhes oferece.}
supostamente conteria em sua sociedade. Em \emph{Notas de inverno sobre
impressões de verão,} o narrador dostoievskiano procura refletir sobre
as tentativas (aporéticas) do Ocidente para derivar a fraternidade da
revolução social:

\begin{quote}
A fraternidade. Ora, este ponto é o mais curioso e, deve"-se confessar,
constitui no Ocidente, até hoje, a principal pedra de toque. O~ocidental
refere"-se a ela como a grande força que move os homens e não percebe que
não há de onde tirá"-la, se ela não existe na realidade. O~que fazer,
portanto? É~preciso criar a fraternidade, custe o que custar.
Verifica"-se, porém, que não se pode fazer a fraternidade, porque ela se
faz por si, concede"-se por si, é encontrada na natureza. Todavia, na
natureza do francês e, em geral, na do homem do Ocidente, ela não é
encontrada, mas sim o princípio pessoal, individual, o princípio da
acentuada autodefesa, da autorrealização, da autodeterminação em seu
próprio Eu, da oposição deste Eu a toda a natureza e a todas as demais
pessoas, na qualidade de princípio independente e isolado, absolutamente
igual e do mesmo valor que tudo o que existe além dele. Ora, uma tal
autoafirmação não podia dar origem à fraternidade (2000b, p. 130-131).
\end{quote}

Liberdade político"-econômica, igualdade jurídico"-política, mas e quanto
à fraternidade? Para Dostoiévski, o Ocidente, no próprio devir da Igreja
Católica como continuação de Roma, já teria arraigado o poder, o ateísmo
e o poder do ateísmo de maneira profunda à carne e ao sangue das pessoas
que, em meio ao avassalador desenvolvimento do capitalismo, só fariam
pensar nos próprios benefícios em detrimento de todos os
demais\footnote{A esse respeito, Kate Holland (2000) argumenta que
  ``Dostoiévski estabeleceu em seu \emph{Diário de um escritor} uma
  visão de mundo que se voltava para o triunfo da `ideia russa'. Sua
  visão utópico"-religiosa está baseada em uma crença na obsolescência do
  cristianismo ocidental. De acordo com Dostoiévski, o catolicismo
  comprometera o ideal cristão com sua ideia de uma união compulsória da
  humanidade sob a autoridade do Papa e, assim, sua base ideológica dera
  vazão ao socialismo e à ideia de união compulsória da humanidade sob a
  bandeira do materialismo, do egoísmo e do racionalismo. A
  contrapartida religiosa do catolicismo, o protestantismo, era definida
  em termos de sua oposição ao catolicismo e, portanto, não tinha
  autoridade espiritual. O~escritor acreditava que essas forças logo
  entrariam em colisão, abrindo caminho para uma nova ordem na Europa, a
  partir da qual ascenderia o espírito redentor da Ortodoxia incorporada
  pelo povo russo, a única religião a conter a `verdadeira
  universalidade' da imagem de Cristo e, portanto, a única religião
  capaz de salvar a Europa da fragmentação'' (p. 102). Na sequência da
  argumentação, falaremos sobre o último trecho da argumentação de
  Holland, o qual se volta para o messianismo russófilo de Dostoiévski
  como (suposto) antídoto para a corrosão ocidental.}. Se o ethos da
divindade já não faz parte dos usos e costumes, a não ser como uma
evocação tradicional cada vez mais contingente -- eis a imagem do fiel
que faz o Pelo Sinal diante de uma catedral e dá graças a Deus por uma
conquista que logrou arrancar ao darwinismo social --, o que poderia
refrear a luta fratricida das classes e o utilitarismo dos indivíduos?
Dostoiévski arremessa Marx contra si mesmo ao dizer que a felicidade de
um é a condição para a infelicidade de todos. O~todo social se erige
pela usurpação da riqueza; assim, a massa desprovida precisa lutar
encarniçadamente para tentar abocanhar o seu quinhão. A~suposta
fraternidade que rege a convivência dos homens em sociedade se torna a
coação da necessidade\footnote{Também seria possível lançar mão da
  escatologia dostoievskiana para refletir sobre as bases que
  estruturariam a sociedade comunista caso a utopia fosse erigida.
  Passaríamos do cotidiano ultracompetitivo à solidariedade com a tomada
  do poder e a reconfiguração material das relações sociais? Certamente
  a competitividade acintosa e estrutural teria menos motivo para
  ocorrer, mas e quanto à carne e ao sangue historicamente configurados?
  Tais pontos vêm sendo discutidos com vários matizes ao longo deste
  capítulo. Assim, gostaria de trazer uma projeção de Hannah Arendt,
  mencionada por Marshall Berman, que arremessa o otimismo iluminista de
  Marx contra as bases etéreas que fundariam a sociedade comunista.
  ``Arendt compreende a extensão do individualismo que subjaz ao
  comunismo de Marx e compreende também os rumos niilistas que esse
  individualismo poderá tomar. Em uma sociedade comunista, onde o livre
  desenvolvimento de cada um é a condição para o livre desenvolvimento
  de todos, \emph{o que poderá manter reunidos esses indivíduos
  livremente desenvolvidos?} {[}grifo meu{]} Eles talvez partilhem a
  busca comum de um infinito bem"-estar experimental; todavia, isso não
  seria `o verdadeiro domínio público, mas apenas atividades privadas,
  soltas no espaço aberto'. Uma sociedade como essa poderia
  perfeitamente vir a experimentar uma sensação coletiva de futilidade:
  `a futilidade de uma vida que não se fixa nem se afirma em qualquer
  objetivo permanente, o qual perdure para além do esforço despendido'"
  (\versal{BERMAN}, 1986, p. 124). Os~esquerdistas bem poderíamos arremessar
  Arendt contra si mesma e lhe perguntar se o intento de transformação
  não valeria a pena em face da atual esfera pública aglutinada pela
  usurpação e pela privação. Ainda assim, a escatologia dostoievskiana
  de Arendt é bastante crítica para fazer com que a utopia tenha que
  pensar a contrapelo de si mesma. E~se a esfera pública de Utópolis se
  desagregasse porque o indivíduo, que já não precisa se preocupar com a
  privação, agora quer apenas rolar pela grama? Este rapaz ainda seria
  uma alma tranquila. E~quanto àqueles que não desdenhariam do poder
  como herança histórica da humanidade? Que tipo de poder seria
  desencadeado se a privação fosse arrefecida? Não haveria a revolta por
  parte dos ressentidos como a nostalgia da hierarquia? Nesse sentido, o
  comentário de Marshall Berman (1986, p. 338) à escatologia de Hannah
  Arendt parece prolongar as aporias e contradições da utopia:
  ``Observe"-se que, de acordo com o pensamento de Marx, \emph{o domínio
  público dos valores e discursos comuns subsistiria e floresceria
  enquanto o comunismo tivesse a característica de movimento de
  oposição}; \emph{ele perderia seu vigor somente onde esse movimento
  tivesse triunfado e se esforçado} (em vão, sem nenhum domínio público)
  por inaugurar uma sociedade comunista'' {[}grifos meus{]}.}.

Mas qual seria, então, a base para a fraternidade? O~narrador
dostoievskiano de \emph{Notas de inverno sobre impressões de verão}
prossegue:

\begin{quote}
Na fraternidade autêntica, não é uma personalidade isolada, um Eu, que
deve cuidar do direito de sua equivalência e equilíbrio em relação a
tudo o \emph{mais}, e sim todo este o \emph{mais} é que deveria chegar
por si a essa personalidade que exige direitos, a esse Eu isolado, e
espontaneamente, sem que ele o peça, reconhecê"-lo equivalente e de
iguais direitos a si mesmo, isto é, a tudo o mais que existe no mundo.
Mais ainda, esta mesma personalidade revoltada e exigente deveria
começar por sacrificar todo o seu Eu, toda a sua pessoa, à sociedade, e
não só não exigir o seu direito, mas, pelo contrário, cedê"-lo à
sociedade, sem quaisquer condições. \emph{Mas a personalidade ocidental
não está acostumada a um tal desenvolvimento dos fatos} {[}grifo meu{]}:
ela exige à força o seu direito, ela quer \emph{participar}, e disso não
resulta fraternidade. Está claro: ela pode transformar"-se. Mas
semelhante transformação leva milênios, porque tais ideias devem antes
entrar na carne e no sangue para se tornarem realidade (2000b, p.
131)\footnote{Nesse sentido, Robert Louis"-Jackson (1981) discorre sobre
  as fontes e derivações da utopia dostoievskiana: ``O ideal de
  Dostoiévski -- estabelecido em \emph{Notas de inverno sobre impressões
  de verão} -- estabelecia que tanto o indivíduo quanto a sociedade
  fizessem suas demandas absolutas de forma recíproca. O~ideal social de
  Dostoiévski era influenciado pelo ideal cristão de amor e
  autossacrifício; e, é claro, ele partia do \emph{Contrato social}, de
  Rousseau, na medida em que rejeitava categoricamente a noção de que,
  na sociedade ideal -- conforme Rousseau a concebia --, o malfeitor
  ``talvez se veja forçado a ser livre''. \emph{Nada é mais alheio a
  Dostoiévski do que a compulsão e a utopia reacionária, que poderia ser
  concebida em termos racionalistas ou teológicos} {[}grifo meu{]}.
  (\ldots) O~ideal social e espiritual de Dostoiévski era certamente
  utópico, no melhor sentido da palavra, mas era o único pelo qual ele
  sentia que valia a pena lutar'' (p. 8). É~importante frisar que,
  segundo Jackson, Dostoiévski seria totalmente contrário à utopia
  compulsória e reacionária tanto em seu matiz utilitário"-racional
  (diatribe contra a \emph{intelligentsia} de esquerda) quanto em seu
  matiz teológico (diatribe contra os eslavófilos que se contrapunham à
  modernidade e à modernização da Rússia), situação que dificulta o
  posicionamento unilateral do escritor em um exclusivo espectro
  político"-social.}.
\end{quote}

Dostoiévski contrapõe a luta secular por direitos -- da qual ele
inclusive participou ao ser adepto do Círculo de Petrachévski -- à
\emph{organicidade} que torna recíprocos direitos e deveres dos
indivíduos em meio ao todo social. O~russo Dostoiévski não vê tal
princípio como base do desenvolvimento ocidental. A~luta -- ou, para
usar um termo mais próprio à análise vindoura de \emph{Os irmãos
Karamázov}, a \emph{revolta} -- só levaria à dissonância entre as vozes
dos indivíduos e a polifonia social. O~desenvolvimento da
\emph{personalidade} no Ocidente, para este Dostoiévski eslavófilo, não
levaria à superação do caráter centrífugo das lutas. É~bem verdade que a
fortuna crítica mais conservadora se esquece da menção de Dostoiévski à
efetiva possibilidade de mudança das identidades historicamente
configuradas, mas o escritor, rente ao processo mais contraditório de
transformação dos usos e costumes, entrevê que o Ocidente levaria
milênios para encontrar uma reciprocidade que fizesse com que a demanda
das partes fratricidas -- e não fraternas -- se transformasse em
abnegação mútua. Ambos perdemos para ambos ganharmos. Não, tudo teria
que ser arrancado à força, de modo revolucionário, ainda que a revolução
logo se transformasse no movimento que se volta contra si mesmo para
conter o ímpeto de mudanças quando os revolucionários tomassem o poder
-- não é esta uma síntese para a autofagia das revoluções francesa e
russa? Mas, neste momento, é preciso perguntar a Dostoiévski sem mais:
onde seria possível encontrar tal fraternidade orgânica em contraposição
ao fratricídio ocidental?

Eis, então, que o bufão Liébediev volta à tona para nos contar uma
estória medieval, uma estória de um tempo em que a divindade lastreava
as relações e se confundia com a consciência moral do todo e de cada uma
de suas partes. Trata"-se da trajetória de um monge que, em uma época de
extrema inanição em meados do século \versal{XII}, havia comido 60 pessoas. Ao
contrário do espírito moderno e individualista que levaria o monge a se
eximir de seus atos -- o anacronismo da culpa e do castigo em face do
crime contumaz --, o religioso acaba se entregando. Liébediev, então,
quer entender o ímpeto de autoexpiação do monge e a completa
excepcionalidade de tal ato em nossos tempos.

\begin{quote}
Que tormentos o esperavam naquele tempo, que rodas, fogueiras e fogos?
Quem o impeliu a ir denunciar"-se? Por que não parar simplesmente na casa
dos sessenta, mantendo o segredo até o último alento? Por que não largar
simplesmente o clero para lá e viver arrependido como um ermitão? Aí
está a decifração. Quer dizer que houve algo mais forte do que as
fogueiras e o fogo e até mais forte que um hábito de vinte anos!
\emph{Quer dizer que havia um pensamento mais forte do que todas as
desgraças, más colheitas, torturas, lepra, maldições e toda sorte de
inferno que a humanidade não suportaria sem um pensamento que
concatenasse, orientasse o coração e fertilizasse as fontes da vida!}
{[}grifo meu{]} Mostrem"-me os senhores algo semelhante a tal força em
nosso século de vícios e estradas de ferro\ldots isto é, é preciso dizer,
em nosso século de vícios e estradas de ferro, porque eu sou um
beberrão, porém justo! Mostrem"-me uma ideia que ligue e agregue a atual
sociedade humana ao menos com a metade daquela força que havia naqueles
séculos. E~atrevam"-se a dizer, por fim, que não se debilitaram, que não
se turvaram as fontes da vida sob essa ``estrela'', sob essa rede que
prende os homens. (\ldots) Há mais riquezas, porém menos força; não resta
mais uma ideia agregadora (2002, p. 423).
\end{quote}

Liébediev nos leva a um Dostoiévski menos elíptico, menos negativo, e
agora começamos a entrever o que significa a vida \emph{viva}, a
natureza \emph{viva}, o processo \emph{vital}\footnote{Leitor contumaz
  de Dostoiévski, Nietzsche interpreta o vitalismo de Liébediev de forma
  bastante distinta daquela realizada pelo escritor russo. Como veremos,
  a noção de vida \emph{viva}, para Dostoiévski, provém justamente da
  fusão deste mundo com a espiritualidade, do sentido histórico de
  realização do espírito e do sentido espiritual de realização
  histórica. O~vitalismo nietzscheano, ao se distanciar da
  espiritualidade de Dostoiévski, afirma a vida \emph{hic et nunc} e a
  transforma em sentido a ser superado ao longo deste que seria o único
  tempo de que dispomos -- na verdade, Nietzsche está totalmente ciente
  de que é o tempo que dispõe de nós, daí o vitalismo pulsional a se
  vincular de forma entranhada com a vida tangível a ser erigida como
  uma obra de arte, o simulacro último do sentido agora compreendido
  apenas em sua dimensão imanente.}\emph{.} A~noção medievalista a que
Liébediev se refere diz respeito a um princípio de solidariedade social
que remete cada uma das partes constitutivas da sociedade à (re)produção
moral do todo. Para além do princípio de sobrevivência, do princípio de
perpetuação do eu -- manutenção do ego contra os e para além dos demais
--, o medievo se estruturava, segundo Liébediev, pela coesão orgânica do
todo. Se \emph{não matarás} é a norma, e o Sermão da Montanha traz o
sentido da redenção, como viver fora de tais fronteiras? Como cruzar os
limites sem ser arrebatado por um profundo arrependimento? Liébediev
transforma o monge em uma personagem típica capaz de congregar a
cosmovisão de uma época. Os~suplícios da crucificação se mostraram
menores do que o alívio moral daquele que volta a se encontrar com o
princípio que o socializou.

Em contrapartida -- e aqui assumimos momentaneamente a voz de Liébediev
--, que princípio poderia refrear a contumácia utilitária na moderna
sociedade reprodutora de mercadorias? Os~jovens Marx e Engels, no
\emph{Manifesto comunista}, estavam fortemente impressionados com o
poder revolucionário da burguesia:

\begin{quote}
A burguesia, em seu reinado de apenas um século, gerou um poder de
produção mais massivo e colossal do que todas as gerações anteriores
reunidas. Submissão das forças da natureza ao homem, maquinário,
aplicação da química à agricultura e à indústria, navegação a vapor,
ferrovias, telegrafia elétrica, esvaziamento de continentes inteiros
para o cultivo, canalização de rios, populações inteiras expulsas de seu
\emph{habitat} -- que século, antes, pôde sequer sonhar que esse poder
produtivo dormia no seio do trabalho social? (2002, p. 49)
\end{quote}

A base de ação do capitalismo é a contínua revolução de todas as
relações sociais, a profanação do sagrado como relíquia ossificada, o
niilismo constante que sempre deve morder a própria cauda. Ora, o que
isso significa em termos das relações intersubjetivas? Se o todo me
obriga a me suplantar continuamente e se as partes -- seres humanos como
eu -- também são estimuladas a fazê"-lo, o que me refreará quando as
colisões forem necessárias, quando a infelicidade de todos for a
condição para a minha felicidade? Ainda uma vez, a Teoria Crítica
apreende as limitações da regra de ouro kantiana para fundar uma ética
universal em meio ao mundo que prescinde de Deus: ``O burguês que
deixasse escapar um lucro pelo motivo kantiano do respeito à mera forma
da lei não seria esclarecido, mas supersticioso -- um tolo'' (\versal{ADORNO} e
\versal{HORKHEIMER}, 1985, p. 85). Emancipação como progresso e regresso --
revolução material como niilismo moral. Liébediev nos faz pensar, a
contrapelo do Iluminismo e na esteira de seu progresso também
regressivo, que o ego liberto pelo capitalismo não encontrou um
princípio que o levasse de volta à sociedade como a organicidade para
além do cálculo utilitário de Nicolau Maquiavel e Thomas Hobbes.

Diante desse quadro tétrico, eis que o desespero escatológico de
Dostoiévski faz com que o escritor diagnostique a Rússia tsarista como a
contrapartida social para a autofagia do Ocidente. Conforme argumenta
Susan McReynolds (2002), ``a nova Rússia e o mundo russo que Dostoiévski
entrevê se organizam como totalidades harmônicas e orgânicas'' (p. 91).
Eis a Rússia e sua (suposta) organicidade de classes/estamentos; a
Rússia embasada sobre a comunidade rural de trabalho coletivo e
abençoada pela Igreja Ortodoxa; a Rússia cuja resultante vetorial aponta
para o tsar. Em suma, o povo teóforo que, em sua constituição autóctone,
em seu desenvolvimento ainda não maculado pelo capitalismo que já tomara
o Ocidente, poderia apresentar uma nova palavra ao mundo, um novo
sentido de desenvolvimento\footnote{Há uma verdadeira pletora de fontes
  para o debate entre ocidentalistas e eslavófilos na Rússia
  oitocentista de Dostoiévski. Em termos polares, os ocidentalistas
  propugnavam pela reconfiguração da Rússia segundo os modelos da
  modernidade ocidental, ao passo que os eslavófilos consideravam que a
  Rússia tinha que seguir um caminho autóctone para desenvolver sua
  civilização. Tais posições extremadas são matizadas por uma série de
  correntes que buscavam sínteses tensas e contraditórias entre o
  ocidentalismo e a eslavofilia. O~próprio Dostoiévski pode ser
  entendido como um escritor que buscou narrar as (im)possibilidades de
  conciliação entre a trajetória do Ocidente e o devir autóctone da
  Rússia. Para mais detalhes sobre os diálogos e tensões entre
  ocidentalistas e eslavófilos, conferir, sobretudo, o primeiro e o
  terceiro tomos da obra que Joseph Frank escreveu sobre a vida e a obra
  de Dostoiévski em estreita correlação com seu campo
  histórico"-cultural: \emph{Dostoiévski: as sementes da revolta,
  1821-1849.} São Paulo: Edusp, 1999 e \emph{Dostoiévski: os efeitos da
  libertação, 1860-1865}. São Paulo: Edusp, 2002. Conferir, também, os
  seguintes capítulos da obra de Andrzej Walicki, \emph{A History of
  Russian Thought: From Enlightenment to Marxism} (Stanford: Stanford
  University Press, 1979): 4. \emph{Anti"-Enlightenment trends in early
  Nineteenth Century}; 6. \emph{The Slavophiles}; 8. \emph{Belinksy and
  different variants of Westernism}; 10. \emph{The origins of ``Russian
  socialism''}. Uma terceira fonte para consulta em relação ao debate
  entre ocidentalistas e eslavófilos é a obra \emph{Pensadores russos,}
  de Isaiah Berlin (São Paulo: Companhia das Letras, 1988). Os~capítulos
  1. \emph{Russia and 1848} e a primeira seção do capítulo 4. \emph{Uma
  década notável}, 4.1. \emph{O nascimento da intelligentsia russa,}
  estabelecem um profícuo panorama do contexto em questão.}. Nesse
sentido, Chátov, um dos intelectuais revolucionários de \emph{Os
demônios}, procura discernir a força essencial de desenvolvimento dos
povos:

\begin{quote}
Os povos se constituem e são movidos por outra força que impele e
domina, mas cuja origem é desconhecida e inexplicável. Essa força é a
força do desejo insaciável de ir até o fim e que ao mesmo tempo nega o
fim. É~a força da confirmação constante e incansável do seu ser e da
negação da morte. O~princípio da vida, como dizem as Escrituras, são
``rios de água viva'' com cujo esgotamento o \emph{Apocalipse} tanto
ameaça (\versal{DOSTOIÉVSKI}, 2004, pp. 250-251).
\end{quote}

Walter Benjamin (2011), em um breve ensaio sobre o romance \emph{O
idiota}, discorre sobre o messianismo russófilo de Dostoiévski em termos
que se aproximam das e aprofundam as colocações de Chátov:

\begin{quote}
Dostoiévski retrata o destino do mundo em meio ao destino de seu próprio
povo. Esse ponto de vista é típico dos grandes nacionalistas, para quem
a humanidade só pode se desenvolver em meio a uma herança nacional
particuar. A~grandiosidade do romance é revelada pela maneira como as
leis metafísicas que governam o desenvolvimento da humanidade e as leis
que governam o desenvolvimento das nações se mostram absolutamente
interdependentes (p. 275).
\end{quote}

A força da confirmação constante e incansável do ser russo -- o
messianismo da nação -- seria mostrar que há uma possibilidade outra em
relação à autofagia ocidental, uma possibilidade em que cada classe
social reconhece sua posição referendada historicamente e clama por
direitos em termos orgânicos. A~hierarquia russa -- mais complexa,
dinâmica e contraditória do que os auspícios de Dostoiévski queriam
projetar -- poderia fornecer um princípio de refreamento da luta
intestina e inexorável entre as classes e os indivíduos, precisamente
porque a Rússia havia sido fundada sobre a purificação do sofrimento, a
\emph{kenosis}\footnote{``\emph{Kenosis}~é um conceito da~teologia
  cristã~que trata do~\emph{esvaziamento}~da vontade própria de uma
  pessoa e da aceitação da vontade de Deus. É~encontrado no Novo
  Testamento~como o esvaziamento de~Jesus. Os~primeiros cristãos da era
  apostólica procuravam viver segundo esse conceito de autoesvaziamento
  (\ldots). A~principal noção que lhe subjaz é a ideia de abandonar um
  estilo de vida egocêntrico para adotar uma vida altruísta, viver rumo
  a Cristo e se doar em serviço aos irmãos de fé. Os~primeiros cristãos
  incorporaram esse conceito com tamanha visceralidade, que chegaram a
  vender bens materiais e distribuir o valor arrecadado para os que
  tinham necessidade. Devido ao enraizamento da \emph{kenosis}, muitos
  cristãos perderam a vida na pregação do evangelho, pois mesmo diante
  das ameaças de morte os fiéis não conseguiam deixar de falar Daquele
  para quem viviam'' (\versal{ARMSTRONG}, 2008, p. 403). Victor Terras (2003),
  nesse sentido, argumenta que ``o Príncipe Míchkin participa da
  \emph{kenosis} de Cristo, esvaziando"-se de todas as honrarias e até
  mesmo da dignidade humana'' (p. 112), na medida em que procura
  conciliar, contra a sua própria sanidade, o perdão para Rogójin,
  assassino de Nastácia Filíppovna, com a justiça diante do assassínio.
  A~análise que fizemos anteriormente do desenlace de \emph{O idiota,}
  mediada pela colocação de Terras, dá um novo matiz à \emph{idiotia} de
  Míchkin com a noção cristã de \emph{kenosis. }} de Jesus Cristo. Nesse
sentido, acompanhemos o desdobramento teológico da \emph{kenosis} para
entendermos por que Dostoiévski entrevia afinidades fundamentais entre a
Rússia e o sentido social da abnegação:

\begin{quote}
Escrevendo a seus conversos de Filipos, na Macedônia, em meados da
década de 50 -- cerca de 25 anos após a morte de Jesus --, Paulo cita um
dos primeiros hinos compostos pelos cristãos, que mostra que, desde o
início, eles viam a missão de Jesus como uma \emph{kenosis} (Filipenses,
2, 6-11). O~hino começa assinalando que Jesus, como todos os seres
humanos, existia à imagem de Deus e, no entanto, não se ateve a esse
alto \emph{status}, \emph{porém se esvaziou/ para assumir a condição de
escravo.} (\ldots)/ \emph{E foi ainda mais humilde, a ponto de aceitar a
morte, morte na cruz.} (\ldots) Assim, {[}Paulo{]} apresenta o hino com
esta instrução aos cristãos de Filipos: ``Em vossa mente, deveis ser
iguais a Cristo Jesus''. Deviam tirar do coração o egoísmo e o orgulho.
Deviam estar unidos no amor, ``com um propósito comum e um pensamento
comum'' (Filipenses, 2, 5). ``Não deve haver competição entre vós, nem
vanglória; mas sede todos humildes. Sempre considerai o outro melhor que
vós, \emph{para que ninguém pense primeiro nos próprios interesses, mas
todos pensem nos interesses dos outros}'' (Filipenses, 2, 2-4).
Reverenciando seus semelhantes com esse altruísmo, entenderiam o
\emph{mythos} da \emph{kenosis} de Jesus. (\ldots) O~paradoxo ``amai vossos
inimigos'' (Sermão da Montanha) (\ldots) requer \emph{kenosis},
\emph{porque se deve ter benevolência sem esperança de retribuição}
{[}grifo meus{]} (\versal{ARMSTRONG}, 2008, pp. 404-407)\footnote{Segundo Victor
  Terras (2003), a mensagem fundamental de Dostoiévski é a de que ``a fé
  em Cristo é a única maneira de salvar o mundo, a Rússia e todos as
  pessoas do caos sem Deus em que estão chafurdados. A~fé em Jesus deve
  assumir a forma de uma \emph{imitatio Christi}'' (p. 111). Assim
  entendemos, em sentido ainda mais expandido, a colocação de Terras
  sobre a \emph{kenosis} realizada pelo Príncipe Míchkin.}.
\end{quote}

A teleologia teológica da Rússia, nesse sentido, sua missão
histórico"-messiânica, seria a encarnação de uma sociedade baseada na
\emph{kenosis}, uma nação vasta a acolher sua população e os demais
povos com o princípio da reciprocidade orgânica\footnote{Vale mencionar,
  ainda uma vez, que os países vizinhos ao imperialismo russo sabem
  historicamente que o \emph{acolhimento} denega a si mesmo em prol da
  \emph{anexação. }}. Dostoiévski pôde descobrir as raízes de tais
características autóctones dos russos justamente ao lidar tanto com os
prisioneiros da \emph{casa dos mortos} quanto com os abnegados que
dedicavam suas vidas a aplacar o sofrimento dos cativos. Os~mais cruéis
assassinos se persignavam diante da mensagem cristã dos piedosos que os
ajudavam. Ouvimos ainda uma vez os laivos de Liébediev: onde seria
possível encontrar em meio à sociedade moderna -- sobretudo em sua
versão de vanguarda no Ocidente -- tal princípio de altruísmo que passa
por sobre o próprio ego, que transforma a cessão ao outro em princípio
vital? Essa seria a sucessão \emph{vital}, o sentido da natureza
\emph{viva} que perpassa a obra dostoievskiana e que transforma a
história, neste momento, em coincidência com os rumos de uma nação que,
como Cristo, tinha uma nova palavra, uma boa nova a ser espraiada para a
humanidade\footnote{Nesse sentido, Birgit Harress (1999) argumenta que,
  ``pelo prisma de Dostoiévski, somente a Rússia está em posição de
  salvar o mundo com uma visão cristã. São as pessoas comuns que, com
  seu sentido de fraternidade, devoção e amor, podem levar a uma
  verdadeira vida de congregação. O~povo russo tem a função de ser um
  guia para aqueles que estão em perigo de se desviar. Dostoiévski
  acredita que a maioria se distancia de Deus e preferiria seguir o
  espírito da destruição. A~ênfase do escritor se volta para aqueles
  que, com seriedade, lutam para alcançar o caminho correto, mas que,
  devido a obstáculos exteriores, não conseguem encontrá"-lo. Trata"-se
  dos jovens russos [a \emph{intelligentsia} revolucionária] de sua
  época. O~autor se dirige a eles, tentando lhes dizer que procurem
  respostas em seu próprio povo, e não em culturas estrangeiras. A
  escolha entre as ideias orientais e ocidentais é idêntica a uma
  decisão pró ou contra Cristo. Se essa decisão seguir a palava dos
  evangelhos, então é possível falar em termos da renovação do homem''
  (pp. 19-20). Kate Holland (2000), como que a desdobrar a colocação de
  Harress, argumenta que ``o isolamento da \emph{intelligentsia} russa,
  de acordo com Dostoiévski, tinha origem no fato de que ela se apartara
  do \emph{народ}
  (\emph{narod}), o povo russo. O~escritor argumentava que o posso russo
  acreditava em um ideal que transcendia todas as utopias terrenas da
  \emph{intelligentsia}: o sacrifício salvífico de Cristo e a
  possibilidade de salvação eterna após a morte'' (p. 101). Veremos, nos
  próximos dois capítulos, que a salvação eterna após a morte -- ou, por
  outra, a noção de vida após a morte -- pode assumir, em Dostoiévski,
  matizes bastante distintos da vinculação à fé ortodoxa do povo russo.}.
A evolução histórica, nesse sentido, levaria em conta as construções e
contradições próprias aos povos para apresentar suas respectivas
contribuições. E, no caso da Rússia, Victor Terras (2003) retoma o
argumento do Príncipe Míchkin, para quem

\begin{quote}
o sentimento religioso do povo russo não pode ser racionalmente
definido, apesar de ser prontamente reconhecível. Ele compreende o amor
incondicional e absoluto a Deus e Sua criação, o perdão a todas as
injustiças sofridas e a firme crença de que todas as faltas de um
pecador arrependido serão perdoadas. A~crença coincide com a mensagem
messiânica, unitária e penitencial da Igreja Ortodoxa russa (p.
111)\footnote{O próximo capítulo se distanciará da conexão unívoca que
  Terras estabelece entre a Ortodoxia russa e a espiritualidade em
  Dostoiévski. A~sequência da argumentação trará o prenúncio da antítese
  em relação à tese russófila e ortodoxa.}.
\end{quote}

Em \emph{Notas de inverno sobre impressões de verão}, a trajetória de
consecução da missão nacional aparece em seus momentos constitutivos
fundamentais: ``Em primeiro lugar, se necessita de natureza; depois,
ciência; em seguida, de uma vida independente, telúrica, incontida e de
uma crença nas próprias forças nacionais'' (2000b, p. 98). Vemos, então,
que a racionalidade não é proscrita por Dostoiévski, mas ocupa um lugar
entre as demais forças que pretendem reverberar a fraternidade. O
cálculo sem enraizamento não passa de mero cálculo, ao passo que a
solidariedade telúrica reparte o pão enquanto a razão procura provar,
teoricamente, que a partilha é fundamental para suplantar o utilitarismo
entranhado na carne e no sangue ocidentais. O~credo nas forças
nacionais, então, refere"-se à percepção de que a Rússia, relativamente
ao Ocidente capitalista, poderia se estruturar em função da natureza
\emph{viva} por conter em seu desenvolvimento histórico as premissas
messiânicas da \emph{kenosis}. O~Cristo russo, segundo Dostoiévski,
seria o grande legado da nação aos demais povos, o princípio de sua
filosofia da história que entrelaçava teleologia e teologia\footnote{Serguei
  Hackel (1982) afirma que ``a pessoa ou, na verdade, a imagem de Cristo
  se torna uma das principais pedras de toque da perfeição moral na obra
  de Dostoiévski'' (p. 11).}. Seria possível, então, acompanhar o devir
da \emph{kenosis} como a organicidade a ser espraiada pelo \emph{pathos}
da cruz:

\begin{quote}
É preciso tornar"-se uma personalidade, e mesmo num grau muito mais
elevado do que o daquele que se definiu agora no Ocidente.
Compreendam"-me: o sacrifício de si mesmo em proveito de todos, um
sacrifício autodeterminado, de todo consciente e por ninguém obrigado, é
que constitui, a meu ver, o sinal do mais alto desenvolvimento da
personalidade, de seu máximo poderio, do mais elevado autodomínio, da
mais completa liberdade de seu arbítrio. Somente com o mais intenso
desenvolvimento da personalidade se pode sacrificar voluntariamente a
vida por todos, ir por todos para a cruz, para a fogueira. Uma
personalidade fortemente desenvolvida, plenamente cônscia do seu direito
de ser personalidade, que já não tem qualquer temor por si mesma, não
pode fazer outra coisa de si, isto é, dar"-se outra aplicação, senão se
entregar completamente a todos, para que todos os demais também sejam
personalidades igualmente plenas de direitos e felizes. É~uma lei da
natureza, o homem tende normalmente para isto. (\ldots) É~uma desgraça
fazer, neste caso, o menor cálculo sequer, no sentido da vantagem
pessoal (2000b, pp. 131-132).
\end{quote}

Não à toa a epígrafe de \emph{Os irmãos Karamázov} retoma a seguinte
passagem do \emph{Evangelho segundo João:} ``Em verdade, em verdade vos
digo: se o grão de trigo, caído na terra, não morrer, fica só; se
morrer, produz muito fruto. Quem ama a sua vida, perdê"-la"-á; mas quem
odeia a sua vida neste mundo, conservá"-la"-á para a vida eterna'' (12,
24-25). Desenvolver"-se significa \emph{des"-envolver} os laços que me
atam ao meu próprio ego, apreender as mediações que me configuram como
um ser socialmente relacional -- \emph{nós} ao invés do eu, a razão
solidária e emancipada que põe Ivan, Piotr e Gustave em comunicação, ao
invés da apreensão reificada de que Ivan, Piotr e Gustave só podem
existir em contraposição uns aos outros. Ademais, conforme veremos nos
dois próximos capítulos deste livro, ao longo dos quais investigaremos o
sentido da transcendência para Dostoiévski, a natureza \emph{viva}, o
processo \emph{vital}, suprime a morte em face da eternidade -- a nação
constituiria um momento fundamental de socialização da \emph{kenosis},
da sociedade como liturgia e como vivência da solidariedade.

Neste momento, a história, da qual Dostoiévski jamais procurou se
apartar, nos obriga a lançar os auspícios messiânicos do escritor contra
o efetivo devir da Rússia. À~luz do tsarismo feudal e de sua corrupção e
desigualdade sociais fortemente entranhadas, como não dizer que tal
Dostoiévski estaria fadado à lata de lixo da história se não pensasse a
contrapelo e para além do chauvinismo russo? De fato, Liébediev toca em
um ponto nevrálgico: a sociedade moderna solapou sua base moral e a
tornou fluida e passível de contínuos questionamentos que, no limite e
até mesmo no cotidiano, podem levar à mais completa anomia. Mas, ora, o
que queria nosso caro Liébediev? Que a humanidade não se constituísse e
continuasse a \emph{sobreviver}, ou pior, a \emph{subsistir} em feudos?
Que o caráter cíclico e estático das colheitas continuasse a determinar
por quem os sinos dobram? A~escatologia dostoievskiana, sempre a
caminhar pelo \emph{negativo} da dialética, comporta a antítese
histórica que busca um patamar de contradição mais universal e
universalizável do que o messianismo russo de instituições arcaicas,
anacrônicas e reacionárias que, felizmente, foram solapadas por Outubro
de 1917\footnote{Para perspectivas teóricas que se embasam no
  medievalismo de Dostoiévski para interpretá"-lo como um profeta
  antimoderno -- a despeito de a dialética dostoievskiana comportar o
  \emph{negativo} como um momento fundamental de sua constituição para
  continuar a caminhar pelas contradições do processo de totalização --,
  conferir Nikólai Berdiaev, \emph{L'esprit de Dostoïevski.} Paris:
  Stock, 1974, e Luiz Felipe Pondé, \emph{Crítica e profecia: a
  filosofia da religião em Dostoiévski}. São Paulo: Leya, 2013.}. Se o
homem pretende buscar um sentido para suas atividades, que o niilismo
seja um momento constitutivo fundamental para que da crise possa
despontar um novo nível de consciência. A~meu ver, então, se quisermos
que a filosofia dostoievskiana da história continue a projetar o devir
de modo escatológico, precisamos libertá"-la de seus momentos mais
datados e afeitos ao \emph{ethos} de seu tempo para que ela continue a
se metamorfosear à revelia de seus anseios \emph{positivos}, isto é, de
suas expectativas de enraizamento para que a roda dialética da história
ofereça algum alívio ao náufrago que não mais tem em que se escorar.
Nesse sentido, as \emph{Notas de inverno sobre impressões de verão}
sintetizam o esforço da comunhão:

\begin{quote}
É preciso sacrificar"-se justamente de tal modo que se entregue tudo e
até se deseje não receber nada de volta, e que ninguém se afane por
nossa causa. (\ldots) Que fazer então? Não se pode fazer nada, mas é
preciso \emph{que tudo se faça por si, que exista na natureza}, que seja
compreendido inconscientemente na natureza de todo um povo, numa
palavra, que haja um princípio fraterno, de amor: é preciso amar. É
preciso que se tenda instintivamente à fraternidade, à comunhão, à
concórdia e que se tenda apesar de todos os sofrimentos seculares da
nação, apesar da rudez bárbara e da ignorância, que se enraizaram nessa
nação, apesar da escravidão secular, das invasões estrangeiras, numa
palavra, que a necessidade da comunhão fraterna faça parte da natureza
do homem, que este nasça com ela ou tenha adquirido tal hábito através
dos séculos (2000b, pp. 132-133).
\end{quote}

Quando arremessamos o Dostoiévski moderno contra seu momento arcaizante,
a noção de \emph{natureza} mostra suas afinidades eletivas com a noção
de \emph{contingência} -- \emph{necessidade} e \emph{liberdade} se
tornam momentos recíprocos. A~Rússia feudal foi varrida do mapa. A
espontaneidade da nação, então, precisa lidar com o esfacelamento do
messianismo em face do sistema"-mundo do capital. Dostoiévski teria que
discorrer também sobre as contiguidades historicamente trágicas entre o
messianismo nacional e o etnocentrismo imperialista. Ademais, a
modernidade desconstrói a \emph{natureza} para transformá"-la em
\emph{liberdade determinada.} Se o homem do subsolo, paradigma do devir
ideológico das personagens dostoievskianas, trouxe à tona o ímpeto
indômito da personalidade para buscar a própria expressão -- a vantagem
das vantagens, no limite, como a própria desvantagem --, seria preciso
buscar a teleologia para além das limitações do solo pátrio, seguir a
história em suas novas totalizações, levar em conta a peculiaridade do
particular como um momento de verdade do universal que não se moveria
para subsumi"-lo, mas para encontrá"-lo como voz na polifonia do todo. O
Cristo russo, nesse sentido, seria um momento menor de Dostoiévski em
face da \emph{necessidade} universal da \emph{kenosis} pregada por
Paulo, o apóstolo dos gentios, o apóstolo do cristianismo para além dos
judeus, o apóstolo do universal para além do local, da igualdade para
além da diferença. Ora, basta sobrevoar o devir histórico de que
Dostoiévski não queria se distanciar para, na primeira escaramuça
nacional, perceber que o Cristo russo se transforma no Anticristo armado
em face da resistência dos poloneses em aceitar a subsunção de seu país
em função do processo \emph{vital} que não lhe é autóctone.

O desafio da (e à) teleologia teológica de Dostoiévski seria expandir o
Sermão da Montanha para a estepe da história, de modo que o \emph{amai o
próximo como a ti mesmo} se superasse como o amor ao \emph{distante} --
o \emph{exótico,} na linguagem da iminência neocolonial a que
Dostoiévski também se filiou. Dado o acúmulo de experiências que entrevê
as afinidades históricas entre o conceito de totalidade e a dominação
político"-econômica e dado o maior conhecimento do \emph{modus vivendi}
dos diferentes povos, o desafio da (e à) teleologia passa a ser a
construção de uma nova síntese que componha o concerto da diferença como
momento constitutivo da totalidade histórica. Já não falaríamos em
inevitabilidade, o processo vital de Dostoiévski comportaria avanços e
recalcitrâncias. Mas, para tanto, o tempo de cicatrização dos indivíduos
e dos povos precisaria se dilatar -- como construir a abnegação com o
horizonte limitado de uma única vida? Como refrear a autofagia que viola
o pacto entre as gerações se o \emph{homo economicus} não vislumbra nada
além de seu próprio desejo?

Dostoiévski vislumbra a fraternidade -- o sentido do desenvolvimento
histórico da humanidade, sua teleologia teológica -- como a
personalidade que diz à sociedade,

\begin{quote}
\emph{sem a menor coação, sem buscar qualquer vantagem} {[}grifo meu{]}:
``Somente unidos seremos fortes; tomai"-me todo, se precisais de mim, não
penseis em mim ao promulgar vossas leis, não vos preocupeis sequer,
entrego"-vos todos os meus direitos e, por favor, disponde de mim. Eis a
minha felicidade suprema: sacrificar"-vos tudo, e que isto não vos traga
qualquer desvantagem. Vou destruir"-me, fundir"-me com toda a indiferença,
contanto que a vossa fraternidade floresça e não morra''. E~a
fraternidade, ao contrário, deve dizer: ``Estás nos dando demais. Não
temos o direito de não aceitar de ti aquilo que nos entregas, pois tu
mesmo dizes que nisso consiste toda a tua felicidade; mas o que fazer se
nos dói incessantemente o coração por essa tua felicidade? Toma, pois,
tudo de nós também. Vamos esforçar"-nos constantemente, com todas as
forças, para que tenhas o máximo de liberdade individual, o máximo de
autoexpressão. Não temas agora nenhum inimigo, quer entre as pessoas,
quer na natureza. Colocamo"-nos todos em tua defesa, todos nós garantimos
a tua segurança, cuidamos incansavelmente de ti, porque somos irmãos,
somos todos teus irmãos, e somos numerosos e fortes; sê plenamente
tranquilo e de ânimo rijo, não temas nada e confia em nós''. Depois
disso, naturalmente, não há mais o que dividir, tudo se dividirá por si.
Amai"-vos uns aos outros, e tudo isto vos será concedido. Mas, realmente,
que utopia, meus senhores! (2000b, p. 133)
\end{quote}

Como dizer, então, que Dostoiévski rompeu de forma inexorável com a
utopia socialista? Bem vemos que, para o escritor russo, o socialismo
pode ser visto como um momento fundamental da teleologia histórica, que
o ímpeto socialista vislumbra o sentido para o qual a humanidade
\emph{pode} caminhar. No próximo capítulo, articularemos o socialismo
como momento essencial de outra visão da teologia teleológica de
Dostoiévski. Por ora, resta entender que o enraizamento da fraternidade
requer uma nova compreensão do que seria a dimensão histórica da
humanidade -- algo que as investigações espirituais de Dostoiévski
trouxeram de forma apenas latente. Que o falanstério não seja
obrigatório, mas que, ainda assim, seja o sentido histórico para além da
monadologia burguesa -- que o falanstério recupere a revolução
progressista do capitalismo que forjou a humanidade pela exploração, mas
que lança as bases para a integração mundial para além dos
provincianismos. Que Dostoiévski, ao ser arremessado contra seus
auspícios nacionalistas, pudesse perceber que a Rússia tendia para o
Ocidente -- que o mundo tendia para o capitalismo total. (Daí,
provavelmente, o desespero do Dostoiévski mais datado para transformar o
último quinhão do solo russo, escatologicamente, em bastião messiânico.)
E a reflexão dostoievskiana prossegue para trazer à tona uma aporia
fundamental da fraternidade não"-telúrica, a partilha \emph{a priori}, a
partilha coercitiva realizada pelos líderes que antes hasteavam a
bandeira da revolução:

\begin{quote}
Vendo que não há fraternidade, o socialista põe"-se a convencer as
pessoas à fraternidade. Ele quer produzir, compor a fraternidade. (\ldots)
Que fraternidade pode haver quando antecipadamente se faz a partilha e
se determina quanto cada um merece e o que cada qual deve fazer? (2000b,
p. 134)
\end{quote}

Se Dostoiévski projetou personagens"-antíteses em relação ao torpor e ao
sadismo (sadomasoquismo) do homem do subsolo, as noções de devir e de
superação lhe eram fundamentais. Assim, como superar o altruísmo
revolucionário que pretende impor a lógica da solidariedade e da
partilha? O~homem do subsolo prolonga a dúvida e volta a contrapor o
princípio indômito de expressão da personalidade ao socialismo das
formigas:

\begin{quote}
É certo que, em liberdade, espancam"-no, não lhe dão trabalho, ele {[}o
homem{]} morre de fome e não tem no fundo nenhuma liberdade, mas, apesar
de tudo, o original pensa que viver à sua vontade é sempre melhor.
Naturalmente, resta ao socialista apenas cuspir e dizer"-lhe que é um
imbecil, que não cresceu suficientemente, não amadureceu e não
compreende a sua própria vantagem; que uma formiga, uma insignificante
formiga, privada do dom da palavra, é mais inteligente que ele, pois no
formigueiro tudo é tão bom, tudo está arrumado e distribuído, todos
estão alimentados, felizes, cada qual conhece a sua tarefa; numa
palavra, o homem ainda está longe do formigueiro (2000b, p. 135).
\end{quote}

O formigueiro socialista, como ideia e mesmo como práxis deformada, já é
a superação do formigueiro medieval -- a modernidade forja a igualdade
\emph{negativa} sobre os antigos escombros da hierarquia orgânica,
imutável e abençoada. Igualdade \emph{negativa} para além do privilégio
aristocrático do não"-trabalho: agora, somos todos passíveis de
exploração. Mas a crise daí decorrente não gera apenas remorso e
paralisia. O~que poderia nos guiar para além do entorpecimento? Como
podemos voltar a nos mover?

Eis, então, que o \emph{demônio} Stiepan Trofímovitch, à beira da morte,
desponta de nossa letargia histórica. Ele sabe que a ideia de um outro
mundo sempre existiu, que a história como realização do sentido
pressupõe o alargamento completo de nosso corredor polonês -- o infinito
como ideia, o infinito como busca, a ideia de busca como o infinito.

\begin{quote}
Uma ideia que sempre existiu, segundo a qual existe algo infinitamente
mais justo e mais feliz do que eu, já me preenche todo com um
enternecimento infinito e -- com a glória -- oh, quem quer que eu tenha
sido, o que quer que tenha feito! Para o homem, muito mais necessário
que a própria felicidade é saber e, a cada instante, crer que em algum
lugar existe uma felicidade absoluta e serena, para todos e para tudo\ldots
Toda a lei da existência humana consiste apenas em que o homem sempre
pôde inclinar"-se diante do infinitamente grande. Se os homens forem
privados do infinitamente grande, não continuarão a viver e morrerão no
desespero. \emph{O desmedido e o infinito são tão necessários ao homem
como o pequeno planeta que ele habita\ldots} {[}grifo meu{]} (2004, p.
641).
\end{quote}

E se a crença puder dar lugar ao conhecimento? E~se o infinito for
tangível e disser respeito à \emph{natureza histórica} de que
Dostoiévski tanto fala ao longo de sua obra? ``Em verdade, em verdade
vos digo: se o grão de trigo, caído na terra, não morrer, fica só; se
morrer, produz muito fruto. Quem ama a sua vida, perdê"-la"-á; mas quem
odeia a sua vida neste mundo, conservá"-la"-á para a vida eterna'' (\versal{JOÃO},
12, 24-25). E~se a perda da vida for apenas o (re)encontro com uma nova
morada? ``Na casa de meu Pai há muitas moradas'' (\versal{IDEM}, 14, 2). A
teologia teleológica, então, teria que se confrontar com seus próprios
limites -- o Dostoiévski oitocentista teria que ser esgarçado por seu
próprio devir escatológico para que o Sermão da Montanha, agora com uma
nova consciência de época, voltasse a vislumbrar a estepe da história.

O próximo capítulo, ``O Sermão da Estepe: Ivan Karamázov e a filosofia
dostoievskiana da história'', estabelecerá uma discussão sobre a
teleologia teológica (e a teologia teleológica) de Dostoiévski a partir
de bases distintas daquelas que se vinculam ao messianismo russófilo e
ortodoxo. Nariman Skakov (2009), nesse sentido, discorre sobre as
leituras religiosas distintas e contrapostas que se estabelecem a partir
da obra de Dostoiévski:

\begin{quote}
A busca espiritual de Dostoiévski nunca foi tema de uma única
interpretação dominante -- na verdade, a diferença de opiniões teve
início enquanto o escritor vivia. O~pensador cristão russo Konstantin
Leont'ev escreveu o seguinte em 1880 a respeito do discurso de
Dostoiévski sobre Púchkin: ``O tom sobremaneira róseo que o discurso de
Dostoiévski introduz no cristianismo é uma novidade para a Igreja
Ortodoxa, a qual não espera que nada decente surja da humanidade no
futuro''. Um ano depois, no entanto, o obituário publicado no jornal
oficial da Igreja \emph{Strannik} (\emph{Peregrino}) descreveu
Dostoiévski como um ``cristão russo genuinamente fiel e profundo'', que
fora acusado erroneamente de misticismo. O~texto também reivindicou que
Dostoiévski demonstrava a ``grandiosidade'' da Ortodoxia russa. Essas
interpretações divergentes constituem tentativas de dar cabo e se
apropriar do discurso espiritual do escritor -- incluí"-lo na moldura
ortodoxa. No entanto, as colocações de Dostoiévski sobre religião são
notoriamente diversas e, por vezes, confusas. Sua totalidade reside na
própria incompletude (pp. 124-125).
\end{quote}

Tendo em vista a incompletude contraditória a que Skakov se refere,
procuraremos, nos próximos dois capítulos, estabelecer novas vinculações
histórico"-filosóficas entre os sermões da montanha e da estepe com base
na espiritualidade de Dostoiévski.

\chapter*{Capítulo 4\\
\bigskip
\emph{O Sermão da Estepe: Ivan Karamázov e a filosofia dostoievskiana da
história}}

\addcontentsline{toc}{chapter}{Capítulo 4\\\scriptsize{\emph{O Sermão da Estepe: Ivan Karamázov e a filosofia dostoievskiana da
história}}}
\hedramarkboth{Capítulo 4}{}

\ \ \quad \ 
\begin{minipage}[c]{0.84\textwidth}
\scriptsize\emph{E clamavam em alta voz, dizendo: ``Até quando tu, que és o Senhor, o
Santo, o Verdadeiro, ficarás sem fazer justiça e sem vingar o nosso
sangue contra os habitantes da terra?}

\smallskip
\hspace*{\fill}-- \emph{Apocalipse}, 6, 10
\end{minipage}

\bigskip

\, \ 
\begin{minipage}{0.84\textwidth}
\scriptsize\emph{Sem a guerra, as pessoas tornam"-se torpes em suas riquezas e conforto e
perdem o poder de pensar e sentir nobremente; tornam"-se brutas, retornam
à barbárie. Não falo de indivíduos, mas de raças inteiras. Sem a dor,
não há como compreender a felicidade. Os~ideais são purificados pelo
sofrimento, como o fogo purifica o ouro. A~humanidade deve lutar por seu
paraíso.}

\smallskip
\hspace*{\fill}-- Dostoiévski, Dresden, 17 de agosto de 1870.\\
\hspace*{\fill}\emph{Carta para a sobrinha Sofia Aleksandrovna}\footnotemark
\end{minipage}
\footnotetext{\emph{Dostoiévski:
  correspondências, 1838-1880}. Tradução de Robertson Frizero. Porto
  Alegre: 8Inverso, 2011, p. 185.}

\bigskip

\, \ 
\begin{minipage}{0.84\textwidth}
\scriptsize\emph{O ateísmo completo está no penúltimo degrau da fé mais perfeita (se
subirá esse degrau já é outra história).}

\smallskip
\hspace*{\fill}-- Clérigo Tíkhon, \emph{Os demônios}\footnotemark
\end{minipage}
\footnotetext{São Paulo: Editora 34, 2004,
  p. 662.}

\thispagestyle{empty}

\section{4.1. Se no princípio era o Verbo, como conjugar Suas aporias?}

Que a dúvida sobre a existência de Deus tenha atormentado Fiódor
Dostoiévski é matéria recorrente entre a fortuna crítica do escritor
russo. De um lado postam"-se os críticos cristãos, prontos a batizar o
\emph{Evangelho segundo Dostoiévski} sob a bênção do Príncipe Míchkin
(\emph{O idiota}) e de Aliócha Karamázov (\emph{Os irmãos Karamázov}).
Míchkin, o Cristo redivivo, fusão dostoievskiana de Jesus e Dom Quixote.
Aliócha, a bondade e o perdão catequizados sob o hábito monástico. Do
outro lado do \emph{front}, niilistas e revolucionários dos mais
diversos matizes arregimentam as personagens mundanas para converter o
\emph{Evangelho segundo Dostoiévski} na \emph{Apologia de Judas
Iscariotes}. Os~pedófilos Svidrigáilov (\emph{Crime e castigo}) e
Stavróguin (\emph{Os demônios}), os homicidas Raskólnikov (\emph{Crime e
castigo}) e Rogójin (\emph{O idiota}), os parricidas Ivan Karamázov e
Smierdiákov (\emph{Os irmãos Karamázov}), segundo as leituras
revolucionárias, só fariam escavar as narrativas dostoievskianas a
reboque do rancor, do ressentimento e da revolta originalmente
irradiados pelo homem do subsolo (\emph{Memórias do subsolo}). Ora, se o
criador do beberrão Marmieládov (\emph{Crime e castigo}) tivesse
entornado um sem"-número de doses da legítima vodca russa ao lado de
Oscar Wilde, poderíamos imaginar que, ao fim e ao cabo, ambos
caminhariam trôpegos pela Avenida Niévski, a artéria de São Petersburgo,
entoando reiteradamente um aforismo em homenagem aos críticos
partidários e parciais: ``Quando os críticos discordam entre si, o
artista concorda consigo mesmo'' (\versal{WILDE}, 2006, p. 69).

Não pretendo simplesmente afirmar ou negar a profunda religiosidade de
Dostoiévski ou mesmo considerá"-lo um revolucionário inequívoco. A~meu
ver, a obra de Dostoiévski enforma"-se segundo o movimento da
contradição. Teses e antíteses são postas e pressupostas, entrechocam"-se
sem solução, de modo que uma síntese parcial tende a subsumir o caráter
irresoluto dos embates dialógicos em função do hasteamento da bandeira
de uma determinada ideologia. Eis o que Sophie Ollivier (1994) chamou de
``a dialética sem síntese na obra de Dostoiévski'' (p. 54). Ao invés de
separar de modo imiscível e polar cristianismo e socialismo, procurarei
demonstrar de que modo tais polaridades \emph{a priori} antípodas
tornam"-se mutuamente recíprocas na filosofia da história subjacente a
``O grande inquisidor'', quinto capítulo do Livro \versal{V} de \emph{Os irmãos
Karamázov}, o último romance de Dostoiévski. Ao analisar o movimento
contraditório da filosofia dostoievskiana da história, espero lançar luz
sobre as aporias que aproximam as teses do grande inquisidor de
conflitos que, a meu ver, ainda possuem atualidade.

As três tentações a que Jesus Cristo foi submetido em sua quarentena no
deserto, segundo o grande inquisidor, sintetizariam as contradições da
natureza humana em seu movimento através da história. Cristo seria o
ômega, isto é, a culminância do processo de desenvolvimento da
humanidade, ao fim do qual o eu, no ápice de sua liberdade, renunciaria
voluntariamente à própria vontade para se doar em função dos demais. As
primeiras pessoas do singular e do plural seriam coincidentes, eu sou
porque nós somos. O~socialismo seria um momento de desagregação -- a
revolta contra a injustiça atávica -- e de agregação -- a reconstrução
do real sobre novas bases -- para que a humanidade buscasse a
universalidade. Mas, adverte o inquisidor que já há muito deixou de
acreditar em Deus, a revolução que apreende o movimento da transformação
apenas como a saciedade dos homens com o pão usurpado de cada dia está
fadada ao sensualismo mais comezinho e deixará de alicerçar a evolução
moral da humanidade. Assim, os homens apenas buscarão alguém diante de
quem se inclinar, com a condição servil de que os estômagos sejam
forrados com a serragem do trabalho usurpado. A~liberdade incriada que
tanto assusta a humanidade ainda distante da autonomia poderá, ainda uma
vez, ser silenciada pela santíssima trindade mundana: o milagre, o
mistério e a autoridade.

Ao rechaçar a síntese entre a tese socialista e a antítese cristã, o
grande inquisidor nos faz pensar sobre o atual momento histórico, em
meio ao qual as noções de totalidade e universalidade são preteridas em
nome do existente tal como ele se apresenta. Assim, a contrapelo da
resignação atual, a retomada da filosofia dostoievskiana da história
pretende esboçar os prolegômenos de um pensamento dialético que traga
novamente à tona o conceito de humanidade como contínua superação e
agregação, ao invés de compreender a contemporaneidade como a
justaposição protobeligerante dos diferentes povos sem que houvesse nada
mais a fazer a não ser ressoar a ode ao existente.

\section{4.2. Para além do Sermão da Montanha}

\begin{quote}
Na verdade, a mesa de Ivan, perto da janela, estava protegida por um
simples biombo dos olhares indiscretos. Encontrava"-se ao lado do balcão,
na primeira sala, em que os garçons circulavam a todo instante. Somente
um velhinho, militar reformado, bebia chá num canto. Nas outras salas,
ouvia"-se o barulho habitual dos botequins: chamadas, garrafas que se
desarrolhavam, os choques das bolas no bilhar. Um órgão fazia"-se ouvir
(\versal{DOSTOIÉVSKI}, 1971, p. 177).
\end{quote}

Estamos diante de um dos últimos diálogos escatológicos da obra de
Dostoiévski. Os~irmãos antípodas Aliócha e Ivan Karamázov encontram"-se
em uma típica taverna dostoievskiana, o reiterado ambiente mundano em
que as personagens se digladiam sob os vapores etílicos que as instigam
e inebriam a ponto de, no ápice da discussão, as ideias se embaralharem
como se não pertencessem a seus sujeitos iniciais; como se elas, as
ideias e as personagens, tivessem suas identidades imiscuídas, como se o
\emph{eu sou} se tornasse fluido para dizer \emph{nós somos}, vale
dizer, o princípio lógico"-formal de identidade se esvai diante da
entrevisão dostoievskiana de que as ideias e as personagens mais
antípodas dialogam segundo um veio de profunda contiguidade.

Ivan, o ateu. Aliócha, o monge. Irmãos. Desde o início do romance, a
trama prenunciava o diálogo polar entre o lobo e o cordeiro. Mas seria
Ivan um militante ateu que não admitiria qualquer possibilidade de
transcendência?

\begin{quote}
Admito Deus, não só voluntariamente, mas ainda sua sabedoria, seu fim
que nos escapa; creio na ordem, no sentido da vida, na harmonia eterna,
na qual se pretende que nos fundiremos um dia: creio no Verbo para o
qual propende o Universo que está em Deus e que é ele próprio Deus, até
o infinito. Estou no bom caminho? Imagina que, em definitivo, esse mundo
de Deus, eu não o aceito e, embora saiba que ele existe, não o admito.
Não é Deus que repilo, nota bem, mas a criação; eis o que me recuso a
admitir. (\versal{IBIDEM})
\end{quote}

Ora, o problema de Ivan não se refere à gênese do universo, mas aos
desdobramentos da criação. ``Tenho essencialmente o espírito de
Euclides: terrestre. De que serve resolver o que não é deste mundo?''
(\versal{IBIDEM}) Se abstrairmos por um breve momento a voz de Ivan e buscarmos
os fios autorais que entretecem a teia romanesca, conseguiremos desvelar
a engenhosidade de Dostoiévski. O~escritor desloca o centro de gravidade
de seu embate teológico da esfera metafísica intangível para a imanência
que se desespera pelo esvaziamento cada vez mais patente da
transcendência. O~mundo é o campo de batalha que Ivan quer perscrutar.
Não se trata de analisar o \emph{Gênesis}. Ivan não quer o começo, mas o
fim, o último livro do Pentateuco, o \emph{Deuteronômio}, a discussão
sobre a lei, a investigação sobre se a teologia, em face da modernidade,
ainda pode ser ética.

\begin{quote}
Se todos devem sofrer, a fim de concorrer com seu sofrimento para a
harmonia eterna, qual o papel das crianças? Não se compreende por que
deveriam sofrer, também elas, em nome da harmonia. Por que serviriam de
matéria para prepará"-la? Compreendo bem a solidariedade do pecado e do
castigo, mas não pode ela aplicar"-se aos inocentinhos, e se na verdade
são solidários com os malfeitos de seus pais, é uma verdade que não é
deste mundo e que eu não compreendo. (\ldots) Os~carrascos sofrerão no
inferno, dir"-me"-ás tu. Mas de que serve esse castigo, uma vez que as
crianças tiveram também o seu inferno? Aliás, que vale essa harmonia que
comporta um inferno? Quero o perdão, o beijo universal, a supressão do
sofrimento. E, se o sofrimento das crianças serve para perfazer a soma
das dores necessárias à aquisição da verdade, afirmo desde agora que
essa verdade não vale tal preço. Não quero que a mãe perdoe ao carrasco,
não tem esse direito. Que lhe perdoe seu sofrimento de mãe, mas não o
que sofreu seu filho estraçalhado pelos cães. Ainda mesmo que seu filho
perdoasse, não teria ela o direito. Se o direito de perdoar não existe,
que vem a tornar"-se a harmonia? Há no mundo um ser que tenha esse
direito? (\versal{IDEM}, p. 184)
\end{quote}

Ivan descarta o transcendente que não esteja enraizado na imanência do
mundo para falar sobre Deus e a criação. Ivan nos pede que falemos sobre
o que há abaixo do céu. Assim, quando constata as contradições mais
latentes em termos éticos, a aporia não pode recorrer ao Deus que apenas
paira sobre nós para ser dirimida. \emph{Não é possível aceitar que as
crianças sejam cúmplices dos malfeitos de seus pais sem recorrer a leis
que transcendam a noção de justiça que se foi cristalizando ao longo da
historia.} Neste momento, entrevemos uma cisão no pensamento de Ivan,
cisão que o aproxima e o distancia de Aliócha de modo eminentemente
contraditório.

Ivan não pode admitir a criação divina, o mundo, segundo as bases
teológicas que lhe foram legadas. Por esse prisma, Ivan é um ateu
convicto e não poderia estar mais distante de Aliócha. Porque o irmão de
Ivan, o monge, sintetiza a teologia que se funda sobre o mistério, que
quer a prática do amor recíproco sem que haja bases racionais de
apreensão do universo e suas leis de desenvolvimento. Ivan transforma"-se
em um restelo que faz terra arrasada da herança cristã que não consegue
dialogar com o devir da razão, que pretende pregar a fé sem que o
intelecto se expanda. A~lógica de argumentação está profundamente
afinada com o sentido da época, com os primórdios da modernidade. Por
que a relação com Deus deveria embotar a razão se o espírito do tempo
requer homens e mulheres que tenham uma relação cada vez mais
intelectiva com a realidade? Se a fé não pode dialogar com a razão, Ivan
torna"-se um militante ateu quiçá à espera de um prisma espiritual que
expanda as fronteiras da teologia para além de suas contradições
contumazes que, historicamente, não puderam ser questionadas diante da
autoridade do mistério imposto pelo clero.

Se tentarmos capturar a cauda fugidia do argumento de Ivan,
acompanharemos a contradição em seu movimento sub"-reptício. Segundo
Ivan, a mãe só pode perdoar ao carrasco com base em seu próprio
sofrimento materno. A~mãe não pode perdoar ao algoz com base no
sofrimento do filho. O~filho, vale recordar, está morto. Assim, ele já
não pode perdoar. Neste momento, parece"-me fundamental questionar por
que Aliócha não interpela Ivan sobre a vida após a morte. Pois, se
houver a imortalidade da alma, o filho poderá perdoar ao algoz. Se
levarmos tal argumento às últimas consequências, o perdão do filho
poderá fazer com que o próprio algoz, um dia, quiçá consiga perdoar a si
mesmo. Por que Aliócha se cala diante do argumento do irmão? Ora,
Dostoiévski é consequente na radicalidade da crítica que pretende
questionar os limites tanto do pensamento ateu quanto do pensamento
cristão. Aliócha não se pronuncia porque a teologia que abraça, em
essência, não se distingue do pensamento de Ivan. O~catolicismo e o
cristianismo ortodoxo do monge Aliócha não perscrutam o além"-mundo. O
que há após a morte? Silêncio, mistério. A~imortalidade da alma
transforma"-se em uma mera projeção para dirimir as contradições de haver
o mal e o sofrimento no mundo. Mais importante para tal pensamento
teológico, na verdade, é a noção de inferno. A~punição, a teologia
taliônica. Mas Ivan já não pode aceitar um deus que se funda sobre a dor
universal. Como Aliócha se cala, percebemos a engenhosidade de
Dostoiévski em fazer com que a teologia do mistério seja emparedada em
suas contradições mais limítrofes. Aliócha e Ivan, neste momento, não se
aproximam apenas porque a lógica de seus pensamentos se faz contígua. Na
verdade, se acompanharmos a dialogia dostoievskiana em seu constante
devir, as identidades de Aliócha e Ivan embaralham"-se, pois enquanto o
monge se choca constantemente contra o muro sem que seu hábito religioso
o leve a escalá"-lo, Ivan, já no cume das contradições, esgarça a
teologia oficial e, por meio da negatividade, da não"-cooptação do
pensamento diante da autoridade da fé, abre caminho para novas
indagações a respeito da natureza de Deus, dos homens e da história.
Neste momento, em face do monge Aliócha, Ivan torna"-se, dialeticamente,
um ateu espiritual.

Se o filho puder perdoar ao carrasco, o direito ao perdão volta a
existir. Mas se houver apenas uma vida para que o perdão seja concedido,
não será possível acompanhar as transformações que se projetam sobre as
relações humanas. Esgarcemos os argumentos de Ivan para acompanharmos o
movimento da contradição: ora, como é possível perdoar univocamente se a
sociedade está fundada sobre a lógica de Talião? O~perdão, no limite,
transforma"-se em um esquecimento, em um deixar para lá, e não em
verdadeiro convívio. Seria possível dizer que, ao longo da história, o
perdão nunca existiu. Apenas houve a distância entre a família da vítima
e o algoz. A~filha do condenado nunca chegou a desposar o carrasco de
seu pai. Mas e se a filha do condenado pudesse se tornar a mãe do
carrasco em outra oportunidade? E~se o condenado, com a cabeça
restituída ao corpo, escolhesse voltar como o irmão do carrasco? Ao
invés de um sistema total em que há apenas uma oportunidade de perdão,
ou, por outra, de reconciliação, falamos de uma totalidade aberta em que
o papel da liberdade determinada é levado às últimas consequências.
Afinal, a dialética negativa de Ivan abre espaço para que a razão
arcaica seja superada por uma racionalidade transcendente que entrevê o
mundo como uma transição, como mais uma etapa evolutiva, já que ``na
casa de meu Pai há muitas moradas'' (\versal{JOÃO}, 14, 2).

``Há no mundo um ser que tenha o direito de perdoar?'', pergunta Ivan a
Aliócha. O~monge exulta ao mencionar Jesus Cristo. A~tradicional
interpretação de que Cristo é o cordeiro de Deus, de que o Messias foi o
enviado do Pai para expiar o pecado dos filhos que ainda não entenderam
a parábola que narra o retorno do filho pródigo. Mas Ivan, em sua
espiritualidade negativa, transcenderá a lógica da expiação unívoca e
tentará concluir um ensaio que seu pai autoral tentou escrever durante
toda a vida. ``Das relações entre cristianismo e socialismo''\footnote{Para
  maiores detalhes sobre o processo de gênese dostoievskiana das ideias
  que correlacionam cristianismo e socialismo, ler o terceiro tomo da
  biografia de Dostoiévski escrita pelo crítico norte"-americano Joseph
  Frank. \emph{Dostoiévski: os efeitos da libertação, 1860-1865}.
  Tradução de Geraldo Gerson de Souza. São Paulo: Edusp, 2002. O~tema
  concentra"-se entre as páginas 505 e 509. Também é possível encontrar
  uma discussão sobre o entrelaçamento entre cristianismo e socialismo
  em Dostoiévski no artigo de Geraldo J. Sabo, ```The Dream of a
  Ridiculous Man': Christian Hope for Humanity''. In: \emph{Dostoevsky
  Studies}. Volume 13. Tübingen: Attempto Verlag, 2009, pp. 47-58. No
  próximo capítulo, Sabo será um dos interlocutores em nossa análise de
  ``O sonho de um homem ridículo'', conto fundamental que Dostoiévski
  escreveu em 1877.}. A~partir de agora, procuraremos desvelar o diálogo
dentro do diálogo. Ivan e Aliócha se desdobram no grande inquisidor e em
Jesus Cristo. Voltaremos à Espanha do século \versal{XVI}. No auge da inquisição
ibérica, iremos a Sevilha. O~poema de Ivan, ``O grande inquisidor'',
transformará Jesus Cristo e o demônio nos intérpretes da história, as
duas personagens que, com a tríade de tentações e refutações, de
perguntas e respostas, sintetizam o devir humano em sua transformação da
particularidade insciente e heterônoma à universalidade consciente e
autônoma, sendo a civilização e seus estertores o processo dialético de
construção das bases sobre as quais o eu volta a se fundir ao todo.

\section{4.3. O Sermão da Estepe}

Após o batismo de Cristo com a cuia de João Batista embebida pelas águas
do Jordão, ``Jesus foi conduzido pelo Espírito ao deserto para ser
tentado pelo demônio'' (\versal{MATEUS}, 4, 1). Parece"-me no mínimo curioso --
\emph{misterioso}, se quisermos utilizar um adjetivo católico e ortodoxo
por excelência -- que não se questione por que Cristo deveria ser
tentado pelo demônio. Não se trata do filho de Deus? Não se trata da
perfeição encarnada? Por que submeter o Messias a uma tentação que não
poderá desviá"-lo? Ora, mas e se Jesus não fosse idêntico ao Cristo desde
o princípio? E~se discordássemos de João e disséssemos que no princípio
era não o Verbo, mas o silêncio passível de evolução? Se assim o
fizermos, Jesus voltará a ser o filho do carpinteiro José. Todos somos
ou fomos filhos, assim como Jesus, a princípio insciente sobre a
\emph{via crucis} que o levaria ao Gólgota e consumaria a Paixão. A
quarentena no deserto, além de sintetizar uma filosofia da história, tem
um sentido eminentemente pedagógico. Jesus precisou evoluir como todos
nós. Não se trata mais do Escolhido, dos escolhidos, mas de um caminho
que todos poderemos percorrer, cada um a seu tempo.

Enquanto esteve na Sibéria, na prisão que ficou conhecida posteriormente
como a \emph{casa dos mortos}, só era permitida a Dostoiévski e aos
demais detentos a leitura da bíblia\footnote{Para mais detalhes sobre a
  estada de Dostoiévski na Sibéria após a condenação por ter participado
  do Círculo de Petrachévski, grupo revolucionário que se contrapunha ao
  regime tsarista, ler o livro escrito pelo autor quando de seu retorno
  à vida literária de São Petersburgo. \emph{Recordações da Casa dos
  Mortos.} Tradução de José Geraldo Vieira. Rio de Janeiro: Francisco
  Alves, 1982. Ler também o segundo tomo da biografia de Dostoiévski
  escrita por Joseph Frank. \emph{Dostoiévski: os anos de provação,
  1850-1859.} Tradução de Vera Pereira. São Paulo: Edusp, 1999.}.
Podemos pressupor, então, que a imaginação analítica de Dostoiévski
perscrutava os fatos aterradores com que ia entrando em contato ao mesmo
tempo em que o ímpeto sintético do autor buscava a universalidade dos
fenômenos, a unidade da pluralidade caótica, em estreito diálogo com a
mediação fornecida pelos textos bíblicos. Eis, a meu ver, o pano de
fundo biográfico que desvela a gestação paulatina da obra dostoievskiana
do período pós"-siberiano. Dostoiévski fizera parte de um grupo
revolucionário, o Círculo de Petrachévski, que se contrapunha à
autocracia tsarista. Joseph Frank sustenta a tese de que a participação
do escritor no círculo socialista não renegava sua formação cristã. E
mais: Dostoiévski estaria vinculado ao círculo revolucionário sobretudo
por seu repúdio encarniçado à servidão da gleba. De qualquer forma, os
anos siberianos alimentados pela leitura bíblica e por recorrentes
reflexões sobre o sentido de suas filiações e pensamentos podem indicar
um prenúncio do que viria a ser ``O grande inquisidor'': uma síntese
entre cristianismo e socialismo tanto em seu movimento regressivo e
restritivo quanto em sua possibilidade de superação a partir de um
pensamento que caminha pelo negativo e tensiona as bases da modernidade
para acompanhar o movimento da humanidade em sua articulação ulterior.

Após quinze séculos, Cristo volta ao nosso plano, caminha entre a
multidão sevilhana, realiza milagres, e a massa o aplaude. Não demora
até que o grande inquisidor ordene aos gorilas eclesiásticos que levem o
Messias redivivo à masmorra da inquisição. No cubículo gradeado e escuro
ocorrerá o diálogo, ou melhor, o monólogo dialogado, entre o
clérigo"-mor, representante do braço armado da Igreja, e o Cristo
novamente aprisionado. Logo no início do poema narrado por Ivan, o
inquisidor nonagenário, a reboque do exílio siberiano de Dostoiévski,
entrevê, na singularidade de cada uma das tentações que compõem a tríade
do deserto, a síntese da história humana:

\begin{quote}
Se jamais houve na terra um milagre autêntico e retumbante foi o dia
daquelas três tentações. O~simples fato de terem sido formuladas aquelas
três perguntas constitui um milagre. Suponhamos que tenham desaparecido
das Escrituras, que seja preciso reconstituí"-las, imaginá"-las de novo
para substituí"-las ali, e que se reúnam para esse efeito todos os sábios
da terra, homens de Estado, prelados, sábios, filósofos, poetas,
dizendo"-lhes: imaginai, redigi três perguntas que não somente
correspondam à importância do acontecimento, mas ainda exprimam em três
frases toda a história da humanidade futura -- acreditas que esse
areópago da sabedoria humana poderia imaginar nada de tão forte e de tão
profundo como as três questões que te propôs então o poderoso espírito?
Essas três questões provam por si sós que se tem de ver com o espírito
eterno e absoluto e não com um espírito humano transitório. Porque
resumem e predizem ao mesmo tempo toda a história ulterior da
humanidade, são as três formas em que se cristalizam todas as
contradições \emph{insolúveis} da natureza humana {[}grifo meu{]}
(\versal{DOSTOIÉVSKI}, 1971, p. 188).
\end{quote}

Decretar o caráter eternamente \emph{irresoluto} das contradições
próprias à natureza humana e à história faz todo o sentido quando
pensamos no inquisidor e em sua instituição como entraves seculares para
a disseminação da liberdade incriada. Entraves que, contraditoriamente,
possibilitaram a legitimação do cristianismo como campo disseminado do
saber e das práticas sociais. Por mais que o inquisidor diga respeito a
algo que deve ser superado, segundo a dialética negativa de Ivan que
tentamos trazer à tona, trata"-se de uma negação determinada, vale dizer,
o cristianismo e a espiritualidade não teriam se plasmado ao transcurso
histórico do Ocidente não fosse a Igreja Católica. Denegá"-la cabalmente,
então, implica fazer tábula rasa de um passado de que somos tributários,
sobretudo porque a antítese negativa de Ivan não teria se estruturado
sem a tese consuetudinária do cristianismo de Aliócha. Dostoiévski se
embebeu da tradição em meio à qual foi socializado para, com mais ou
menos consciência, escavar os textos bíblicos segundo o espírito da
época de cujos debates participou até o fim de sua vida.

As questões, por si sós, foram transitórias e efêmeras, mas, segundo o
inquisidor, o sentido que as embasou procurou sintetizar a história
humana. A~quarentena de Cristo no deserto, assim, transforma"-se em um
microcosmo que concentra a trajetória evolutiva do espírito, sua
liberdade incriada. E~quem seria Satanás? O~contumaz anjo decaído,
Lúcifer e sua legião malévola? Ou estaríamos diante do caráter dialético
da liberdade, isto é, da possibilidade de regressão diante da
consciência em expansão? Tentemos apreender o entrechoque entre as teses
e antíteses cristãs no movimento das tentações desérticas.

``Se és Filho de Deus, ordena que estas pedras virem pães'' (\versal{MATEUS}, 4,
3). A~humanidade ainda estava longe de se constituir. O~Pai Nosso de
Cristo fazia todo o sentido. Para as comunidades que precisavam lutar
pela própria sobrevivência, era preciso orar pelo pão nosso de cada dia.
Se Cristo transformasse as pedras em pães, um enorme suplício dos homens
estaria satisfeito. Mas que resultaria daí? ``Reduzi"-nos à servidão,
contanto que nos alimenteis'' (\versal{DOSTOIÉVSKI}, 1971, p. 189). Os~homens
ainda pouco diferenciados, membros de comunidades não de todo fixadas,
em cujas memórias ainda ressoava o atavismo nômade, seriam congregados
em função do pão, orbitariam ao redor do ser místico que os alimentasse
sem mais. O~ego, que nem de longe havia despontado, continuaria amorfo e
plasmado ao espírito coletivo. ```Nos estágios primitivos da sociedade',
escreve Dostoiévski, `Deus é a ideia coletiva de humanidade, da massa,
de cada \emph{um}. Quando o homem vive em massa (nas comunidades
patriarcais primitivas, sobre as quais foram deixadas muitas lendas),
então o homem vive \emph{espontaneamente}'" (\versal{FRANK}, 2002, p. 505). A
espontaneidade, nos primórdios das comunidades humanas, refere"-se não ao
desenvolvimento da liberdade, mas à vivência contumaz da necessidade.
Sem o desenvolvimento técnico, o homem não consegue cortar o cordão
umbilical que o aguilhoa à natureza. O~eu pode se ver, no ápice de sua
trêmula consciência, como um pseudópode do todo. A~baixíssima divisão
social do trabalho ata os poucos elos produtivos e quase não os
distancia geograficamente. O~caçador entrega o espólio de sua jornada
diretamente àqueles que deverão executar os trabalhos secundários. O
aproveitamento da carne e da pele. Quando o excedente produtivo começa a
permitir o ócio reverencial, os primeiros sacerdotes abençoam o fruto do
trabalho alheio. Deus e seus representantes, assim, fundem"-se à
alienação do trabalho, à consciência de que um Outro que não a
comunidade embasa as realizações do ego amorfo, vale dizer, do todo
social pouco diferenciado. Eis os primórdios do totem, do fetiche. Eis o
ícone diante do qual a humanidade deve se ajoelhar. Dostoiévski apreende
o caráter \emph{negativo} do movimento que faz coincidir a evolução com
a saciedade da fome. Nas comunidades iniciais, o eu não tem consciência
de si. Um nós indiferenciado transforma os animais sociais em extensões
recíprocas da comunidade. Os~homens apenas percebem os contornos de seus
corpos. A~saciedade imediata cristalizaria a heteronomia. Pensemos, por
exemplo, na passagem judaica do Maná do Deserto. Moisés, em seu
transcurso pelo deserto, não tinha com que alimentar os filhos de Judá.
O milagre dos pães de mel pôde aquietar as tribos centrípetas. A
passagem é reiterada no Novo Testamento. Cristo e a multiplicação dos
pães e dos peixes. A~religiosidade, em termos quintessenciais, sempre
caminhou a reboque de suas contradições. Quando Cristo disse que não
veio para negar a lei, mas para pô"-la em prática, entrevemos a superação
e o atavismo em suas palavras. Senão, vejamos: o amor recíproco aumenta
a liberdade humana, mas a reiteração dos milagres ainda uma vez
catequiza os homens segundo a lógica suscetível do mistério. Assim,
descobrimos as contradições que os próprios evangelistas animaram. Um
Cristo mágico, santo e milagreiro, o passado atávico, ao lado de um
Cristo novo e dialético, senhor de uma liberdade que pode ser
transmitida a todos e a cada de um nós.

As pedras tornadas pães acorrentariam os homens a um estágio de
consciência que geraria um ego animalesco. Um ego eminentemente ligado
ao estômago. Não se trata de negar a importância da supressão das
carências materiais. Trata"-se de correlacioná"-la com o sentido de
desenvolvimento do todo. Um homem assim pouco diferenciado veria, mágica
e instantaneamente, o rompimento dos laços da necessidade. A~liberdade
lhe cairia no colo. As~comunidades se estilhaçariam, o todo se tornaria
a somatória centrífuga dos eus recém"-forjados pela saciedade do
estômago. A~nova liberdade, insciente sobre si mesma, traria como
consequência a completa dependência dos seres diante do deus fornecedor.
E se os pães voltassem a ser pedras? Por mais trêmula que fosse, a
consciência apreenderia o retrocesso. Os~pães remanescentes trariam de
volta as guerras. A~consciência seria aguçada, então, não por um sentido
de conexão entre o eu e o todo, mas por um ímpeto de beligerância.
\emph{Eu} preciso sobreviver contra os \emph{demais}. Apenas um Messias
mágico poderia restabelecer a paz pela transformação de novas pedras em
pães.

\begin{quote}
Compreenderão por fim que a liberdade e o pão da terra à vontade para
cada um são inconciliáveis, porque jamais saberão reparti"-los entre si!
(\ldots) Tal é o sentido da primeira pergunta que te foi feita no deserto,
e eis o que rejeitaste em nome da liberdade, que punhas acima de tudo.
No entanto, ocultava ela o segredo do mundo. Consentindo no milagre dos
pães, terias acalmado a eterna inquietação da humanidade -- indivíduos e
coletividade --, isto é: ``Diante de quem se inclinar?'' Porque não há,
para o homem que fica livre, preocupação mais constante e mais ardente
do que procurar um ser diante do qual se inclinar. Mas só quer ele
inclinar"-se diante de uma força incontestada, que todos os humanos
respeitem por consenso universal. (\ldots) Porque essa necessidade da
comunidade na adoção é o principal tormento de cada indivíduo e da
humanidade inteira, desde o começo dos séculos. (\versal{DOSTOIÉVSKI}, 1971, pp.
189-190)
\end{quote}

Neste momento, o inquisidor revela a síntese de suas apreensões sobre os
dilemas que configuram a história humana. Deus assume a forma da
autoridade. Deus Pai, portanto. Deus é quem provê, Deus é quem realiza.
A humanidade devolve a Deus todo o fruto de seu trabalho para Dele
receber a produção com a bênção divina. Deus, nesse sentido, é não
apenas a transcendência inatingível a partir da metáfora atávica do céu.
Deus se transforma na própria produção social deificada pelos dominantes
que a usurpam da coletividade. Sem uma base de verdadeira reciprocidade
entre o eu e o todo, um dos polos da dialética sempre buscará a
constituição contra o outro. Ora o eu arredio buscará se isolar,
hipostasiando uma vida livre em meio à solidão, quando na verdade sua
suposta independência deriva da divisão social do trabalho, ora o todo
procurará padronizar as identidades com uma lógica de manada que tenta
evitar a constituição do ego centrífugo. Vemos aqui, a milênios de
distância, o dilema que enredou o socialismo. Há milênios, a consciência
baça e trêmula mal podia pedir algo para além do pão. A~imagem de um
deus provedor afinava"-se com o espírito da época. Mas eis que os
socialistas revisitam a liberdade a partir do estômago. Seus melhores
representantes, entre os quais Karl Marx e Oscar Wilde, entreveem que a
condição para a liberdade de um é a liberdade de todos. Só assim o homem
estaria livre para o exercício de suas máximas faculdades. Que a fome
fosse extirpada, que a humanidade se visse livre, efetivamente, do jugo
da primeira natureza. Que a emancipação como potencial da segunda
natureza aproximasse os desígnios da imaginação da prática cotidiana da
humanidade. Mas a história traz em seu bojo a consciência atávica. O
otimismo revolucionário vai ficando para trás. O~filho dileto da
modernidade, o capitalismo, prega que todos e cada um podem maximizar
suas satisfações pessoais -- com os demais e, sempre que for preciso,
contra os demais. A~multiplicação dos pães e dos peixes, a acumulação
das riquezas, pressupõe a divisão do trabalho em escala competitiva.
Sendo assim, que fizeram os líderes comunistas diante do capitalismo
estatal soviético? O~ímpeto por mais consumo era calado
autoritariamente. A~sociedade \emph{devia} produzir até chegar a seus
estertores, mas o novo Deus, o Partido, e o novo Filho, o Guia Genial
dos Povos, decidiriam como se daria a partilha. Vemos, então, que o
socialismo se vê enredado por dilemas análogos aos que emparedaram a
religiosidade primitiva. A~liberdade se vê transformada no culto da
necessidade, ainda que o desenvolvimento material tenha potencializado
sobremaneira a consciência dos homens.

\emph{``Não só de pão vive o homem, mas de toda palavra que procede da
boca de Deus''} (\versal{MATEUS}, 4, 4). Curioso: o pão teria vindo de Cristo,
filho de Deus. Por que a palavra que procede da boca de Deus não pode
coincidir com a saciedade material? Ora, o sensualismo coincide com um
eu autossuficiente no sentido mais prosaico do termo. A~coletividade
havia se diferenciado dos animais ao viver para além da mera
contingência. Mas o eu entregue a si mesmo sem a consciência expandida
não progrediria para além do ímpeto de se saciar. Sendo assim, por que
voltar ao trabalho? Qual o sentido da atividade social quando o eu se
liberta das correntes que o condenavam à vida em comunidade? O~movimento
da dialética nos sugere que pode haver uma utilização conservadora da
noção de que o eu não deve receber os frutos de seu trabalho à exaustão.
Assim, sempre deveria haver uma instância de decisão que repartisse de
forma justa, ainda que desigual, os frutos da riqueza social. Assim tem
sido historicamente. Mas a humanidade alcançou o cume do desenvolvimento
que mostra uma inequívoca reciprocidade entre o eu e os outros? A
palavra que procede da boca de Deus visaria, então, à diminuição da
distância entre o todo e o eu. Cristo chegaria a dizer que todos somos
deuses. Mas o movimento de totalização não é imediato. A~liberdade
não"-mediada pode transformar"-se em autocracia. E, aqui, não se trata de
corroborar a desigualdade atual com vistas a uma aposta em uma melhoria
futura. Trata"-se de perceber a permanência das aporias que não conseguem
transformar o ego humano apenas com a administração dos recursos
produtivos. Para seres que estão acostumados a viver sob o signo da
distinção, a fundamental supressão das carências materiais tende a
elevar o acirramento dos egos. É~essa elevação, bastante desigual entre
os povos, que constitui a civilização\footnote{Nesse sentido, Zsuzsanna
  Bjørn Andersen (2000), em ``Os conceitos de dominação e poder em `A
  dócil', de Dostoiévski'', argumenta que ``um relacionamento entre
  seres humanos sempre se baseia, em alguma medida, em um balanço de
  poder, não importa o caráter que a relação assuma. Nem mesmo o amor
  está livre da luta por poder. Em termos gerais, há duas
  possibilidades. A~luta por poder que acontece entre os seres humanos
  pode ser definida como uma questão de pura autopreservação. Em outra
  palavras, a luta passa a ser uma questão de se obter uma parte dos
  recursos materiais com os quais cada indivíduo consegue assegurar sua
  própria sobrevivência. Essa forma de pensamento sobre o poder é
  encontrada nas obras de Hobbes, Nietzsche, Foucault e Max Stirner,
  {[}\emph{Der Einzige und sein Eigentum} (\emph{O único e sua
  propriedade}){]}, com cujas ideias Dostoiévski certamente tinha
  familiaridade'' (p. 54). Na sequência, Andersen desenvolve uma
  argumentação derivada da dialética hegeliana do reconhecimento,
  argumentação que será muito importante no próximo capítulo, quando
  analisarmos a cicatrização do espírito na trajetória do homem
  ridículo, protagonista de ``O sonho de um homem ridículo'' (1877):
  ``No entanto, há outra forma de pensamento sobre o poder mais
  diferenciada e sofisticada, forma que aparece nas primeiras obras de
  Hegel (especialmente, na \emph{Fenomenologia do Espírito} e nas obras
  anteriores chamadas `Textos de Jena'). De acordo com Hegel, a luta por
  poder não se volta apenas para a garantia material de recursos às
  expensas das outras pessoas, mas também, e até mesmo de forma mais
  importante, se volta para a conquista de \emph{status} imaterial e
  espiritual -- ele chama isso de `Anerkannstein' e `Anerkennen'
  (\emph{thymos}), reconhecimento. E~essa é a base necessária do que nós
  chamamos de autoestima e autoconfiança. Sem o reconhecimento, não há
  sociabilidade e, o que é mais impressionante, também não há
  individualidade. Assim, a noção de reconhecimento não pode ser
  formulada sem o pensamento dialético. Hegel enfatiza que há uma
  diferença significativa entre a noção da luta por poder entendida como
  uma luta pela autopreservação e a luta pela honra do reconhecimento e
  autoconfiança'' (p. 55). Veremos como, em ``O sonho de um homem
  ridículo'', as noções de reconhecimento e autoconfiança se entrelaçam
  à superação do niilismo com vistas ao reencontro do eu com a
  totalidade como movimento do espírito.}.

\begin{quote}
Mas ``então vem o período de transição, isto é, mais desenvolvimento,
isto é, a civilização. (A civilização é um período de transição.) Nesse
maior desenvolvimento ocorre um fenômeno, um fato novo, do qual ninguém
pode fugir: o desenvolvimento da consciência pessoal e a negação das
ideias e leis espontâneas (leis autoritárias, patriarcais das massas)''.
(\versal{DOSTOIÉVSKI} apud \versal{FRANK}, 2002, p. 505)
\end{quote}

Moisés desceu do Sinai com as tábuas da lei. O~imperativo categórico
ético petrificado e paternalista, do Criador para as criaturas, de cima
para baixo. O~todo divino impunha ao eu aquilo que deveria ser feito. O
Pai se confundia com o profeta e suas normas. Mas o desenvolvimento
material pressupõe o refinamento das profissões, isto é, a maior
especialização dos ramos produtivos, ou seja, uma divisão social do
trabalho mais complexa. O~comerciante, mediador inequívoco, se interpõe
entre aquele que produz e aquele que compra. O~ego, em estreito diálogo
material com a história, já apreende muito mais do que o mero contorno
original de seu corpo, a consciência se aguça. O~ego se funde ao artesão
e ao agricultor, às profissões que o identificam diante dos demais. É
assim que o senhor Smith, hábil ferreiro, e o senhor Schumacher, exímio
sapateiro, orgulham"-se de vender aos demais o fruto de seu trabalho em
troca da retribuição devida. O~eu, agora, já pode morar algo distante,
não precisa necessariamente das aglomerações primordiais que sobreviviam
sob a batuta do profeta patriarca. O~ego pede a Deus que continue a lhe
dar forças, pois o trabalho provê aquilo de que ele precisa.
Anteriormente longínquo e inacessível, Deus agora passa a habitar a
terra, ainda que os olhos reverenciais orem aos céus. Quando o
protestante, por exemplo, pede um contato mais íntimo com a divindade
sem a mediação do clero, tal intimidade pressupõe mais igualdade, do
contrário seria necessária a mediação ritual para que o eu fosse tocado
pela divindade em comunhão com os demais. O~devir da civilização, assim,
aproxima Deus dos homens, mas o faz segundo a lógica de emancipação
parcial do ego. O~eu que antes se ajoelhava agora quer caminhar com as
próprias pernas, ele quer mandar em si mesmo e, potencialmente, comandar
os demais.

\begin{quote}
O demônio transportou"-o {[}a Cristo{]} à Cidade Santa, colocou"-o no
ponto mais alto do templo e disse"-lhe: ``Se és Filho de Deus, lança"-te
abaixo, pois está escrito: \emph{Ele deu a seus anjos ordens a teu
respeito; proteger"-te"-ão com as mãos, com cuidado, para não machucares o
teu pé em alguma pedra}'' (\versal{MATEUS}, 4, 5-6).
\end{quote}

A síntese realizada pelas tentações traga o devir do tempo. Quantos
séculos não se passaram desde a primeira tentação para que as
comunidades amorfas pudessem erigir templos e observar do cume mais
elevado o resultado de seu desenvolvimento? Na primeira tentação, Cristo
paira sobre os homens famintos. A~transformação milagrosa das pedras em
pães saciaria a fome animalesca do homem e estagnaria sua consciência
como uma função da massa que só faria ajoelhar"-se para receber as
dádivas. Quando da segunda tentação, porém, o homem civilizado já tem o
poderio de um deus. Seu desenvolvimento já lhe permite escalar e
colonizar montanhas, uma vez que todas as estepes já foram percorridas.
O homem se deslumbra com o próprio poder. Se Deus antes se assemelhava à
natureza indômita, agora a divindade passa a ter barbas e músculos tal
como Michelangelo a eternizou no teto da Capela Sistina. Deus passa a
ser medido à imagem e à semelhança do homem. Chega o momento, então, de
destacar o eu do todo. O~indivíduo, invenção precípua da civilizada
divisão do trabalho, quer se destacar da e na multidão.

Falemos primeiramente daquele que quer se destacar da multidão. O
centrífuga por excelência. O~ermitão. O solitário. Ele testará o poder
de Deus para ver até que ponto seu eu se faz independente. Os~produtos à
mão, com a única condição de trocá"-los mediante dinheiro, lhe dão a
ilusão de que, a qualquer momento, tudo lhe pode ser oferecido. Aos
poucos, o eu se sente não apenas superior. Ele se sente só. O~suicídio,
assim, tenta restituir a união primordial perdida. Aquele que se joga do
penhasco procura braços etéreos que contenham sua queda. Mas a massa
primitiva que só caminhava pelas estepes já se foi há muito tempo. A
consciência expandida que se volta apenas sobre e para si mesma não
encontra o diálogo que tanto busca em meio àqueles cujo trabalho provê a
independência fictícia do artista e do intelectual. Quem tenta se
destacar completamente da multidão encontra a coletividade do cemitério
como a reintegração ao todo. E~assim acontecerá, em suas sucessivas
reencarnações, até que se dê conta de que a individualidade, sentido
máximo da evolução, não é uma identidade, um ente em si mesmo, mas uma
relação polar, uma configuração relacional que pressupõe o outro, os
outros, a sociedade. Quando o adulto altivo afaga a criancinha com
carinho e saúda o idoso com reverência, ele traz à tona o atavismo
imemorial dos períodos de desenvolvimento da humanidade. A~criança
lembra a massa heterônoma, o idoso a reboque da bengala lembra ao adulto
bípede a contingência de sua altivez. Toda independência é determinada,
mas a civilização e suas mediações diluem a solidariedade social e a
transformam em mero e frio profissionalismo. A~segunda tentação aponta o
sentido para aquele que já não se reconhece no seio da sociedade que o
gerou. A~solidão, o exílio, o suicídio. ``Também está escrito: \emph{Não
tentará o Senhor teu Deus}'' (\versal{MATEUS}, 4, 7). Cristo reverbera ainda uma
vez a contradição entre o atavismo e a consciência expandida. Àqueles
que se moldam à divisão do trabalho e a seu panorama ideológico, o
centrífuga parece um louco. Sendo assim, é melhor não atentar contra
Deus. Por outro lado, em nosso transcurso dialético que une, em meio ao
movimento da contradição, a preservação e a superação, Cristo também
aponta para o eu que volta ao todo. Sim, porque despencar sobre o abismo
não é a única saída para aquele que descobre as determinações. O~livre
arbítrio pressupõe o distanciamento, a fuga, a completa negação. O~eu
irá se suicidar reiteradamente enquanto compreender a divindade como um
mero solipsismo. Enquanto o outro lhe parecer um instrumento, por que
não manter contatos apenas pela troca de produtos? No limite, o eu se
encerra em sua fortaleza -- o solipsista -- ou, de forma ainda mais
coerente, vai em busca do universo unívoco que sua individualidade
projeta -- o suicídio que busca a integração mais irrestrita do eu em si
mesmo. Mas a consciência expandida não pressupõe apenas a fuga. A
liberdade também pressupõe a negação da negação, a percepção de que o eu
que se autonomiza apreende os momentos heterônomos de sua expansão. O~eu
como construção social. O~eu como parte do nós, não mais pela união
primitiva, mas sim pela consciência do todo. Não tentar a Deus, nesse
sentido, refere-se ao movimento centrípeta do eu que retorna,
engrandecido, ao sentido da criação. A~humanidade toma forma em seus
entes conscientes que se percebem relacionais. A~quarentena de Cristo,
na verdade, durou quarenta séculos, milênios e eras. O~tempo individual
da maturação do espírito.

O socialismo, em sua possibilidade de ser apreendido como capitalismo de
Estado, e o capitalismo de mercado contrapõem necessidade e liberdade
como identidades inequivocamente antípodas, como se não houvesse uma
determinação recíproca que as imbricasse de modo dinamicamente
relacional. Quanto mais necessidade, menos liberdade, até o ponto em que
o todo se estilhaça porque o nível de consciência total se retrai.
Alguns grupos entendem a dinâmica do poder e, com a consciência
pragmática, porém embotada, só fazem planejar golpes para que também
eles sejam idolatrados. Quanto mais liberdade, menos necessidade, até o
ponto em que o eu se estilhaça porque o nível de consciência pessoal se
retrai. Alguns indivíduos entendem a dinâmica do isolamento e, com a
consciência embotada, porém pragmática, só fazem planejar a mônada já
estéril e a fuga para que também eles permaneçam sós. Ainda não chegamos
à fusão entre liberdade e necessidade. O~eu transita do amorfo para o
morfológico, mas sente muita dificuldade em abrir mão do que lhe é
próprio. Que fazem, então, as religiões nos momentos de fratura? Pregam
o milagre, o mistério e a autoridade.

\begin{quote}
Não há, repito"-te, preocupação mais aguda para o homem que encontrar o
mais cedo possível um ser a quem delegar esse dom da liberdade que o
infeliz traz consigo ao nascer. Mas, para dispor da liberdade dos
homens, é preciso dar"-lhes a paz da consciência. (\ldots) Não há nada de
mais sedutor para o homem do que o livre arbítrio, mas também nada de
mais doloroso. (\ldots) Aumentaste a liberdade humana em vez de confiscá"-la
e assim impuseste para sempre ao ser moral os pavores dessa liberdade.
Em lugar da dura lei antiga, o homem devia doravante, com o coração
livre, discernir o bem e o mal, não tendo para se guiar senão a tua
imagem, mas não previas que ele repeliria afinal e contestaria mesmo tua
imagem e tua liberdade, esmagado sob essa carga terrível: a liberdade de
escolher? (\versal{DOSTOIÉVSKI}, 1971, p. 190)
\end{quote}

Discernir entre o bem e o mal, eis a alcunha teológica para a expansão
da consciência. O~bem e o mal, tenso equilíbrio entre a necessidade e a
liberdade do eu autoconsciente que vive e convive em sociedade. Para
aquele que teme a liberdade e quer esmagá"-la sob o jugo da necessidade,
sempre haverá a opção por não mais optar. A~razão pode se esconder de si
mesma, ou, por outra, o livre pensamento pode caminhar em círculos,
esferas e mundos que diluam o movimento da contradição com muros que
obliterem a superação. Assim, Cristo volta a ser o cordeiro de Deus,
aquele que se ofereceu em holocausto para expiar nossos pecados. Devemos
sentir culpa em face de Jesus, não devemos entrevê"-lo como mais um entre
nós. A~quarentena e as tentações são meras peças alegóricas, Cristo
ainda poderá descer da cruz. Se o fizer, o milagre voltará a nos
orientar. Sim, a Graça, a intromissão ao léu, o regozijo do Deus Pai de
quem tanta falta sentimos. Fundir a liberdade à necessidade é muito
árduo, por que não voltar a cindir tais instâncias? Ter que escolher a
cada instante é como tatear no escuro. O~mundo depende do ímpeto da
consciência, mas é mais fácil ter que se referir a uma totalidade
fechada, isto é, totalitária. Quando não soubermos o que dizer, a Igreja
lá estará. Concílios atrás de concílios farão o crivo do que deve ser
crível. Eis um movimento possível para quem, dialeticamente, é livre
para retroagir. A~eternidade como palco -- e como máscara.

Resta"-nos a última tentação, aquela que convoca o eu centrífugo para o
fardo dos grandes líderes, a liberdade que se degenera em irrestrita
vontade de poder, a liberdade que se transforma em necessidade apenas
para os demais que se ajoelham diante do trono. ``O demônio
transportou"-o, uma vez mais, a um monte muito alto, lhe mostrou todos os
reinos do mundo e a sua glória e disse"-lhe: `Dar"-te"-ei tudo isto se,
prostrando"-te diante de mim, me adorares'" (\versal{MATEUS}, 4, 10). Assim como
Deus se funde ao sentido do todo, o demônio também se transforma em
totalidade. A~totalidade do poder. O~cume é ainda mais alto, local
ocupado apenas pelos maiores conquistadores. Cristo seria não apenas o
milagreiro da massa; Jesus Cristo se transformaria no Pai, com a única
condição de que incorresse no parricídio. Quem são os grandes líderes
senão os brados atávicos contra a nova ordem que eles mesmos
sintetizaram? Os~grandes líderes são a síntese contraditória do novo
atávico, isto é, um novo momento histórico que possibilita outras bases
de desenvolvimento em estreita conexão com práticas retrógradas
profundamente arraigadas ao transcurso identitário dos homens. ``Para
trás, Satanás, pois está escrito: \emph{Adorarás o Senhor teu Deus, e só
a ele servirás}'' (\versal{IDEM}, 4, 11). Mas, ora, quem é Deus? Por que Cristo
ficou em silêncio diante do tribunal de Pôncio Pilatos? Quando o romano
lhe perguntou ``o que é a verdade?'', Cristo permaneceu em silêncio.
Poderia ter se eximido, poderia ter trazido uma explicação parcial e
tática, mas a consciência expandida e irrestrita quer dar a cada um
conforme o momento de necessidade e liberdade. Se Cristo houvesse se
batido contra Pilatos, o sangue voltaria a tingir a porta dos templos.
Podemos até mesmo pressupor que a revolta e a revolução já haviam sido
tentadas por outras metamorfoses de Cristo, sem a consequente libertação
pela qual todos ansiavam. Se Jesus houvesse obrigado Pilatos a acreditar
em seus ideais, a transformação se veria reduzida a uma mera coação, sem
que o governante romano pudesse transvalorar seus próprios valores
mediante uma nova consciência. Cristo seria, então, um agente da
inquisição, alguém que busca uma retratação formal e exterior, sem a
dialética entre o ímpeto moral e seu espraiamento pelo mundo.
Paradoxalmente, o carrasco não conseguiria se libertar de sua vítima.
Pilatos estaria, de uma forma ou de outra, preso ao exercício da
dominação. Pôncio não poderia deixar de ser Pilatos. Assim, o silêncio
de Cristo pressupõe a liberdade, o caminho plural, o reconhecimento da
universalidade no caminho de todos e cada um de nós. ``A humanidade'',
reconhece o grande inquisidor, ``teve sempre tendência no seu conjunto
para organizar"-se sobre uma base universal'' (\versal{DOSTOIÉVSKI}, 1971, p.
193). Mas a universalidade via de regra se constituiu como o chicote
sobre o dorso do dominado e do diferente. Assim, os egos nacionais que
se odeiam de modo encarniçado dentro de suas fronteiras conseguem
projetar uma aliança temporária para hastearem o ódio ainda mais atávico
contra o selvagem além"-mar. Lembremos que, para a autoridade máxima da
Igreja, as contradições históricas da natureza humana são totalmente
insolúveis. O~homem que quer voltar à espontaneidade da massa deve se
inclinar diante do grande líder. Em meio à logicidade
histórico"-filosófica de Dostoiévski, aquele que apreende a
especificidade narcísica do próprio eu pode recorrer ao suicídio como
consequência de seu isolamento ou atentar contra a vida alheia para
afirmar o poderio estéril de sua própria individualidade. No primeiro
caso, será lembrado através da carta que deixou aos familiares. No
segundo, receberá a pecha de homicida, será considerado culpado e
voltará a ser isolado do convívio social que inicialmente renegou. O
homicida apenas passa a ser exaltado quando age mediatamente, quando ora
segundo a cartilha de Maquiavel, quando aglutina o ódio geral em função
do poder que exerce a violência legítima segundo a delegação aos pares
-- os carrascos, os únicos que devem sujar as mãos. Vale dizer que os
individualistas da segunda tentação também podem procurar emprego entre
as fileiras dos verdugos; se assim o fizerem, conseguirão constituir uma
síntese contraditória entre o ímpeto individual e a volta ao todo: o
carrasco se individualiza radicalmente ao executar o outro que não o
reconhece ao mesmo tempo em que volta à comunidade geral obedecendo à
risca às leis do ressentimento coletivo.

\begin{quote}
{[}Tu, Cristo,{]} Terias podido então tomar o gládio de César. Por que
repeliste esse derradeiro dom? Seguindo esse terceiro conselho do
poderoso espírito, realizavas tudo quanto os homens procuram na terra:
um senhor diante de quem se inclinar, um guarda de sua consciência e o
meio de se unirem finalmente na concórdia em uma comunidade de
formigueiro, porque a necessidade da união universal é o terceiro e
derradeiro tormento da raça humana (\versal{IDEM}, p. 192).
\end{quote}

O grande inquisidor sintetiza a heteronomia a que o cristianismo
clerical condena a humanidade. Que significa ajoelhar"-se diante de um
ícone, não tomar decisões por si mesmo e unir"-se de modo amorfo e
autoritário a uma comunidade que desfigura o indivíduo? Ora, estamos
diante da estagnação autoritária e infantil da humanidade. A~Igreja do
mistério pretende reter o fluxo dialético da contradição.
Historicamente, a síntese pacificadora sempre foi abençoada pela
comunhão que coage. Como sucedâneo centrífuga, o indivíduo baço, trêmulo
e desorientado tateia pelas aporias que, tradicionalmente, o impediram
de desenvolver uma consciência \emph{autônoma e determinada}.

\begin{quote}
Em consequência desse individualismo, que nega a antiga lei das massas,
o homem ``sempre perde a fé em Deus''. Todas as civilizações terminam
nessa fase de irreligiosidade, e Dostoiévski acreditava que a Europa
chegara a esse estágio de declínio. ``A desintegração das massas em
personalidades, ou civilização, é um estado doentio e conduz, com
relação ao indivíduo, à perda de uma ideia viva sobre Deus e a uma
condição em que o homem sente"-se mau, está triste, perde a fonte da vida
viva, não conhece sensações espontâneas e está consciente de tudo''
(\versal{DOSTOIÉVSKI} apud \versal{FRANK}, 2002, p. 505).
\end{quote}

Dostoiévski se embebeu das aporias de seu tempo para tentar levar às
últimas consequências o movimento do espírito. A~Europa em fins do
século \versal{XIX} hasteava bandeiras totais. O~positivismo prometia o
mapeamento científico do mundo. Darwin reescreveu o \emph{Gênesis}. O
socialismo chegara a tomar de assalto a capital do mundo com a Comuna de
Paris. A~crise oitocentista parecia estar restrita ao subsolo que
Dostoiévski sempre se preocupou em escavar. As~aporias parecem tão
cristalizadas e contumazes a ponto de não mais ser perceptível o
movimento da contradição. Afinal, conforme Ivan arremata seu diálogo
monologado com Aliócha,

\begin{quote}
é uma felicidade medíocre atingir a liberdade perfeita, quando milhões
de criaturas permanecem para sempre desgraçadas, demasiado fracas para
usar de sua liberdade, (\ldots) {[}já que{]} esses revoltados débeis não
poderão jamais terminar sua torre {[}de Babel{]}, e não é para tais
gansos que o grande idealista sonhou sua harmonia (\versal{DOSTOIÉVSKI}, 1971, p.
195).
\end{quote}

A resignação nunca foi a síntese da dialética. A~felicidade medíocre da
indústria cultural, segunda -- ou mesmo primeira -- natureza da
atualidade, tenta estancar o movimento da contradição com a ode
irrestrita ao existente. Mas a depressão cotidiana é um sintoma de que a
promessa histórica não foi esquecida. A~patologia inocula os germes de
uma normalidade que transcende a apatia. A~individualidade contratual
que apenas respeita o outro tática e juridicamente mostra suas garras
tão logo a inadimplência e a compressão materiais não mais mantenham os
padrões pequeno"-burgueses. Dostoiévski via em Cristo o ideal da
liberdade daquele que, no ápice do desenvolvimento de sua personalidade,
doa"-se para a humanidade.

\begin{quote}
E a lei desse novo ideal {[}de Cristo{]} consiste na volta à
espontaneidade, às massas, mas de uma maneira livre. Não de forma
obrigatória, mas, ao contrário, voluntária e conscientemente no mais
alto grau. Está claro que essa maior voluntariedade é ao mesmo tempo uma
maior renúncia da vontade (\versal{DOSTOIÉVSKI} apud \versal{FRANK}, 2002, p. 506).
\end{quote}

Cristo, Buda, Maomé, Kardec. Universalidade verdadeiramente ecumênica. A
renúncia à vontade como nova espontaneidade claramente transcende o
estado original das massas. Eis o desafio, a síntese da liberdade que se
vê imiscuída e compromissada com a necessidade. Uma totalidade mecânica
e fechada torna recíprocos e necessários quaisquer incidentes na vida de
uma pessoa. A~interpretação conservadora se impõe: o que ocorreu e
ocorre sempre deveria ter sido e deve ser assim\footnote{No próximo
  capítulo, a filosofia da história de Hegel acompanhará a trajetória de
  cicatrização do espírito do homem ridículo. Neste momento, no entanto,
  nossa análise sobre a teleologia teológica como possível apologia do
  real como já sendo racional -- e não como um vir a ser racional, isto
  é, a utopia -- pode municiar"-se da crítica que Theodor Adorno (2013)
  faz ao conceito de realidade racional (e racionalidade real) em Hegel,
  de modo a entrevermos a profunda (e dialética) diferença entre a
  totalidade como um sistema fechado, autorreferencial e que corrobora o
  existente tal como ele já existe e a totalidade aberta, processual e a
  entrever as \emph{possibilidades} do real em devir: ``Ela {[}a
  filosofia de Hegel{]} gostaria de justificar o real como sendo de todo
  racional e dispensar a reflexão que a ele se opõe com aquela
  superioridade que mostra o quão difícil o mundo é, tirando disso a
  sabedoria segundo a qual ele não pode ser mudado. Se Hegel foi burguês
  em algum ponto, foi aqui. Mas julgá"-lo a esse respeito seria ainda uma
  atitude subalterna. O~elemento mais questionável de sua doutrina, e
  por isso também o mais divulgado, aquele segundo o qual o real é
  racional, não era simplesmente apologético. \emph{Para ele, a razão se
  encontra na mesma constelação da liberdade. Liberdade e razão, uma sem
  a outra, são absurdas. Apenas na medida em que o real deixa
  transparecer a ideia da liberdade, a autodeterminação real da
  humanidade, ele pode valer como racional} {[}grifos meus{]}.
  Escamotear essa herança que Hegel tem do esclarecimento e pretender
  que sua lógica na verdade nada tem que ver com a disposição racional
  do mundo é falsificá"-lo. Mesmo lá, em seu período tardio, no qual
  Hegel defende o positivismo, aquilo que simplesmente é e que ele
  atacava em sua juventude, há um apelo à razão, que compreende o que
  existe para além daquilo que existe, entendendo"-o a partir da
  perspectiva da consciência"-de"-si e da autoemancipação do homem'' (pp.
  122-123). A~percepção das profundas afinidades eletivas entre razão e
  liberdade, em Hegel, pode ser claramente relacionada ao entendimento
  dostoievskiano de que a cicatrização do espírito pressupõe a
  \emph{liberdade determinada} para que a razão -- vale dizer, a
  reconciliação do que existe, a superação (\emph{Aufhebung}) do atual
  estado de coisas -- possa, então, reger, isto é, libertar a realidade.}.
A consciência se vê ainda mais premida. Por que não pensarmos, a reboque
da dialética negativa de Ivan, em uma totalidade aberta e contingente
que dê tempo e espaço para que a necessidade se imiscua cada vez mais
com a liberdade? Que pode fazer a condenação mundana ou espiritual senão
resvalar o arrependimento por meio do encarceramento físico ou
reencarnado? A~consciência do algoz não precisa apenas do perdão alheio.
Ela precisa perdoar a si mesma. Tal movimento de introjeção do amor
pressupõe a ressignificação completa da culpa. Talião deixa de ser o
motor da história para oferecer a outra face. O~ressentimento se
estilhaça como vestígios da memória, as diferentes vidas do espírito em
seu transcurso da heteronomia à autonomia.

O peso que recai sobre a liberdade em meio a uma totalidade aberta não
poderia ser maior. Sempre há o risco de que se projete a eternidade como
o terreno idílico para a síntese do que não pode ser reconciliado neste
mundo. Mas Dostoiévski não concebia a vida terrena como a única
possibilidade para o desenvolvimento do espírito.

\begin{quote}
Dostoiévski obviamente comunga com aqueles que (\ldots) não conseguiam
imaginar uma transformação genuinamente apocalíptica dentro dos limites
da vida na terra. Seja como for, acolhe a possibilidade do que chama uma
``volta às massas'', uma cura para a doença do individualismo e da
civilização, seguindo o exemplo de Cristo e aceitando o ideal que ele
trouxe para a humanidade. Isso suscitaria a restauração de qualquer
unidade que ainda é realizável na terra; e essa nova unidade, em
concordância com o mito, também se realizaria num plano mais alto -- não
mais seria uma unidade do instinto, mas uma unidade realizada livremente
mediante uma renúncia consciente da vontade. (\ldots) A~ênfase que
Dostoiévski, nessa nova unidade a ser alcançada através de Cristo,
coloca na ``liberdade'' já prefigura um dos motivos centrais da Lenda do
grande inquisidor (\versal{FRANK}, 2002, p. 508).
\end{quote}

Dialeticamente, o grande inquisidor de Ivan faz o papel da antítese para
voltar a animar o movimento da contradição. O~clérigo cínico e
autoritário tem um momento de verdade a contrapelo de si mesmo. Do
instinto à consciência, da necessidade à liberdade determinada, o
movimento do espírito se supera em suas finitudes reiteradas
transformando o finito no infinito determinado. A~filosofia
dostoievskiana da história aparentemente subsume a pluralidade dos mais
diversos povos em função das grandes linhas de força que sintetizariam a
base universal sobre a qual a humanidade, segundo o escritor, sempre
procurou se assentar. Mas a síntese dostoievskiana, a reboque e a
contrapelo da teologia de Ivan, não coage a diferença a uma
universalidade unívoca. A~renúncia à vontade como o ápice da vontade
pressupõe o tempo precípuo para o processo e a pluralidade dos caminhos.
Ao fim e ao cabo, a renúncia à própria vontade transforma o indivíduo
total e o todo individual em deuses criadores. Não seria o ápice da
bondade racional de Deus a permissão para que houvesse deuses, a
permissão para que a bondade se espraiasse? A~liberdade, inclusive, para
que a bondade fosse revogada. Eis a síntese dinâmica entre a ordem e o
caos.

Munidos, então, dos resultados de nossa análise sobre a filosofia da
história dostoievskiana a partir do grande inquisidor a dialogar com as
três tentações no deserto a que Cristo foi submetido, passemos à
trajetória de cicatrização do espírito que levou o homem ridículo,
protagonista de ``O sonho de um homem ridículo'', do ímpeto pelo
suicídio ao transbordamento daquilo que a personagem vivenciou como o
reencontro com a verdade e o sentido. Para usarmos uma expressão
sobremaneira dostoievskiana, \emph{a carne e o sangue} do homem ridículo
desvelarão os momentos teleológicos da história -- suas contradições,
aporias e possibilidades de superação em face do atual estado de coisas.

\chapter*{Capítulo 5\\
\bigskip
\emph{A utopia como cicatrização do espírito}}

\addcontentsline{toc}{chapter}{Capítulo 5\\\scriptsize{\emph{A utopia como cicatrização do espírito}}}
\hedramarkboth{Capítulo 5}{}

\section{5.1. Preâmbulo}

Após a análise da filosofia da história dostoievskiana a partir de uma
interpretação das três tentações no deserto a que Cristo foi submetido,
é preciso compreender como o escritor russo narra a trajetória de
cicatrização do espírito de uma de suas personagens mais emblemáticas:
acompanharemos o homem ridículo, protagonista de ``O sonho de um homem
ridículo'' (1877), do (anti)clímax de seu niilismo à beira do suicídio
até a retomada de sua vinculação histórico"-espiritual com a vida. Nesse
sentido, a filosofia da história em Dostoiévski ganhará \emph{alma e
corpo}.

Desde o início, vale frisar que o conto ``O sonho de um homem ridículo''
desponta como uma narrativa peculiar em meio à obra agônica de Fiódor
Dostoiévski. Como o homem do subsolo, protagonista das \emph{Memórias do
subsolo}, o homem ridículo não é nomeado. Por sinal, Rudolf Neuhäuser
(1993) ressalta que

\begin{quote}
``O sonho'' desponta com a mesma estrutura da primeira parte de
\emph{Memórias do subsolo}, na qual a personagem falara sobre a doença
do homem contemporâneo -- o indivíduo \emph{hiperconsciente.} O~homem
ridículo é um descendente direito do homem do subsolo, o que o
transforma em um ``moderno progressista russo e um cidadão deplorável de
São Petersburgo'' (p. 181).
\end{quote}

No entanto, à diferença de \emph{Memórias do subsolo} -- e de boa parte
da obra de Dostoiévski --, ``O sonho de um homem ridículo'' narra uma
trajetória redentora para o protagonista. O~homem ridículo, a princípio
um niilista à iminência do suicídio, encontra, por intermédio de um
sonho escatológico, uma verdade espiritual que o alça para além do nada
que quase o levara a dar um tiro na têmpora direita. Sergei Hackel
(1982), nesse sentido, menciona cinco tópicos que Dostoiévski chegara a
esboçar em suas anotações para o romance \emph{O idiota}, tópicos que
bem podemos utilizar como uma síntese para a trajetória de ``O sonho de
um homem ridículo'':

\begin{quote}
(i) O `centro moral' {[}Deus{]} pode ter sido corroído e esquecido; (ii)
meramente retratar sua erosão não é o bastante; (iii) a ideia principal
requer uma nova justificativa; (iv) pode ser difícil defini"-la ou até
mesmo retratá"-la; (v) ainda assim, alguns sinais da ideia poderiam e
deveriam ser apresentados.
\end{quote}

Não à toa, Bakhtin (2008) considera a narrativa em questão uma suma das
questões que transpassam a obra de Dostoiévski. Rudolf Neuhäuser (1993)
afirma que os

\begin{quote}
estudos desse conto têm se concentrado em suas óbvias implicações
filosóficas e em sua significância para a visão de mundo e a poética de
Dostoiévski. Central para o conto é a visão do herói sobre a Idade de
Ouro, interpretada por Horst"-Jürgen Gerigk como uma ``cura da falsa
consciência''. Vladimir Zakharov chegou a uma conclusão similar, ao ver
no conto uma ``estória da iluminação do herói, de sua descoberta da
verdade, e o sonho se torna um mito sobre os destinos históricos da
humanidade''. Robert"-Louis Jackson, por outro lado, enfatizou a
significância do texto para a poética do autor, supondo que a visão de
Dostoiévski fosse a seguinte: ``A arte última se torna revelação
(profecia); ela não explora somente a realidade social do homem, mas
escava a realidade última do espírito humano e de seu destino, esse
mundo invisível de `fins e começos' que permanece para o homem um
domínio do fantástico''. {[}Por essa perspectiva, ``O sonho''
expressa{]} ``uma crença em uma ordem moral e em uma realidade
transcendental''. Konrad Onasch, por sua vez, considera ``O sonho'' um
diálogo entre ``Dostoiévski \versal{I} e Dostoiévski \versal{II}, i.e., o religioso e o
cético -- ambas as posições levadas às últimas consequências'' (pp.
175-176).
\end{quote}

Entrevista a pletora de temas fundamentais de Dostoiévski movimentados
por ``O sonho de um homem ridículo''\emph{,} também é importante frisar
que, no capítulo 4 deste livro, nossa análise da filosofia da história
em Dostoiévski dialogou com o ensaio ``Socialismo e cristianismo'', que
Dostoiévski escreveu em 1864. Nesse sentido, Gerald J. Sabo (2009)
também considera importante

\begin{quote}
para a apreciação cristã de ``O sonho'' a intertextualidade com o ensaio
``Socialismo e cristianismo'', cujas três fases do desenvolvimento
social e pessoal são replicadas na vida do homem ridículo no duplo do
planeta Terra e, então, em sua existência após o sonho. De fato, a
aparição sequencial de todos esses vários elementos na narrativa de ``O
sonho'' são marcadores progressivos do processo que transforma o homem
ridículo de um ser indiferente e solipsista em uma pessoa dinânica que,
sob a influência regeneradora de Cristo e do pensamento cristão, se
compromete apaixonadamente a persuadir os demais a se transformar como
ele e, ao fazer isso, a transformar a sociedade -- e isso sem se
importar com o caráter ridículo que sua pregação possa ter para os
ouvintes/leitores (pp. 47-48).
\end{quote}

Além dos resultados da análise de ``O Sermão da Estepe: Ivan Karamázov e
a filosofia dostoievskiana da história'' serem muito importantes para
que entendamos o sentido da redenção que configura ``O sonho de um homem
ridículo''\emph{,} um filósofo alemão fundamental para a formação de
Dostoiévski nos ajudará a ressignificar a trajetória do homem ridículo e
de sua cosmovisão: falamos sobre Georg Wilhelm Friedrich Hegel,
sobretudo a partir de sua obra \emph{Filosofia da história}\footnote{Tradução
  de Maria Rodrigues. Brasília: Editora da UnB, 1995.}. Sobre a
influência da filosofia alemã para o contexto artístico"-intelectual de
que Dostoiévski fazia parte, Susan McReynolds (2002) afirma que

\begin{quote}
os estudiosos de Dostoiévski concordam que ele chegou à maturidade
intelectual em um ambiente saturado com as ideias de Kant, Schelling e
Hegel e que, independentemente dos textos lidos pelo escritor, as ideias
de tais pensadores permearam sua arte e pensamento. A~filosofia alemã
frequentemente chegava a Dostoiévski por meio de suas relações com os
intelectuais russos, entre os quais Vissarion Bielinski, que dominava
vários dos textos originais (p. 94).
\end{quote}

Apresentadas, em termos gerais, as vinculações entre a filosofia
dostoievskiana da história, ``O sonho de um homem ridículo'' como a
superação (\emph{Aufhebung}) do niilismo e a influência da filosofia
alemã -- notadamente, da dialética histórica de Hegel -- para a obra do
escritor russo, passemos a acompanhar, \emph{in media res}, a trajetória
do homem ridículo como a imbricação dialética envolvendo a cicatrização
do espírito e a utopia.

\section{5.2. Só existe uma ação verdadeiramente niilista: o suicídio\protect\footnotemark}
\footnotetext{Variação dostoievskiana da frase que abre o ensaio
  \emph{O mito de Sísifo}, de Albert Camus: ``Só há um problema
  filosófico verdadeiramente sério: o suicídio'' (Paris: Gallimard,
  1942, p. 17).}
  

\begin{quote}
Eu sou um homem ridículo. Agora eles me chamam de louco. Isso seria uma
promoção, se eu não continuasse sendo para eles tão ridículo quanto
antes. Mas agora já nem me zango, agora todos eles são queridos para
mim, e até quando riem de mim -- aí é que são ainda mais queridos. Eu
também riria junto -- não de mim mesmo, mas por amá"-los, se ao olhar
para eles não ficasse tão triste. Triste porque eles não conhecem a
verdade, e eu conheço a verdade. Ah, como é duro conhecer sozinho a
verdade! Mas isso eles não vão entender. Não, não vão entender
(\versal{DOSTOIÉVSKI}, 2003, p. 91).
\end{quote}

Poderíamos considerar, a princípio, que o homem ridículo, narrador de
nossa estória, lança mão do mesmo açoite sadomasoquista com que o homem
do subsolo se flagela -- e nos flagela. Ambas as personagens se mutilam
diante da alteridade espectral -- alteridade, não nos esqueçamos, que é
derivada do solipsismo paradoxal do subsolo, uma solidão que não deixa
de ser transpassada pelo outro sequer por um instante, mas uma solidão
que não convive, efetivamente, com a alteridade. O~homem ridículo se
chicoteia -- ``ridículo'', ``louco''. Como já conhecemos a (tauto)lógica
do subsolo, sabemos que o rebaixamento de si (tese) pressupõe o
movimento posterior que achincalha o outro espectral que se pensava
superior a partir da voz narrativa autodepreciativa (antítese). Em um
paradoxo deveras dostoievskiano, a vítima passa a golpear o algoz --
algoz que a própria vítima \emph{projetara} -- com as feridas que o
carrasco lhe impõe.

Estaríamos, assim, diante de mais uma narrativa ultracompetitiva ao
longo da qual todas e quaisquer relações passariam pelo crivo do poder e
da disputa -- o ego, ainda uma vez, como o princípio único de todas as
coisas. Ocorre que, em ``O sonho de um homem ridículo'', encontramos um
raro momento na obra de Dostoiévski em que, como pretendemos demonstrar,
o ego deixa de orbitar ao redor do niilismo. O~``ridículo'' da
personagem passará por uma completa transfiguração ao final de sua
jornada: \emph{ridículo} será tudo aquilo que o niilismo proscreve;
\emph{ridículo} será o ímpeto por uma humanidade totalmente outra -- a
utopia; \emph{ridículo} será o sentido como eternidade -- Deus. Assim, a
colocação ``(\ldots{}) agora já nem me zango, agora todos eles são
queridos para mim, e até quando riem de mim -- aí é que são ainda mais
queridos'' não pode ser lida apenas como uma queda de braço
dostoievskianamente às avessas -- o derrotado só faz açoitar o vencedor
com o fardo de sua vitória, de modo a atormentá"-lo com a culpa. O
ressentimento que Nietzsche frequentemente entreviu nos humilhados e
ofendidos de Dostoiévski não dá conta de uma tensão que se afrouxa e que
já não se quer mais como vontade de poder. ``Eu também riria junto --
não de mim mesmo, mas por amá"-los, se ao olhar para eles não ficasse tão
triste''. Diferentemente do homem do subsolo, o homem ridículo já não
quer rir de si mesmo. Ele quer rir \emph{conjuntamente.} Ocorre que o
fato de \emph{eles} não conhecerem a verdade o entristece sobremaneira.
Nesse momento, poderíamos pensar em uma personagem que se quer ainda uma
vez superior por se dizer detentora da verdade. A~vaidade estaria aí.
Mas ``(\ldots) como é duro conhecer sozinho a verdade!'' A~verdade não se
consuma sem a partilha, a verdade precisa ser \emph{compartilhada.} No
início de ``O sonho'', nosso herói já prenuncia a potência e os limites
de sua utopia. Ele pressente que os demais não vão entendê"-lo, que será
muito difícil repartir o pão da verdade, mas, se isso não acontecer, sua
utopia não se tornará história. Assim, em um momento literário que se
configura para além da agonia, Dostoiévski reverte a dialogia utilitária
em um ímpeto fundamental pela comunhão.

O primeiro parágrafo de nossa estória se apresenta, assim, como uma
síntese para toda a trajetória -- algo como o alfa e o ômega, a moldura
da transvaloração do ridículo. O~ímpeto para que o \emph{rir do outro}
se converta em \emph{rir com o outro.} O~homem ridículo já passara por
sua experiência fundamental -- o sonho redentor. Há quanto tempo ele vem
tentando desvelar a verdade aos homens e mulheres? Não podemos precisar,
mas veremos que, ao longo da estória, a curvatura do tempo, sua
vertiginosa aceleração, será um fator para entendermos a dimensão
cicatrizante das experiências.

Após a apresentação panorâmica do sentido de sua narrativa, o homem
ridículo passa sua vida em revista. Nascimento, escola, universidade --
tudo sempre lhe pareceu ridículo, vale dizer, completamente desprovido
de sentido. E~o ridículo, nesse momento em que o homem ridículo narra
com uma consciência retraída -- isto é, uma consciência que procura
retomar seu horizonte embotado e anterior ao sonho --, significa
autoflagelação, competição e orgulho.

\begin{quote}
A cada ano aumentava e se fortalecia em mim essa mesma consciência do
meu aspecto ridículo em todos os sentidos. Todos riam de mim, o tempo
todo. Mas ninguém sabia nem suspeitava que, se havia na terra um homem
mais sabedor do fato de que sou ridículo, esse homem era eu, e era justo
isso que mais me ofendia, que eles não soubessem disso, mas aqui o
culpado era eu mesmo: sempre fui tão orgulhoso que por nada no mundo
jamais iria querer confessar o fato a ninguém. Esse orgulho cresceu em
mim ao longo dos anos, e se acontecesse de me deixar confessar, diante
de quem quer que fosse, que sou ridículo, creio que imediatamente, nessa
mesma noite, estouraria os miolos com um revólver (\versal{IDEM}, pp. 91-92).
\end{quote}

Vemos em funcionamento, no fragmento em questão, a dialética
dostoievskiana que imiscui, visceralmente, masoquismo e sadismo. Todos
os demais -- e todos os demais, por não serem o eu, se tornam
competidores -- humilham o homem ridículo. Ele, a quintessência do que é
o ego deformado pelo individualismo burguês, sabe que, ainda assim, eles
não o conhecem suficientemente, ele se sabe ainda mais ridículo do que o
juízo que todos os demais fazem sobre ele próprio. Assim, a
hiperconsciência de sua própria humilhação o eleva em relação aos demais
-- \emph{e eles não têm consciência disso.} Como os outros tendem a ser
derivações e projeções do homem ridículo, os ignorantes zombam do que
não entendem. Ocorre que o inferior ridículo, para ser superior,
resguarda para si a hiperconsciência de sua inferioridade. E~o orgulho,
nesse sentido, é o mecanismo que o aparta dos demais na mesma medida em
que, supostamente, eleva o homem ridículo de forma secreta. Eleva"-o em
sua miséria, eleva"-o para isolá"-lo. Essa dinâmica profundamente doentia
é a imagem mais rematada do niilismo que nos torna ilhas de nós mesmos
em meio ao arquipélago da sociedade. O~orgulho -- a bem dizer, a vaidade
mórbida -- do homem ridículo não sabe como abrir mão das relações
reduzidas a duelos, a não ser com o próprio naufrágio, isto é, o
suicídio.

\begin{quote}
Mas desde que me tornei moço, apesar de reconhecer mais e mais a cada
ano a minha horrível qualidade, por um motivo qualquer fiquei um pouco
mais tranquilo. (\ldots) Talvez porque na minha alma viesse crescendo uma
melancolia terrível por causa de uma circunstância que já estava
infinitamente acima de todo o meu ser: mais precisamente -- ocorrera"-me
a convicção de que no mundo, em qualquer canto, \emph{tudo tanto faz.}
(\ldots) Senti de repente que para mim \emph{dava no mesmo} que existisse
um mundo ou que nada houvesse em lugar nenhum. Passei a perceber e a
sentir com todo o meu ser que \emph{diante de mim não havia nada.} (\ldots)
Antes também não havia nada. (\ldots) Pouco a pouco me convenci de que
também não vai haver nada jamais. Então de repente parei de me zangar
com as pessoas e passei a quase nem notá"-las (\versal{IDEM}, pp. 92-93).
\end{quote}

Nesse momento, não é o ímpeto por comunhão que faz com que o homem
ridículo pare de se zangar com as pessoas, mas o mais rematado
solipsismo. Como \emph{tudo tanto faz}, já não lhe importa que os demais
o considerem ridículo -- e também já não lhe importa a queda de braço
sádica entre senhor e escravo. Presente, passado e futuro se embaralham
e não significam nada. O~passado não pode ser redimido, o presente só se
esvai e o futuro não passa de uma mera ilusão. \emph{Tudo tanto faz} --
à exceção da possibilidade de tornar coerente a pasmaceira do nada, à
exceção do suicídio. Ocorre que o homem ridículo, à iminência do
suicídio, encontra, em um dia 3 de novembro e por intermédio de um sonho
escatológico, uma verdade espiritual que o alça para além do nada que
quase o levara a dar um tiro na têmpora direita.

É bem verdade que Raskólnikov, protagonista de \emph{Crime e castigo},
passa por um longo e doloroso processo de conversão após cometer um
duplo assassinato. O~ex"-estudante de Direito vertera sangue alheio para
testar, pragmaticamente, se era possível matar o \emph{não matarás} e
ocupar o trono que a morte histórica de Deus deixara vago. Ocorre que
Raskólnikov, contra seu ímpeto hedonista, contra sua vontade de poder,
não consegue suportar o fardo de Caim: o criminoso sucumbe diante da
culpa, o crime se ajoelha diante do castigo. {[}Nietzsche bem poderia
dizer que o homicida, \emph{no momento em que comete o assassínio}, está
para além do bem e do mal; as decorrências do crime, no entanto, fazem
com que o (suposto) super"-homem volte a rastejar entre os homens
ordinários, para os quais nem tudo -- ou pior, quase nada -- é
permitido.{]} Sônia Marmieládova, uma síntese dostoievskiana entre Maria
e Maria Madalena, uma fusão inconsútil entre a santa e a prostituta,
entre o céu e a terra, encarna o sentido da redenção para Raskólnikov. O
homicida encontra em Sônia uma companheira que se dispõe a abandonar a
si mesma para segui"-lo durante o degredo siberiano -- para consolá"-lo
durante o processo prisional de expiação e cicatrização de sua culpa. É
assim que, ao fim de \emph{Crime e castigo}, deparamos, em linhas
gerais, com o processo de redenção de Raskólnikov:

\begin{quote}
Ela {[}Sônia{]} passou todo esse dia intranquila e à noite chegou até a
adoecer. Mas ela estava tão feliz que quase se assustava com a sua
felicidade. Sete anos, \emph{apenas} sete anos! No começo da sua
felicidade, em outros instantes, os dois estavam prontos para considerar
esses sete anos como sete dias. Ele não sabia nem que essa nova vida não
lhe sairia de graça, que ainda deveria pagar caro por ela, pagar por ela
com um grande feito no futuro\ldots

\noindent Mas aqui já começa outra história, a história da renovação gradual de um
homem, a história do seu paulatino renascimento, da passagem progressiva
de um mundo a outro, do conhecimento de uma nova realidade, até então
totalmente desconhecida. Isto poderia ser o tema de um novo relato --
mas este está concluído (\versal{DOSTOIÉVSKI}, 2001, p. 561).
\end{quote}

Aquilo que \emph{Crime e castigo} apenas esboça, vale dizer, o processo
de transvaloração de Raskólnikov, constitui a quintessência de ``O sonho
de um homem ridículo'' -- ``a história da renovação gradual de um homem,
a história do seu paulatino renascimento, da passagem progressiva de um
mundo a outro, do conhecimento de uma nova realidade, até então
totalmente desconhecida'' --, com a diferença de que, como veremos, o
sonho escatológico do homem ridículo desponta não como uma renovação
gradual, mas como uma explosão efetivamente dostoievskiana.

O homem ridículo é o antípoda por excelência de uma gama de personagens
dostoievskianas que encarnam as diversas fenomenologias da vontade de
poder. Como já dissemos, o homem do subsolo lança mão de sua consciência
hipertrofiada para diagnosticar o niilismo de sua época e se regozijar,
de forma sadomasoquista, com a impossibilidade de estabelecer relações
verdadeiramente humanas com os demais. O~homem do subsolo se vangloria
por poder se resguardar em sua cripta, por ter um refúgio de onde pode
observar o mundo -- sem entrar em contato efetivo com a vida. O~prazer
do subsolo se confunde com a dor do recluso. Já Svidrigáilov
(\emph{Crime e castigo}) e Stavróguin \emph{(Os demônios}) auferem
prazer da sexualidade desenfreada que não poupa sequer as criancinhas.
Se Deus não existe e tudo é permitido, apenas a manada respeita as
normas que não têm repercussão alguma senão aqui e agora. Se não há
eternidade -- raciocinam as personagens dostoievskianas --, toda e
qualquer transgressão só terá repercussão se a sanha do meu prazer for
castrada; em outras palavras, se eu não for ardiloso o suficiente e me
deixar punir pelos hipócritas da lei -- aqueles que querem nos fazer
acreditar que \emph{suas} leis, isto é, as leis de \emph{seus}
interesses e de \emph{sua} classe têm validade universal.

Imbuído do ímpeto do suicídio, o homem ridículo diria que o homem do
subsolo, Svidrigáilov e Stavróguin não estão sendo suficientemente
radicais em seu niilismo. Ser um sádico, ser um assassino, ser um
pedófilo, ser o Todo"-Poderoso, tudo isso é agir em um mundo que, após o
crepúsculo dos deuses, já não faz sentido algum. Imbuído da dialética
dostoievskiana, o homem ridículo sentencia que querer romper a norma,
ainda que por intermédio da destruição, significa ovacionar o mundo.
Para o homem ridículo, a única conclusão escatologicamente coerente para
o desterro transcendental dos homens é o suicídio. Toda e qualquer
tergiversação após a morte de Deus -- à direita e à esquerda, para a
conservação ou a revolução, a fim de destruir ou criar -- não passa de
mera covardia. Quando o homem ridículo compreende que o flagelo do tempo
é a substância de todas e quaisquer experiências, o suicida ressoa
Mefistófeles, para quem tudo o que existe merece perecer. Os~entes
queridos, a amada, os amigos, nada e, sobretudo, ninguém permanecem.
Sendo assim -- prossegue a (escato)lógica do homem ridículo --,
\emph{tudo tanto faz}. Tanto faz se ela diz que me ama em meio à
felicidade da nudez ou se fico a espreitar a neve pela janela para
driblar a solidão; tanto faz se o filho (não) perdoou ao pai moribundo;
tanto faz se a mãe se despediu dos filhos. \emph{Tanto faz, tudo tanto
faz}. Tudo passa a ser indiferente. Em verdade, em verdade, o homem
ridículo nos diz: só há apenas uma ação verdadeiramente niilista -- o
suicídio. Nesse sentido, Kate Holland (2000) cita um fragmento do
\emph{Diário de um escritor} datado de 1876 -- um ano antes, portanto,
de Dostoiévski escrever ``O sonho de um homem ridículo'': ``Eu estou
absolutamente convencido de que a maioria desses suicídios, direta ou
indiretamente, acontece por causa da mesma doença espiritual: a ausência
de um ideal de existência mais elevado na alma dessas pessoas'' (p.
101).

É assim que o homem ridículo nos leva a uma gélida noite de São
Petersburgo, ``a mais tenebrosa noite que pode haver'' (\versal{DOSTOIÉVSKI},
2003, p. 93). Nosso herói caminha a esmo e, quando olha para o céu e
espreita uma estrelinha brilhante, ele decide que é chegado o momento de
o niilismo existencial se transformar em túmulo. Naquela noite, sem
mais, ele se suicidaria.

\begin{quote}
Fazia dois meses que isso já estava firmemente decidido, e, apesar de
ser pobre, comprei um belo revólver e carreguei"-o naquele mesmo dia. Já
se tinham passado dois meses, porém, e ele ainda jazia na gaveta; mas
para mim tudo era a tal ponto indiferente que me deu vontade, afinal, de
arranjar um minuto em que tudo não fosse assim tão indiferente, para quê
-- não sei. E, desse modo, durante esses dois meses, a cada noite eu
voltava para casa pensando que me mataria. Só esperava o minuto. E~agora
essa estrelinha me trouxe a ideia, e decidi que seria \emph{sem falta}
nessa mesma noite. Mas por que a estrelinha me trouxe essa ideia -- não
sei (\versal{IDEM}, p. 94).
\end{quote}

Apesar de ser pobre, o homem ridículo nos diz que comprara um belo
revólver e o carregara naquele mesmo dia. Ora, mesmo o niilismo mais
rematado precisa de uma despedida solene -- afinal, o solipsista,
criador do universo, precisa de um suicídio à altura. Essa derivação da
vaidade não nos deve fazer esquecer que, sob a indiferença contumaz, se
esgueirava a vontade de que houvesse ``um minuto em que tudo não fosse
assim tão indiferente, para quê -- não sei''\footnote{A esse respeito,
  Allan Kardec (1944), um autor com o qual dialogaremos mais adiante,
  faz a seguinte colocação: ``E, se as consequências {[}do niilismo{]}
  não são tão desastrosas quanto poderiam ser, é, em primeiro lugar,
  porque na maioria dos incrédulos há mais jactância que verdadeira
  incredulidade, mais dúvida que convicção -- possuindo eles mais medo
  do nada do que pretendem aparentar --, o qualificativo de espíritos
  fortes lisonjeia"-lhes a vaidade e o amor"-próprio; em segundo lugar,
  porque os incrédulos absolutos se contam por ínfima minoria e sentem a
  seu pesar os ascendentes da opinião contrária, mantidos por uma força
  material'' (p. 13).}.

A estrelinha lhe traz a ideia do suicídio -- a mesma estrelinha que,
dentro em breve e em meio ao sonho, o homem ridículo verá em sua viagem
intergaláctica conduzida pelo espírito"-guia. Como nosso narrador já
vivenciou seu sonho e nos conta suas experiências retrospectivamente, é
possível dizer que o passado caótico e (supostamente) indiferente é
ressignificado por meio do \emph{acaso objetivo.} Tudo o que lhe parecia
disperso e desprovido de sentido -- a manifestação fenomenológica do
niilismo, o paradoxo do nada encarnado -- passa a compor a miríade dos
momentos que, \emph{a posteriori}, são apreendidos como mediações para
comunicar ao nosso herói que havia algo além (e não aquém) do fundo do
poço. Não é à toa, então, que, logo em seguida à entrevisão da
estrelinha que lhe fizera chegar à decisão cabal do suicídio, o homem
ridículo depara com uma menininha em desespero que começa a forjar seu
processo de conversão.

Se estivéssemos diante de uma narrativa bíblica -- se estivéssemos, por
exemplo, diante do altar que o patriarca Abraão preparara para imolar
seu filho Isaac \emph{ad majorem Dei gloriam} --, um anjo do Senhor
apareceria no momento derradeiro para impedir a consecução da tragédia.
Se Dostoiévski quer refletir, narrativamente, sobre os estertores do
niilismo, isto é, sobre o nada como a quintessência do
homem"-que"-se"-sabe"-para"-a"-morte, é preciso criar uma situação limítrofe
que submeta o desprezo e a indiferença a um último desafio -- a um
desafio derradeiro. Assim será possível descobrir, vivencialmente, se
tudo o que existe merece perecer. Súbito, enquanto caminha pelo ocaso
noturno de São Petersburgo, uma menininha pálida e desesperada resvala a
indiferença do homem ridículo.

\begin{quote}
A menina tinha uns oito anos, de lencinho e só de vestidinho, toda
encharcada (\ldots{}). De repente ela começou a me puxar pelo cotovelo e
a me chamar. Não chorava, mas soltava entre gritos umas palavras que não
conseguia pronunciar direito, porque tremia toda com tremedeira miúda de
calafrio. Estava em pânico por alguma coisa e berrava desesperada:
``Mámatchka! Mámatchka!'' Voltei o rosto para ela, mas não disse uma
palavra e continuei andando, só que ela corria e me puxava (\ldots{}).
Embora ela não articulasse bem as palavras, entendi que a sua mãe estava
morrendo em algum lugar, ou que alguma coisa acontecera lá com elas, e
ela fora correndo chamar alguém ou achar alguma coisa para ajudar a mãe.
Mas não fui atrás dela e, de repente, me veio a ideia de enxotá"-la.
Primeiro lhe disse que fosse procurar um policial. Mas ela de repente
juntou as mãozinhas e, soluçando, sufocando, corria sem parar ao meu
lado e não me largava. Foi então que bati o pé e dei um grito. Ela
apenas gritou bem forte: ``Senhor, senhor!\ldots{}'' (\versal{IDEM}, p. 95).
\end{quote}

O homem ridículo enxota a menininha como a um vira"-lata, mas ela acabara
de lhe trazer a semente da discórdia. Já de volta ao cubículo humilde
que alugava, ele se senta à mesa em silêncio. ``Tirei o revólver e o
coloquei à minha frente. (\ldots{}) {[}Então,{]} perguntei a mim mesmo:
`É assim?' Ou seja, vou me matar. (\ldots{}) E~é claro que teria me
matado, se não fosse aquela menina'' (\versal{IDEM}, p. 97). Aquela menininha
indefesa a lhe pedir ajuda, na verdade, e sem o saber, estendera a mão
(e a vida) ao homem ridículo.

\section{5.3. A mão que fere é a mesma mão que pode curar\protect\footnotemark}
\footnotetext{``The hand that inflicts the wound is alone the hand that can heal it'' (Hegel). Citado por Marshall Berman em \emph{The Politics of Authenticity: Radical Individualism and the Emergence of Modern Society.} New York: Verso, 2014, p. 145.}

Ora, se tudo tanto faz e se tudo lhe é diferente, o homem ridículo não
pode se apiedar pela menininha que clama pela mamãe; o homem ridículo
não pode pensar que, sem a proteção da mamãe, a menininha acabaria
caindo nas garras de pedófilos e cafetões como Svidrigáilov e
Stavróguin.

\begin{quote}
Me irritei em consequência da conclusão de que, se eu já tinha decidido
que nessa mesma noite me mataria, então, por isso, tudo no mundo, agora
mais do que nunca, deveria me ser indiferente. Por que é que eu fui
sentir de repente que nem tudo me era indiferente e que eu tinha pena da
menina? Lembro que tive muita pena dela, quase até o ponto de uma
estranha dor, aliás completamente inverossímil na minha situação.
(\ldots{}) Se eu vou me matar (\ldots{}) daqui a duas horas, então o que
é que me importa a menina? (\ldots{}) Eu me transformo num nada, num
nada absoluto. Se o nada é a verdade, não há lugar para a piedade (\versal{IDEM},
pp. 99-100).
\end{quote}

Ocorre que o homem ridículo, assim como Raskólnikov, sente a náusea do
outro -- a dor conjunta, a compaixão. {[}Nietzsche diria que o niilista,
ainda uma vez, não está à altura de seu niilismo. Se o pensador alemão
assim o fizesse, o homem ridículo poderia perguntar a Nietzsche se a
gaia ciência conseguiria esboçar uma ideia de convívio social para além
da competitividade utilitária da vontade de poder. É~como se o homem
ridículo descobrisse, após o encontro com a menininha, que Nietzsche e
sua gaia ciência pressupõem, unilateralmente, a altivez do jovem e a
(suposta e contingente) independência do burguês, seu egoísmo por
excelência. Em verdade, em verdade, o homem ridículo nos diz: a lógica
do senhor, Nietzsche, é a lógica daquele que tem medo de se tornar
escravo. Ocorre, Nietzsche, que, a despeito da magnanimidade de
Zaratustra -- \emph{ecce homo}, eis o homem! --, o tempo não apenas
envilece os homens como também os envelhece. Se até mesmo Júlio César
foi posto de joelhos, Nietzsche, que dizer do jovem diante de seu vir a
ser senil? Que dizer do retrato de Zaratustra enquanto velho? Apoiado em
sua bengala, o outrora super"-homem logo voltará a usar fraldas como um
bebê de colo. (Se a vontade de poder só pode suplantar o niilismo
negativo do suicídio com um niilismo momentâneo de afirmação jovial da
vida -- Svidrigáilov e Stavróguin afirmam que o homem como que engana a
letalidade do tempo enquanto o desejo se mantém ereto --, a escatologia
do homem ridículo sentencia que, após o êxtase do prazer que não
perdura, o niilista orgíaco deve se matar.){]}

No limite máximo do niilismo, já à beira do penhasco de seu exílio, o
homem ridículo reencontra um bastião de humanidade em sua compaixão pela
menininha desesperada e indefesa. Nesse sentido, Robert Louis"-Jackson
(2002) observa que

\begin{quote}
o verbo russo para ``sentir compaixão'', notemos aqui, é
сострадать
(``sostradat'') -- ``sofrer \emph{conjuntamente}'', como em alemão
existe o substantivo ``Mitleid'' ou o verbo correlato, ``mitfühlen''.
Dostoiévski certa vez escreveu, ``a compaixão -- eis o cristianismo'', o
que significa, literalmente, ``sofrer com os outros -- eis o
cristianismo''. Sofrer, sofrer com os outros, o problema do sofrimento,
a ideia da ``salvação por meio do sofrimento'', preocupou Dostoiévski em
suas obras. O~sofrimento, acreditava o autor, era uma necessidade moral
e espiritual para o indivíduo, introduzindo equilíbrio em uma natureza
que não nascera para o amor\footnote{Assim, Sergei Hackel (1982) lembra
  que o homem do subsolo, irmão mais velho do nosso homem ridículo,
  ``também levanta dúvida sobre a adequação da famosa colocação de
  Descartes, `Penso, logo existo'. Tem sido bem enfatizado que o homem
  do subsolo a substituiria por `Sofro, logo existo'. (\ldots) Assim
  Dostoiévski escreveu em um de seus cadernos de anotações para
  \emph{Crime e castigo}: `O homem não nasceu para a felicidade'. E~ele
  acrescentou: `O homem conquista a felicidade -- e sempre por meio do
  sofrimento. Pois o sofrimento é a base do ser; o sofrimento e a
  liberdade caminham de mãos dadas; o sofrimento é o preço da
  liberdade'" (pp. 16-17).}. Para Dostoiévski, sem o sofrimento, não há
verdadeira felicidade. O~ideal, acreditava o escritor, é testado pelo
sofrimento como o ouro em meio às chamas\footnote{A dialética a
  apresentar o sofrimento que reencontra a compaixão como um momento
  fundamental da cicatrização do espírito para além do niilismo foi
  sintetizada por Theodor Adorno (2013) com a seguinte citação
  hegeliana: ``Somente a lança que causou a ferida pode curá"-la'' (p.
  158).} (pp. 19-20).
\end{quote}

Eis a maestria de Dostoiévski em encontrar pontos de inflexão
existenciais: se o homem ridículo tivesse sido interpelado por um
mendigo (o enésimo maltrapilho\ldots{}), sua comiseração nem de longe
teria sido tensionada em comparação com o choque que nos traz a
fragilidade de uma criança indefesa (uma menininha ainda por cima!) a
clamar pela mamãe. E~não se trata de qualquer amor, mas do amor de mãe,
um dos últimos redutos de verdade em meio ao mundo rasgado pelo niilismo
do cálculo utilitário. Assim, podemos argumentar, ao lado de Zsuzsanna
Bjørn Andersen (2000), que, como um momento primeiro e essencial para a
transvaloração de seu niilismo, o homem ridículo descobre

\begin{quote}
por meio do reconhecimento que o homem deixa de ser um indivíduo
isolado. Em outras palavras, por meio do reconhecimento nós descobrimos,
ao mesmo tempo, a dependência do homem como um animal social e sua
independência como um eu (\emph{self}). Em uma fórmula simples, Hegel
{[}e Dostoiévski{]} exprimem aqui a unidade entre sociabilidade e
individualidade (pp. 55-56).
\end{quote}

O niilismo solipsista e monadológico -- o princípio burguês desprovido
da vontade de poder e levado às últimas consequências -- começa a
encontrar uma antítese para o suicídio inequívoco com o caráter
quintessencial da compaixão, a compaixão que transforma o outro em
mediação para o eu. \emph{Nós}, a bem dizer. Assim, a logicidade que
aproxima Dostoiévski de Hegel (2008) anuncia a dor conjunta como a
possibilidade para a superação (\emph{Aufhebung}) da dor:

\begin{quote}
A miséria externa tem que se tornar a dor do homem em si mesmo: ele
precisa se sentir como o negativo de si mesmo; ele precisa reconhecer
que seu infortúnio é o infortúnio de sua natureza, que ele é, em si
mesmo, a separação e o rompimento (p. 272).
\end{quote}

O homem ridículo poderia dizer, a partir de Hegel, que a miséria externa
tem que se tornar a dor do homem em si e para além de si mesmo: ele
precisa se sentir como o negativo de si mesmo -- e como o positivo da
menininha, a mão que já não a enxota, a mão que lhe estende a cura; ele
precisa reconhecer que seu infortúnio é o infortúnio da \emph{nossa}
natureza, que nós somos, em nós e por nós mesmos, a separação, o
rompimento -- e a possibilidade de reconciliação. É~assim que Hegel
prossegue dizendo que, ``por meio da perda, da realidade meramente
externa, o espírito é impelido para si {[}o homem ridículo bem poderia
acrescentar -- e para além de si{]}. O~lado da realidade é purificado
para a universalidade pela {[}retomada da{]} relação com o Uno. (\ldots) O
infortúnio sabe"-se algo necessário para a mediação da unidade do homem
{[}e da menininha{]} com Deus'' (\versal{IDEM}, p. 274; p. 275). Esse tipo de
compaixão, em Dostoiévski e em Hegel, tem uma forte conotação espiritual
-- e político"-social. Esse será um ponto muito importante que
exploraremos no decorrer da nossa análise, mas, como o escritor russo e
o filósofo alemão bem sabiam, a compaixão para com os humilhados e
ofendidos, se permanecer ilhada em uma ação contingente, tende a não se
enraizar em uma proposta para a transformação moral e social da
realidade. É~assim que Theodor Adorno (2009) entenderia o resgate do
homem ridículo de seu \emph{bunker} niilista a partir da compaixão -- e
da esperança.

\begin{quote}
A consciência não poderia de modo algum se desesperar quanto ao cinza se
ela não cultivasse o conceito de uma cor diferente cujo traço errático
não faltasse ao todo negativo {[}e niilista{]}. Esse traço provém
constantemente do passado {[}e do ímpeto pelo futuro{]}, a esperança
nasce do seu oposto, daquilo que precisou cair ou é condenado {[}e
daquilo que quer voltar a se levantar{]}; uma tal interpretação estaria
completamente de acordo com a última frase do texto de Benjamin sobre as
\emph{Afinidades eletivas} {[}de Goethe{]}\emph{:} ``A esperança só nos
é dada em nome dos desesperançados'' {[}fiz os adendos em diálogo com o
\emph{pathos} transvalorado do homem ridículo{]} (p. 313).
\end{quote}

Mas, em meio ao princípio de reconciliação suscitado pela compaixão,
Dostoiévski faz com que o homem ridículo submeta, dialeticamente, o teor
de verdade de seu altruísmo a um novo teste do niilismo relativista e
corrosivo:

\begin{quote}
Se eu vivesse antes na lua, ou em Marte, e lá cometesse o ato mais
canalha e mais desonesto que se possa imaginar, e lá fosse achincalhado
e desonrado como só se pode imaginar às vezes dormindo, num pesadelo, e
se, vindo parar depois na terra, eu continuasse a ter consciência do que
cometi no outro planeta e, além disso, soubesse que nunca mais, de jeito
nenhum, voltaria para lá, então, olhando a lua da terra -- tudo me
\emph{seria indiferente ou não?} Sentiria vergonha por aquele ato ou
não? (\versal{DOSTOIÉVSKI}, 2003, pp. 100-101)
\end{quote}

O raciocínio é dostoievskianamente ardiloso: será que o homem ridículo
sentiu pena da menininha que clamava pela mãe moribunda por conta do
\emph{hábito} que nos impele a estender a mão aos demais (apenas) quando
alguém se encontra em estado de extrema penúria? (É bem verdade que os
mendigos seriam os primeiros a testemunhar que o teor de verdade de tal
\emph{hábito}, cotidianamente reproduzido em nossa sociedade como algo
contingente e reificado, não se lhes aplica.) Quando tal \emph{hábito} é
infringido, a condenação alheia se converte em vergonha para aquele que
incorre no desvio. Assim, o homem ridículo, tentando resistir às
invectivas da verdade da compaixão para proceder a seu suicídio, se
pergunta (e nos pergunta), em suma, se a moralidade é totalmente
contingente, se ela depende, \emph{in toto}, das peculiaridades dos usos
e costumes de determinada época, ou se, ao contrário, haveria um teor de
verdade que, a despeito de todas as transformações fenomênicas,
permaneceria {[}e seria \emph{elevado} (\emph{aufgehoben}){]} através da
história. Quando começarmos a analisar o sonho escatológico, veremos que
a referência à lua ou ao planeta Marte não é nada fortuita. O~homem
ridículo descobrirá aquilo que passará a vivenciar como verdade em um
planeta outro que é o duplo da Terra. Neste momento, interessa"-nos a
pergunta afiada como um punhal: se o homem for entregue a si mesmo --
ainda que, em termos dialéticos, não haja qualquer pessoa que exista em
si e por si mesma --, isto é, se os juízos alheios se transformarem em
um espectro, se apenas o infrator da norma tiver que lidar com o fardo
de sua infração, tudo lhe será indiferente ou não? Se o arrependimento
-- vale dizer, a purificação -- tiver que partir do próprio eu, do homem
como juiz de si próprio, do homem como consciência de si, ou melhor, do
homem como consciência entre os homens, haverá o ímpeto do castigo em
face do crime ou o homem soterrará a culpa no subsolo de suas memórias?
Já sabemos que Raskólnikov não conseguiu suportar o fardo de tal
pergunta -- e foi justamente isso que fez sua vaidade sofrer, pois o
intelectual niilista se deu conta de que não estava à altura (reificada)
da indiferença moral de um Napoleão.

Com o retorno da possibilidade de um sentido a lhe atormentar, o homem
ridículo se dá conta de que ``essa menina me salvou, porque com as
questões eu adiei o tiro'' (\versal{IBIDEM}). Então, extenuado por conta de seus
conflitos limítrofes, o sono, como há muito tempo não acontecia, acaba
por arrebatá"-lo. O~homem ridículo adormece diante do revólver.

Antes de nos conduzir pela purificação do suicídio mediada pelo sonho
redentor, nosso herói esboça uma teoria sobre os sonhos, teoria que,
além de nos fazer entender por que Sigmund Freud era um leitor contumaz
de Dostoiévski, nos apresenta a (escato)lógica que nos levará,
narrativamente, à superação dos marcos de pensamento e vivência do
niilismo:

\begin{quote}
Os sonhos, como se sabe, são uma coisa extraordinariamente estranha: um
se apresenta com assombrosa nitidez, com minucioso acabamento de
ourivesaria nos pormenores, e em outro, como que sem se dar conta de
nada, você salta, por exemplo, por cima do espaço e do tempo. Os~sonhos,
ao que parece, move"-os não a razão, mas o desejo, não a cabeça, mas o
coração, e no entanto que coisas ardilosas produzia às vezes a minha
razão em sonho! (\versal{IDEM}, p. 101)
\end{quote}

Para além da apreensão freudiana que faz os sonhos jorrarem do matiz
sexual do desejo, a contraposição dostoievskiana entre razão e desejo,
cabeça e coração -- contraposição totalmente quintessencial para a
transfiguração do niilismo de nossa personagem, como logo veremos --
parece nos sugerir que o pensamento e a vida limitados pelo marco da
existência niilista e, consequentemente, finita não têm condições de
apreender, \emph{na carne e no sangue}, o sentido do universo -- a
infinitude. É~nesse preciso sentido que o salto por cima do espaço e do
tempo a que o homem ridículo se refere aproxima, ainda uma vez,
Dostoiévski e Hegel:

\begin{quote}
Costuma"-se dar muita importância aos limites do pensamento, da razão
etc., e se afirma mesmo que esses limites não podem ser transgredidos.
Nessa afirmação reside a ausência da consciência de que no fato mesmo de
algo ser determinado como limite já se transgridem esses limites. Pois
uma determinidade, uma fronteira, não é determinada como limite senão em
oposição ao seu outro em geral, em oposição ao seu ilimitado; o outro de
um limite é justamente o para"-além desse limite mesmo (\versal{HEGEL} apud
\versal{ADORNO}, 2009, p. 317).
\end{quote}

É possível dizer, ainda que com alto grau de tensão após ``O sonho de um
homem ridículo'', que a obra de Dostoiévski respeita, dialeticamente, os
marcos de representação e verossimilhança do realismo. Respeito que,
hegelianamente, pressupõe a transgressão com o salto dialético da
quantidade para a qualidade, isto é, com a narração de situações
escatológicas e limítrofes que, a partir do extravazamento de suas
próprias dinâmicas, requerem a expansão dos marcos de apreensão da
realidade. É~assim que a obra de Dostoiévski leva às últimas
consequências o niilismo para arremessá"-lo contra si mesmo, para
comprimir a finitude até que, no ápice da retração -- eis uma síntese
para o ímpeto de suicídio do homem ridículo --, ela acabe sendo
explodida para superar os marcos que antes a limitavam. É~nesse preciso
sentido que, no capítulo anterior, retomamos uma colocação hegeliana de
que o finito é o infinito determinado: Dostoiévski não faz o homem
ridículo cruzar galáxias distantes por cima do espaço e do tempo fora do
sonho -- se bem que, com a profunda evolução tecnológica, a narração de
tal evento já não precisaria se dar em meio a um sonho para respeitar,
em termos gerais, os marcos do realismo. O~finito como infinito
determinado volta a ter consciência sobre si mesmo na medida em que, do
ápice do torpor, advém o resgate. A~jornada tem (re)início com a tensão
na suma paralisia. Para o homem ridículo, a vida se revigorou com a
vertigem do penhasco. Em uma belíssima passagem, Hegel fala sobre o mar
e a infinitude da mesma maneira que o homem ridículo discorre sobre os
sonhos:

\begin{quote}
O mar nos fornece a representação do indeteminado, do ilimitado e do
infinito; e quando o homem se sente nesse infinito, isso o estimula a
transcender o limitado. (\ldots) A~terra, na região do vale, fixa o homem
ao solo, tornando"-o infinitamente dependente; o mar o conduz para além
desses limitados círculos de pensamento e ação (2008, pp. 80-81).
\end{quote}

É assim que, ao perguntar ``como é que não me espanto com o fato de que,
embora esteja morto, mesmo assim ele {[}o irmão de nosso herói, falecido
havia 5 anos{]} está aqui ao meu lado e se atarefa junto comigo? Por que
o juízo admite isso?'' (\versal{DOSTOIÉVSKI}, 2003, pp. 101-102), o homem
ridículo -- que, lembremos, narra sua estória após ter vivenciado o
sonho -- já não consegue aceitar o existente tal como a modernidade
niilista o apresenta, já que a vida após a morte e o infinito, como
veremos, se insinuam como a base transcendental que atravessa a história
e alça (\emph{aufheben}) o teor de verdade da moralidade e do sentido
para além das contingências relativistas de espaço e tempo. Assim, à
beira do penhasco que, para fazer nosso herói voar à velocidade da luz,
acabou por incitá"-lo à queda, chegamos, agora, ao sonho de um homem
ridículo, sonho que, para implodir o realismo, anuncia a verdade
enquanto a razão dorme (e se expande):

\begin{quote}
Mas por acaso não dá no mesmo, seja isso um sonho ou não, já que esse
sonho me anunciou a Verdade? Pois, se você uma vez conhece a verdade e a
enxerga, então sabe que ela é a verdade e que não há outra e nem pode
haver, esteja você dormindo ou vivendo. Ora, que seja um sonho, que
seja, mas essa vida que vocês tanto exaltam, eu queria extingui"-la com o
suicídio, e o meu sonho, o meu sonho -- ah, ele me anunciou uma vida
nova, grandiosa, regenerada e forte! (\versal{IBIDEM})
\end{quote}

\section{5.4. Só há um problema existencial verdadeiramente sério: a
eternidade }

\begin{quote}
De repente sonhei que apanho o revólver e, sentado, aponto"-o direto para
o coração -- para o coração, e não para a cabeça; e eu que antes tinha
determinado que meteria sem falta um tiro na cabeça, mais precisamente
na têmpora direita. Apontando"-o para o peito, esperei um ou dois
segundos, e a minha vela, a mesa e a parede diante de mim começaram de
repente a se mexer e a balançar. Puxei depressa o gatilho (\versal{IDEM}, p.
103).
\end{quote}

Já vimos que a oposição entre razão e desejo, cabeça e coração é
fundamental para a expansão da própria vida -- ou, para usarmos uma
expressão sumamente dostoievskiana (expressão que logo reaparecerá em
nossa análise), para a expansão da \emph{vida viva.} Em relação à
decisão do homem ridículo de atirar contra o coração, e não contra a
cabeça, Gerald J. Sabo (2009) observa que

\begin{quote}
em uma carta de 31 de outubro de 1838 para seu irmão, Dostoiévski
exprimiu o papel do coração na cognição e na vida humanas: ``Para
conhecer a natureza, a alma, Deus, o amor\ldots Tudo isso é conhecido pelo
coração, não pela mente. (\ldots) Se o objetivo do conhecimento for o amor
e a natureza, então se abre uma clareira para o \emph{coração}''. Em ``O
sonho'', o homem ridículo observa que, ``aparentemente, não a razão, mas
o \emph{desejo}, não a cabeça, mas o \emph{coração} dirige o sonho''
(pp. 51-52).
\end{quote}

Diferentemente de Gerald J. Sabo, não interpretamos e não
interpretaremos tal oposição como uma ode dostoievskiana ao
irracionalismo e/ou à fé cristã que se cinde da apreensão racional do
(além-)mundo. Como vimos argumentando ao longo deste livro, a crítica
dostoievskiana à razão é \emph{dialeticamente determinada} e se vincula
ao momento de superação do ego, ou, por outra, da razão instrumental.
Assim, a despeito do que pregam os estudiosos que filiam Dostoiévski,
inequivocamente, ao partido do mistério e da fé, a relação do autor com
o devir do pensamento como momento de elevação humana é quintessencial
para entendermos a cicatrização do espírito. Ficará claro, ao fim da
análise, que o embotamento do conhecimento sobre o e no mundo não faz
parte da redenção do homem ridículo -- não voltaremos a acender velas de
cera diante de altares para resgatarmos relações com a espiritualidade.
O tiro contra o coração, então, também dispara contra a razão, pois,
como vimos anteriormente, para Dostoiévski, a vida compreende a razão e
todas as estruturas que lhe são vivificantes. Razão e desejo, cabeça e
coração, assim como as personagens"-ideias dostoievskianas, não são
mônadas, mas polos relacionais. Se o coração é o cerne da \emph{vida
viva}, a razão, para muito além do cálculo utilitário, é uma das fontes
que o municiam.

Pois muito bem: no início de seu sonho, o homem ridículo dispara contra
o próprio coração e, portanto, consuma o suicídio. Após assistir ao
próprio enterro, nosso herói diz que jazia em sua sepultura úmida e fria
e ``nada esperava, aceitando sem discussão que um morto nada tem a
esperar'' (\versal{DOSTOIÉVSKI}, 2003, p. 104). Súbito, através da campa do
caixão, se infiltra uma goteira que passa a lhe torturar o olho
esquerdo, com toda a indiferença do universo, a cada 60 segundos. Além
de descobrir que continua vivo após a morte, o homem ridículo sente o
gotejamento infinito como um castigo pelo suicídio -- ao que, em sumo
desespero, ele suplica:

\begin{quote}
-- Seja você quem for, mas se você é, e se existe alguma coisa mais
racional do que o que está acontecendo agora, então permita a ela que
seja aqui também. Se você se vinga de mim pelo meu suicídio insensato
com a hediondez e o absurdo da continuação da existência, saiba que
nunca nenhum tormento que eu venha a sofrer vai se comparar ao desprezo
que eu vou sentir calado, nem que seja durante milhões de anos de
tortura!\ldots{} (\versal{IDEM}, pp. 104-105)
\end{quote}

Com a goteira eterna contra o olho do suicida encalacrado em seu caixão,
Dostoiévski nos oferece mais uma imagem aterradora do niilismo que,
antes de se voltar contra si mesmo e para além de si mesmo, precisa
descer aos mais tenebrosos círculos de Dante. O~inferno da tortura
chinesa a que Dostoiévski submete o homem ridículo só parece superado
pela eternidade concebida como mal absoluto pelo lascivo Svidrigáilov,
personagem de \emph{Crime e castigo: }

\begin{quote}
Para mim a eternidade é uma idéia impossível de compreender, algo de
enorme, imenso. Mas por que há de ser precisamente enorme? E, de
repente, em vez disso, imagine (\ldots{}) que existe aí um quarto, no
gênero duma sala de banho em pleno campo, negra de fumo e com aranhas
por todos os lados, e que a isso se resumisse a eternidade. Olhe, eu
imagino"-a muitas vezes assim (2002, p. 268).
\end{quote}

Se for possível levar tal horror dostoievskiano ainda além, fará sentido
a colocação de Theodor Adorno (2009) a dialogar com Samuel Beckett,
quando o autor irlandês reage à

\begin{quote}
situação do campo de concentração, uma situação que ele {[}Beckett{]}
não nomeia, como se ela estivesse submetida à interdição das imagens. O
que é se mostra, segundo ele, como um campo de concentração. Em um certo
momento, ele fala de uma pena de morte perpétua. A~única esperança
emerge do fato de não haver mais nada (p. 315).
\end{quote}

Ocorre que, após o seu clamor em desespero -- e como mais um evento do
\emph{acaso objetivo} que paira como um guia a estruturar a sucessão dos
fenômenos parciais para além de si mesmos, isto é, com vistas à
constituição dinâmica da totalidade --, o homem ridículo passou a
acreditar, paradoxal, ``imensa e inabalavelmente, que agora sem falta
tudo mudaria'' (\versal{DOSTOIÉVSKI}, 2003, p. 105). De repente, o caixão se
rompe e uma criatura desconhecida toma nosso herói nos braços. Tem
início a odisseia no tempo e no espaço.

\begin{quote}
Voávamos no espaço já longe da terra. (\ldots) Não lembro quanto tempo
voamos, nem posso imaginar: tudo acontecia como sempre nos sonhos,
quando você salta por cima do espaço e do tempo e por cima das leis da
existência e da razão e só pára nos pontos que fazem o coração delirar.
Lembro que de repente avistei na escuridão uma estrelinha. (\ldots) {[}Ao
que a criatura que o levava, após a interpelação do homem ridículo,
respondeu{]}: ``(\ldots) Essa é a mesma estrela que você viu entre as
nuvens quando voltava para casa'' (\versal{IDEM}, pp. 105-106).
\end{quote}

Como já havíamos dito no decorrer deste capítulo, o sonho propriamente
dito começa a duplicar e a tornar repletos de sentido os momentos
narrativos anteriores que, aparentemente, estavam submetidos à
indiferença e ao caos que apenas apontavam para o suicídio. Já não
parece aleatório o fato de que nosso herói tenha outrora decidido se
matar justamente ao olhar para a estrelinha que, agora, demarca o início
da jornada que propiciará as experiências para o homem ridículo
vivenciar sua verdade. Dostoiévski faz queda e elevação, suicídio e
redenção se imbricarem de forma inconsútil. Tese e antítese se enlaçam
em busca de uma possível síntese.

Mas quem seria a criatura que conduz o homem ridículo em sua odisseia
espaço"-temporal?

\begin{quote}
Eu sabia que ela possuía como que um rosto humano. Coisa estranha, não
gostava dessa criatura, sentia mesmo uma aversão profunda. Esperava o
não"-ser absoluto e, por isso, dei um tiro no coração. E~eis que estou
nos braços de uma criatura, não humana, é claro, mas que \emph{é},
existe: ``Ah, então há também uma vida além"-túmulo!'' (\ldots{}) ``E se
é preciso \emph{ser} novamente -- pensei eu -- e viver mais uma vez pela
vontade inelutável de seja lá quem for, então não quero que me dominem e
que me humilhem!'' -- ``Você sabe que eu tenho medo de você e, por isso,
me despreza'' -- disse eu de repente ao meu companheiro de viagem, não
conseguindo conter uma pergunta humilhante, que trazia uma confissão em
si, e sentindo, como uma picada de alfinete, a humilhação no coração.
Ele não respondeu à minha pergunta, mas senti de repente que não me
desprezam e não riem de mim, que nem mesmo se compadecem de mim
(\ldots{}) (\versal{IBIDEM}).
\end{quote}

Se estivéssemos imbuídos da tradição católica, poderíamos chamar a
criatura de anjo da guarda. Quiçá um espírito afim -- um espírito"-guia.
É fundamental frisar que, em meio ao sonho, irrompe a vida após a morte
e, consequentemente, o homem ridículo passa a vivenciar o além"-mundo.
Não é nada gratuito que o recém"-suicida projete, em sua relação com o
guia interestelar, o mesmo duelo sadomasoquista que frequentemente
caracteriza as personagens dostoievskianas. Nosso herói não aceita ser
conduzido -- a vaidade mórbida do suicida fora ferida, uma vez que o
não"-ser não existe. Se ele estava errado, é provável -- segundo a lógica
do duelo -- que a criatura queira humilhá"-lo pelo medo profundo que o
homem ridículo sente do que lhe é desconhecido, isto é, de tudo aquilo
que ele não consegue subjugar. Mas, assim como no início de nossa
narrativa, uma lógica outra se insinua para além da tensão e do
conflito. O~homem ridículo diz que não há risos e comiserações contra
ele -- nosso herói tenta reencontrar a consciência de quando ainda não
sabia que as relações entre os seres podem ocorrer em comunhão e
fraternidade.

A odisseia intergaláctica continua de forma vertiginosa. ``Fazia tempo
que já não via as constelações familiares ao olho. Sabia que há nos
espaços celestes certas estrelas cujos raios só alcançam a terra depois
de milhares e milhões de anos. Talvez já tivéssemos voado por esses
espaços''\footnote{A esse respeito, Allan Kardec (2009) afirma que ``a
  matéria existe em estados que vos são desconhecidos. Pode ser, por
  exemplo, tão etérea e sutil que nenhuma impressão vos cause aos
  sentidos; entretanto, é sempre matéria, embora para vós não o seja.
  (\ldots{}) Na matéria, (\ldots{}) {[}há{]} duas propriedades
  essenciais: a força e o movimento (\ldots{}). {[}Os espíritos gastam
  algum tempo para percorrer o espaço?{]} Sim, porém, rápido como
  pensamento'' (p. 40; p. 43; p. 57).} (\versal{IDEM}, pp. 106-107). Súbito, o
homem ridículo depara com um duplo do Sol -- ele sabia que não se
tratava do \emph{nosso} Sol, que já havia ficado há milhões de anos"-luz.

\begin{quote}
Um sentimento doce, invocatório, começou em êxtase a ressoar na minha
alma: a força matriz do universo, desse mesmo universo que me deu à luz,
pulsou no meu coração e o ressuscitou, e eu pude sentir a vida, a vida
de antes, pela primeira vez desde a minha sepultura.

\noindent -- Mas se esse -- é o sol, se esse sol é exatamente igual ao nosso --
gritei eu --, então onde está a terra? -- E~o meu companheiro me apontou
uma estrelinha que reluzia na escuridão com um brilho de esmeralda.
Estávamos voando diretamente para ela.

\noindent -- Serão possíveis tais repetições no universo, será possível que seja
assim a lei da natureza?\ldots{} (\versal{IBIDEM})
\end{quote}

O homem ridículo se refere à força matriz do universo, Deus, a força que
lhe deu à luz -- o fiat lux do \emph{Gênesis} --, e então seu coração
\emph{ressuscitou.} Nesse momento, o homem ridículo não apenas nos fala
que, segundo a experiência de seu sonho, a vida é eterna. Ele nos diz
que a vida, ``a vida de antes'', é \emph{retomada. }

Diante do duplo do Sol, nosso herói pede à criatura que lhe mostre a
Terra. Eis, então, que o sonho se encaminha para o logradouro próprio da
conversão do homem ridículo -- o duplo da Terra. E~me parece fundamental
para a compreensão do que seja a \emph{retomada} da vida após o suicídio
a pergunta sobre se tais \emph{repetições} existem no universo, sobre se
esta seria a \emph{lei da natureza}.

Neste momento, precisamos interromper a análise propriamente dita para,
mediados por Rudolf Neuhäuser, estabelecermos alguns pontos de diálogo
entre Dostoiévski e Hippolyte Léon Denizard Rivail (1804-1869), mais
conhecido como Allan Kardec, o codificador da doutrina espírita. Veremos
que a noção de cicatrização do espírito que Dostoiévski e Hegel
partilham e desenvolvem se fará ainda mais lógica com a mediação do
espiritismo. Tentaremos, nesse sentido, responder à pergunta postulada
pelo homem ridículo a respeito da \emph{retomada da vida} e das
\emph{repetições no universo.} Quando retornarmos à análise de nossa
narrativa, veremos como as contribuições de Kardec -- já espectrais
desde o capítulo anterior deste livro -- passarão a nos acompanhar.

\subsection{5.4.1. Um espectro ronda Dostoiévski, o espectro do espiritismo}

Rudolf Neuhäuser (1993) afirma que

\begin{quote}
em três edições de seu \emph{Diário de um escritor} (janeiro, março e
abril de 1876), Dostoiévski escreveu sobre o espiritismo. (\ldots)
Começando nos anos 1860, Dostoiévski parece ter se sentido
crescentemente atraído pelo espiritismo. Ele ficou muito impressionando
com as experiências mediúnicas, mas, ainda assim, se opôs firmemente ao
espiritismo por razões religiosas, até mesmo o colocando em um nível
igual ao do niilismo. ``Não, eu prefiro o puro ateísmo ao espiritismo!''
E ele deu suas razões: ``Mas eu odeio apenas a hipótese nauseante dos
espíritos e de manter relações com eles''. Dostoiévski entendia por
``espíritos'' sobretudo ``demônios'' (``tcherti''), conforme ele disse,
de maneira chistosa, na edição de janeiro de seu \emph{Diário de um
escritor.} A~ideia de que pessoas mortas pudessem reaparecer na Terra
como espíritos era, para o autor, uma ``kochtchunstvo'' (blasfêmia), uma
``hipótese nauseante''. Outro aspecto que parece ter perseguido o
escritor era a possibilidade de que o espiritismo se transformasse em
uma fé sectária que levasse a posteriores ``obosoblenie''
(``isolamento'') and ``raz'edinenie'' (``separação'') na sociedade
russa. Por esse ponto de vista, ele tinha medo dos aspectos místicos dos
ensinamentos espirituais (p. 186; p. 187).
\end{quote}

As colocações iniciais de Neuhäuser sobre as relações de Dostoiévski com
o espiritismo mostram uma aproximação bastante tensa e contraditória.
Primeiramente, Neuhäuser menciona que os fenômenos mediúnicos
impressionavam sobremaneira o escritor. Isso não nos deve surpreender,
uma vez que o autor, que era epiléptico, considerava que os surtos lhe
expandiam a consciência e propiciavam uma profunda ressignificação da
realidade -- já discutimos, neste capítulo, a noção de que, para
Dostoiévski, os marcos do \emph{realismo}, na verdade, embotavam a
verdadeira e essencial \emph{realidade}, que incluía o além"-mundo que a
modernidade ateia passara a considerar \emph{fantástico.} Nessa medida,
Neuhäuser prossegue dizendo que

\begin{quote}
Dostoiévski tinha um interesse pronunciado pela natureza e pelos efeitos
de estados psicológicos anormais de consciência. A~``hiperconsciência''
poderia levar o homem ao crime {[}como, por exemplo, no caso de
Raskólnikov, em \emph{Crime e castigo}{]}, mas ela também poderia trazer
à tona um renascimento espiritual levando o homem a uma compreensão mais
profunda sobre o mundo e sua própria existência {[}como, por exemplo, no
caso do Príncipe Míchkin e do homem ridículo{]}. De qualquer forma, a
hiperconsciência expunha o indivíduo a inflluências que o levavam para
além dos limites da existência física. Dostoiévski tinha considerável
interesse pela relação tênue entre a realidade cotidiana, por um lado, e
a esfera intangível porém ``real'' do que está além do alcance dos
sentidos e da nossa mente. Ele até mesmo clamava que, sob certas
circunstâncias, tais como em estados de consciência elevada, sejam eles
causados por \emph{doenças ou sonhos} {[}grifo meu{]}, o homem poderia
estabelecer contato com um mundo intangível que existiria paralelamente
ao nosso mundo visível e tangível (1993, p. 188).
\end{quote}

Mencionemos, então, o seguinte fragmento em que Svidrigáilov, personagem
de \emph{Crime e castigo}, discorre a respeito das relações com o
além"-mundo:

\begin{quote}
As aparições são, por assim dizer, pedaços ou fragmentos de outros
mundos, o seu princípio. É~claro que o homem são não tem motivo para
vê"-las, porque o homem são é o homem mais terreno e deve viver uma vida
terrestre, atendendo à harmonia e à ordem. Mas quando adoece, ou quando
a ordem terrena se altera no organismo, começa imediatamente a
mostrar"-se a possibilidade de outro mundo, de maneira que, \emph{quando
morre completamente, o homem vai direto para esse mundo.} Já há muito
tempo que medito nisso. Se o senhor acredita na outra vida, pode
acreditar também nesse raciocínio {[}grifo meu{]} (2002, p. 268).
\end{quote}

\emph{Crime e castigo} foi escrito em 1866, 11 anos antes de ``O sonho
de um homem ridículo'', mas, como vemos pelas colocações de
Svidrigáilov, a temática essencial já rondava o imaginário de
Dostoiévski, e o tema da expansão da consciência rente a estados
patológicos volta a aparecer em \emph{O idiota} (1869). Se Svidrigáilov
é um nobre lascivo que extorque a irmã de Raskólnikov para a consecução
de seus prazeres e chega a flertar com a pedofilia, a próxima personagem
a discorrer sobre a expansão da consciência a tangenciar o além"-mundo é
o espiritualizado Príncipe Míchkin, que, como o próprio Dostoiévski,
sofre ataques epilépticos. Eis algumas considerações de Míchkin sobre os
instantes que antecedem os ataques:

\begin{quote}
Qual é o problema se essa tensão é anormal, se o próprio resultado, se o
minuto da sensação lembrada e examinada já em estado sadio vem a ser o
cúmulo da harmonia, da beleza, de uma sensação inaudita e até então
inesperada de plenitude, de medida, de conciliação e de fusão extasiada
e suplicante com a mais suprema síntese da vida? (\ldots{}) Esses
instantes eram, justamente, só uma intensificação extraordinária da
autoconsciência (\ldots), da autoconsciência e ao mesmo tempo da
autossensação do imediato no mais alto grau. (\ldots{}) Nesse momento me
fica de certo modo compreensível a expressão insólita: \emph{não haverá
demora}\footnote{O Príncipe Míchkin faz referência à seguinte passagem
  do \emph{Apocalipse} (10, 5-7): ``Então o anjo, que eu vira de pé
  sobre o mar e a terra, levantou a mão direita para o céu e jurou por
  aquele que vive pelos séculos dos séculos, que criou o céu e tudo o
  que há nele, a terra e tudo o que ela contém, o mar e tudo o que
  encerra, \emph{que não haveria mais tempo}; mas nos dias em que soasse
  a trombeta do sétimo anjo, se cumpriria o mistério de Deus, de acordo
  com a boa nova que confiou a seus servos, os profetas''.} (2002, p.
261; p. 262).
\end{quote}

Ainda que tal temática tenha perpassado a obra de Dostoiévski, Neuhäuser
nos apresentou acima colocações do autor que associam o espiritismo ao
niilismo. E~o motivo para tanto -- algo tão contraditório quanto o
próprio escritor -- é que, apesar de suas personagens pressentirem o
contato com o e discorrerem sobre o além"-mundo, Dostoiévski afirma que
``eu odeio apenas a hipótese nauseante dos espíritos e de manter
relações com eles (\versal{DOSTOIÉVSKI} apud \versal{NEUHÄUSER}, 1993, p. 187), pelo fato
de, como mencionamos mais acima, o autor entender que os espíritos eram
demônios. Tal consideração nos apresenta um Dostoiévski ao mesmo tempo
para além do e enraizado em seu tempo, na medida em que a aproximação
vanguardista com o espiritismo convive com os preconceitos tanto do
catolicismo quanto do cristianismo ortodoxo em relação ao além"-mundo --
lembremos que, para tais tradições religiosas, tudo o que diz respeito
ao além"-mundo está na ordem do \emph{mistério} {[}mistério a ser mediado
(e obscurecido) pelo monopólio litúrgico do clero{]}. A~respeito da
associação entre espíritos e demônios, a passagem intitulada ``Expulsão
dos demônios'', presente nos evangelhos de Mateus (8, 28-34), Marcos (5,
1-20) e Lucas (8, 26-39), nos apresenta uma boa indicação sobre o temor
e o tremor de Dostoiévski:

\begin{quote}
No outro lado do lago, na terra dos gadarenos, dois possessos de
demônios saíram de um cemitério e vieram"-lhe {[}a Jesus{]} ao encontro.
Eram tão furiosos que pessoa alguma ousava passar por ali. Eis que se
puseram a gritar: ``Que tens a ver conosco, Filho de Deus? Vieste aqui
para nos atormentar antes do tempo?'' Havia, não longe dali, uma grande
manada de porcos que pastava. Os~demônios imploraram a Jesus: ``Se nos
expulsas, envia"-nos para aquela manada de porcos''. -- ``Ide'',
disse"-lhes. Eles saíram e entraram nos porcos. Neste instante toda a
manada se precipitou pelo declive escarpado para o lago e morreu nas
águas. Os~guardas fugiram e foram contar na cidade o que se tinha
passado e o sucedido com os endemoninhados. Então a população saiu ao
encontro de Jesus. Quando o viu, suplicou"-lhe que deixasse aquela região
(\versal{MATEUS}, 8, 28-34).
\end{quote}

O medo de Dostoiévski, então, parecia estar associado à tradicional
interpretação litúrgica da expressão \emph{espírito de porco.} Mas, para
além de temores obscurantistas e apesar de o homem ridículo, como logo
veremos, voltar à Terra após a morte, a ideia de os mortos poderem
reaparecer na terra como espíritos, para Dostoiévski, era uma
```kochtchunstvo' (blasfêmia), uma `hipótese nauseante''', a qual
poderia estar associada à noção de que o espiritismo se tornasse uma fé
sectária ``que levasse a posteriores ``obosoblenie'' (isolamento) e
``raz'edinenie'' (separação) na sociedade russa. Por esse ponto de
vista, ele tinha medo dos aspectos místicos dos ensinamentos
espirituais'' (\versal{DOSTOIÉVSKI} apude \versal{NEUHÄUSER}, 1993, p. 187). O~receio de
Dostoiévski de que o espiritismo poderia trazer a semente da discórdia à
sociedade russa assentada sobre o obscurantismo tsarista e ortodoxo não
nos é novidade: no fim do terceiro capítulo desta tese, analisamos as
contradições que envolviam, de um lado, o Dostoiévski eslavófilo,
reacionário e arcaizante, e, de outro, o autor afinado com as
transformações racionais do espírito do tempo, transformações que
propugnavam por um diálogo dialética e qualitativamente distinto entre
razão e espiritualidade. Neuhäuser nos apresenta uma contradição
fundamental em Dostoiévski e em sua obra -- contradição que, conforme
vimos discutindo, também se comunica à fortuna crítica do escritor. O
medo de que a Rússia viesse a se cindir parecia uma blasfêmia ao
nacionalista ortodoxo que via o imperialismo do tsar como um antídoto
para a corrosão individualista do Ocidente. Mas, como já disséramos no
capítulo anterior, não devemos nos deter diante das contradições
dostoievskianas que, em grande medida, refrearam o próprio Dostoiévski.
Lancemos Dostoiévski contra si mesmo para que a obra do autor se abra
para contradições históricas e espirituais de que seus textos estão
repletos.

Assim, Neuhäuser (1993) prossegue dizendo que Dostoiévski

\begin{quote}
tinha razão em ter medo, porque ele mesmo admitiu ter sido afetado pelo
espiritismo. Uma sessão espírita em fevereiro de 1876, disse o escritor,
``exerceu uma poderosa impressão sobre mim'', e ele explicou que a
``fé'' dos espíritas que encontrou por lá tinha sido a fonte para a sua
primeira oposição ao espiritismo -- apesar do óbvio apelo emocional para
ele. Dostoiévski se forçou -- com base em um cálculo racional -- a
rejeitar o espiritismo: ``Eu me recusei completamente a acreditar -- de
modo que nenhum tipo de prova me tocará novamente''. Certamente, este é
um tipo de argumento irracional que cabe a uma personagem de Dostoiévski
como o homem do subsolo. Obviamente, Dostoiévski compartilhava as
descobertas da comissão do Prof. Mendeleiev, que investigara as
reivindicações mediúnicas, entre setembro de 1875 e maio de 1876, e
concluíra: ``O ensinamento espírita é uma superstição'' (p. 187).
\end{quote}

Quando Neuhäuser faz menção ao argumento irracional de Dostoiévski
contra o espiritismo que bem se parece com os argumentos irascíveis de
suas personagens -- notadamente, o homem do subsolo --, podemos perceber
quanta tensão \emph{subterrânea} há sob a colocação de que o espiritismo
é uma superstição, uma vez que, conforme vimos demonstrando e
continuaremos a demonstrar, há forte vinculação entre a
\emph{logicidade} espírita e a cicatrização do espírito narrada por
Dostoiévski.

Estabelecidos os contatos mais imediatos de Dostoiévski com o
espiritismo, reiteremos a questão que o homem ridículo fez à criatura
que o conduzia pela viagem intergaláctica para que possamos analisá"-la à
luz da doutrina de Kardec.

\begin{quote}
-- Mas se esse -- é o sol, se esse sol é exatamente igual ao nosso --
gritei eu --, então onde está a terra? (\ldots{}) Serão possíveis tais
repetições no universo, será possível que seja assim a lei da
natureza?\ldots{} (\versal{DOSTOIÉVSKI}, 2003, p. 107)
\end{quote}

Rudolf Neuhäuser (1993) menciona o fato de que a reiteração dos planetas
em conexão com a pluralidade de existências era um interesse fundamental
de Dostoiévski:

\begin{quote}
O interesse de Dostoiévski pelos aspectos físicos e astronômicos da
matéria é testemunhado pelo fato de que sua biblioteca continha dois
livros do astrônomo francês Nicolas Camille Flammarion (1842-1925)
traduzidos para o russo. Flammarion também estudara a questão da
pluralidade dos mundos e tinha um forte interesse por fenômenos
parapsicológicos!\footnote{Eis a nota de rodapé que Rudolf Neuhäuser
  (1993) apõe para falar sobre o astrônomo francês: ``Flammarion,
  fundador da Sociedade Francesa de Astronomia, tinha um sério interesse
  por fenômenos parapsicológicos. Ele organizou sessões espíritas e
  compilou relatos de eventos paranormais, alguns dos quais publicou em
  forma de livro com o título de \emph{O desconhecido e os problemas
  psíquicos} (Paris: 1900). Outros livros de sua autoria são \emph{A
  pluralidade dos mundos habitados} (1862) e \emph{Os mundos imaginários
  e os mundos reais} (1862)'' (p. 185).} (pp. 184-185)
\end{quote}

Ademais, Neuhäuser nos informa que Dostoiévski tinha em sua biblioteca o
livro \emph{Sobre os céus, o mundo espiritual e o inferno}, do
espiritualista sueco Emanuel Swedenborg (1688-1772)\footnote{Em 1859, na
  \emph{Revista espírita} (Sorocaba, \versal{SP}: Edicel, 2014), Allan Kardec fez
  os seguintes comentários sobre Swedenborg: ``Como todos os homens que
  professam ideias contrárias à maioria {[}referência à heterodoxia de
  Swedenborg em relação às interpretações então correntes de catolicismo
  e protestantismo{]}, ideias que ferem certos preconceitos, ele teve e
  ainda tem os seus contraditores. Se estes se tivessem limitado a
  refutá"-lo, estariam no seu direito. Mas o facciosismo nada respeita, e
  as mais nobres qualidades não são reconhecidas por eles, os
  contraditores. Swedenborg não poderia ser uma exceção'' (p. 332). Ao
  comentar um aspecto da doutrina de Swedenborg, Kardec observa: ``Um
  ponto fundamental repousa naquilo que ele chama de \emph{as
  correspondências}. Na sua opinião, estando os mundos, espiritual e
  natural, ligados entre si, como o interior ao exterior, resulta que as
  coisas espirituais e as coisas naturais constituem uma unidade, por
  influxo, e há, entre elas, uma correspondência. Eis o princípio; mas o
  que deve ser entendido por essa correspondência e esse influxo: eis o
  que é difícil apreender'' (p. 334). Veremos, na sequência de nossa
  análise, como o homem ridículo, quando depara com o paraíso no planeta
  que mimetiza a Terra, vivenciará a unidade entre espírito e natureza
  como o encontro orgânico da bondade. Por ora, é importante frisar que,
  nos ``Prolegômenos'' com que Kardec inicia \emph{O livro dos
  espíritos}, o nome de Swedenborg aparece ao lado de Sócrates e Platão,
  que são tidos como precursores do cristianismo e do espiritismo.}, no
qual há o seguinte o fragmento:

\begin{quote}
Todos os planetas visíveis aos olhos no nosso sistema solar são como a
Terra, e, além deles, o universo está repleto de incontáveis planetas,
os quais são povoados por habitantes precisamente da mesma maneira
(\ldots). O~homem poderia convencer a si mesmo sobre a pluralidade dos
planetas no universo pelo fato de os céus estrelados não terem limites e
estarem repletos de inumeráveis estrelas, cada uma das quais com seu
lugar e sistema -- não se pode deixar de acreditar que, onde quer que
haja um planeta, haverá também seres humanos (1993, p. 185).
\end{quote}

Vemos, assim, que Neuhäuser nos traz as possíveis gêneses ideológicas
que foram consubstanciando ``O sonho de um homem ridículo'' pelas
leituras de Dostoiévski. O~estudioso de Dostoiévski chega a afirmar que,
``no volume 1 de sua \emph{Opera Philosophica et Mineralia}, Swedenborg
argumentou que qualquer mundo recém"-surgido no universo seria similar ao
nosso mundo em sua juventude!'' (1993, p. 185). Como veremos na
sequência de nossa análise, a pluralidade habitável e habitada dos
planetas e a noção de que tais mundos, em sua juventude, se assemelham à
infância da Terra são vivências efetivas do homem ridículo quando ele
adentra o paraíso. Neste momento, no entanto, já é possível estruturar
uma resposta, com base no espiritismo, para a pergunta do homem ridículo
sobre se as duplicações e repetições planetárias no universo
constituiriam a lei da natureza. Assim, Neuhäuser afirma que há no homem
ridículo

\begin{quote}
uma paráfrase das visões apresentados por Allan Kardec, em seus livros
\emph{O que é o espiritismo?} (1857) e \emph{O livro dos espíritos}
(1859). Rivail é o fundador do primeiro periódico espírita francês, a
\emph{Revista espírita}, e autor de obras como o \emph{Livro dos
médiuns} e \emph{O céu e o inferno.} Ele afirmava que havia diversas
``estrelas'' no universo colonizadas por seres humanos em diferentes
estágios de desenvolvimento espiritual, sendo que a Terra ocupava uma
posição inferior como um tipo de purgatório que, nas palavras de Kardec,
não estava longe do inferno. Ele a considerava um local de purificação
para os espíritos pecadores. Por detrás dessa visão havia a
pressuposição de que planetas distantes servem como locais de vida para
os espíritos dos mortos (1993, pp. 181-182).
\end{quote}

Quando o homem ridículo fala em lei da natureza, vemos que as leituras
que Dostoiévski vinha fazendo -- e, sobretudo, sua relação com o
espiritismo -- o influenciaram para refletir sobre a espiritualidade
para muito além da fé no mistério sobre a vida após a morte. A
pluralidade dos planetas (e das existências) já fora ecoada por Cristo,
quando Ele disse que

\begin{quote}
não se turbe o vosso coração. Crede em Deus, crede também em mim.
\emph{Na casa de meu Pai há muitas moradas} {[}grifo meu{]}. Não fora
assim, e eu vos teria dito; pois vou preparar"-vos um lugar. Depois de ir
e vos preparar um lugar, voltarei e tomar"-vos"-ei comigo, para que, onde
eu estou, também vós estejais. E~vós conheceis o caminho para ir aonde
vou (\versal{JOÃO}, 14, 1-4).
\end{quote}

Em \emph{O evangelho segundo o espiritismo}, Kardec (2004) fala que ``a
casa do Pai é o universo; as diversas moradas são os mundos que circulam
no espaço infinito e oferecem aos espíritos encarnados estâncias
adequadas ao seu adiantamento'' (p. 44), uma vez que, para o
espiritismo, a evolução espiritual é a lei que preside o universo.
Observemos que Jesus afirma que aqueles que seguem seus preceitos
estarão onde Cristo está -- e veremos ao fim de ``O sonho de um homem
ridículo'' que a máxima ``amarás o teu próximo como a ti mesmo''
(\versal{MARCOS}, 12, 31), proferida ao longo do Sermão da Montanha, encerra (e
liberta) o teor de verdade descoberto por nosso herói. Ao invés, então,
de uma oposição hierárquica quintessencial entre Deus e os demais seres,
Kardec diria que Cristo fala sobre a possibilidade de evolução para
todos e cada um de nós. Daí o \emph{instruí"-vos} além do amai"-vos uns
aos outros -- ``e vós conheceis o caminho para ir aonde vou'' (\versal{JOÃO}, 14,
4). Assim, para Kardec (2004), decorre da lei da evolução espiritual
que,

\begin{quote}
conforme {[}o espírito{]} esteja mais ou menos purificado e desprendido
dos laços materiais, \emph{variarão ao infinito o centro em que se
encontra, o aspecto das coisas, as sensações que experimenta, as
percepções} {[}grifo meu para ressaltar a contiguidade entre Kardec e a
lógica ridículo"-dostoievskiana da expansão da consciência{]}. Enquanto
uns não se podem afastar da esfera em que viveram, \emph{outros
elevam"-se e percorrem o espaço e os mundos} {[}grifo meu para ressaltar
a frase como síntese para a transubstanciação do homem ridículo{]}.
(\ldots) Também aí há diferentes moradas, embora não localizadas nem
circunscritas {[}isto é, pode haver fluidez e trânsito entre os mundos
para os seres, de acordo com sua evolução{]}. (\ldots) Os~diversos mundos
estão em condições muito diferentes uns dos outros, quanto ao grau de
adiantamento e de inferioridade {[}moral{]} de seus habitantes. (\ldots)
Ainda é possível, considerando seu estado, destino e matizes mais
destacados, dividi"-los, de um modo geral, em: mundos primitivos,
reservados às primeiras encarnações da alma humana; mundos de expiação e
de provas, onde predomina o mal; mundos regeneradores, onde as almas que
nada têm a expiar adquirem nova força, descansando das fadigas da luta;
mundos felizes, onde o bem sobrepuja o mal; e mundos celestes ou
divinos, morada dos espíritos purificados, onde o bem reina sem qualquer
mistura. A~Terra pertence à categoria dos mundos de expiação e de
provas, razão por que o homem nela está sujeito a tantas misérias.

\noindent Os espíritos encarnados em um mundo não estão a ele sujeitos
indefinidamente, nem tampouco nele cumprem todas as fases progressivas,
que devem ser percorridas até chegar à perfeição. Quando em um mundo
alcançaram um grau de aperfeiçoamento nele permissível, passam a outro
mais adiantado {[}isto é, morrem e reencarnam, sendo a morte um fenômeno
muito distinto conforme a evolução espiritual{]} e assim sucessivamente
até o estado de espíritos puros (pp. 44-45).
\end{quote}



Munidos de tais pressupostos, poderíamos nos perguntar sobre qual seria,
segundo o espiritismo, o elemento quintessencial para a cicatrização ou
depuração dos espíritos em sua trajetória evolutiva pelo universo. Em
\emph{O livro dos espíritos}, Kardec (2009) reafirma que

\begin{quote}
os espíritos não pertencem perpetuamente à mesma ordem. Todos progridem,
passando por diferentes graus de hierarquia espírita\footnote{Vale
  frisar, sempre em diálogo com Kardec, que a hierarquia espírita,
  fluida e dinâmica conforme a evolução dos espíritos, é de ordem moral.
  Nesse sentido, conforme afirma Kardec (2009), ``não há faltas
  irremissíveis e que não possam ser apagadas pela expiação. O~homem
  encontra o meio, nas diferentes existências, que lhe permite avançar,
  segundo seus desejos e esforços, na senda do progresso e na direção da
  purificação que é seu objetivo final'' (p. 17). Eis, então, à luz do
  espiritismo, a colocação de Cristo de que ``os últimos serão os
  primeiros, e os primeiros serão os últimos'' (\versal{MATEUS}, 20, 16),
  colocação revolucionária que se contrapõe ao atual estado de coisas
  como se ele fosse imutável e como se os valores e as práticas que
  reproduzem o existente -- a razão instrumental a presidir a reificação
  -- tivessem que ser adulados apenas por serem dominantes. Nesse
  sentido, seria possível, dialeticamente, fazer Jesus se contrapor a
  Cristo, uma vez que a continuação de sua colocação parece nomear
  eleitos para o Reino de Deus: ``Muitos serão os chamados, mas poucos
  os escolhidos'' (\versal{MATEUS}, 20, 16). Ao que o espiritismo poderia
  redarguir que todos serão chamados, aos poucos ou mais rapidamente,
  conforme sua evolução. Assim, todos seremos escolhidos.}. (\ldots) Esse
progresso ocorre pela {[}re{]}encarnação, que é imposta a uns como
expiação {[}em relação às faltas das vidas passadas{]} e a outros como
missão {[}de acordo com a lei da fraternidade em estreita correlação com
a depuração espiritual{]}. A~vida material é uma prova que {[}os
espíritos{]} devem suportar por várias vezes, até que hajam alcançado a
perfeição absoluta. É~uma espécie de exame severo ou depurador, de onde
eles saem mais ou menos purificados (p. 15).
\end{quote}

A partir de agora, o homem ridículo entrará, efetivamente, em seu
processo de depuração -- ou, para retomarmos a expressão hegeliana,
nosso herói aprofundará a cicatrização de seu espírito. O~processo é tão
dostoievskianamente vertiginoso que o sonho fará o tempo sobrepujar a si
mesmo, de modo que a queda rediviva do paraíso sobrevoe a história
humana para apreender seus momentos dialéticos de (de)formação. Em
poucas páginas, ``O sonho de um homem ridículo'' transpassa a nervura
milenar da humanidade para encontrar os patamares de contradição que
alçam os homens para além de si mesmos na mesma medida em que a
reconciliação para além do ego ainda não se processou. A~eternidade em
Dostoiévski, mediada por Hegel e Kardec, nos mostrará como o ímpeto de
síntese, que jamais esteve ausente da obra do escritor {[}mesmo (e
sobretudo) em seus momentos mais niilistas{]}, constitui o que há de
mais essencial ao movimento contraditório da história.

\section{5.5. Do Éden à queda}

Assim que o homem ridículo pergunta ao guia de sua jornada
intergaláctica se as repetições dos mundos configuram a lei do universo,
a criatura lhe diz que ``você vai ver tudo -- (\ldots) e um certo pesar se
fez ouvir na sua voz'' (\versal{DOSTOIÉVSKI}, 2003, p. 107). O~pesar se deve ao
fato de que a criatura, a vivenciar, com o ápice da consciência, um
tempo expandido, já conhece o que acontecerá a partir do momento em que
o homem ridículo chegar ao paraíso. Apesar de nosso herói deparar com a
concretude do além"-mundo mediada por seu sonho, ele ainda considera que

\begin{quote}
na nossa terra não podemos amar de verdade senão com o tormento e só
pelo tormento! De outro modo não sabemos amar e não conhecemos amor
diferente. Eu quero o tormento para poder amar. Eu tenho desejo, eu
tenho sede, neste exato instante, de beijar, banhado em lágrimas,
somente aquela terra que deixei, e não quero, não admito a vida em
nenhuma outra!\ldots (\versal{IDEM}, p. 108)
\end{quote}

Dostoiévski pode ser tido como um dos autores que mais escavaram o
subsolo da síndrome de Estocolmo, o amor da vítima pelo carrasco, o
apego do prisioneiro pelas grades de sua cela. No mito da caverna
dostoievskiano, os cativos se debatem não para arrebentar os elos das
correntes que os submetem, mas para apedrejar todos aqueles que tentam
demovê"-los do charco de suas sombras, do regozijo em sentir prazer com o
embotamento da realidade. É~assim que tudo aquilo que desponta como algo
terno e desprovido de confronto -- algo que tem o sentido de converter a
dor em purgação -- parece flagelar o niilista com o mesmo rancor que
pauta suas (não-)relações. A~reconciliação lembra ao homem ridículo a
necessidade de se haver com suas próprias faltas, a necessidade de se
cicatrizar, ao passo que a inércia tautológica da dor começa a sentir
prazer com o punhal a escarafunchar a ferida purulenta. O~passo
seguinte, como logo veremos, é transmitir aos felizes (e inscientes) a
consciência (e a liberdade) não como compaixão, mas como partilha da dor
-- se a dor flagela o homem ridículo, que ela passe então a flagelar a
todos os demais. Eis o que o niilismo concebe como convivência.

O homem ridículo já vê os oceanos, ele consegue discernir os contornos
da Europa -- súbito, o guia de sua viagem já não está mais ao seu lado.

\begin{quote}
Eu me achava, ao que parecia, numa daquelas ilhas que formam na nossa
terra o Arquipélago Grego, ou em algum lugar na costa do continente
vizinho a esse Arquipélago. Ah, tudo era exatamente como na nossa terra,
mas parecia que por toda a parte rebrilhava uma espécie de festa e um
triunfo grandioso, santo, enfim alcançado (\versal{IBIDEM}).
\end{quote}

As imagens idílicas pululam. O~vaivém das ondas beija e afaga as
margens, as copas coloridas das árvores ensinam a olhar para o céu, os
pássaros libertam o homem ridículo da cadeia alimentar e, sem qualquer
receio, pousam sobre os ombros de nosso herói e ressoam o cântico de
reconciliação que a relva aromática já prenunciara. Até que,
``finalmente, eu vi e conheci os habitantes dessa terra feliz'' (\versal{IDEM},
p. 109). O~homem ridículo se vê cercado não pela horda, mas pela
comunhão. Eles o beijam e saúdam, seus olhos não são crispados pela
dúvida, ``nas palavras e nas vozes dessa gente soava uma alegria de
criança'' (\versal{IBIDEM}).

\begin{quote}
Foram"-lhe, então, apresentadas algumas criancinhas para que pusesse as
mãos sobre elas e orasse por elas. Os~discípulos, porém, as afastavam.
Disse"-lhes Jesus: ``Deixai vir a mim estas criancinhas e não as
impeçais, porque o Reino dos céus é para aqueles que se lhes
assemelham'' (\versal{MATEUS}, 19, 13-14).
\end{quote}

Cristo se refere à noção de pureza das crianças, ao ímpeto desarmado, à
maturidade do adulto que recupera a seriedade da criança ao brincar
(Nietzsche), à comunhão que sorri. E, então, o homem ridículo compreende
para onde seu sonho o levou:

\begin{quote}
Ah, imediatamente, (\ldots) entendi tudo, tudo! Essa era a terra não
profanada pelo pecado original, nela vivia uma gente sem pecado, vivia
no mesmo paraíso em que viveram, como rezam as lendas de toda a
humanidade, os nossos antepassados pecadores, apenas com a diferença de
que aqui a terra inteira era em cada canto um único e mesmo paraíso.
Essas pessoas, rindo alegremente, se achegavam a mim e me afagavam;
levaram"-me consigo; e cada uma delas queria me apaziguar (\versal{DOSTOIÉVSKI},
2003, p. 109).
\end{quote}

O paraíso perdido sempre rondou o imaginário de Dostoiévski. De acordo
com Sergei Hackel (1982),

\begin{quote}
ao lado do ideal cristão, Dostoiévski partilhava o ideal de uma Idade de
Ouro, o ideal de uma harmonia cósmica há muito perdida, à qual a
humanidade sofredora ainda poderia aspirar, o distante objetivo para o
qual a contemplação da beleza ainda poderia impelir a humanidade mesmo
hoje. (\ldots) Algumas das personagens de Dostoiévski vivenciam uma visão
em sonho de tal Idade de Ouro nostálgica como um paraíso terrestre
perdido; outras a vivenciam como um paraíso a ser reconquistado. Cristo
e a Idade de Ouro, de modo separado, fornecem uma alternativa e um
antídoto ao presente estado de coisas, no qual tudo se dirige à
fragmentação e no qual é muito difícil falar sobre um movimento
centrífugo, uma vez que há não um centro a partir do qual a fugacidade
se inicia (p. 12).
\end{quote}

As colocações do homem ridículo sobre o paraíso, em meio ao qual os
seres animados e inanimados se relacionam de forma orgânica e
reconciliada, nos vão fazendo subir até o penhasco íngreme que logo vai
precipitar a alegria insciente dos primeiros homens e mulheres para a
queda. Nosso herói não consegue entendê"-los -- já sabemos que, ``para o
moderno progressista russo'' e ``petersburguês sórdido'', cuja
racionalidade egoico"-instrumental só fazia coincidir entendimento e
duelo, nem de longe era possível compreender como ``eles não desejavam
nada e eram serenos, {[}como{]} não ansiavam pelo conhecimento da vida
como nós ansiamos por tomar consciência dela'' (\versal{DOSTOIÉVSKI}, 2003, p.
111). O~homem ridículo conseguia vislumbrar que, para eles, a plenitude
era algo dado, algo simplesmente vivenciado.

\begin{quote}
A sua sabedoria era mais profunda e mais elevada que a da nossa ciência;
uma vez que a nossa ciência busca explicar o que é a vida, ela mesma
anseia por tomar consciência da vida para ensinar os outros a viver; ao
passo que eles, mesmo sem ciência, sabiam como viver, e isso eu entendi,
mas não conseguia entender a sua sabedoria. Eles me apontavam as suas
árvores, e eu não conseguia entender o grau de amor com que as olhavam:
era como se falassem com seres semelhantes a eles. E, sabem, talvez eu
não esteja enganado se disser que falavam com elas! Sim, eles
descobriram a sua língua, e estou certo de que elas os entendiam (\versal{IDEM},
pp. 111-112).
\end{quote}

Como é que nós poderíamos compreender o paraíso orgânico vivenciado pelo
homem ridículo, se nossa época só fez aprofundar as aporias niilistas?
Assim, em contraposição à nossa própria época civilizada, para Gerald J.
Sabo (2009), a partir do ensaio de Dostoiévski ``Socialismo e
cristianismo'' que analisamos no capítulo anterior, o homem ridículo
vivencia o paraíso como

\begin{quote}
a condição primordial do patriarcado -- a vida espontânea em meio à
massa: {[}Dostoiévski afirma que{]} ``Deus é a ideia da humanidade
coletiva, das massas, de tudo/todos. Quando um ser humano vive em meio
às massas, então ele vive \emph{espontaneamente}''. Entre eles -- as
crianças do Sol, e então o homem ridículo --, a comunicação é espontâneo
e intuitiva, com uma inefável aceitação amorosa. Na verdade, isso ocorre
até mesmo com as criaturas não"-humanas nesse duplo da terra. A~interação
do homem ridículo com as crianças do Sol de forma tão espontânea e
amorosa indica a presença de Deus entre eles conforme se verifica nas
palavras dos versículos 7 e 8 do capítulo 4 na ``Primeira epístola de
João'' sublinhada à caneta às margens do Novo Testamento russo de
Dostoiévski: ``Amados,~amemo"-nos~uns aos outros, porque o~amor~é de
Deus, e qualquer um que ama é~nascido de Deus~e conhece a Deus. Aquele
que não ama não~conhece~a Deus, porque Deus é amor'' (pp. 53-54).
\end{quote}

É assim que, em meio ao Éden, por meio do amor comunal que integra as
partes como completas extensões do Todo, que é Deus, não há esforço para
a obtenção e a partilha do fruto do trabalho, não há propriamente
\emph{meu} e \emph{teu}, a integração orgânica sequer precisa separar o
que é \emph{nosso}, pois o Verbo divino faz com que Adão e Eva só se
conjuguem como \emph{nós}. O~homem ridículo nos diz que não há ciúme e
inveja, mesmo o núcleo familiar ali não existia, já que os filhos eram
filhos de todos. Será que há morte no paraíso? Sim, mas um sentido inato
de eternidade faz com que a morte seja vivenciada como um até breve. A
integração, sempre segundo o homem ridículo, era tão plena, que eles
``não tinham fé, mas em troca tinham a firme noção de que, quando a sua
alegria terrena se plenificasse até os limites da natureza terrena,
então começaria para eles, tanto para vivos quanto para mortos, um
contato ainda mais amplo com o Todo do universo'' (\versal{DOSTOIÉVSKI}, 2003, p.
113). Se nos lembrarmos da tipologia panorâmica dos mundos espirituais
mencionada por Kardec, será possível dizer que tal paraíso idílico
estaria no tempo primevo da humanidade, em que o mundo -- a Terra e seus
infinitos duplos -- estava repleto de seres ``simples e inscientes''
(\versal{KARDEC}, 2009, p. 67), cuja alegria orgânica se contrapunha, entretanto,
à ausência de liberdade.

Quando o homem ridículo sentencia que ``esses homens não se esforçavam
por fazer com que eu os entendesse, amavam"-me assim mesmo, mas em
contrapartida eu sabia que eles também jamais me entenderiam, e por isso
quase não lhes falava da nossa terra'' (\versal{DOSTOIÉVSKI}, 2003, p. 112), não
há apenas ressentimento na colocação de nosso herói. É~bem verdade que o
rancor do homem ridículo o impede de sequer imaginar relações que não
estejam sob o signo da desconfiança e do desentendimento recíprocos.
Ainda assim, a hiperconsciência de nosso herói, filha de uma civilização
muito mais complexa e contraditória do que a organicidade primeva,
pressupõe o ímpeto por uma reconciliação que supere o individualismo
encarniçado com novos marcos de liberdade\footnote{Nesse sentido,
  Geraldo J. Sabo (2009) nos lembra, no ensaio ``Socialismo e
  cristianismo'', de que Dostoiévski considera o patriarcado como ``a
  condição primeva/primordial. A~civilização é uma condição
  intermediária, de transição'' (p. 52).}.

Nesse sentido, podemos utilizar, \emph{mutatis mutandis}, duas
categorias de que Hegel lança mão para apreender o movimento dialético
que envolve a \emph{liberdade substancial} (tese), algo próximo ao
patriarcado orgânico que o homem ridículo ora nos narra, e a
\emph{liberdade subjetiva} (antítese), que pressupõe um longo transcurso
histórico para o surgimento de estruturas sociais mais complexas, de
modo que o eu adquira consciência de si e, consequentemente, possa se
perceber como um ente distinto da imediatez do todo. Nas palavras de
Hegel (2008), ``a liberdade substancial é a razão da vontade existente
em si'', de forma orgânica, ou sob o jugo de um Estado patriarcal.
``Todavia, essa determinação da vontade não contém ainda sabedoria e
vontade próprias'' (p. 93). Os~filhos do Sol que circundam o homem
ridículo são plenos, mas não são livres, vale dizer, eles não têm
consciência da dimensão de sua própria existência. A~vida até pode
jorrar de uma fonte viva, como querem os laivos mais conservadores,
nostálgicos e anacrônicos presentes em Dostoiévski, mas o ônus para tal
retomada do Éden idílico seria o completo embotamento das conquistas
histórico"-racionais da humanidade. Aceitar tal fator limitado e
limitante para a interpretação da obra de Dostoiévski seria o mesmo que
pedir ao escritor que não compusesse suas narrativas, uma vez que elas
aguçam, justamente, a hiperconsciência do ego solapado pelo niilismo.
Assim, ao invés de um recuo, precisamos de um salto qualitativo para
além da integração orgânica, em meio à qual

\begin{quote}
leis e ordenamentos são consistentes em si mesmos, e diante deles os
sujeitos comportam"-se com perfeita subordinação. Ora, tais leis não
devem ser conformes à vontade individual, e os sujeitos assemelham"-se a
crianças obedientes a seus pais, sem vontade nem julgamento próprios
(\versal{HEGEL}, 2008, p. 93).
\end{quote}

O estado de integração orgânica é ainda mais simples do que a liberdade
substancial hegeliana, porque, como \emph{ainda} não houve a queda,
seres humanos, animais, vegetais e seres inanimados não se distinguem
entre si. (Lembremos que os filhos do Sol viviam em tamanha simbiose com
a natureza que conseguiam conversar, \emph{imediatamente}, com as
árvores.) A~liberdade subjetiva, por sua vez, ``só se determina no
indivíduo, constituindo a reflexão dele em sua consciência''. Assim,
prossegue Hegel,

\begin{quote}
quando se produz a liberdade subjetiva {[}e, aqui, o filósofo alemão
acelera o tempo histórico e a transformação das relações sociais da
mesma forma que o homem ridículo o fará quando sobrevoar a história
humana a partir da queda{]} e o homem desce da contemplação da realidade
exterior para a sua própria alma, surge o contraste sugerido pela
reflexão, envolvendo a negação da realidade {[}isto é, o destacamento do
ego do todo imediato{]}. De fato, sair do presente já forma uma
antítese, da qual um lado é Deus -- ou o divino -- e o outro o sujeito,
como um indivíduo (2008, pp. 93-94).
\end{quote}

A liberdade substancial que, em face da liberdade subjetiva, é, na
verdade, uma não"-liberdade, pressupõe a ``moralidade objetiva
espontânea'', em meio à qual ``a vontade individual do {[}não-{]}sujeito
adapta"-se imediatamente aos costumes, aos hábitos jurídicos e às leis. O
{[}não-{]}indivíduo está, portanto, em unidade espontânea com o fim
universal'' (\versal{IDEM}, p. 95). Em contrapartida, a moralidade objetiva deve
ser ``conquistada na luta pela liberdade subjetiva em seu
\emph{renascimento}'' {[}grifo que sintetiza o sentido da cicatrização
do espírito e que, em diálogo com Dostoiévski e Kardec, aponta para a
dialética ao longo da eternidade.{]} A~moralidade objetiva precisa se
elevar, ``pela purificação, até a livre subjetividade'' (\versal{IBIDEM}). Kardec
(2009) diria que ``a sabedoria de Deus'' -- sabedoria dialeticamente
distinta do Deus Pai e orgânico, uma vez que os homens e as mulheres já
não são idênticos a Adão e Eva -- ``está na liberdade que ele deixa a
cada um de escolher, porque cada um {[}em meio à liberdade subjetiva{]}
tem o mérito de suas obras'' (pp. 67-68). Como veremos até o fim de
nossa análise, para os homens ridículos, a negação da liberdade
subjetiva -- individualidade que, como Hegel nos demonstrou, é uma
negação primeira do ego que se aparta do todo orgânico para poder
existir como consciência de si e para si --, se se quiser emancipatória,
precisa se consubstanciar, historicamente, como uma negação determinada,
isto é, como a negação da negação da liberdade subjetiva com vistas à
reconciliação dos egos hedonistas em meio a uma sociedade de sujeitos
emancipados.

Segundo Theodor Adorno (2013), o temor do Hegel tardio e conservador --
aquilo que viria a ser derivado como hegelianismo de direita -- se
relacionava à projeção de que a esfera por excelência da liberdade
subjetiva, a sociedade civil e suas relações antagônicas, não pudesse
superar (\emph{aufheben}) a guerra intestina de todos contra todos. Daí
``a idolatria do Estado por Hegel'' (p. 104), já que o Estado, como
instituição a se erigir para além dos sujeitos individuais em colisão,
(supostamente) representaria o universal em meio à guerra dos
particulares da sociedade civil. Se considerarmos a situação de uma
Alemanha ainda não unificada e transpassada por principados antagônicos
em meados do século \versal{XIX}, veremos como o princípio hegeliano do Estado
como a representação do universal dizia respeito à superação
(\emph{Aufhebung}) da dispersão fratricida, aspecto que matiza a
atribuição de um conservadorismo unívoco a Hegel. Ainda assim, Adorno
nos faz pensar que o Hegel mais próximo do Estado prussiano seria o
Hegel para quem o real, tal como ele \emph{já} se apresenta, é racional,
assim como o racional, ainda que comporte a irracionalidade da autofagia
individualista e centrífuga, \emph{deve ser} real, isto é, \emph{deve
corroborar} os poderes existentes tais como eles se apresentam (e nos
oprimem). Poderíamos utilizar o mesmo raciocínio adorniano para o temor
de Dostoiévski que, a despeito de sua pujança racional que flerta com
uma espiritualidade dialeticamente outra, ainda tenta encontrar
antítodos para a modernidade por meio de soluções anacrônicas, idílicas
e pré"-modernas. Assim, se substituirmos a menção que Adorno faz a Hegel
no trecho seguinte, entenderemos, com mais profundidade, a necessidade
de negação determinada da razão instrumental com vistas a uma síntese
verdadeiramente racional, isto é, emancipatória -- e não, portanto,
regressiva, nostálgica e irracionalista:

\begin{quote}
Hegel manifestou que a sociedade reificada e racionalizada da era
burguesa, na qual a razão dominadora da natureza se consuma, poderia se
tornar uma sociedade digna dos homens, não através da regressão a
estágios mais antigos e irracionais, anteriores à divisão do trabalho,
mas ao aplicar sua racionalidade a si mesma, em outras palavras, ao
reconhecer salutarmente as marcas do irracional em sua própria razão
{[}a tensão entre a Razão que tende ao universal e a razão utilitária,
particularista e excludente{]}, assim como do rastro do racional no
irracional {[}a utopia do universal que subsiste em meio à desrazão
utilitária{]} (2013, p. 158).
\end{quote}

Nesse sentido, Richard Peace (1982), a meu ver, se equivoca ao dizer que
a utopia de Dostoiévski está univocamente baseada ``não no progresso da
razão humana, mas na retenção dos sentimentos inocentes e primitivos;
para o escritor, a Idade de Ouro não está no futuro, mas no passado''
(p. 73). Se há em Dostoiévski a nostalgia edênica, Richard Peace não
acompanha a dialética dostoievskiana que, desde o ateísmo espiritual de
Ivan Karamázov -- como vimos no capítulo anterior --, abre a
possibilidade para que a utopia se projete para o futuro, de modo que
sejam superadas, histórica e espiritualmente, as contradições
encarniçadas da civilização egoica que a modernidade levou às últimas
consequências.

Por motivo análogo, também preciso discordar, dialeticamente, de Birgit
Harress (1999), quando ela afirma que ``Dostoiévski enfatiza que o homem
governado por sua própria vontade acabará descendo ao nível dos
animais'' (p. 24). Dostoiévski certamente estaria de acordo com Hegel
(2008), para quem ``o conceito de uma relação da essência de Deus como
substância universal da ação humana'' {[}fundamenta{]} a ``moralidade
objetiva'' (p. 137). O~aspecto aterrador da modernidade, para
Dostoiévski, dizia respeito à perda da conexão com a eternidade -- isto
é, Deus -- por conta do acirramento dos egos. Conforme Gerald J. Sabo
(2009) pontua ao retomar uma passagem do ensaio de Dostoiévski
intitulado ``Socialismo e cristianismo'', trata"-se do segundo estágio do
desenvolvimento humano e social:

\begin{quote}
Nesse desenvolvimento posterior {[}em meio à civilização{]} surge um
fenômeno, um fato novo, do qual ninguém consegue escapar -- trata"-se do
desenvolvimento da consciência pessoal e da negação das ideias e leis
espontâneas (leis autoritárias e patriarcais das massas). (\ldots) Essa
condição, que implica a desintegração das massas em personalidades, como
civilização, é insalubre. A~perda da ideia vivencial sobre Deus dá
testemunho disso. Nessa condição, as pessoas se sentem mal, ficam
deprimidas, perdem a fonte da vida viva, não conhecem sensações
espontâneas e estão conscientes de tudo (p. 53).
\end{quote}

A perda da relação com a eternidade, para Dostoiévski, significava o
embotamento da realidade em seu sentido quintessencial -- precisamente
aquilo que a modernidade ateia passaria a proscrever como
\emph{fantástico} e/ou \emph{maravilhoso} -- e o solapamento da base
consuetudinária para as ações morais. É~assim que Lyudmila Parts (2009),
com um argumento que se aproxima do sentido desenvolvido ao longo deste
livro, compreende que

\begin{quote}
não é contra a ciência ou a lei que Dostoiévski dirige seus argumentos;
na verdade, o autor as vê como ineficazes sem uma base cristã. (\ldots)
Assim, se a piedade é necessária para a comunidade social e se a piedade
é inerentemente cristã, então a coesão social é impossível sem o
cristianismo. (\ldots) Um dos objetivos de Dostoiévski em \emph{Crime e
castigo} é mostrar as limitações da piedade secular e a impossibilidade
da moralidade secular. Para esse fim, ele mostra como a tentativa de
eliminar a religião da fundação da sociedade leva à degradação moral (p.
62; p. 70).
\end{quote}

Concordo com Birgit Harress quando a autora fala sobre o terror de
Dostoiévski em relação ao homem entregue a si mesmo, isto é, o temor em
relação ao homem sem vinculação com a fonte da \emph{vida viva}, vale
dizer, Deus. No paraíso edênico, o homem ridículo pôde entrever que
``eles continuavam em contato com os seus mortos mesmo depois de sua
morte, (\ldots) {[}pois{]} a morte não rompia a ligação terrena entre eles.
(\ldots) Estavam tão inconscientemente convictos dela {[}da vida eterna{]}
que isso não constituiria para eles uma questão'' (\versal{DOSTOIEVSKI}, 2003, p.
113). Nesse sentido, Kate Holland (2000) também entreviu que

\begin{quote}
as visões de mundo lineares da \emph{intelligentsia} {[}os
revolucionários{]} se baseavam no triunfo da razão e da lógica e em um
paraíso terrestre, mas, segundo Dostoiévski, não passavam de abstrações,
porque elas não satisfaziam as necessidades morais e espirituais do
homem, e, em particular, elas não respondiam à questão que Dostoiévski
considerava a mais crucial para o homem: a possibilidade de vida eterna
para além do túmulo\footnote{Nesse sentido, podemos entrever com que
  avidez Dostoiévski teria lido a seguinte passagem de Hegel (2008): ``A
  ideia de que o espírito é imortal inclui a posse pelo indivíduo humano
  de um infinito valor em si. O~mero natural parece isolado, é pura e
  simplesmente dependente de outro e tem a sua existência em outro: com
  a imortalidade, manifesta"-se a concepção de que o espírito é infinito
  em si mesmo'' (pp. 180-181).} (p. 101).
\end{quote}

Sem jamais deixarmos de considerar o caráter fundamental da eternidade
para Dostoiévski, podemos proceder à negação determinada da colocação de
Birgit Harress, uma vez que não se trata de relegar os homens ao estado
primevo de massa insciente, mas de fazer com que a humanidade, ao se
tornar autoconsciente, possa transformar a consciência de si em
consciência entre si, entre os homens -- em consciência partilhada em
meio a uma sociedade emancipada. Já Hegel (2008) falara que, na
liberdade substancial -- isto é, em meio à não"-liberdade, se tivermos em
mente a antítese da futura conformação da liberdade subjetiva --, falta
``a vontade que cumpre as ordens por convicções exteriores'', já que o
espírito, ``por não ter atingido a interioridade'', isto é, a condição
(autor)reflexiva, ``mostra"-se como mera espiritualidade natural'' (p.
101). A~decorrência do argumento de Hegel nos mostra por que o então
jovem e materialista Karl Marx iniciou sua trajetória intelectual como
um hegeliano de esquerda:

\begin{quote}
A mundaneidade deve ser adaptada ao princípio espiritual, mas
\emph{deve}, apenas: a força profana, abandonada pelo espírito, logo se
esvai, necessariamente, diante da força espiritual, enquanto aquela,
mergulhando na primeira, perde com a sua determinação também a sua
força. (\ldots) A~forma mais elevada do pensamento racional {[}é{]} o
espírito de novo refluindo sobre si próprio, produzindo a sua obra sob a
forma de pensamento e tornando"-se capaz de realizar o racional graças
unicamente ao princípio de mundaneidade. Acontece que, em virtude da
eficácia de determinações universais, que têm o seu fundamento no
princípio do espírito, \emph{o reino do pensamento é engendrado no
real.} (\ldots) \emph{Isso é o resultado final a que o processo histórico
deve chegar}, e nós temos que percorrer o longo caminho que acaba de ser
sumariamente indicado. \emph{Mas a extensão do tempo é algo muito
relativo e o espírito pertence à eternidade}; para ele, não há
propriamente extensão {[}grifos meus{]} (2008, p. 97).
\end{quote}

A história, tanto para Hegel (e o jovem Marx) quanto para Dostoiévski,
se transforma no terreno primordial para a cicatrização do espírito,
para que ``ele se produza e se transforme no que é''. O~espírito é
``originalmente livre, e a liberdade é sua natureza e seu conceito''
(\versal{IDEM}, p. 209). Mas Hegel e Dostoiévski diriam que não se trata de
afirmar, em si e para si, a liberdade original do espírito, o estado
insciente e primordial dos homens, pois

\begin{quote}
não podemos considerar tal estado de selvageria como algo sublime,
cometendo talvez o erro de Rousseau, que imaginou a situação dos
selvagens da América como aquela na qual o homem estivesse no domínio da
verdadeira liberdade. O~selvagem desconhece uma grande parte do
infortúnio e da dor, mas isso é apenas negativo, enquanto a liberdade
tem que ser essencialmente afirmativa. Os~benefícios da liberdade
afirmativa são os benefícios da consciência sublime (\versal{IBIDEM}).
\end{quote}

Como já havíamos discutido no capítulo anterior, a liberdade em
Dostoiévski constitui a quintessência da trajetória do espírito. Vale
frisar, ainda uma vez, que se trata de uma liberdade determinada, uma
vez que a liberdade completamente centrífuga -- o ego que apenas se
volta para e sobre si mesmo -- bem poderia redundar em fratura social.
(Também já vimos os desdobramentos que a obra de Dostoiévski apresenta
nesse sentido, por meio do homicídio, da extorsão, da pedofilia e do
suicídio.) Em diálogo com Hegel e com a trajetória do homem ridículo,
começaremos a compreender, a partir de agora, que o movimento da
história, com a deformação advinda da queda dos homens do paraíso
edênico, apresenta, dialeticamente, a possibilidade de superação da
insciência original por meio das feridas como um momento fundamental da
liberdade -- as feridas do ego individualizado e não mais organicamente
integrado pressupõem o ímpeto pela cicatrização, o ímpeto pela
reconciliação em uma nova totalidade para além da manada original.

De volta à nossa trajetória narrativa, o homem ridículo quer saber por
que, em meio aos habitantes do Éden que o recebem como a um irmão, ``não
consigo odiá"-los, se não os amo, por que não consigo deixar de
perdoá"-los, e ainda assim no meu amor por eles há melancolia: por que
não consigo amá"-los, se não os odeio?'' Nosso herói sentencia, então,
que ``a sensação de plenitude da vida me tirava o fôlego, e eu os
adorava calado'' (\versal{DOSTOIÉVSKI}, 2003, p. 114; p. 115). Ainda uma vez, a
tensão entre o passado orgânico e o futuro com potencial de liberdade se
estabelece para apreendermos a utopia da reconciliação em Dostoiévski.
Estamos a um passo de descobrirmos que o homem ridículo fará as vezes da
serpente demoníaca do \emph{Gênesis}. Nosso herói será o artífice da
queda, isto é, ele propiciará a síntese radical para o transcurso da
história que vai da massa insciente e suas formações sociais orgânicas e
simples à complexidade da civilização e sua divisão social trabalho que
ensejam o indivíduo e a autoconsciência egoica. Assim, dialeticamente, o
homem ridículo sente a dor de não poder se irmanar,
\emph{imediatamente}, à comunhão paradisíaca. Mas, conforme vimos
argumentando, tal melancolia implica uma evolução, um movimento
qualitativamente distinto: é duríssimo não poder amá"-los, é um horror
entrever que o desamor egoico levou o homem ridículo ao suicídio, mas
não é possível retornar a momentos que a história superou sem mutilar as
mediações que foram conformando a identidade humana através do
transcurso das gerações. Preciso concordar com Richard Peace (1982),
quando o autor diz que

\begin{quote}
é o homem a se apartar dos demais que constitui a causa do declínio da
Idade de Ouro em ``O sonho de um homem ridículo''. (\ldots) Nos romances de
Dostoiévski os homens que se apartam dos demais, tais como o homem do
subsolo, Raskólnikov, Kiríllov, se tornam propensos a ideias
extremistas, negativas e destrutivas (p. 72).
\end{quote}

Ocorre que, conforme discutimos no capítulo anterior, a reintegração dos
egos não se dará, como quer Aliócha Karamázov, por meio da clausura da
humanidade, ainda uma vez, em feudos e/ou monastérios, não humanizaremos
a humanidade ao embotarmos o nível de consciência social historicamente
erigido. O~temor e o tremor dostoievskianos fazem o autor levar a razão
utilitária tanto ao ápice dialético de sua superação pela razão
emancipatória quanto ao retrocesso pela desrazão que se volta para o
embotamento do paraíso orgânico e/ou para o misticismo do cristianismo
ortodoxo. A~obra de Dostoiévski testemunha, nesse sentido, uma colocação
lapidar do clérigo Tíkhon, personagem de \emph{Os demônios}: ``O ateísmo
completo está no penúltimo degrau da fé mais perfeita (se subirá esse
degrau já é outra história)'' (2004, p. 662). Por nutrir essa
contradição agônica até o fim de sua vida -- contradição que, na
verdade, ultrapassa Dostoiévski e se alça da subjetividade do escritor
para o caráter objetivo de sua época, como espírito do tempo --, o autor
fomentou em sua obra a tese positiva do retrocesso orgânico, histórico e
espiritual e a antítese negativa e radical da espiritualidade dialética
a acompanhar a razão. A~angústia sobre a possibilidade histórica de
negação da negação (a síntese) -- angústia que, como já vimos, comporta
um Dostoiévski revolucionário para além do Fiódor nacionalista e
ortodoxo cerceado por sua época -- tornou a contradição o elemento
histórico"-motor da obra do escritor russo, de modo a não antecipar (e
hipostasiar) ``a reconciliação como um atentado contra a conciliação
real'' (\versal{ADORNO}, 2013, pp. 102-103), uma vez que a síntese ainda não
adveio -- e a dúvida lancinante do niilismo que sempre rondou
Dostoiévski suspeita que ela pode não advir. Nesse sentido, Sergei
Hackel (1982) nos relata que, quando um amigo de Dostoiévski o intimou a
fornecer respostas para as críticas de Ivan Karamázov a Deus, em sua
\emph{Lenda do grande inquisidor}, o escritou assim redarguiu:

\begin{quote}
Precisamente, e toda a minha preocupação e ansiedade residem nisso
agora. O~próximo livro trará uma resposta a todo o \emph{lado negativo.}
E eu fico até trêmulo com isso: será que chegarei a uma resposta
\emph{suficiente?} Especialmente porque a resposta, na verdade, não será
direta, não será uma resposta ponto a ponto em relação às questões
levantadas anteriormente, mas ela virá de forma indireta. Isso é o que
me perturba, isto é, será que chegarei a uma resposta compreensível e
será que alcançarei sequer uma parte do meu objetivo? (p. 13)
\end{quote}

Como Dostoiévski faleceu em 1881, dois anos após a publicação de
\emph{Os irmaõs Karamázov}, não veio à tona a obra que poderia se
contrapor, ainda que obliquamente, às diatribes do grande inquisidor --
diatribes que, conforme vimos argumentando, procuram resguardar,
dialeticamente, a possibilidade de a espiritualidade acompanhar a razão.
Ainda assim, ``O sonho de um homem ridículo'', a meu ver, desempenha um
papel \emph{positivo}, na medida em que, ao transitar da agonia para a
possibilidade de redenção, ilumina as contradições objetivas de sua obra
e nos fornece a possibilidade de apreender o ímpeto por uma nova
totalidade através das fraturas e caminhos exíguos abertos pela história
que \emph{ainda} não se reconciliou.

É assim que, no ápice da integração edênica da qual o homem ridículo já
não pode fazer parte, nosso herói nos conta o segredo que já
antecipáramos: ``Tudo isso, talvez, não tenha sido sonho coisa
nenhuma!'' Afinal, a espiritualidade proscrita pela modernidade como um
sonho fantástico, para Dostoiévski, é a realidade elevada à máxima
potência. ``Porque aqui {[}no Éden{]} se passou uma coisa tal, uma coisa
tão horrivelmente verdadeira, que não poderia ter surgido em sonho (\ldots)
O fato é que eu\ldots perverti todos eles!'' (\versal{DOSTOIÉVSKI}, 2003, pp.
115-116)

\section{5.6. Da queda à reconciliação?}

\begin{quote}
Sim, sim, o resultado foi que eu perverti todos eles! (\ldots) O~sonho
atravessou um milênio voando e deixou em mim apenas a sensação do todo.
Só sei que a causa do pecado original fui eu (\versal{IDEM}, p. 117).
\end{quote}

A última parte de ``O sonho de um homem ridículo'' volta aos primórdios
do \emph{Gênesis} com a reedição da queda de Adão e Eva. Assim, em uma
passagem que um leitor desavisado poderia atribuir a Dostoiévski, Hegel
(2008) nos conta que

\begin{quote}
o homem, criado à imagem e semelhança de Deus, perdeu o seu
contentamento absoluto ao comer da árvore do conhecimento do bem e do
mal. O~pecado aqui reside no conhecimento: ele é pecaminoso, e por sua
causa o homem perdeu a felicidade natural. É~bem verdade que o mal
reside na consciência, pois os animais não são nem bons e nem maus, da
mesma forma que o homem natural. Só a consciência possibilita a divisão
do eu, depois de sua infinita liberdade como arbitrariedade e do puro
conteúdo da vontade, do bem. O~conhecimento da elevação da unidade
natural é o pecado, não sendo uma história acidental, mas a história
eterna do espírito, pois esse estado da inocência, esse estado
paradisíaco, é animalesco. O~paraíso é um jardim onde só os animais
podem permanecer, não os homens, pois o animal forma uma unidade com
Deus, mas apenas por si. Só o homem é espírito, ou seja, para si mesmo.
Esse ser para si, essa consciência, é, todavia, a divisão do espírito
universal divino. Se me oponho, em minha liberdade abstrata, contra o
bem, então essa é a posição do mal. O~pecado original é, por isso, o
eterno mito do homem, pelo qual ele se torna humano (p. 273).
\end{quote}

A questão dostoievskiana por excelência, conforme vimos argumentando, é
saber se a humanidade conseguirá se humanizar \emph{para além} do pecado
original, se ela conseguirá se religar com a eternidade, já que a
estagnação no pecado original, na individuação hedonista, ``é o mal;
(\ldots) {[}eis{]} o sentimento de dor sobre si e a ânsia que encontramos
em Davi, quando ele canta: `Senhor, concede"-me um coração puro e um
espírito novo'" (\versal{IDEM}, p. 273). O~homem ridículo dera um tiro em seu
coração, e Cristo já dissera que o Reino de Deus pressupõe espíritos
puros como as crianças. A~pureza, aqui, se refere à reforma moral, sobre
a qual falaremos mais adiante, mas não corrobora o embotamento da
consciência. O~coração puro, no cântico dos cânticos de Davi, quer se
aliar a um espírito dialeticamente novo -- eis uma síntese para a
trajetória do homem ridículo.

Nosso herói nos diz que a perversão por ele inoculada entre os filhos do
Sol logo dá vazão à mentira. O~sobrevoo quintessencial pelos elos de
queda e conexão da humanidade segue de forma vertiginosa -- cada palavra
e cada frase sintetizam séculos de transcurso da história: a mentira
enseja o gosto pela mentira, a beleza começa a se irmanar a tudo o que é
lúgubre, logo chegamos à volúpia, por sua vez contígua ao ciúme -- se já
há o ciúme, estamos diante da ruptura da não"-propriedade, isto é, da
produção social como o todo comunal. O~ciúme pressupõe a cisão entre o
\emph{meu} e o \emph{seu}, o \emph{nosso} se fratura e já aponta para os
primórdios da luta de classes. ``Bem depressa respingou o primeiro
sangue'' (\versal{DOSTOIÉVSKI}, 2003, p. 117).

\begin{quote}
Caim disse (\ldots) a Abel, seu irmão {[}ambos filhos de Adão e Eva{]}:
``Vamos ao campo''. Logo que chegaram ao campo, Caim atirou"-se sobre seu
irmão e matou"-o. O~Senhor disse a Caim: ``Onde está teu irmão Abel?'' --
Caim respondeu: ``Não sei! Sou porventura eu o guarda do meu irmão?'' O
Senhor disse"-lhe: ``Que fizeste! Eis que a voz do sangue do teu irmão
clama por mim desde a terra. De ora em diante, serás maldito e expulso
da terra, que abriu sua boca para beber de tua mão o sangue do teu
irmão. Quando a cultivares, ela te negará os seus frutos. E~tu serás
peregrino e errante sobre a terra'' (\versal{GÊNESIS}, 4, 8-12).
\end{quote}

O \emph{Gênesis} nos diz que Abel era pastor e Caim, lavrador (4, 2). A
simbologia é profunda: Abel, nosso antepassado mais puro e longínquo,
ainda desempenhava atividades nômades e rudimentares. Ademais, o pastor
pressupõe o rebanho, forma pela qual os filhos do Éden eram chamados em
sua integração e indiferenciação orgânicas. O~lavrador Caim pressupõe a
sedimentação oriunda da agricultura. Quando Deus condena o primeiro
assassino mítico à errância e à peregrinação, os homens e mulheres
decaídos sentimos o início da contradição de termos nos destacado do
todo amorfo, de termos nos reconhecido como criaturas, de termos nos
sedimentado, mas, ainda assim, de não mais conseguirmos nos reconciliar.
A ironia que Caim desfere contra Deus mostra a altivez do ego que ousa
se contrapor Àquele que o forjou. Quando o onisciente pergunta a Caim
onde está Abel, o assassino Lhe responde: ``Não sei! Sou porventura eu o
guarda de meu irmão?'' Ora, Deus mesmo já deveria saber. O~tema é
dostoievskiano e hegeliano por excelência: a liberdade subjetiva dos
novos homens pressupõe o desvio; a liberdade, no limite, pressupõe a
mimese das aniquilações divinas por parte dos homens -- Caim toma para
si o gládio e, ao matar seu irmão Abel, sentencia que não apenas Deus é
capaz de afogar suas criaturas com o dilúvio. Diante da cumplicidade de
Deus com o assassínio, Kardec (2009) diria que ``Deus não criou
espíritos maus, criou"-os simples e ignorantes, isto é, com aptidão tanto
para o bem quanto para o mal {[}com capacidade para a perfectibilidade
divina{]}. Aqueles que são maus {[}como Caim o foi{]} assim se tornaram
por sua vontade'' (p. 67). Como ainda somos mais descendentes de Caim do
que de Abel, Dostoiévski, Hegel e Kardec diriam que ``a sabedoria de
Deus está na liberdade que ele deixa a cada um de escolher, porque cada
um tem o mérito de suas obras'' (\versal{IDEM}, pp. 67-68). Do contrário,
seríamos felizes e autômatos, reconciliados e inscientes. Em suma, mais
\emph{animais} que \emph{racionais. }

Após o assassínio de Abel, os homens e mulheres, à imagem e à semelhança
de Caim, ``começaram a se dispersar, a se dividir. Surgiram alianças,
mas dessa vez umas contra as outras'' (\versal{DOSTOIÉVSKI}, 2003, p. 117). As
epígrafes dostoievskianas sobrevoam movimentos ctônicos da história
humana. O~jovem Marx de \emph{A ideologia alemã} (1998) desponta, então,
para estabelecermos mediações entre a integração original e orgânica e o
surgimento das alianças facciosas e encarniçadamente antípodas:

\begin{quote}
A primeira forma da propriedade é a propriedade tribal. Ela corresponde
àquele estágio rudimentar da produção em que um povo se alimenta da caça
e da pesca, do pastoreio ou, eventualmente, da agricultura {[}tenhamos
em mente as imagens de Abel e Caim{]}. (\ldots) Nesse estágio, a divisão do
trabalho é ainda muito pouco desenvolvida e representa apenas uma
extensão maior da divisão natural que ocorre na família. A~estrutura
social se limita, por isso mesmo, a uma extensão da família: chefes da
tribo patriarcal, abaixo deles os membros da tribo e os escravos. A
escravidão latente na família só se desenvolve paulatinamente com o
aumento da população e das necessidades, com a extensão dos intercâmbios
externos, tanto da guerra como do comércio (pp. 12-13).
\end{quote}

As famílias vão se tornando maiores, o excedente da produção coletiva
passa a ser apropriado pela família mais numerosa e poderosa, as cisões
ensejam o nomadismo de Caim -- a agricultura talvez ainda não tenha
atingido a pujança total para a fixação. A~dispersão dos homens e
mulheres pressupõe a divisão do trabalho social a partir da divisão do
trabalho no seio familiar -- o masculino dominante e o feminino
subalterno para além da distinção biológica, o regime da casa e a busca
por alimento. Marx prossegue dizendo que

\begin{quote}
essa divisão do trabalho encerra ao mesmo tempo a repartição do trabalho
e de seus produtos, distribuição \emph{desigual}, na verdade, tanto em
quantidade quanto em qualidade. Encerra portanto a propriedade, cuja
primeira forma, o seu germe, reside na família em que a mulher e os
filhos são escravos do homem (\versal{IDEM}, p. 27).
\end{quote}

Nessa nova guerra de todos contra todos que se retroalimenta do micro ao
macrocosmo, o surgimento de alianças facciosas e encarniçadamente
contrapostas pressupõe a expansão dos clãs e a formação de nações. O
homem ridículo nos diz, então, que surgem ``as acusações, as censuras.
Conheceram a vergonha, e a vergonha erigiram em virtude. Nasceu a noção
de honra, e cada aliança levantou a sua própria bandeira'' (\versal{DOSTOIÉVSKI},
2003, p. 117). Vemos, aqui, um marco dialético importante: a superação
da (não-)liberdade substancial do todo orgânico pela liberdade subjetiva
do ego passa a aprofundar a lei do mais forte. Assim, as alianças se
assemelham à ideia do contrato social hobbesiano: para que os egos
fratricidas não se aniquilem completamente, a nova liberdade começa a
alicerçar mecanismos de coesão e autoridade sociais que possuem teores
de verdade para além da lei do mais forte. Ainda que os grupos mais
poderosos prevaleçam, a vergonha e a honra, vistas pelo homem ridículo
como uma perda em relação à integração original, representam,
dialeticamente, a retomada do ímpeto de integração para os homens e
mulheres que já não podem ser subsumidos pelo rebanho do Éden. Os~homens
passam a falar línguas diferentes, a torre de Babel ganha contornos e se
irradia com as grandes navegações, a ciência irrompe e se retroalimenta
da manufatura e da indústria vindouras -- ``na indústria, o homem é o
seu próprio objeto e trata a natureza como algo subjugado a ele,
imprimindo o selo de sua atividade'' (\versal{HEGEL}, 2008, p. 163), de modo que,
conforme Marx (1998), logo

\begin{quote}
encontramos a oposição entre cidade e campo e, mais tarde, a oposição
entre os Estados que representam o interesse das cidades e aqueles que
representam o interesse dos campos (\ldots). A~partir do instante em que o
trabalho começa a ser dividido, cada um tem uma esfera de atividade
exclusiva e determinada, que lhe é imposta e da qual ele não pode fugir;
ele é caçador, pescador, pastor ou crítico e deverá permanecer assim se
não quiser perder seus meios de sobrevivência. (\ldots) A~divisão do
trabalho implica também a contradição entre o interesse do indivíduo
isolado ou da família isolada e o interesse coletivo de todos os
indivíduos que mantêm relações entre si (pp. 13-14; pp. 27-28).
\end{quote}

Quando as contradições se exacerbam de maneira a quase estilhaçar o todo
frágil que só se mantém pela exploração das desigualdades e pela
necessidade de sobrevivência,

\begin{quote}
quando todos se tornaram maus, começaram a falar em fraternidade e
humanidade e entenderam essas ideias. Quando se tornaram criminosos,
conceberam a justiça e prescreveram a si mesmos códigos inteiros para
mantê"-la, e para garantir os códigos instalaram a guilhotina. Mal se
lembravam daquilo que perderam, não queriam acreditar nem mesmo que um
dia foram inocentes e felizes (\versal{DOSTOIÉVSKI}, 2003, p. 118).
\end{quote}

A dialética avivou o processo de antítese no homem ridículo justamente
quando o suicídio já lhe fazia entrever a lápide de sua sepultura.
Assim, o ápice do processo de autofagia social enseja a eclosão de
ideias totalizantes que alcem os homens para além de si mesmos, que
estruturem o movimento global da história como a tensão entre o
particular e o universal, a exploração e a emancipação. Era nesse
sentido que Hegel (2008) compreendia o empreendimento das grandes
navegações:

\begin{quote}
O oceano convida o homem à conquista e à pilhagem, mas igualmente ao
comércio e ao lucro. A~terra, na região do vale, fixa o homem ao solo,
tornando"-o infinitamente dependente; o mar o conduz para além desses
limitados círculos de pensamento e ação. Aqueles que cruzam os mares
visam também ao lucro; mas os meios que utilizam para realizar esse
intento são paradoxais: arriscam a propriedade e a própria vida. Os
meios são, portanto, o oposto daquilo que tencionam (pp. 80-81).
\end{quote}

Que a guilhotina irrompa como um símbolo bárbaro em meio à tentativa de
humanização da humanidade para além da barbárie absolutista e medieval
não deveria embasbacar nem a escatologia do homem ridículo e nem seu pai
criador, uma vez que Fiódor Dostoiévski sempre acompanhou de forma rente
as contradições da história e dos homens. Nesse momento, se quiséssemos
tornar unilateral a apreensão da narrativa dostoievskiana, diríamos que
o escritor possui um profundo materialismo crítico quando se trata de
apreender as aporias dos movimentos esquerdistas de transformação da
humanidade, criticidade que se arrefece e se torna algo idealista quando
Dostoiévski narra a distância entre a história material dos homens e a
Ideia de reconciliação com o paraíso perdido. Como este livro não
apreende Dostoiévski como um nostálgico -- apesar de analisarmos os
momentos em que tal perspectiva está presente em sua obra --, trata"-se
de insuflar dialética para entendermos como a mão que fere também pode
curar, como o punho cerrado do soco pode se distender para estender a
mão para a amizade e o afago. Hegel, nesse sentido, estabelece a
mediação para voltarmos a arremessar o homem ridículo contra si mesmo,
para que tentemos entrever as cabeças que voltam a se unir ao corpo após
a guilhotina: ``A escravidão é, em si e por si, injustiça, pois a
essência humana é liberdade. Mas, para chegar à liberdade, o homem tem
que amadurecer. Portanto, a abolição progressiva da escravidão é algo
mais apropriado e correto do que a sua abrupta anulação'' (2008, p. 88).
Mas, prossegue o homem ridículo, ``se pelo menos fosse possível que eles
voltassem àquele estado inocente e feliz do qual se privaram, e se pelo
menos alguém de repente o mostrasse a eles de novo e lhes perguntasse:
querem voltar? -- eles certamente recusariam'' (\versal{DOSTOIÉVSKI}, 2003, p.
118). Ao que responderíamos ao homem ridículo: não é possível retornar
ao Éden insciente sem mutilar a história, sem condenar a humanidade à
prostração diante de um Deus Pai supremo que trata seus filhos como
crianças eternas. A~recusa é dialética: a volta ao Éden não se dará como
um movimento linearmente circular, ao fim do qual o ponto de chegada
coincide com o ponto de partida. Isso significaria a supressão das
conquistas históricas que brotaram do solo da pilhagem, significaria o
embotamento da memória transmitida através das gerações. Só a completa
barbárie -- uma hecatombe nuclear como a imagem da totalidade a se
confundir com o nada -- nos faria voltar a acender velas de cera diante
de ícones silenciosos. Como Hegel bem o sabia desde Heráclito de Éfeso,
o retorno se dá por meio de um movimento em espiral -- há paralelismo
entre pontos de consciência e desenvolvimento análogos (a memória sempre
nos trará nostalgia), mas não se trata das mesmas situações, pois o rio
de Heráclito faz a quantidade transbordar pelas margens para alcançar
uma nova qualidade. Assim, a sociedade irreconciliada sente saudade da
Idade de Ouro, uma saudade, na verdade, projetada dialeticamente para o
futuro, uma saudade que quer recuperar o teor de verdade do Éden (a
integração) para alçá"-lo em convivência reconciliada dos novos homens e
mulheres cientes de si e entre si.

Ocorre que -- assim dizem os outrora filhos do Sol ao homem ridículo --
nós

\begin{quote}
temos a ciência, e por meio dela encontraremos de novo a verdade, mas
dessa vez a usaremos conscientemente, o entendimento é superior ao
sentimento, a consciência da vida -- é superior à vida. A~ciência nos
dará sabedoria, a sabedoria revelará as leis, e o conhecimento das leis
é superior à felicidade''. Era o que eles me diziam, e nem podiam fazer
diferente. Cada um tornou"-se tão cioso de sua individualidade que não
fazia outra coisa senão tentar com todas as forças humilhar e diminuir a
dos outros, e a isso dedicava a sua vida (\versal{DOSTOIÉVSKI}, 2003, p. 119).
\end{quote}

O fragmento em questão enuncia o sentido dialético da utopia: a elevação
racional dos homens para além da razão utilitária -- como os autores
mais conservadores da fortuna crítica dostoievskiana tendem a
interpretar as diatribes do escritor como se elas fossem críticas
\emph{in toto} (e, portanto, não dialéticas), o contrário supostamente
reconciliado da sociedade racional seria a organicidade idílica e
sentimental. Entretanto, já Theodor Adorno (2009) se perguntara se

\begin{quote}
as coisas seriam tão diferentes em épocas que se presumem como sob uma
abóbada celeste metafísica, épocas que o jovem Lukács {[}da \emph{Teoria
do romance}{]} denominava as épocas prenhes de sentido. (\ldots) O~caráter
fechado das culturas, a imperatividade coletiva de concepções
metafísicas, o seu poder sobre a vida não garantem a sua verdade. A
possibilidade de uma experiência metafísica é antes irmanada com a
possibiliddade da liberdade, e, dessa liberdade, somente o sujeito
desenvolvido é capaz, o sujeito que destruiu os laços louvados como
sagrados (p. 328).
\end{quote}

Como Dostoiévski pode ser tido como um dos autores que, dialeticamente,
mais arremessaram a razão contra si mesma por meio do próprio movimento
racional, não é possível enquadrá"-lo na fileira dos irracionalistas (por
mais que algumas tendências do próprio Dostoiévski assim o quisessem) e
não é possível aceitar o ninho idílico do homem ridículo sem abortar a
liberdade a que Hegel e Adorno (e o próprio Dostoiévski) fazem menção.
Os filhos do Sol que se tornaram ciosos da própria individualidade vivem
em um contexto histórico"-social que, a administrar de forma cada vez
mais rente (e introjetada) todas e cada uma de nossas ações, arremessa
os sujeitos uns contra os outros em meio à sanha pela sobrevivência e
apresenta a irracionalidade fundamental a reproduzir a sociedade
socializada. Então -- prossegue o homem ridículo --, nesse contexto de
escravidão coletiva (e até mesmo de escravidão voluntária) e de
apedrejamento dos justos,

\begin{quote}
surgiram pessoas que começaram a imaginar: como fazer com que todos se
unam de novo, de modo que cada um, sem deixar de amar a si mesmo mais do
que aos outros, ao mesmo tempo não perturbe ninguém e possam viver assim
todos juntos como que numa sociedade cordata (\versal{DOSTOIÉVSKI}, 2003, p.
119).
\end{quote}

Neste momento em que o eco do socialismo ressoa no sobrevoo vertiginoso
da história em ``O sonho de um homem ridículo''\emph{,} a voz autoral do
ex"-socialista, ex"-membro do Círculo de Petrachévski e ex"-prisioneiro
político Fiódor Dostoiévski pulsa sob a colocação de nosso herói. Susan
McReynolds (2003) sintetiza a tensão entre transformação moral e social
em Dostoiévski ao afirmar que

\begin{quote}
ao longo de sua via após o exílio, o autor insiste que o progresso
comunal só é alcançado por meio do avanço moral dos indivíduos. ``O
homem não muda por razões \emph{externas}, mas apenas por transformações
\emph{morais}'', escreve Dostoiévski em seu caderno de anotações em
1863. Assim, de acordo com o escritor, a melhoria das formas externas e
institucionais da vida social não resulta das mudanças nos papéis
políticos, mas das transformações da condição interior e moral dos
indivíduos'' (p. 82).
\end{quote}

A noção de que a transformação interior e moral \emph{precede} a
transformação externa e político"-social pôde ser vivenciada por
Dostoiévski no frêmito da comutação da pena de morte a que os
integrantes do Círculo de Petrachévski haviam sido condenados. Robert
Louis"-Jackson (2002) menciona um trecho de uma carta de Dostoiévski a
seu irmão Mikhail, trecho marcado por um extraordinário senso de
descoberta espiritual:

\begin{quote}
Meu irmão, eu não perdi a coragem e não me sinto desmotivado. Vida é
vida em qualquer lugar, a vida está em nós mesmos, e não no exterior.
Haverá pessoas ao meu redor, e ser um homem entre os homens, permanecer
assim para sempre, não perder a esperança e desistir, não importa quão
duras as coisas possam se tornar -- isso é a vida, e esse é seu
propósito. Eu acabei descobrindo isso. Essa ideia agora se tornou parte
da minha carne e do meu sangue (p. 20).
\end{quote}

O trecho é extremamente tocante e nos mostra, ainda uma vez, o talento
de Dostoiévski para extrair o sentido fundamental da vida a partir de
situações escatológicas -- a realidade efetiva e superior que jorra da
realidade embotada pelo cotidiano. Ainda assim, precisamos arremessar o
homem ridículo e Dostoiévski contra si mesmos, uma vez que não há uma
vida em si e para si que não esteja dialeticamente relacionada às
condições materiais e à história. Com Dostoiévski, seria bem possível
dizer que a transformação das instituições e das relações sociais não
esgota a cicatrização do espírito -- a revolução política, assim, seria
apenas o começo de uma revolução muito mais profunda e encarniçada, a
revolução do coração dos homens, uma vez que a humanidade liberta da
esfera da necessidade passaria a viver a exponencialização dos desejos
segundo novas (e imprevisíveis) possibilidades. Mas a discussão sobre a
moralidade como uma esfera autônoma, como se o sujeito ético pudesse
legislar sobre suas ações com coerência em meio a um contexto
contraditório e fraturado, aponta para um solipsismo que enfraquece a
crítica dostoievskiana ao atual estado de coisas e não desvela as
imbricações e implicações entre a parte e o todo, o indivíduo e a
sociedade. É~nesse sentido que Theodor Adorno (2013) traz à tona uma
consideração de Hegel que nos vale como uma negação determinada da
moralidade em Dostoiévski:

\begin{quote}
A frase de Hegel de que não existe realidade moral alguma não é um mero
momento de transição para a sua doutrina da eticidade objetiva. Nessa
frase transparece já o reconhecimento de que o moral não se conhece de
forma alguma a partir de si mesmo, de que a consciência não garante a
ação justa e de que o Eu que se retrai em direção a si mesmo para saber
aquilo que ele deveria ou não fazer se perde no irracional e na vaidade.
Hegel continua perseguindo um impulso do esclarecimento radical. À~vida
empírica, ele não opõe o bom como um princípio abstrato, como uma ideia
que se satisfaria a si mesma, mas ele o liga segundo seu conteúdo
próprio à produção de um todo justo -- àquilo que, na \emph{Crítica da
razão prática}, aparece com o nome de humanidade. Com isso, Hegel
transcende a separação burguesa entre o \emph{ethos} como uma
determinação que obriga incondicionalmente, mas que vale tão somente
para o sujeito, e a objetividade pretensamente empírica da sociedade (p.
127).
\end{quote}

A vinculação \emph{necessária} entre moralidade e sociedade, ética e
história irrompe, então, como uma implicação recíproca. E~se, por um
lado, o Dostoiévski que escapa da pena de morte após o perdão do tsar
enfatiza, de forma algo unilateral e escatológica, a dimensão da
internalidade como o bastião de tudo aquilo que é vívido e moral, também
é preciso arremessar Hegel e Adorno contra si mesmos, à luz das
tragédias históricas perpetradas pelos regimes tidos como socialistas,
por meio da noção de que a moral não se conhece de forma alguma a partir
de si mesma, isto é, de que não há um teor de verdade na integração a
partir da ideia de uma bondade que deva ser socialmente partilhada. Se a
moralidade em si e por si mesma padece de abstrações que são
neutralizadas pelo mal reproduzido cotidianamente em termos sociais, a
ética concebida meramente como ideologia das classes dominantes a serem
destronadas e como entrave anacrônico para a nova sociedade pode gerar
aquilo que Robert Louis"-Jackson (2002), em estreito diálogo com a obra
de Dostoiévski, chamou de ``os problemas dos crimes contra a humanidade
em nome da humanidade'' (p. 13). Assim, ao considerarmos Dostoiévski um
autor que se imbrica à (escato)lógica da história de modo a levar as
ideias às últimas consequências em termos de seus desdobramentos futuros
e \emph{possíveis}, não deixa de ser totalmente \emph{premonitória} a
seguinte consideração do homem ridículo à luz dos campos de concentração
sino"-soviéticos, dos campos de extermínio no Camboja e do \emph{paredón}
cubano:

\begin{quote}
Os beligerantes acreditavam firmemente e ao mesmo tempo que a ciência, a
sabedoria e o sentimento de autopreservação {[}sem a fonte da vida viva
que, para Dostoiévski, se nutria da eternidade{]} vão afinal obrigar o
homem a se unir numa sociedade cordata e racional, e assim, enquanto
isso, para apressar as coisas, os ``sábios'' esforçavam"-se o mais
depressa possível por exterminar todos os ``não sábios'' que não
entendiam a sua ideia, para que não interferissem no triunfo dela
(\versal{DOSTOIÉVSKI}, 2003, p. 119).
\end{quote}

É assim que o homem ridículo logo entrevê que o sentimento de
autopreservação vai se embotando, na medida em que os homens, seres
sociais por excelência, só se veem privados da felicidade em meio à
solidão compartilhada com os demais, na medida em que os homens não veem
qualquer sentido em uma vida fadada à morte e que apenas pode
recompensá"-los -- se é que o pode fazer -- nos breves anos de juventude.
O sofrimento se torna a beleza, ``já que só no sofrimento existe razão''
(\versal{IDEM}, p. 120). Eis, aqui, uma síntese para a obra de Dostoiévski se a
considerarmos apenas pelo prisma do niilismo, isto é, dos homens a viver
sob o crepúsculo de Deus. O~homem ridículo se apieda, ele se sente
responsável pela queda dos filhos do Sol, nosso herói logo quer ser
crucificado, a dor é enorme, ``mas eu não conseguia, não tinha forças
para me matar sozinho, (\ldots) {[}eu{]} queria tomar deles os suplícios,
estava sedento de suplícios, sedento de que nesses suplícios o meu
sangue fosse derramado até a última gota'' (\versal{IBIDEM}). Se a vaidade
mórbida não pode fazer com que Narciso -- uma das máscaras do homem
ridículo e do homem do subsolo e, a bem dizer, de grande parte das
personagens agônicas de Dostoiévski -- seja canonizado por sua beleza,
que o culpado pela queda dos homens se ofereça em holocausto e, ainda
uma vez, se torne o centro das atenções. Não deixa de ser sintomático,
nesse sentido, que o mesmo homem ridículo que descobre o egoísmo
historicamente reproduzido como um mal fundamental a ser superado seja o
herói que não consegue vivenciar as situações para além do egocentrismo.
Quando os filhos decaídos do Sol começaram a ameaçar o homem ridículo
com o exílio no hospício se ele não deixasse de tentar lembrá"-los sobre
o paraíso perdido -- paraíso que poderia ser dialeticamente recuperado
--, ``a dor entrou na minha alma com tanta força que o meu coração se
oprimiu e eu senti que estava prestes a morrer, e foi aí\ldots bem, foi aí
que eu acordei'' (\versal{IDEM}, p. 121).

O sonho propriamente dito termina, e quando o homem ridículo depara com
o revólver diante de si, nosso herói o repele com a sensação de que fora
novamente ungido pela vida.

\begin{quote}
Sim, a vida e -- a pregação! Naquele mesmo minuto decidi que iria
pregar, e é claro que pelo resto da minha vida! Eu vou pregar, eu quero
pregar -- o quê? A~verdade, pois eu vi, eu a vi com os meus próprios
olhos, eu vi toda a sua glória! (\ldots) Todos seguem em direção a uma
única e mesma coisa, pelo menos todos anseiam por uma única e mesma
coisa, do mais sábio ao último dos bandidos, só que por caminhos
diferentes. (\ldots) Eu vi a verdade, eu vi e sei que as pessoas podem ser
belas e felizes, sem perder a capacidade de viver na terra. Não quero e
não posso acreditar que o mal seja o estado normal dos homens (\versal{IDEM}, p.
121; p. 122).
\end{quote}

Gerald J. Sabo (2009) analisa as implicações do termo utilizado pelo
homem ridículo para se referir à verdade,
истина
(\emph{istina}):

\begin{quote}
Que \emph{istina}/verdade é um sinônimo para Cristo foi notado por Diane
Oenning Thompson em uma análise de \emph{Os demônios.} Aqui, o velho
Vierkhoviénski faz a distinção entre \emph{pravda} e \emph{istina}: ``Eu
menti a vida inteira. Mesmo quando eu falava a verdade
[\emph{pravdu}], eu nunca falava a verdade [\emph{istinu}], mas
somente para mim mesmo''. De acordo com Thompson, então, ``a verdade
própria a Stiepan, \emph{pravda}, que geralmente designa a verdade
terrena ou a justiça, agora se opõe à \emph{istina} (verdade), um termo
bíblico comum e um sinônimo para Cristo''\footnote{``Problems of the
  Biblical Word in Dostoevsky's Poetics''. In: \emph{Dostoevsky and the
  Christian Tradition}. Eds. George Pattion and Diane Oenning Thompson.
  Cambridge: Cambridge \versal{UP}, 2001, pp. 69-99; aqui, p. 78.} (p. 58).
\end{quote}

Em discussão com o ensaio de Dostoiévski \emph{Socialismo e
cristianismo}, Sabo nos lembra de que ``o cristianismo é o terceiro e
último estágio do ser humano, mas, aqui, o desenvolvimento termina, e o
ideal é alcançado, {[}já que há{]} uma \emph{vida futura}
{[}\emph{buduschaia jizn'}{]}'' (\versal{IDEM}, p. 52). Poderíamos dizer que,
para Dostoiévski, cicatrização do espírito e eternidade se pressupõe
reciprocamente. Assim, se torna compreensível a colocação de Robert
Louis"-Jackson (1981), quando o autor diz que

\begin{quote}
Deus criou o homem à Sua própria imagem. Mas essa imagem se torna
obscura e até mesmo desfigurada. {[}Trata"-se da queda que faz com que o
homem adquira sua própria liberdade para, dialeticamente, voltar ao
sentido divino em um novo patamar, liberdade que, como vimos
argumentando, pode pressupor as maiores transgressões.{]} Ela nunca se
perde completamente; no entanto, a imagem precisa ser redescoberta,
``restaurada'' -- em termos teológicos, redimida -- em toda a sua pureza
original. As~preocupações fundamentais de Dostoiévski em sua arte estão
sempre relacionadas a essa tarefa de restauração (p. 18).
\end{quote}

Os niilistas e resignados que leem a obra de Dostoiévski têm muita
dificuldade em lidar com a colocação do homem ridículo de que ``não
quero e não posso acreditar que o mal seja o estado normal dos homens''
(\versal{DOSTOIÉVSKI}, 2003, p. 122). Mas, então, como se daria a trajetória de
\emph{restauração}, ou, por outra, \emph{cicatrização,} não só do homem
ridículo, mas também de seus semelhantes? Quando nosso herói diz que
``eu vou seguir, vou seguir, ainda por mais mil anos!'' (\versal{IBIDEM}), não
devemos associar a noção do milênio a uma mera hipérbole, mas à noção
que aproxima Dostoiévski de Kardec: a tentativa de disseminar a verdade
para além do espírito que se curou pressupõe, \emph{necessariamente}, a
eternidade. ``E, quanto àquela menininha, eu a encontrei\ldots E~vou
prosseguir! E~vou prosseguir!'' (\versal{IDEM}, p. 123) Seria factível que o
homem ridículo a tivesse (re)encontrado nesta vida, após o sonho, ou
então que a eternidade tivesse possibilitado o (re)encontro sob novas
formas. Resta saber, no entanto, como disseminar a verdade. ``Mas como
instaurar o paraíso -- isso eu não sei, porque não sou capaz de
transmitir isso em palavras. (\ldots) Ah, como é duro conhecer sozinho a
verdade!'' (\versal{IDEM}, p. 123; p. 91) Neste momento, nos damos conta do
caráter parcial e não"-reconciliado da verdade descoberta pelo homem
ridículo. Toda a trajetória de conversão e cicatrização fez com que
nosso herói saísse de sua (monado)lógica e buscasse a reintegração
dialeticamente outra com os demais. Mas, no momento em que tenta
retornar à convivência que ele antes só fazia repelir, a solidão se
ressignifica, o homem ridículo não consegue fazer suas palavras ecoarem.
Não se trata apenas de uma aporia subjetiva a lhe solapar a pregação.
Trata"-se da profunda falta de ressonância de seu teor de verdade em meio
às ações e reações que embasam o convívio social. Não se ignora
impunemente a dialética entre decisão moral e reverberação
pragmático"-social, parte e todo, indivíduo e sociedade.

A respeito da síntese dostoievskiana, Robert Louis"-Jackson (2002) retoma
uma citação do escritor em um de seus cadernos de anotações, na qual
Dostoiévski afirma que a lei de Cristo, como o ideal para a humanidade,
consiste em uma ``volta para um senso de imediatez, para a massa das
pessoas, mas -- algo livre, e não por meio da vontade, não por meio da
razão, mas por meio da consciência, por meio de um sentimento direto,
terrivelmente forte e incontornável de que \emph{isso} é muito
\emph{bom}'' (pp. 22-23). Tal colocação parece estar totalmente de
acordo com o que diz o próprio homem ridículo ao fim de sua narrativa:

\begin{quote}
Num dia qualquer, \emph{numa hora qualquer --} tudo se acertaria de uma
vez só! O~principal é -- ame aos outros como a si mesmo, eis o
principal, só isso, não é preciso nem mais nem menos: imediatamente você
vai descobrir o modo de se acertar. E~no entanto isso é só -- uma velha
verdade, repetida e lida um bilhão de vezes, e mesmo assim ela não
pegou! ``A consciência da vida é superior à vida, o conhecimento das
leis da felicidade -- é superior à felicidade'' -- é contra isso que é
preciso lutar! E~é o que eu vou fazer. Basta que todos queiram, e tudo
se acerta agora mesmo (\versal{DOSTOIÉVSKI}, 2003, p. 123).
\end{quote}

Em um dos trechos mais problemáticos da obra de Dostoiévski, o homem
ridículo atribui à descoberta espiritual e filosoficamente
\emph{idealista} a transformação e a reconciliação imediatas das
contradições encarniçadas da realidade, como se um \emph{deus ex
machina} pudesse saltar por sobre a condição humana historicamente
configurada. Assim, o teor de verdade da descoberta onírico"-espiritual
-- o ímpeto pelo amor recíproco que já fora prenunciado pelo Sermão da
Montanha proferido por Jesus Cristo -- se vê emparedado pelo
aprofundamento radical da reificação hedonista em nossa sociedade. O
fato de a velha verdade, repetida e esvaziada um bilhão de vezes nas
missas, cultos e celebrações das igrejas, sinagogas, mesquitas,
terreiros, templos e centros espíritas não ter aderido ao real é
suficientemente tenso e contraditório para nos fazer perguntar, ainda
uma vez, por que a emancipação humana ainda não adveio. Assim, há um
problema profundíssimo na noção de que, \emph{numa hora qualquer,
bastando que todos queiram, tudo se acerta agora mesmo.} Já Hegel
dissera, segundo Theodor Adorno (2013), que ``não há nada entre o céu e
a terra que não seja `mediado', nada que, por conseguinte, mesmo
determinado como existência simples, não contenha reflexão, um momento
espiritual: `a imediatez é ela própria essencialmente mediada'" (p.
138). O~problema se aprofunda ainda mais diante da impossibilidade de
imediatez da vida em uma sociedade em que as estruturas de administração
e opressão já se fazem internalizadas a ponto de se confundirem com as
próprias categorias da consciência. Hegel (2008) concorda com o teor de
verdade do Sermão da Montanha, mas, para o filósofo, a imediatez
insciente (a natureza) precisa ser dialeticamente superada pela
\emph{mediação} da relação com Cristo:

\begin{quote}
A essência do princípio cristão que já foi abordada anteriormente é o
princípio da mediação. O~homem só se torna realmente um ser espiritual
quando supera a sua naturalidade. Essa superação só é possível a partir
do pressuposto de que as naturezas humana e divina formam em si e por si
uma unidade, e que o homem, sendo espírito, contém a essencialidade e a
substancialidade que pertencem ao conceito de Deus. A~mediação depende
da consciência dessa unidade, e a intuição dessa unidade foi dada ao
homem \emph{in Christo.} Portanto, o essencial é que o homem alcance
essa consciência e que ela seja constantemente despertada nele (p. 317).
\end{quote}

O homem ridículo nos fala, justamente, sobre o despertar constante dessa
consciência \emph{in Christo.} Entretanto, tanto em diálogo com
Dostoiévski quanto em diálogo com Hegel, já havíamos descoberto como a
tese da unidade do homem com Deus pressupõe a antítese da desunião,
antítese mediada pela liberdade própria à queda, para que a síntese do
reencontro entre o homem e Deus se dê pela ascensão da criatura ao
Criador por meio de sua cicatrização. Ocorre que tal processo,
explosivamente contraditório e a pressupor a eternidade, se vê entravado
por uma série de barreiras objetivas. Ora, a noção de que basta a todos
querer se faz profundamente polifônica em sociedades de classe tão
desiguais como a russa, a brasileira e a norte"-americana, por exemplo. O
oferecimento da outra face se vê emparedado pela necessidade de o eu
\emph{ter que} competir, cotidianamente, por sua sobrevivência, ainda
que, subjetivamente, a deslealdade possa incomodá"-lo sobremaneira. É
assim que a sociedade socializada (re)produz a esquizofrenia como
polifonia identitária objetiva: uma pessoa religiosa pode amar a
humanidade como um todo, de forma abstrata, durante a celebração
espiritual. Ela pode se sentir ungida pelo \emph{religare} cósmico.
Ainda assim, à saída de seu \emph{locus} sagrado, se um mendigo a
interpelar por uma ajuda substantiva que ultrapasse a indiferença da
esmola, o homem ridículo \emph{terá que} sentir a impotência da
compaixão contingente diante do atual estado de coisas.

Nesse momento, entrevemos as fagulhas do entrechoque do Dostoiévski
idealista com o escritor dialético. É~assim que Hegel (2008), para quem
a história e a eternidade são os terrenos por excelência da cicatrização
do espírito, aponta para o potencial de transformação moral em completa
imbricação com a necessidade de transformação social que há nos
evangelhos, ao mencionar duas passagens de Cristo. A~primeira narra os
dissabores de um jovem rico:

\begin{quote}
Um jovem aproximou"-se de Jesus e lhe perguntou: ``Mestre, que devo fazer
de bom para ter a vida eterna?'' Disse"-lhe Jesus: (\ldots) ``Se queres
entrar na vida, observa os mandamentos''. -- ``Quais?'', perguntou ele.
Jesus respondeu: ``Não matarás, não cometerás adultério, não furtarás,
não dirás falso testemunho, honra teu pai e tua mãe, amarás teu próximo
como a ti mesmo''. Disse"-lhe o jovem: ``Tenho observado tudo isto desde
a minha infância. Que me falta ainda?'' Respondeu Jesus: ``Se queres ser
perfeito, vai, vende teus bens, dá"-os aos pobres e terás um tesouro no
céu. Depois, vem e segue"-me!'' Ouvindo estas palavras, o jovem foi
embora muito triste, porque possuía muitos bens. Jesus disse então aos
seus discípulos: ``Em verdade vos declaro: é difícil para um rico entrar
no Reino dos céus! Eu vos repito: é mais fácil um camelo passar pelo
fundo de uma agulha do que um rico entrar no Reino de Deus'' (\versal{MATEUS},
19, 16-24).
\end{quote}

O potencial e os limites da espiritualidade do homem ridículo estão
contidos nessa passagem. À~exceção do mandamento radical da caridade --
caridade em sua elevação para muito além do materialismo contumaz e da
esmola --, o jovem rico era moralmente probo. Ocorre que, como Cristo
desvela com muita argúcia, a probidade do jovem abastado se assentava
sobre a exploração dos mais pobres. Assim, Cristo é bastante efetivo: o
amor ao próximo como a si mesmo -- o mandamento mais universal de todos,
do qual tudo deriva -- \emph{precisa} revolucionar a sociedade, do
contrário sempre haverá impedimentos objetivos para que mais pessoas
possam exercer a bondade.

Suponhamos, no entanto, que o jovem rico tivesse legado toda a sua
fortuna aos pobres para seguir Cristo. Tal ato, sem dúvida, se tornaria
um exemplo sumamente formidável, mas, ainda assim, a \emph{estrutura
social reificada} não seria abalada pela ação contingente do rapaz. E
mais: se o jovem assim o fizesse, o que aconteceria com seus filhos?
Seria preciso que houvesse uma profunda reciprocidade entre o todo
social e os indivíduos, de modo que as transformações se
retroalimentassem. Por outro lado, Hegel (2008) faz um comentário
sobremaneira arguto que arremessa a transformação social contra si mesma
para que ela não deixe de se assentar sobre o teor de verdade do homem
ridículo: ``Caso isso tivesse sido seguido {[}caso o jovem rico tivesse
legado sua fortuna aos pobres{]}, teria havido uma inversão: os pobres
teriam se tornado ricos'' (p. 277). Tal comentário de Hegel, além de se
imbricar às possibilidades objetivas do devir, sintetiza,
dostoievskianamente, uma profunda contradição em que se enredariam
muitos movimentos revolucionários vitoriosos: a vanguarda do
proletariado destrona a classe dominante de outrora e, então, passa a se
encastelar no poder com práticas de concentração de riquezas, injustiça
e desigualdade que fariam inveja aos antigos dominantes, caso a maioria
deles não tivesse sido fuzilada. Hegel e Dostoiévski entreveem que não
basta revolucionar o todo sem novos marcos morais -- marcos morais que,
dialeticamente, precisam embasar a e surgir da revolução do todo. Para
que a revolução não configure uma volta (conservadora) sobre si mesma,
Cristo faz o teor de verdade do Sermão da Montanha e da luta de classes
colidirem, de modo que ambos se irmanem para forjar um novo tempo da
história.

A segunda passagem do evangelho que Hegel menciona diz respeito às
instruções que Cristo dá a seus apóstolos. ``Não julgueis que vim trazer
a paz à terra. Vim trazer não a paz, mas a espada. Eu vim trazer a
divisão entre o filho e o pai, entre a filha e a mãe, entre a nora e a
sogra, e os inimigos do homem serão as pessoas de sua própria casa''
(\versal{MATEUS}, 10, 34-36). Hegel (2008) nota então que ``aqui reside uma
abstração de tudo que pertence à realidade, mesmo dos laços morais.
Pode"-se dizer que em nenhum lugar se falou de forma tão revolucionária
como nos evangelhos, pois tudo o que antes tinha validade passa a ser
algo indiferente, desprezível'' (p. 277). Se a esmagadora maioria das
instituições religiosas pensasse em levar a cabo tal instrução de
Cristo, as religiões perderiam seu caráter reificado e liberariam seu
teor de verdade para a transformação da realidade. A~família, forma
nuclear da sociedade, guarda um teor de verdade, na medida em que,
\emph{às vezes}, relações profundamente amorosas e verdadeiras se
estabelecem como antíteses em relação à reificação social. Ainda assim,
a família também configura a célula motriz para o aprendizado da
hierarquia, para a tensão entre pais e filhos, para a competição entre
os irmãos, para a preparação do egocentrismo que arremessa uma família
contra a outra. O~espiritismo entenderia a colocação de Cristo a
respeito da centelha de discórdia em meio à família justamente como a
necessidade de cicatrização, ao longo das reencarnações, para espíritos
que têm muitas afinidades e também muitos conflitos a serem resolvidos.
Mas, nas passagens em questão, Cristo mostra que todas as formas
ossificadas, conforme a consciência do homem evolui, apontam para a sua
própria diluição com vistas à integração da humanidade.

Agora que vimos as afinidades profundamente eletivas entre o Sermão da
Montanha e a estepe da história humana, resta"-nos saber como a
cicatrização do espírito pode assumir matizes coletivos e totais para
além das trajetórias cristãs e contingentes do Príncipe Míchkin e dos
irmãos Aliócha e Ivan Karamázov. O~homem ridículo pressupõe a pregação,
a conversão que se propaga. Mas, como já vimos, há estruturas e agentes
sociais encarniçadamente refratários a mudanças. Nesse sentido, quando
Cristo fala em desembainhar a espada -- ``Não julgueis que vim trazer
paz à terra. Vim trazer não a paz, mas a espada'' (\versal{MATEUS}, 10, 34) --, o
oferecimento pacífico da outra face se vê preterido, em momentos
históricos de transformação, pelo exercício da violência renovadora, de
modo que, para permanecer idêntico a si mesmo, o teor de verdade da
cicatrização do espírito precisa se revolucionar. Entretanto, quando a
violência perde o caráter revolucionário e sua teleologia passa a se
enredar a um fim em si mesmo, a banalidade do mal desafia a noção de
sentido e transforma a onisciência divina em cumplicidade, de modo que
já não consigamos entrever um sentido dialético de purgação em meio ao
sofrimento. Conseguimos entrever sentido na trajetória de cicatrização
do espírito própria ao homem ridículo, mas como seria possível apreender
a purgação espiritual em meio ao navio negreiro ou ao holocausto? Eis o
núcleo da argumentação de Adorno (2009) para tensionar a metafísica após
os campos de concentração nazistas -- tensão histórica que ora fazemos
se voltar também contra Dostoiévski:

\begin{quote}
Depois de Auschwitz, comete"-se uma injustiça contra as vítimas com toda
afirmação de \emph{positividade da existência} {[}isto é, Deus{]}; uma
afirmação (\ldots{}), com toda tentativa de \emph{arrancar de seu
destino um sentido qualquer} {[}isto é, Deus{]}, por mais exíguo que
seja, (\ldots{}) {[}condena{]} ao escárnio a construção de um sentido da
imanência que emane \emph{de uma transcendência positivamente
posicionada} {[}isto é, de Deus{]}. (\ldots{}) A~faculdade metafísica é
paralisada porque o que aconteceu {[}em Auschwitz, no navio negreiro, em
Hiroshima, em Nagasaki e no Gulag soviético{]} destruiu para o
pensamento metafísico especulativo {[}o pensamento sobre a
espiritualidade{]} a base de sua unificabilidade com a experiência
{[}isto é, com a história{]} (p. 299).
\end{quote}

Após tragédias históricas da proporção de Auschwitz, a (suposta) bondade
de Deus é exumada de sua sepultura construída no século \versal{XIX} para que a
divindade também seja asfixiada em uma câmara de gás. Ainda assim,
Dostoiévski poderia dizer a Adorno, como bem podemos depreender, que a
cisão definitiva da história em relação ao Absoluto, isto é, em relação
à eternidade, não apenas estanca a fonte da \emph{vida viva}, mas também
relega os homens e as mulheres ao caos do acaso, à
consciência"-para"-a"-morte -- e ao niilismo. Ocorre que o mesmo Adorno que
certa vez chegara a sentenciar a impossibilidade da poesia (e da)
metafísica após Auschwitz volta a insuflar dialética no ápice do horror
para tentar movimentar o sentido perfilado pelo extermínio industrial
dos campos de concentração:

\begin{quote}
No que diz respeito àqueles para os quais o desespero não é uma palavra
vã, é possível perguntar se não seria melhor que não houvesse
absolutamente nada além de algo. Mas tampouco é possível dar uma
resposta geral a essa questão. Para um homem em um campo de
concentração, se é que um outro homem que escapou na hora certa tem o
direito de julgar, seria melhor não ter nascido. Apesar disso, diante do
brilho de um olhar, sim, diante do cachorro que abana um pouco o rabo
porque alguém lhe deu para morder algo que ele logo esquece, desvanece o
ideal do nada (\versal{IDEM}, p. 315).
\end{quote}

A tentativa adorniana de reconstruir o sentido histórico em meio à suma
beleza de um momento -- algo como a tentativa de eternizar o gozo e a
felicidade em meio ao instante a ser ceifado pelo tempo, isto é, uma
releitura adorniana da máxima de Albert Camus de que é preciso imaginar
Sísifo feliz -- se vê emparedada tanto pela finitude quanto pela
premente impossibilidade de vivenciar o belo como algo sagrado e eterno
após a destruição e o horror a partir dos quais não advém sentido algum.

O homem ridículo vivenciou a dialética da cicatrização do espírito por
meio do sofrimento -- e da iminência da destruição de si. Mas e quanto
aos milhões de condenados da terra? Deus os teria esquecido? Estaríamos
diante do \emph{pesadelo dos homens ridículos?} Ora, Adorno já falara
que a esperança deve se dar sempre em nome dos desesperançados. Assim,
se o homem ridículo desvelou a eternidade, que sentido cicatrizante
poderia haver em tragédias rematadas?

Nesse momento, Kardec (2009) se insinua para que tentemos entrever se
haveria alguma dialética entre destruição e renascimento:

\begin{quote}
{[}A destruição é uma lei da Natureza?{]} É~preciso que tudo se destrua
para renascer e se regenerar, porque o que chamais destruição não é
senão uma transformação que tem por objetivo a renovação e o
melhoramento dos seres vivos. (\ldots{}) A~destruição é necessária para
a regeneração moral dos espíritos, que adquirem, a cada nova existência,
um novo grau de perfeição. É~preciso ver o fim para lhe apreciar os
resultados. Não os julgueis senão sob o vosso ponto de vista pessoal e a
chamais de flagelo por causa do prejuízo que vos ocasionam. Mas esses
transtornos são, frequentemente, necessários para fazer alcançar, mais
prontamente, uma ordem melhor de coisas e, em alguns casos, o que
exigiria séculos (p. 233; p. 235).
\end{quote}

Está claro que os sobreviventes de atrocidades e chacinas e, muitas
vezes, os filhos que lograram sobreviver ao extermínio dos pais teriam
muita dificuldade em abrir mão do \emph{ponto de vista pessoal} para
depreender sentidos de seus flagelos. Ainda assim, o homem ridículo e
Kardec pedem que \emph{o processo} seja visto como um movimento rumo à
reconciliação da totalidade, que a chaga mais própria seja cotejada com
o sentido de cicatrização que se confundiria com a história. A
destruição -- isto é, a morte --, pelo prisma da eternidade, não existe
como aniquilação, mas, conforme o homem ridículo e Kardec, constitui uma
transformação, o movimento da renovação. Se a lei do universo é a
perfectibilidade moral (e social), o ato de desencarnar estaria
relacionado ou à expiação de determinada(s) falta(s) cometida(s) nesta
vida ou em vidas passadas, ou então à noção de que, no plano em que se
encontra, o espírito já teria vivenciado todas as provações para a sua
cicatrização. Mas reparemos que Kardec menciona a questão de que
convulsões mais traumáticas teriam o papel de acelerar a renovação
espiritual que, sem os turbilhões da história, levaria séculos para se
consumar.

É bem verdade que Hegel já sentenciara que a coruja de Minerva só alça
voo ao entardecer, de modo que, dialeticamente, é o crepúsculo dos
processos que nos permite entrever os momentos de sentido a partir de
sua aurora. Ainda assim, duas questões fundamentais se impõem quando
tentamos cotejar a cicatrização do homem ridículo com as tragédias
coletivas da história: (i) por que acontecem expiações coletivas que,
muito potencialmente, imiscuem justos e injustos em meio à punição que
pressupõe a possibilidade de purgação?; (ii) como entender a renovação
da humanidade por meio da guerra que parece ceifar o ímpeto humano por
comunhão?

Kardec (2008) procura responder à questão sobre a expiação coletiva da
seguinte maneira:

\begin{quote}
O carrasco expia para com a sua vítima, seja achando"-se em sua presença
no espaço, seja vivendo em contato com ela numa ou várias existências
sucessivas, até a reparação de todo o mal cometido. Ocorre o mesmo
quando se trata de crimes cometidos solidariamente, por um certo número;
as expiações são solidárias, o que não aniquila a expiação simultânea
das faltas individuais.

\noindent Em todo homem há três caracteres: o do indivíduo, do ser em si mesmo; o
de membro de família, e, enfim, o de cidadão; sob cada uma dessas três
faces pode ser criminoso ou virtuoso, quer dizer, pode ser virtuoso como
pai de família, ao mesmo tempo em que criminoso como cidadão, e
reciprocamente; daí as situações especiais que lhe são dadas em suas
existências passadas.

\noindent Salvo exceção, pode"-se admitir como regra geral que todos aqueles que
têm uma tarefa comum reunidos numa existência já viveram juntos para
trabalharem pelo mesmo resultado e se acharão reunidos ainda no futuro,
até que tenham alcançado o objetivo, quer dizer, expiado o passado, ou
cumprido a missão aceita.

\noindent (\ldots{}) Compreendeis agora a justiça das provas que não resultam de
atos da vida presente, porque já vos foi dito que é a quitação de
dívidas do passado; por que não ocorreria o mesmo com as provas
coletivas? Dissestes que as infelicidades gerais atingem o inocente como
o culpado; mas sabeis que o inocente de hoje pode ter sido o culpado de
ontem? (\ldots{}) E, depois, como dissemos, há faltas do indivíduo e do
cidadão; a expiação de umas não livra da expiação das outras, porque é
necessário que toda dívida seja paga (\ldots{}). {[}As{]} faltas
coletivas (\ldots{}) são expiadas coletivamente pelos indivíduos que
para elas concorreram, os quais se reencontram para sofrerem juntos a
pena de talião, ou ter a ocasião de repararem o mal que fizeram
(\ldots{}). O~que é incompreensível, inconciliável com a justiça de
Deus, sem a preexistência da alma, se torna claro e lógico pelo
conhecimento dessa lei (pp. 152-153).
\end{quote}

Quando o homem ridículo deparou com o paraíso como o duplo da Terra,
nosso herói pôde acompanhar, em outro plano espiritual, uma série de
vidas pregressas da humanidade -- lembremo"-nos do sobrevoo vertiginoso
pelo tempo histórico -- e perceber as correlações entre a queda, o
ímpeto pela liberdade, o desvio egocêntrico e o ímpeto por
reconciliação. Sem a eternidade, já vimos que Dostoiévski e Kardec
concordam em relação ao niilismo mais contumaz. Kardec, entretanto, vai
além de Dostoiévski em relação às explicações sobre a dialética que
envolve destruição e renovação espiritual. Poderíamos considerar a lei
de Deus sumamente taliônica e draconiana, mas a própria noção de
\emph{evolução} pressupõe a renovação dos próprios parâmetros da
renovação, isto é, a transformação dos movimentos destrutivos em face da
cicatrização da humanidade, pois ``a necessidade de destruição se
enfraquece entre os homens, na medida em que o espírito se sobrepõe à
matéria, e é por isso que vedes o horror à destruição seguir o
desenvolvimento intelectual e moral''\footnote{Hegel (2008) falou em
  termos muito próximos aos de Kardec, quando afirmou que ``a forma de
  luta {[}para a cicatrização do espírito{]} se modifica bastante quando
  o fundamento é outro, e a reconciliação é verificada na realidade. O
  caminho do sofrimento ruiu (mais tarde ele ressurge, mas sob outra
  forma), pois a consciência despertou e encontra"-se no homem como
  elemento de uma condição moral. O~momento da negação é certamente
  necessário ao homem, mas conserva agora a forma calma da educação, e
  com isso desaparece todo o pavor da luta interior'' (p. 337).}
(\versal{KARDEC}, 2009, p. 234).

Os leitores de Dostoiévski e todos aqueles que conhecemos as atrocidades
do século \versal{XX} -- atrocidades cometidas a partir das mesmas premissas de
desenvolvimento técnico que, em germe, poderiam emancipar a humanidade
da inanição e do trabalho reificado -- já sabemos que não é possível
interpretar a evolução moral a que o homem ridículo, Kardec e Hegel
fazem referência como um sentido historicamente linear. As
recalcitrâncias e regressões se imbricam ao processo histórico, as
resistências dos poderosos em relação à igualdade e à verdadeira
democracia entravam o processo de construção de uma nova humanidade. É
assim que desponta a segunda pergunta a que fizemos menção mais acima:
há alguma possibilidade de, a partir da guerra, sentidos ossificados
serem renovados para que cheguemos a novos patamares de reconciliação?

Susan McReynolds (2002) cita um Dostoiévski belicoso, a partir das
páginas do \emph{Diário de um escritor}, a discorrer sobre o egoísmo que
se recrudesce com a paz:

\begin{quote}
Uma paz prolongada sempre dá vazâo á crueldade, à covardia, ao egoísmo
grosseiro e inflado e, acima de tudo, à estagnação intelectual. A~paz
gera uma terrível ânsia pela aquisição e acumulação de dinheiro, cada um
se isola e se aparta, e o resultado é que uma paz prolongada e burguesa
quase sempre gera um desejo pela guerra (p. 98).
\end{quote}

Nas páginas do \emph{Diário de um escritor}, em 1876, Dostoiévski
publicou um diálogo intitulado \emph{Um paradoxalista}\footnote{\emph{A
  Writer's Diary, Volume 1, 1873-1876}. Evanston, Illinois: Northwestern
  University Press, 1994, pp. 452-457.}, no qual um dos interlocutores,
justamente o paradoxalista, procura demonstrar que o egoísmo,
dialeticamente, seria tensionado com a agregação que a guerra propicia:

\begin{quote}
Afirmo categoricamente que um longo período de paz endurece o coração do
povo. Durante o interminável período de paz, a balança social sempre
pende para o lado do que é estúpido e grosseiro na humanidade,
principalmente em direção à riqueza e ao capital. Imediatamente após uma
guerra, a honra, a filantropia e o autossacrifício são respeitados,
valorizados e altamente resguardados; quanto mais a paz se estende, mais
tais valores nobres e belos se tornam pálidos e insípidos, até que
desapareçam, enquanto todos e cada um estão obcecados pela riqueza e
pelo espírito da aquisição. (\ldots{}) Uma paz prolongada produz apatia,
ideais medíocres, depravação e um arrefecimento das paixões. Os~prazeres
não se mostram refinados e se tornam mais e mais envilecidos. (\ldots{})
A guerra obriga as nações ao respeito mútuo. A~guerra renova os povos. O~amor
ao próximo chega ao ápice no campo de batalha. (\ldots{}) A~guerra
aumenta o moral do povo e o seu senso de amor"-próprio e dignidade. A~guerra
torna todos e cada um iguais durante as batalhas e reconcilia o
senhor e o escravo na manifestação mais sublime de dignidade humana -- o
sacrifício da vida pela causa comum, por todos e cada um, pela pátria
(1994; p. 453; p. 454; p. 455).
\end{quote}

Quando o paradoxalista diz que o amor ao próximo chega ao ápice no campo
de batalha, nos lembramos do homem ridículo a readquirir vínculos
humanos diante da menininha encharcada e indefesa a suplicar por sua
mãe. Apenas a face tangível da dor tende a romper as muralhas do ego. Os
indivíduos ultracompetitivos e belicosos se unem não em um todo
emancipado, mas sob as fardas dos batalhões. Assim, ao vislumbrar o
caráter de coesão mutilada que a guerra propicia, Dostoiévski se
aproxima ainda uma vez de Allan Kardec (2009), quando o espírita afirma
que ``o egoísmo, longe de diminuir, aumenta com a civilização, que
parece excitá"-lo e entretê"-lo (\ldots{}). Quanto maior o mal, mais ele
se torna hediondo. Era preciso que o egoísmo fizesse muito mal para
fazer compreender a necessidade de extirpá"-lo'' (p. 283). Assim, com uma
verve efetivamente dostoievskiana, Kardec (2009) sintetiza a dialética
da cicatrização do espírito:

\begin{quote}
{[}A história nos mostra uma multidão de povos que, depois dos abalos
que os agitaram, caíram na barbárie; onde está, nesse caso, o progresso?
Onde está a evolução moral?{]}

\noindent Quando tua casa ameaça ruir, tu a derrubas para reconstruí"-la de maneira
mais sólida e mais cômoda; mas até que ela esteja reconstruída, há
perturbação e confusão em tua residência (p. 246).
\end{quote}

Eis as bases dostoievskianas para a cicatrização do espírito,
cicatrização que pressupõe a história e a eternidade em seu processo
dialético que envolve sofrimento e purgação, destruição e renovação,
tensão e reconciliação, nostalgia e futuro, Éden e utopia. Susan
McReynolds (2009) cita uma colocação de Liev Tolstói sobre Dostoiévski,
colocação que, em termos efetivamente dialéticos, isto é, a contrapelo
de si mesma, sintetiza a contradição objetiva que confere atualidade à
obra agônica (e potencialmente redentora) do autor de ``O sonho de um
homem ridículo'':

\begin{quote}
Vocês exageraram sua importância e o fizeram da maneira usual, elevando
ao patamar de profeta e santo um homem que morre no limite inflamado de
sua luta interna entre o bem e o mal. O~escritor nos toca e nos
interessa, mas é impossível colocar em um pedestal um homem inteiramente
composto de luta para edificar a posteridade (p. 20).
\end{quote}

Enquanto a história humana for sumamente dostoievskiana, vale dizer,
enquanto a mão que fere não se converter na mão que afaga, a luta
intestina entre a fratura e a reconciliação não apenas continuará a
fazer de Dostoiévski um escritor atual, mas também prosseguirá com a
escavação (e a erosão) do subsolo para que Pandora desça do cárcere de
seu pedestal.

\chapter*{Considerações finais}

\addcontentsline{toc}{chapter}{\\Considerações Finais}

\hedramarkboth{Considerações finais}{}

Ao fim de nossa leitura dialética das obras de Dostoiévski, façamos uma
recapitulação dos principais pontos desenvolvidos por este estudo, de
modo a que seja possível elencar as conclusões e repropor alguns debates
sobre aspectos que julgamos fundamentais em meio ao legado do escritor
russo.

A primeira parte desta pesquisa (tese), ``Dostoiévski e o fetichismo da
mercadoria'', procurou realizar uma síntese dialética de questões que,
em Mikhail Bakhtin, não alcançaram o âmbito dialético da totalidade. O
caminho através das aporias dialógico"-polifônicas nos revelou que a
incompletude do projeto bakhtiniano se relaciona tanto ao autoritarismo
do contexto estalinista quanto à não"-realização -- isto é, à
impossibilidade de realização -- de uma crítica efetivamente dialética à
polifonia de Dostoiévski.

Por um lado, a construção bakhtiniana da catedral polifônica como a
imagem"-síntese para o coro das vozes dostoievskianas -- imagem que
consideramos limitada e limitante em relação ao caráter efetivamente
contraditório do universo do escritor russo -- tem o potencial de
liberar uma diatribe velada de Bakhtin em relação à total falta de
dialogia e tolerância, equipolência e alteridade democrática por parte
do Estado soviético. (Não à toa, a primeira versão de \emph{Problemas da
poética de Dostoiévski}, publicada em 1929, só receberia uma segunda
versão crítica em 1963, dez anos após a morte de Stálin.) Por outro
lado, o horizonte epistemológico de Bakhtin -- horizonte necessariamente
apequenado e reificado pela ideologia oficial -- não permitiu que o
crítico acompanhasse, com a efetiva radicalidade que a (in)disciplina
das obras requer, o movimento da contradição em Dostoiévski, de modo a
que a dialogia, em suas fraturas e aporias, se tornasse um momento da
dialética polifônica do escritor russo. Como Bakhtin parte da ontologia
dialógica das personagens como se a equipolência das vozes fosse uma
realidade ficcional apriorística, o crítico não consegue acompanhar a
equi"-impotência das vozes em meio ao concerto polifônico reificado.
Conforme procuramos argumentar, não é possível hipostasiar a ontologia
dialógica como se a equipolência das vozes, a tolerância e a democracia
efetiva -- para muito além dos arremedos da ditadura do partido e da
democracia de mercado -- já fossem marcos sociais plenamente
constituídos. Se Bakhtin queria projetar o diálogo emancipado como o
ápice da utopia que desponta com forte latência em Dostoiévski, teria
sido necessário escavar o subsolo das obras para que, a partir das
densas camadas de carvão dos círculos infernais, fosse possível extrair
o diamante da liberdade. A~dialogia plena, isto é, o concerto polifônico
emancipado, configura"-se não como o ponto de partida, mas como o ponto
de chegada das estórias e da história. Assim, a revolução formal que
Bakhtin quis ensejar a partir de Dostoiévski não pode hipostasiar a
efetiva revolução político"-social.

Procuramos insuflar ar dialético através das estruturas aporéticas e
ossificadas da catedral erigida por Bakhtin, de modo a acompanhar o
\emph{movimento de ruína} da arquitetura dostoievskiana. À~catedral
sistêmica, cartesiana e integral soerguida a contrapelo da poética de
Dostoiévski, contrapusemos a trajetória dialética que caminha pelos
\emph{escombros} da obra do escritor. Assim, tensionamos a noção
apriorística de totalidade em Dostoiévski não para que o todo fosse
preterido de nosso horizonte de análise, mas para que o movimento da
contradição abalasse as fundações do subsolo e nos mostrasse que o
caráter aporético e fragmentário da totalidade desponta desde o caráter
irreconciliado das partes. Bakhtin não conseguiu demonstrar como a
liberdade relativa e determinada das vozes dialógicas pode compor um
todo polifônico sem que a totalidade erija, ao fim e ao cabo, uma
síntese monológica a amordaçar e a reificar a ontologia dialógica. Nesse
sentido, este livro procurou totalizar o projeto bakhtiniano na medida
em que, dialeticamente, a análise se propôs a superar (\emph{Aufheben})
o sistema cartesiano e integral de modo a desvelar como o movimento da
contradição enseja, em meio à obra de Dostoiévski, totalidades abertas e
processuais, ao longo das quais a irreconciliação sintetiza,
\emph{negativamente}, o prolongamento das fraturas e a impossibilidade,
ainda atual, de erigir novos abrigos a partir dos escombros da utopia.

Buscamos, então, estabelecer uma analogia entre a poética de Dostoiévski
e o movimento fetichista da mercadoria, tal como Marx o analisa nos
primeiros capítulos do volume 1 de \emph{O capital.} Ao fazermos um
\emph{close reading} de \emph{Memórias do subsolo} -- e, em grande
medida, ao rompermos as fronteiras das obras de Dostoiévski para
ouvirmos os diálogos das vozes dostoievskianas como duelos encarniçados
--, procuramos mostrar \emph{de que forma} a reificação da polifonia
ocorre como extração de mais"-valia poética que se reverte em menos"-valia
humana. A~analogia entre o subsolo dostoievskiano e o devir tautológico
do capital nos permitiu apreender o horizonte bakhtiniano como M -- D --
M -- vender para comprar, isto é, dialogar para comunicar --, momento
metabólico racional em meio à ciranda do capital. Se a humanidade como
um todo pudesse decidir sobre a produção e a distribuição de sua riqueza
social; se, em suma, a ontologia dialógica já se confundisse com a
história, a latência utópica de Dostoiévski já teria libertado as
memórias da masmorra do subsolo. Ocorre que, conforme argumentamos ao
lado do e contra o homem do subsolo, M -- D -- M -- vender para comprar,
dialogar para comunicar --, em termos histórico"-poéticos, configura
apenas um momento da ciranda tautológica do capital, isto é, D -- M --
D' -- comprar para vender, ou melhor, comprar para vender mais caro,
dialogar não para comunicar, mas para silenciar. Nossa análise buscou
desvelar a (tauto)lógica da polifonia como concerto reificado de vozes e
restabeleceu tanto a dialética como \emph{arquitetura formal} do subsolo
quanto Marx e a Teoria Crítica como interlocutores fundamentais de
Dostoiévski e Bakhtin.

Assim, tendo em vista os resultados da primeira parte deste livro -- a
\emph{tese} deste concerto dissonante e dialético --, torna"-se
importante, a meu ver, repropor e expandir os debates entre os estudos
bakhtinianos de Dostoiévski e a dialética de Marx e da Teoria Crítica. A
\emph{tese} deste livro, ``Dostoiévski e o fetichismo da forma
mercadoria'', centrou"-se em \emph{Memórias do subsolo} como o
\emph{locus} primordial a partir do qual a reificação e o ímpeto
encalacrado por reconciliação se irradiam em Dostoiévski. Nesse sentido,
estudos vindouros poderão verificar, com os devidos e respectivos
\emph{close readings}, \emph{de que forma (e conteúdo)} a extração de
mais"-valia poética revertida em menos"-valia humana poderia desvelar,
\emph{in toto}, a realidade ficcional de obras como \emph{Notas de
inverno sobre impressões de verão, Recordações da casa dos mortos, Crime
e castigo, O~idiota, Os~demônios} e \emph{Os irmãos Karamázov}.

A segunda parte desta pesquisa, ``O conteúdo em Dostoiévski como a
cicatrização do espírito rumo à utopia?'', conforme dissemos já no
preâmbulo deste trabalho, configura"-se como uma antítese em relação à
primeira parte. Se o fetichismo da forma em Dostoiévski apreende a
articulação das relações danificadas das personagens e entrevê o
movimento da totalidade poética como o devir de fraturas e contradições
que ainda não foram historicamente reconciliadas, o ímpeto de
cicatrização do espírito rumo à utopia em meio às obras do escritor
procura discorrer sobre uma temática que também não foi suficientemente
analisada pela fortuna crítica de Dostoiévski: o teor de verdade do
conteúdo de suas obras como a articulação de uma filosofia da história.

Argumentamos que socialismo e cristianismo constituem os polos de tal
filosofia da história em Dostoiévski. Em um primeiro momento, analisamos
os fatores que distanciam o escritor da \emph{intelligentsia}
revolucionária -- a geração que antecede os bolcheviques e Outubro de
1917 --, tendo em vista as aporias éticas que Dostoiévski pôde apreender
para a fundação de um novo tempo histórico que buscou fazer tábula rasa
de Deus e da eternidade da alma como salvaguardas para a construção de
uma eticidade objetiva. A~reificação das relações intersubjetivas que
veio à tona na primeira parte de nosso estudo também nos municiou para
acompanhar a possível reversão da utopia em distopia. Criticamos a
percepção de que, ao projetar uma série de características trágicas
daquilo que viria a ser o socialismo real -- sobretudo em sua vertente
soviética --, Dostoiévski seria um \emph{profeta.} Na verdade, ao se
imbricar profundamente com a logicidade de sua época, o escritor pôde
levar às últimas consequências os sentidos e os ressentimentos das
\emph{possibilidades latentes e objetivas da história. }

Vale frisar, ademais, que o filisteísmo -- em termos dostoievskianos, o
\emph{chigaliovismo} -- que o escritor atribui ao socialismo de
formigueiro, a emancipação reificada apenas para a cúpula dirigente, o
rebaixamento dos (e o ressentimento em relação aos) padrões artísticos e
intelectuais elevados como características aristocráticas e o caráter
altamente autoritário da produção e da distribuição da riqueza
socialmente produzida também se aplicam, \emph{mutatis mutandis}, ao
capitalismo tardio e às suas democracias de mercado. Assim, \emph{as
possibilidades latentes e objetivas da história} trouxeram a Dostoiévski
uma projeção das afinidades eletivas entre socialismo e capitalismo para
muito além do dogmatismo bem próprio à \emph{Realpolitik} da Guerra
Fria.

Se o socialismo é submetido pelo ex"-revolucionário e ex"-prisioneiro
político Fiódor Dostoiévski a uma negação determinada, o cristianismo
também tem seu sentido ``revertido'' pelo escritor russo, tendo em vista
as contradições encarniçadas e objetivas para que o Sermão da Montanha
desça à estepe de nossa história. Do estilhaçamento da consciência do
cristão Míchkin a se oferecer em holocausto para perdoar ao assassino
Rogójin e para afagar sua bem"-amada assassinada Nastácia Filíppovna ao
chauvinismo datado segundo o qual a Rússia tsarista e ortodoxa poderia
contrapor barricadas em relação ao espraiamento capitalista do Ocidente,
o oferecimento da outra face, em Dostoiévski, vai se tornando tão
recluso e contingente quanto as efetivas possibilidades de o monge
Aliócha Karamázov disseminar a boa nova para além dos muros e claustros
de seu monastério.

Quisemos demonstrar, dessa maneira, que a negação determinada tanto do
socialismo quanto do cristianismo por parte de Dostoiévski chega a um
novo patamar de síntese e superação (\emph{Aufhebung}) dialéticas por
meio do ateísmo espiritual de Ivan Karamázov e seu Sermão da Estepe. A
partir das três tentações a que Cristo é submetido no deserto, o grande
inquisidor, personagem"-mor do \emph{poema} de Ivan, articula o núcleo
das grandes contradições que a natureza humana historicamente
configurada ainda não conseguiu dirimir. Socialismo e cristianismo,
conforme argumentamos, deixam de ser polos antagônicos para se
imbricarem como momentos \emph{necessários e recíprocos} da teleologia
teológica da história que tem o potencial de cicatrização do espírito de
época. O~teor de verdade de igualdade e justiça do socialismo é
rearticulado com o princípio cristão de eternidade da alma -- segundo
Dostoiévski, a base substancial para a constituição da eticidade
objetiva para além do niilismo que só faz sentenciar que \emph{se Deus
não existe, tudo é permitido} --, de modo a que, do cume ao qual Cristo
é alçado pelo demônio em meio à terceira e última tentação, seja
possível vislumbrar a filosofia da história como o ímpeto de
reconciliação da utopia.

Tal transcurso global nos levou, como arremate deste livro, a analisar
as potencialidades e aporias da utopia como cicatrização do espírito em
diálogo com ``O sonho de um homem ridículo''\emph{.} E, para acompanhar
a trajetória da personagem que vai da iminência do suicídio à verdade
que a redime, Georg Wilhelm Friedrich Hegel e Allan Kardec se mostraram
fundamentais.

Hegel nos municiou com o conceito de ``cicatrização histórica do
espírito'', por meio do qual as relações danificadas passam por um
processo de purgação, através do sofrimento e das rupturas que vivificam
a consciência, para que a possibilidade de reconciliação venha à tona.
Qual um espírito objetivo a pairar sobre a, a transitar pela e a
articular a história, o ímpeto de cicatrização e cura se apresenta ao
homem ridículo no momento em que, à beira do suicídio, a personagem
restabelece um vínculo humano ao sentir compaixão, contra suas próprias
premissas niilistas, por uma criancinha desesperada que, em uma
tenebrosa noite de inverno petersburguês, clama por sua mãe
possivelmente moribunda. A~agonia do niilista ao descobrir que nem tudo
lhe é indiferente e que a negatividade do sofrimento o religa à verdade
da compaixão desvela o sentido da reconciliação entre os escombros. É~esse
o mote que faz com que o homem ridículo, em meio ao sonho
escatológico, percorra o devir da história humana como a reedição da
queda mitológica que teria expulsado os primeiros seres do Éden. É~o
quase"-suicídio reconciliado pela compaixão que faz com que a queda
histórica o leve ao ímpeto de ascensão, assim como, dialeticamente, a
mão que fere é a mesma mão que pode curar.

No momento em que a personagem descobre a vida para além da morte, o
espiritismo desenvolvido por Allan Kardec configura uma mediação
essencial para entendermos as muitas casas e encarnações do universo
como as moradas de cicatrização que a razão transcendental nos
forneceria em meio ao processo de existência como cura e evolução. Como
procuramos argumentar, a aproximação entre Dostoiévski e Kardec, além de
desvelar os veios de profunda contiguidade entre ambos os autores no que
diz respeito à eternidade e à multiplicidade da(s) vida(s), tem o
sentido de estabelecer um diálogo outro sobre a espiritualidade do
escritor russo para além das tradicionais interpretações de determinadas
tendências da fortuna crítica que entreveem a liturgia ortodoxa como o
espectro unívoco de sua obra.

Ainda assim, quando voltamos a sobrevoar a história a partir da
reconciliação vivenciada (\emph{oniricamente}) pelo homem ridículo, o
caráter tenso e aporético da utopia histórico"-transcendental também vem
à tona, uma vez que, da cicatrização individual para a reconciliação
social e coletiva, a dialética histórica ainda teria que dar vazão a
muito sofrimento e a muito belicismo para que a (auto)consciência de
época pudesse superar (\emph{aufheben}) a si mesma e fundar novas
relações entre as pessoas como a práxis do cotidiano.

Assim, este livro, que intitulamos \emph{Dostoiévski e a dialética:
fetichismo da forma, utopia como conteúdo,} busca trazer à tona tomadas
de posições antitéticas não apenas ao se submeter à (in)disciplina
dialeticamente contraditória das obras e reflexões dostoievskianas, mas
também ao repropor debates e temáticas que se contradizem entre si e que
requerem que tendências diversas e antípodas da fortuna crítica de
Dostoiévski passem a analisar a obra do escritor de forma efetivamente
relacional e dialógica, isto é, a contrapelo de suas próprias premissas
poético"-ideológicas. Ora, uma obra dialeticamente polifônica como a de
Dostoiévski requer que as tomadas de posições de Bakhtin e Adorno,
Kardec e Hegel, Cristo e Marx duelem entre si para que o princípio
dialógico de \emph{concórdia} se transforme na \emph{síntese histórica}
que \emph{eleva} o teor de verdade das mais contrapostas teses e
antíteses como a latência da ontologia polifônica que, desde já, nos
leve a imaginar Sísifo livre.

\chapter*{Referências Bibliográficas}
\addcontentsline{toc}{chapter}{\\Referências Bibliográficas}

\hedramarkboth{Referências Bibliográficas}{}


\begin{Parskip}
\versal{ABDULMASSIH}, Fabio Brazolin. \emph{Aulas de literatura russa -- \versal{F}.\versal{M}.
Dostoiévski por V. Nabokov: por que tirar Dostoiévski do pedestal?}
Dissertação de mestrado defendida junto ao Departamento de Letras
Orientais da \versal{FFLCH}-\versal{USP} sob a orientação do Prof. Dr. Homero Andrade de
Freitas.

\versal{ADORNO}, Theodor e \versal{HORKHEIMER}, Max. \emph{Dialética do Esclarecimento}.
Tradução de Guido Antônio de Almeida. Rio de Janeiro: Jorge Zahar
Editor, 1985.

\versal{ADORNO}, Theodor. \emph{Dialética negativa.} Tradução de Marco Antônio
Casanova. Rio de Janeiro: Jorge Zahar Editor, 2009.

\_\_\_\_\_\_\_\_ . ``Idéias para a sociologia da música''. Tradução de
Roberto Schwarz. In: \emph{Os Pensadores}. São Paulo: Abril Cultural,
1980a.

\_\_\_\_\_\_\_\_ . ``Lírica e sociedade''. Tradução de Wolfgang Leo
Maar. In: \emph{Os Pensadores}. São Paulo: Abril Cultural, 1980b.

\_\_\_\_\_\_\_\_ . ``O artista como representante''. Tradução de Jorge
de Almeida. In: \emph{Notas de Literatura \versal{I}}. São Paulo: Editora 34,
2003.

\_\_\_\_\_\_\_\_ . ``O fetichismo na música e a regressão da
audição''\emph{.} Tradução de Rubens Rodrigues Torres Filho. In:
\emph{Os Pensadores.} São Paulo: Abril Cultural, 1980c.

\_\_\_\_\_\_\_\_ . ``Posição do narrador no romance contemporâneo''.
Tradução de Modesto Carone. In: \emph{Os Pensadores}. São Paulo: Abril
Cultural, 1980d.

\_\_\_\_\_\_\_\_ . \emph{Teoria Estética.} Tradução de Artur Morão.
Lisboa: Edições 70, 2012.

\_\_\_\_\_\_\_\_ . \emph{Três estudos sobre Hegel.} Tradução de Ulisses
Vazzante Vaccari. São Paulo: Editora Unesp, 2013.

\versal{ALMEIDA}, Jorge Mattos Brito de. \emph{Crítica dialética em Theodor
Adorno: música e verdade nos anos vinte.} Cotia, \versal{SP}: Ateliê Editorial,
2007.

\_\_\_\_\_\_\_\_ . ``Pressupostos, salvo engano, dos pressupostos, salvo
engano''. In: \emph{Um crítico na periferia do capitalismo: reflexões
sobre a obra de Roberto Schwarz.} Maria Elisa Cevasco e Milton Ohata
(Orgs.). São Paulo: Companhia das Letras, 2007, pp. 44-53.

\_\_\_\_\_\_\_\_ . \emph{Theodor W. Adorno: Indústria cultural e
sociedade.} São Paulo: Paz e Terra, 2002.

\versal{ANDERS}, Günther. \emph{Kafka: pró e contra.} Tradução de Modesto Carone.
São Paulo: Cosac \& Naify, 2007.

\versal{ANDERSEN}, Zsuzsanna Bjørn. ``The Concepts of Domination and
Powerlessness in \versal{F}. \versal{M}. Dostoevsky's `A Gentle Spirit'"\emph{.} In:
\emph{Dostoevsky Studies.} Volume 4. Tübingen: Attempto Verlag, pp.
53-60.

\versal{ARMSTRONG}, Karen. \emph{A grande transformação.} Tradução de Hildegard
Feist. São Paulo: Companhia das Letras, 2008.

\versal{ARSENTIEVA}, Natália. ``El sueño de un hombre ridículo: el viaje hacia la
verdad''\emph{.} In: \emph{Caderno de Literatura e Cultura Russa:
Dostoiévski}. São Paulo: Ateliê Editorial, 2008.

\versal{BAKHTIN}, Mikhail. \emph{Estética da criação verbal.} Tradução de Paulo
Bezerra. São Paulo: Martins Fontes, 1983.

\_\_\_\_\_\_\_\_ . \emph{Problemas da poética de Dostoiévski}. Tradução
de Paulo Bezerra. Rio de Janeiro: Forense Universitária, 2008.

\versal{BALZAC}, Honoré de. \emph{Eugênia Grandet.} Tradução de Antônio Pereira
da Costa. São Paulo: Abril Cultural, 1971.

\versal{BENJAMIN}, Walter. ``A obra de arte na época de suas técnicas de
reprodução''. Tradução de José Lino Grünnewald. In: \emph{Os
Pensadores}. São Paulo: Abril Cultural, 1980a.

\_\_\_\_\_\_\_\_ . ``Dostoevsky's \emph{The Idiot}''. Translated by
Michael Katz. In: \emph{Early Writings: 1910-1917.} Translated by Howard
Eiland and Others. Cambridge, Massachussetts: The Belknap Press of
Harvard University Press, 2011, pp. 275-280.

\_\_\_\_\_\_\_\_ . ``O narrador -- observações sobre a obra de Nikolai
Leskow''. Tradução de Modesto Carone. In: \emph{Os Pensadores}. São
Paulo: Abril Cultural, 1980b.

\versal{BERDIAEV}, Nikolai. \emph{L'esprit de Dostoïevski}. Paris: Éditions
Saint"-Michel, 1929

\versal{BERLIN}, Isaiah. \emph{Pensadores russos.} Tradução de Carlos Eugênio
Marcondes de Moura. São Paulo: Companhia das Letras, 1988.

\versal{BERMAN}, Marshall. \emph{The Politics of Authenticity: Radical
Individualism and the Emergence of Modern Society.} New York: Verso,
2014.

\_\_\_\_\_\_\_\_ . \emph{Tudo que é sólido desmancha no ar.} Tradução de
Carlos Felipe Moisés. São Paulo: Companhia das Letras, 1999.

\versal{BERNARDINI}, Aurora Fornoni. ``Dostoiévski: criação, poesia e
crítica''\emph{.} In: \emph{Caderno de Literatura e Cultura Russa:
Dostoiévski}. São Paulo: Ateliê Cultural, 2008.

\_\_\_\_\_\_\_\_ . ```O grande inquisidor' \emph{e Os irmãos
Karamázov}''\emph{.} In: Novinsky, \versal{A}. \versal{W}.; Carneiro, \versal{M}. \versal{L}. \versal{T}. (Orgs.).
\emph{Inquisição: ensaios sobre mentalidade, heresias e arte.} São
Paulo: \versal{USP}/Expressão e Cultura, 1992, pp. 682-691.

\_\_\_\_\_\_\_\_ . ``Questões de forma e modernidade em Gogol e
Dostoiévski''\emph{.} In: \versal{MUTRAN}, \versal{M}. \versal{H}. e \versal{CHIAMPI}, \versal{I}. (orgs), \emph{A
questão da modernidade}. Caderno 1, Departamento de Letras Modernas,
\versal{FFLCH}/\versal{USP}, 1993.

\versal{BEZERRA}, Paulo. ``Dostoiévski: \emph{Bóbok}, Tradução e análise do
conto''. São Paulo: Editora 34, 2009.

\_\_\_\_\_\_\_\_ . ``Mundos desdobrados, seres duplicados''. In:
\emph{Caderno de Literatura e Cultura Russa: Dostoiévski}. São Paulo:
Ateliê Cultural, 2008.

\emph{Bíblia Sagrada}. São Paulo: Editora Ave"-Maria, 1994.

\versal{BIANCHI}, Maria de Fátima. \emph{Os caminhos da razão e as tramas
secretas do coração: a representação da realidade em} A dócil\emph{, de
Dostoiévski.} Dissertação de mestrado. São Paulo: \versal{FFLCH}-\versal{USP}, 2001.

\_\_\_\_\_\_\_\_ . ``O domínio do homem pelo homem na novela \emph{A
senhoria}, de Dostoiévski''\emph{.} In: \emph{Revista de Literatura e
Cultura Russa.} Volume 2, número 2. São Paulo: 2013, pp. 35-54.

\_\_\_\_\_\_\_\_ . \emph{O ``sonhador'' de} A Senhoria\emph{, de
Dostoiévski.} Tese de doutorado. São Paulo: \versal{FFLCH}-\versal{USP}, 2006.

\versal{CAMUS}, Albert. \emph{Le mythe de Sisyphe.} Paris: Éditions Gallimard,
2006.

\versal{CICOVACKI}, Predrag. ``The Enigmatic Conclusion of Dostoevsky's
\emph{Idiot}: A Comparison of Prince Myshkin and Wagner's Parsifal''.
In: \emph{Dostoevsky Studies.} Volume 9. Tübingen: Attempto Verlag,
2005, pp. 106-114.

\_\_\_\_\_\_\_\_ . ``The Role of Goethe's \emph{Faust} in Dostoevsky's
Opus''\emph{.} In: \emph{Dostoevsky Studies.} Volume 14. Tübingen:
Attempto Verlag, 2010, pp. 153-166.

\versal{CLARK}, Katerina e \versal{HOLQUIST}, \versal{MICHAEL}. \emph{Mikhail Bakhtin}. Tradução de
J. Guinsburg. São Paulo: Perspectiva, 1998.

\versal{DOSTOIÉVSKI}, Fiódor Mikháilovitch. ``A dócil -- narrativa fantástica''.
Tradução de Fátima Bianchi. São Paulo: Editora 34, São Paulo, 2003.

\_\_\_\_\_\_\_\_ . \emph{A Writer's Diary, Volume One, 1873-1876.}
Translated by Michael Katz. Evanston: Northwestern University Press,
1994.

\_\_\_\_\_\_\_\_ . \_\_\_\_\_\_\_\_ , \emph{Volume Two, 1877-1881.}
Translated by Michael Katz. Evanston: Northwestern University Press,
Evanston, 1994.

\_\_\_\_\_\_\_\_ . \_\_\_\_\_\_\_\_ . In: \emph{Fiódor Dostoiévski --
obra completa}. Rio de Janeiro: Editora Nova Aguilar, 2004.

\_\_\_\_\_\_\_\_ . \emph{B'esy: roman v triokh tchastiakh.} Berlin:
Izdatel'stvo \versal{L}.\versal{P}. Ladyjnikova, 1921.

\_\_\_\_\_\_\_\_ . \emph{Brat'ia Kamazovy: roman v tchetyriokh
tchastiakh s epilogom.} Moskva: Sovremennnik, 1981.

\_\_\_\_\_\_\_\_ . \emph{Crime e castigo.} Tradução de Vera Pereira. São
Paulo: Editora Nova Cultural, 2002.

\_\_\_\_\_\_\_\_ . \_\_\_\_\_\_\_\_ . Tradução de Paulo Bezerra. São
Paulo: Editora 34, 2001.

\_\_\_\_\_\_\_\_ . \emph{Correspondências: 1838-1880.} Tradução de
Robertson Frizero. Porto Alegre: 8Inverso, 2011.

\_\_\_\_\_\_\_\_ . \emph{Idiot: roman v tchetyriokh tchastiakh.} Moskva:
Sovestskaia Rossiia, 1981.

\_\_\_\_\_\_\_\_ . \emph{Memórias do subsolo}, Tradução de Boris
Schnaiderman. São Paulo: Editora 34, 2000a/2004.

\_\_\_\_\_\_\_\_ . \emph{Notas de inverno sobre impressões de verão},
Tradução de Boris Schnaiderman. São Paulo: Editora 34, 2000b.

\_\_\_\_\_\_\_\_ . \emph{Os demônios}, Tradução de Paulo Bezerra. São
Paulo: Editora 34, 2004.

\_\_\_\_\_\_\_\_ . \emph{O duplo}. Tradução de Paulo Bezerra. São Paulo:
Editora 34, 2013.

\_\_\_\_\_\_\_\_ . \emph{O idiota}, Tradução de Paulo Bezerra. São
Paulo: Editora 34, 2002.

\_\_\_\_\_\_\_\_ . \emph{Os irmãos Karamázov}. Tradução de Oscar Nunes.
São Paulo: Abril Cultural, 1971.

\_\_\_\_\_\_\_\_ . \_\_\_\_\_\_\_\_ . Tradução de Paulo Bezerra. São
Paulo: Editora 34, 2008.

\_\_\_\_\_\_\_\_ . \emph{O sonho de um homem ridículo}, Tradução de
Vadim Nikitin. São Paulo: Editora 34, 2003.

\_\_\_\_\_\_\_\_ . \_\_\_\_\_\_\_\_ \emph{.} In: \emph{Fiódor
Dostoiévski -- Obra Completa.} Rio de Janeiro: Editora Nova Aguilar,
2004.

\_\_\_\_\_\_\_\_ . \emph{Polnoie sobranie sotchinenii v tridtsati
tomakh}. Leningrado: Ciência, 1985.

\_\_\_\_\_\_\_\_ . \emph{Prestuplenie i nakazanie: kniga dlia tchteniia
s kommentariem.} Moskva: Russkii Iazyk, 1984.

\_\_\_\_\_\_\_\_ . \emph{Recordações da casa dos mortos}. Tradução de
José Geraldo Vieira. Rio de Janeiro: Editora Francisco Alves, 1982.

\_\_\_\_\_\_\_\_ . \emph{Sobranie sotchinenii v desiati tomakh.} Moskva:
Khudojestvennaia Literatura, 1956.

\_\_\_\_\_\_\_\_ . \emph{Son smechnogo tcheloveka -- fantastitcheskii
rasskaz}. Moscou: \versal{OLMA} Media Group, 2008.

\_\_\_\_\_\_\_\_ . \emph{A Paradoxalist.} In: \emph{A Writer's Diary,
Volume 1, 1873-1876}. Translated by Michael Katz. Evanston, Illinois:
Northwestern University Press, 1994, pp. 452-457.

\_\_\_\_\_\_\_\_ . \emph{Zapiski iz miortvovo doma.} Paris: Ymca"-Press,
1945.

\_\_\_\_\_\_\_\_ . \emph{Zapiski iz podpolya.} Saint Petersburg: Azbuka,
2011.

\versal{EVDOKIMOV}, Paul. \emph{L'orthodoxie}. Paris: \versal{DDB}, 1979

\versal{FRANK}, Joseph. \emph{Dostoiévski: as sementes da revolta, 1821-1849.}
Tradução de Vera Pereira. São Paulo: Edusp, 1999.

\_\_\_\_\_\_\_\_ . \emph{Dostoiévski: os anos de provação, 1850-1859.}
Tradução de Vera Pereira. São Paulo: Edusp, 1999.

\_\_\_\_\_\_\_\_ . \emph{Dostoiévski: os efeitos da libertação,
1860-1865}. Tradução de Geraldo Gérson de Souza. São Paulo: Edusp, 2002.

\_\_\_\_\_\_\_\_ . \emph{Dostoiévski: os anos milagrosos, 1865-1871}.
Tradução de Geraldo Gérson de Souza. São Paulo: Edusp, 2003.

\_\_\_\_\_\_\_\_ . \emph{Dostoiévski: o manto do profeta},
\emph{1871-1881}. Tradução de Geraldo Gérson de Souza. São Paulo: Edusp,
2009.

\_\_\_\_\_\_\_\_ . \emph{Dostoevsky: The Mantle of the Prophet,
1871-1881}. New Jersey: Princeton University Press, 2003.

\_\_\_\_\_\_\_\_ . \emph{Pelo prisma russo: ensaios sobre literatura e
cultura}. Tradução de Paula Cox Rolim e Francisco Achcar. São Paulo,
Edusp, 1992.

\versal{GOETHE}, Johann Wolfgang von. \emph{Fausto.} Tradução de Jenny Klabin
Segall. Primeira Parte. São Paulo: Editora 34, 2004.

\_\_\_\_\_\_\_\_ . \_\_\_\_\_\_\_\_ . Segunda Parte. Tradução de Jenny
Klabin Segall. São Paulo: Editora 34, 2007.

\versal{GOMIDE}, Bruno Barretto (Org.). \emph{Antologia do pensamento crítico
russo (1802-1901)}. Vários tradutores. São Paulo: Editora 34, 2013.

\_\_\_\_\_\_\_\_ . \emph{Da estepe à caatinga: o romance russo no Brasil
(1887-1936).} São Paulo: Edusp, 2011.

\versal{GROSSMAN}, Leonid. \emph{Dostoevsky and Balzac.} Translated by Anthony
McGovern. London: Ardis, 1995.

\_\_\_\_\_\_\_\_ . \emph{Dostoiévski artista.} Tradução de Boris
Schnaiderman. Rio de Janeiro: Civilização Brasileira, 1967.

\versal{GRUPO} \versal{KRISIS}. \emph{Manifesto contra o trabalho.} Tradução de Heinz
Dieter Heidemann. Disponível em:
\emph{http://o-beco.planetaclix.pt/mctp.htm}.
Consulta feita no dia 26 de dezembro de 2013.

\versal{HACKEL}, Sergei. ``\versal{F}.\versal{M}. Dostoevsky (1821-1881): prophet manqué?'' In:
\emph{Dostoevsky Studies.} Volume 3. Sem menção a local. 1982, pp. 5-25.

\versal{HARRESS}, Birgit. ``\emph{Besy} als Sendschreiben Dostoevskijs an
Russland''\emph{.} In: \emph{Dostoevsky Studies.} Volume 12. Tübingen:
Attempto Verlag, 2008, pp. 37-50.

\_\_\_\_\_\_\_\_ . ```Glaza nachi vstretilis' -- über die existentielle
Dimension des Blicks in \versal{F}.\versal{M}. Dostoevskijs Erzählung `Krotkaja'"\emph{.}
In: \emph{Dostoevsky Studies.} Volume 4. Tübingen: Attempto Verlag, pp.
117-128.

\_\_\_\_\_\_\_\_ . ``The Renewal of Man: A Poetic Anthropology on
Dostoevsky's Major Novels''\emph{.} In: \emph{Dostoevsky Studies.}
Volume 3. Tübingen: Attempto Verlag, pp. 19-26.

\versal{HEGEL}, Georg Wilhelm Friedrich. \emph{A razão na história: uma
introdução geral à filosofia da história.} Tradução de Beatriz Sidou.
São Paulo: Centauro: 2001.

\_\_\_\_\_\_\_\_ . \emph{Filosofia da História.} Tradução de Maria
Rodrigues. Brasília: Editora da UnB, 1995.

\versal{HERZOG}, Werner. \emph{O enigma de Kaspar Hauser} (Filme). Alemanha:
1974.

\versal{HOLLAND}, Kate. ``The Fictional Filter: ``Krotkaia'' and the \emph{Diary
of a Writer}''. In: \emph{Dostoevsky Studies.} Volume 4. Tübingen:
Attempto Verlag, pp. 95-116.

\versal{HOLQUIST}, Michael. \emph{Dialogism: Bakhtin and his world}. London/New
York: Routledge, 1990.

\_\_\_\_\_\_\_\_ . \emph{Dostoevsky \& the Novel}. Evanston:
Northwestern University Press, 1986.

\versal{HORKHEIMER}, Max. \emph{Anhelo de justicia: teoría crítica e religión.}
Edición y traducción de Juan José Sanchez. Madrid: Editorial Trotta,
2000.

\_\_\_\_\_\_\_\_ . ``Filosofia e Teoria Crítica''. Tradução de Edgard
Afonso Malagodi e Ronaldo Pereira Cunha. In: \emph{Os Pensadores}. São
Paulo: Abril Cultural, 1980a.

\_\_\_\_\_\_\_\_ . ``Teoria Tradicional e Teoria Crítica''\emph{.}
Tradução de Edgard Afonso Malagodi e Ronaldo Pereira Cunha. In: \emph{Os
Pensadores.} São Paulo: Abril Cultural, 1980b.

\versal{JACKSON}, Robert"-Louis. ``Dostoevsky and the Twentieth Century''\emph{.}
In: \emph{Dostoevsky Studies.} Volume 1. Sem menção de lugar. 1980, pp.
3-10.

\_\_\_\_\_\_\_\_ . ``Dostoevsky Today and for All Times''\emph{.} In:
\emph{Dostoevsky Studies.} Volume 6. Tübingen: Attempto Verlag, 2002,
pp. 11-27.

\_\_\_\_\_\_\_\_ . ``Napoleon in Russian Literature''\emph{.} In:
\emph{Yale French Studies,} No. 26, \emph{The Myth of Napoleon,} 1960.

\_\_\_\_\_\_\_\_ . \emph{The Art of Dostoevsky: Deliriums and
Nocturnes}. Princeton: Princeton \versal{UP}, 1981.

\versal{KARDEC}, Allan. \emph{Obras póstumas.} Tradução de Salvador Gentile. São
Paulo: \versal{IDE}, 2008.

\_\_\_\_\_\_\_\_ . \emph{O céu e o inferno.} Tradução de Manuel
Justiniano Quintão. Rio de Janeiro: Federação Espírita Brasileira, 1944.

\_\_\_\_\_\_\_\_ . \emph{O evangelho segundo o espiritismo.} Tradução de
Júlio Abreu Filho. São Paulo: Cultrix, 2004.

\_\_\_\_\_\_\_\_ . \emph{O livro dos espíritos.} Tradução de Salvador
Gentile. São Paulo: \versal{IDE}, 2009.

\versal{KELLY}, Aillen. ``Dostoevskii and the Divided Conscience''. In:
\emph{Slavic Review,} Vol. 47, No. 2, Summer, 1988.

\versal{KURZ}, Robert. \emph{Com todo valor ao colapso.} Tradução de Marcos
Aquino. Juiz de Fora: \versal{UFJF}, Pazulin, 2004.

\_\_\_\_\_\_\_\_ . \emph{Manifesto contra o trabalho}. Tradução de Heinz
Dieter Heidemann. Link:
\emph{http://www.krisis.org/1999/manifesto-contra-o-trabalho}.
Consulta feita em 15/07/15.

\_\_\_\_\_\_\_\_ . \emph{O colapso da modernização: da derrocada do
socialismo de caserna à crise da economia mundial.} Tradução de Karen
Elsabe Barbosa. Rio de Janeiro: Paz e Terra, 2004.

\_\_\_\_\_\_\_\_ . \emph{Os últimos combates.} Tradução de Karen Elsabe
Barbosa. Petrópolis: Vozes, 2004

\versal{LUKÁCS}\emph{,} György. \emph{A Teoria do romance}. Tradução de José
Marcos Mariani de Macedo. São Paulo: Livraria Duas Cidades/Editora 34,
2000.

\_\_\_\_\_\_\_\_ . \emph{Dostoevsky}, in \emph{Dostoevsky: A collection
of critical essays}. Translated by Michael Katz. New Jersey:
Prentice"-Hall, 1962.

\versal{MAN}, Paul de. ``Dialogue and Dialogism''\emph{.} In: \emph{Rethinking
Bakhtin: Extensions and Challenges}. Edited by Gary Saul Morson and
Caryl Emerson. Evanston, Illinois: Northwestern University Press, 1989,
pp. 105-114.

\versal{MARTINSEN}, Deborah. ``Ingratitude and the Underground''\emph{.} In:
\emph{Dostoevsky Studies.} Volume 17. Tübingen: Attempto Verlag, 2013,
pp. 7-22.

\_\_\_\_\_\_\_\_ . ``Shame and Punishment''\emph{.} In: \emph{Dostoevsky
Studies.} Volume 5. Tübingen: Attempto Verlag, 2001, pp. 51-70.

\versal{MARX}, Karl. \emph{A ideologia alemã}. Tradução de Luís Cláudio de Castro
e Costa. São Paulo: Martins Fontes, 1998.

\_\_\_\_\_\_\_\_ . \emph{Contribuição à crítica do Direito de Hegel}.
Tradução de Alex Marins. In: \emph{Manuscritos Econômico"-Filosóficos}.
São Paulo: Martin Claret, 2002.

\_\_\_\_\_\_\_\_ . \emph{O capital}. Volume 1. Tradução de Flávio Kothe.
São Paulo: Nova Cultural, 1998.

\_\_\_\_\_\_\_\_ . \emph{Sobre a questão judaica.} Tradução de Nélio
Schneider e Wanda Nogueira Caldeira Brant. São Paulo: Boitempo, 2010.

Mc\versal{REYNOLDS}, Susan. ``Aesthetics and Politics: The Case of Dostoevsky''.
In: \emph{Literary Imagination}~(2002)~4~(1):~91-104.

\_\_\_\_\_\_\_\_ . ``Dostoevsky in Europe: The Political as the
Spiritual''\emph{.} In: \emph{Partisan Review} 1. 2002.

\_\_\_\_\_\_\_\_ . ``Introduction to Dostoevsky and Christianity''. In:
\emph{The Journal of the International Dostoevsky Society}, Vol. 13.
Tübingen: Attempto Verlag, 2009, pp. 5-22.

\_\_\_\_\_\_\_\_ . ``Schillerian Aesthetic Humanism in \emph{Vremia}''.
In: \emph{Dostoevsky Studies.} Volume 7. Tübingen: Attempto Verlag,
2003, pp. 81-88.

\versal{MONTEFIORE}, Simon Sebag. \emph{Stálin: A corte do tzar Vermelho}.
Tradução de Pedro Maia Soares. São Paulo: Companhia das Letras, 2006.

\versal{MORSON}, Gary Saul. ``Paradoxical Dostoevsky''. In: \emph{The Slavic and
East European Journal,} Vol. 43, No. 3, Autumn, 1999.

\_\_\_\_\_\_\_\_ . ``Rethinking Bakhtin: Extensions and
Challenges''\emph{.} Evanston, Illinois: Northwestern University Press,
1986.

\versal{NATOV}, Nadine. ``Dostoevsky versus Max Stirner''\emph{.} In:
\emph{Dostoevsky Studies.} Volume 6. Tübingen: Attempto Verlag, 2002,
pp. 28-38.

\versal{NEUHÄUSER}, Rudolf. ```The Dream of a Ridiculous Man': Topicality as a
Literary Device''\emph{.} In: \emph{Dostoevsky Studies.} New Series.
Volume 1, Number 2. Sem menção a lugar. Charles Schlacks, Jr.,
Publisher, 1993, pp. 175-190.

\versal{NIETZSCHE}, Friedrich. \emph{Escritos sobre história.} Tradução de Noéli
Correa de Melo Sobrinho. Rio de Janeiro: Editora da \versal{PUC}-Rio/Edições
Loyola, 2005.

\versal{PARTS}, Lyudmila. ``Christianity as Active Pity in \emph{Crime and
Punishment}''. In: \emph{Dostoevsky Studies.} Volume 13. Tübingen:
Attempto Verlag, 2009, pp. 61-76.

\versal{PEACE}, Richard. ``Dostoevsky and The Golden Age''. In: \emph{Dostoevsky
Studies.} Volume 3. Sem menção a local. 1982, pp. 61-78.

\emph{Pocket Oxford Russian Dictionary}. Oxford University Press, Nova
Iorque, 2006.

\versal{PONDÉ}, Luiz Felipe. \emph{Crítica e profecia: a filosofia da religião em
Dostoiévski}. São Paulo: Editora 34, 2003.

\versal{REICH}, Wilhelm. \emph{Psicologia de massas do fascismo.} Tradução de
Francisco Couto Fernandes. São Paulo: Martins Fontes, 1988.

\versal{SABO}, Gerald. \versal{J}.. ```The Dream of a Ridiculous Man': Christian Hope for
Human Society''\emph{.} In: \emph{The Journal of the International
Dostoevsky Society}, Vol. 13, Attempto Verlag, Tübingen, 2009.

\versal{SCHMID}, Wolf. ``Dostoevskijs Erzähltechnik in narratologischer
Sicht''\emph{.} In: \emph{Dostoevsky Studies.} Volume 6. Tübingen:
Attempto Verlag, pp. 63-72.

\versal{SCHNAIDERMAN}, Boris. ``A arte da superação''. In: \emph{Caderno Mais!}.
São Paulo: Folha de S.Paulo, 28 de outubro de 2001.

\_\_\_\_\_\_\_\_ . ``Bakhtin e o Ocidente -- etapas de uma
aproximação''. Prefácio. In: \versal{CLARK}, Katerina e \versal{HOLQUIST}, Michael,
\emph{Mikhail Bakhtin}. São Paulo: Perspectiva, 1998.

\_\_\_\_\_\_\_\_ . ``Bakhtin 40 graus (uma experiência brasileira)''.
In: B. Brait (org.), \emph{Bakhtin, dialogismo e construção do sentido}.
Campinas: Editora da Unicamp, 1997.

\_\_\_\_\_\_\_\_ . ``Deus e nada mais''. In \emph{Caderno Mais!}. São
Paulo: Folha de S.Paulo, 30 de novembro de 2003.

\_\_\_\_\_\_\_\_ . ``Dostoiévski: a ficção como pensamento''. In:
\emph{Artepensamento}. A. Novaes (org.). São Paulo: Companhia das
Letras, 1994.

\_\_\_\_\_\_\_\_ . \emph{Dostoiévski: prosa e poesia}. São Paulo:
Perspectiva, 1982.

\_\_\_\_\_\_\_\_ . ``Os paradoxos políticos de um gigante do
pensamento''. In: \emph{Caderno 2}, O Estado de S.Paulo, São Paulo, 26
de março de 2000.

\_\_\_\_\_\_\_\_ . ``Para Boris Schnaiderman, autor é o
`escritor"-filósofo por excelência'". Entrevista. In: \emph{Caderno
Mais!}. São Paulo: Folha de São Paulo, 6 de maio de 2001.

\_\_\_\_\_\_\_\_ . \emph{Turbilhão e Semente: ensaios sobre Dostoiévski
e Bakhtin}. São Paulo: Duas Cidades, 1983.

\versal{SCHWARZ}, Roberto. \emph{Ao vencedor as batatas: forma literária e
processo social nos inícios do romance brasileiro.} São Paulo: Duas
Cidades, 1977.

\_\_\_\_\_\_\_\_ . \emph{Um mestre na periferia do capitalismo: Machado
de Assis}. São Paulo: Duas Cidades/Editora 34, 2000.

\versal{SHESTOV}, Lev. \emph{Dostoevsky and Nietzsche: The Philosophy of
Tragedy}. Translated by Michael Katz. Ohio: Ohio University Press, 1969.

\versal{STEPÂNIAN}, Karen. ``Os irmãos Karamázov: a hosana de Dostoiévski''. In:
\emph{Caderno de Literatura e Cultura Russa: Dostoiévski}. São Paulo:
Ateliê Editorial, 2008.

\versal{TERRAS}, Victor. ``Dostoevsky's Detractors''\emph{.} In: \emph{Dostoevsky
Studies.} Volume 6. Knoxville, Tennessee: University of Tennessee, 1985,
pp. 165-172.

\_\_\_\_\_\_\_\_ . ``Problems of Human Existence in the Works of the
Young Dostoevsky''\emph{.} In: \emph{Slavic Review,} Vol. 23, No. 1,
March, 1964.

\_\_\_\_\_\_\_\_ . ``Religion and Poetics in Dostoevsky''\emph{.} In:
\emph{Dostoevsky Studies.} Volume 7. Tübingen: Attempto Verlag, pp.
109-114.

\versal{VASSOLER}, Flávio Ricardo (Org.). \emph{Dostoiévski e Bergman: o niilismo
da modernidade.} São Paulo: Intermeios, 2012.

\_\_\_\_\_\_\_\_ . \emph{O evangelho segundo talião.} São Paulo:
nVersos, 2013.

\_\_\_\_\_\_\_\_ . ``Patíbulo e perdão''\emph{.} In: \emph{Revista
Cult}, nº. 200, São Paulo, abril de 2015, pp. 50-55.

\_\_\_\_\_\_\_\_ . \emph{Tiro de Misericórdia.} São Paulo: nVersos,
2014.

\_\_\_\_\_\_\_\_ . ``Nem tudo o que é sólido desmancha no ar''\emph{.}
In: \emph{Jornal Rascunho}, nº. 159, Curitiba, julho de 2013, pp. 20-21.

\versal{VÓLGUIN}, Igor. ``A devolução do bilhete: paradoxos da autoconsciência
nacional''\emph{.} In: \emph{Caderno de Literatura e Cultura Russa:
Dostoiévski}. São Paulo: Ateliê Editorial, 2008.

\versal{WALICKI}, Andrzej. \emph{A History of Russian Thought from the
Enlightenment to Marxism}. Stanford: Stanford University Press, 1979.

\versal{WELLEK}, René. ``Bakhtin's view of Dostoevsky: `Polyphony' and
`Carnivalesque'"\emph{.} In: \emph{Dostoevsky Studies.} Volume 1. Sem
menção a lugar. 1980, pp. 31-40.

\_\_\_\_\_\_\_\_ . ``Introduction: A history of Dostoevsky criticism''.
In \emph{Dostoevsky: A collection of critical essays}. New Jersey:
Prentice"-Hall, 1962.

\versal{WILDE}, Oscar. \emph{A alma do homem sob o socialismo.} Tradução de
Heitor Ferreira da Costa. Porto Alegre: \versal{L}\&\versal{PM} Pocket, 2003.

\_\_\_\_\_\_\_\_ . \emph{Aforismos ou Mensagens Eternas}. Tradução de
Duda Machado. São Paulo: Landy Editora, 2006.

\end{Parskip}

